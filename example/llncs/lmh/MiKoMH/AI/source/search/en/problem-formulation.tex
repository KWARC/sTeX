\documentclass[notes,mh]{mikoslides}
\libinput{preamble}
\begin{document}
\begin{module}[id=problem-formulation]
\begin{frame}
  \frametitle{Problem Formulation}
  \begin{itemize}
  \item
    \begin{definition}[display=flow]
      A \defii{problem}{formulation} models a situation using \defis{state} and
      \defis{action} at an appropriate level of abstraction.\lec{do not model things like
        ``put on my left sock'', etc.}
      \begin{itemize}
      \item it describes the \defii{initial}{state}\lec{we are in Arad}
      \item it also limits the objectives by specifying \defiis{goal}{state}.
        \lec{excludes, e.g. to stay another couple of weeks.}
      \end{itemize}
      A \defi{solution} is a sequence of \trefis{action} that leads from the
      \trefii{initial}{state} to a \trefii{goal}{state}.

      \Defii{problem}{solving} computes \trefis{solution} from
      \trefiis{problem}{formulation}.
    \end{definition}
  \item Finding the right level of abstraction and the required (not more!) information is
    often the key to success.
  \end{itemize}
\end{frame}
\end{module}
\end{document}

%%% Local Variables:
%%% mode: latex
%%% TeX-master: t
%%% End:
