\documentclass[notes,mh]{mikoslides}
\libinput{preamble}
\begin{document}
\begin{module}[id=planning-task]
\importmhmodule[dir=planning/en]{planning-vs-search}
\importmhmodule[dir=search/en]{search-problem-detenv}

\begin{frame}
  \frametitle{How does a planning language describe a problem?}
  \begin{itemize}
  \item
    \begin{definition}
      A \defii{planning}{language} is a logical language for the components of a
      \trefii[search-problem]{search}{problem}; in particular a \emph{logical description}
      of the 
      \begin{itemize}
      \item<1-> possible \trefis[search-problem]{state} (vs.\ blackbox: data
        structures). \lec{E.g.: predicate $Eq(.,.)$.}
      \item<2-> \trefii[search-problem]{initial}{state} $I$ (vs.\ data structures).
        \lec{E.g.: $Eq(x,1)$.}
      \item<3-> \trefii[search-problem-detenv]{goal}{test} $G$ (vs.\ a goal-test
        function). \lec{E.g.: $Eq(x,2)$.}
      \item<4-> set $A$ of \trefis[search-problem]{action} in terms of
        \defis{precondition} and \defis{effect} (vs.\ functions returning applicable
        actions and successor states).  \lec{E.g.: ``increment $x$: pre $Eq(x,1)$, eff
          $Eq(x,2) \land \lnot Eq(x,1)$''.}
      \end{itemize}
      A logical description of all of these is called a \defii{planning}{task}.
    \end{definition}
  \item
    \begin{definition}
      Solution (\defi{plan}) \hateq sequence of actions from $A$, transforming $I$ into a
      state that satisfies $G$. \lec{E.g.: ``increment $x$''.}
    \end{definition}
  \end{itemize}
\end{frame}
\end{module}
\end{document}

%%% Local Variables:
%%% mode: latex
%%% TeX-master: t
%%% End:

%  LocalWords:  search-problem-detenv lnot
