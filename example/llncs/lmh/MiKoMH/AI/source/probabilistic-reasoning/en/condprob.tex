\documentclass[notes,mh]{mikoslides}
\libinput{preamble}
\begin{document}
\begin{module}[id=condprob]
\importmhmodule[dir=probabilistic-reasoning/en]{unconditional-prob}
\symdef[name=CondProb]{CondProbOp}{P}
\symdef{CondProb}[2]{\mixfixii[p=800,pi=\niprec,pii=\niprec]{\CondProbOp(}{#1}\mid{#2})}

\begin{frame}
  \frametitle{Conditional Probabilities: Definition}
  \begin{itemize}
  \item 
    \begin{definition}
      Given propositions $A$ and $B$ where $\notequal{\uProb{b}}0$, the
      \defii{conditional}{probability}, or
      \defii[name=conditional-probability]{posterior}{probability}, of $a$ given $b$, written
      $\CondProb{a}b$, is defined as:
      \[\fundefeq{a,b}{\CondProb{a}b}{\frac{\uProb{\lconj{a,b}}}{\uProb{b}}}\]
    \end{definition} 
  \item 
    \begin{omtext}[title=Intuition]
      The likelihood of having $a$ and $b$, within the set of outcomes where we have $b$.
    \end{omtext}
  \item 
    \begin{example}
      $\uProb{\lconj{\rvval{cavity},\rvval{toothache}}} = 0.12$ and
      $\uProb{\rvval{toothache}} = 0.2$ yield
      $\CondProb{\rvval{cavity} }{\rvval{toothache}} = 0.6$.
    \end{example}
  \end{itemize}
\end{frame}
\end{module}
\end{document}

%%% Local Variables:
%%% mode: latex
%%% TeX-master: t
%%% End:

%  LocalWords:  condprob importmhmodule gimport smglom realarith symdef CondProbOp neq
%  LocalWords:  mixfixii frametitle itemize defii fundefeq a,b lconj a,b omtext mathit
%  LocalWords:  mathit 800,pi niprec,pii niprec rvval rvval
