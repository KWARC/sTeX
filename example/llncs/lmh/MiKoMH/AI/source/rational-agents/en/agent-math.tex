\documentclass[notes,mh]{mikoslides}
\libinput{preamble}
\begin{document}
\begin{module}[id=agent-math]
\importmhmodule[dir=rational-agents/en]{agentenv}
\gimport[smglom/sets]{words}
\symdef{percepts}{\mathcal{P}}
\symdef{agentactions}{\mathcal{A}}
\symdef{agentfunction}[1]{f_{#1}}

\begin{frame}[label=slide.agent-math]
  \frametitle{Modeling Agents Mathematically and Computationally}
  \begin{itemize}
  \item 
    \begin{definition}
      A \defi{percept} is the perceptual input of an \trefi[agentenv]{agent} at a specific instant. 
    \end{definition}
  \item 
    \begin{definition}
      Any recognizable, coherent employment of the \trefis[agentenv]{actuator} of an
      \trefi[agentenv]{agent} is called an \defi{action} .
    \end{definition}
  \item
    \begin{definition}
      The \defii{agent}{function} $\agentfunction{a}$ of an agent $a$ maps
      from \trefi{percept} histories to \trefis{action}:
      \[\fun{\agentfunction{a}}{\kleenestar\percepts}\agentactions\]
    \end{definition}
  \item We assume that \trefis[agentenv]{agent} can always perceive their own
    \trefis{action}.\lec{but not necessarily their consequences} 
  \item 
    \begin{omtext}[title=Problem]
      \trefiis{agent}{function} can become very big\lec{theoretical tool only}
    \end{omtext}
  \item
    \begin{definition}
      An \trefii{agent}{function} can be implemented by an \defii{agent}{program} that
      runs on a physical \defii{agent}{architecture}.
    \end{definition}
  \end{itemize}
\end{frame}
\end{module}

\end{document}
