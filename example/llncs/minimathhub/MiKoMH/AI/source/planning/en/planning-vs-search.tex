\documentclass[notes,mh]{mikoslides}
\libinput{preamble}
\begin{document}
\begin{module}[id=planning-vs-search]
\importmhmodule[dir=search/en]{search-problem}
\gimport[smglom/computing]{API}

\begin{frame}
  \frametitle{Planning}
  \begin{itemize}
  \item
    \begin{omtext}[title=Ambition]
      Write one program that can solve all classical
      \trefiis[search-problem]{search}{problem}.
    \end{omtext}
  \item
    \begin{definition}
      Let $\Pi$ be a \trefii[search-problem]{search}{problem}
      \lec{see \sref[fallback=the search chapter]{sec.search}}
      \begin{itemize}
      \item The \defii{blackbox}{description} of $\Pi$ is an \trefi[API]{API} providing
        functionality allowing to construct the state space: $\text{InitialState}()$,
        $\text{GoalTest}(s)$, \dots
        \begin{itemize}
        \item ``Specifying the problem'' \hateq programming the \trefi[API]{API}.
        \end{itemize}
      \item The \defii{declarative}{description} of $\Pi$ comes in a
        \defiii{problem}{description}{language}. This allows to implement the
        \trefi[API]{API}, and much more.
        \begin{itemize}
        \item ``Specifying the problem'' \hateq writing a problem description.
        \end{itemize}
      \end{itemize}
    \end{definition}
  \item Here, ``\trefiii{problem}{description}{language}'' \hateq planning language.
  \end{itemize}
\end{frame}
\end{module}
\end{document}

%%% Local Variables:
%%% mode: latex
%%% TeX-master: t
%%% End:
