\documentclass[notes,mh]{mikoslides}
\libinput{preamble}
\begin{document}
\begin{module}[id=uncert-logic]
\importmhmodule[mhrepos=MiKoMH/CompLog,dir=pl1/en]{pl1-syntax}
\symdef[name=Symptom]{SymptomOp}{\text{Symptom}}
\symdef{Symptom}[2]{\prefix\SymptomOp{#1,#2}}
\symdef[name=Disease]{DiseaseOp}{\text{Disease}}
\symdef{Disease}[2]{\prefix\DiseaseOp{#1,#2}}
\symdef{toothache}{\text{toothache}}
\symdef{cavity}{\text{cavity}} \symdef{gumdisease}{\text{gingivitis}}

\begin{frame}[label=slide.uncert-logic]
  \frametitle{Uncertainty and Logic}
  \begin{itemize}
  \item
    \begin{omtext}[title=Diagnosis]
      We want to build an expert dental diagnosis system, that deduces the cause (the
      disease) from the symptoms.
    \end{omtext}
  \item Can we base this on logic? \pause
  \item
    \begin{omtext}[title=Attempt 1]
      Say we have a toothache. How's about:
      \[\allcdot{p}{\limpl{\Symptom{p}{\toothache}}{\Disease{p}{\cavity}}}\]
      \begin{itemize}
      \item Is this rule correct? \pause
      \item No, toothaches may have different causes (``cavity'' = ``Loch im Zahn'').
      \end{itemize}
    \end{omtext}
  \item
    \begin{omtext}[title=Attempt 2]
      So what about this:
      \[\allcdot{p}{\limpl{\Symptom{p}{\toothache}}
          {\ldisj{\Disease{p}{\cavity},\Disease{p}{\gumdisease},\ldots}}}
      \]
      \begin{itemize}
      \item We don't know all possible causes.
      \item And we'd like to be able to deduce which causes are more plausible!
      \end{itemize}
    \end{omtext}
  \end{itemize}
\end{frame}

\begin{frame}[label=slide.uncert-logic2]
  \frametitle{Uncertainty and Logic, ctd.}
  \begin{itemize}
  \item
    \begin{omtext}[title=Attempt 3]
      Perhaps a \blue{causal} rule is better?
      \[\allcdot{p}{\limpl{\Disease{p}{\cavity}}{\Symptom{p}{\toothache}}}\]
    \end{omtext}
  \item
    \begin{omtext}[title=Question]
      Is this rule correct? \pause
    \end{omtext}
  \item
    \begin{omtext}[title=Answer]
      No, not all cavities cause toothaches.
    \end{omtext}
  \item
    \begin{omtext}[title=Question]
      Does this rule allow to deduce a cause from a symptom?  \pause
    \end{omtext}
  \item
    \begin{omtext}[title=Answer]
      No, setting $\Symptom{p}{\toothache}$ to true here has no consequence on the truth
      of $\Disease{p}{\cavity}$. \pause
    \end{omtext}
  \item
    \begin{omtext}[title=Note]
      If $\Symptom{p}{\toothache}$ is \emph{false}, we would conclude
      $\lneg{\Disease{p}{\cavity}}$ \dots which would be incorrect, cf.  previous
      question.  \pause
    \end{omtext}
  \item Anyway, this still doesn't allow to compare the plausibility of different causes.
  \item Logic does not allow to weigh different alternatives, and it does not allow to
    express incomplete knowledge (``cavity does not always come with a toothache, nor vice
    versa'').
  \end{itemize}
\end{frame}
\end{module}
\end{document}

%%% Local Variables:
%%% mode: latex
%%% TeX-master: t
%%% End:

% LocalWords:  uncert-logic importmhmodule CompLog,path pl1-syntax symdef gumdisease ctd
%  LocalWords:  frametitle itemize textbf allcdot limpl ldisj ldots emph lneg CompLog,dir

