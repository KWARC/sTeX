\documentclass[notes,mh]{mikoslides}
\libinput{preamble}
\begin{document}
\begin{module}[id=unconditional-prob]
\importmhmodule[dir=probabilistic-reasoning/en]{uncert-logic}
\importmhmodule[mhrepos=MiKoMH/GenCS,dir=logic/en]{semantics-base}
\gimport[smglom/probability]{random-variable}

\symdef{rvval}[1]{\text{#1}}
\symdef{rvvalset}[1]{\mathbf{#1}}
\symdef{randvar}[1]{\text{#1}}
\symdef{randvarset}[1]{\mathbf{#1}}
\symdef[name=uProb]{uProbOp}{P} 
\symdef{uProb}[1]{\prefix[p=800]\uProbOp{#1}}

\begin{frame}
  \frametitle{Unconditional Probabilities, Random Variables, and Events}
  \begin{itemize}
    \item 
      \begin{definition}
        A \drefii[random-variable]{random}{variable} (also called
        \drefii[random-variable?random-variable]{random}{quantity},
        \drefii[random-variable?random-variable]{aleatory}{variable}, or
        \drefii[name=random-variable?random-variable]{stochastic}{variable}) is a variable
        quantity whose \defi{value} depends on possible outcomes of unknown
        \trefis[random-variable?random-variable]{variable} and processes we do not
        understand.
      \end{definition}
  \item
    \begin{definition} 
      If $X$ is a \trefii{random}{variable} and $x$ a possible \trefi{value}, we will
      refer to the fact $X = x$ as an \defi{outcome} and a set of \trefis{outcome} as an
      \defi{event}. The set of possible \trefis{outcome} of $X$ is called the
      \defi{domain} of $X$.
    \end{definition}
  \item The notation \nlex{uppercase ``$X$''} for a \trefii{random}{variable}, and
    \red{lowercase ``$x$''} for one of its values will be used frequently. \lec{Follows
      Russel/Norvig}
  \item
    \begin{definition}
      Given a \trefii{random}{variable} $X$, $\uProb{X = x}$ denotes the
      \defii[name=unconditional-probability]{prior}{probability}, or
      \defii{unconditional}{probability}, that $X$ has value $x$ in the absence of any
      other information.
    \end{definition}
  \item
    \begin{module}[id=cavity-ex]
      \begin{example}
        $\uProb{\randvar{Cavity}=\semtrue} = 0.2$, where $\randvar{Cavity}$ is a random
        variable whose value is true iff some given person has a cavity.
      \end{example}
    \end{module}
  \end{itemize}
\end{frame}

\begin{frame}
  \frametitle{Types of Random Variables}
  \begin{itemize}
  \item
    \begin{definition}
      We say that a \trefii[unconditional-prob]{random}{variable} $X$ is
      \defi{finite-domain}, iff the \trefi{domain} $D$ of $X$ is
      \trefi[finite-cardinality]{finite} and \defi{Boolean}, iff
      $D=\set{\semtrue,\semfalse}$.
    \end{definition}
  \item 
    \begin{omtext}[title=Note]
      In general, \trefiis{random}{variable} can have arbitrary \trefis{domain}. In
      \useSGvar{courseacronym}, we restrict ourselves to \trefi{finite-domain} and
      \trefi{Boolean} \trefiis[unconditional-prob]{random}{variable}.
    \end{omtext}
  \item
    \begin{module}[id=weather-ex]
      \symdef{sunny}{\rvval{sunny}} \symdef{rain}{\rvval{rain}}
      \symdef{cloudy}{\rvval{cloudy}} \symdef{snow}{\rvval{snow}}
      \begin{example}
        Some \trefiis[?unconditional-probability]{prior}{probabilities}
        \begin{align*}
          \uProb{\randvar{Weather}=\sunny}      &= 0.7 \\
          \uProb{\randvar{Weather}=\rain}       &= 0.2 \\
          \uProb{\randvar{Weather}=\cloudy}     &= 0.08 \\
          \uProb{\randvar{Weather}=\snow}       &= 0.02 \\
          \uProb{\randvar{Headache}=\semtrue}      &= 0.1
        \end{align*}
        Unlike us, Russel and Norvig live in California \dots\ :-( :-(
      \end{example}
    \end{module}
  \item
    \begin{omtext}[title=Convenience Notations]
      \begin{itemize}
      \item By convention, we denote \trefi{Boolean}
        \trefiis[unconditional-prob]{random}{variable} with $A$, $B$, and more general
        \trefi{finite-domain} \trefiis[unconditional-prob]{random}{variable} with $X$,
        $Y$.
      \item For a \trefi{Boolean} \trefii[unconditional-prob]{random}{variable}
        $\randvar{Name}$, we write $\rvval{name}$ for the
        \trefi[unconditional-prob]{outcome} $\randvar{Name} = \semtrue$ and
        $\lneg{\rvval{name}}$ for $\randvar{Name} = \semfalse$. \lec{Follows
        Russel/Norvig as well}
    \end{itemize}
  \end{omtext}
\end{itemize}
\end{frame}
\end{module}
\end{document}

%%% Local Variables:
%%% mode: latex
%%% TeX-master: t
%%% End:
