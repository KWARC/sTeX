\documentclass[notes,mh]{mikoslides}
\libinput{preamble}
\begin{document}
\begin{module}[id=envtypes]
\importmhmodule[dir=rational-agents/en]{agent-math}
\begin{frame}
  \frametitle{Environment types}
  \begin{itemize}
  \item 
    \begin{assertion}[type=observation]
      Agent design is largely determined by the type of \trefi[agentenv]{environment} it
      is intended for. 
    \end{assertion}
  \item 
    \begin{omtext}[title=Problem]
      There is a vast number of possible kinds of \trefis[agentenv]{environment} in AI.
    \end{omtext}
  \item 
    \begin{omtext}[title=Solution]
      Classify along a few ``dimensions'' \lec{independent characteristics}
    \end{omtext}
  \item 
    \begin{definition}
      For an agent $a$ we classify the \trefi[agentenv]{environment} $e$ of $a$ by its
      \defi{type}, which is one of the following. We call $e$
      \begin{enumerate}
      \item \defii{fully}{observable}, iff the $a$'s sensors give it access to the
        complete state of the \trefi[agentenv]{environment} at any point in time, else
        \defii{partially}{observable}.
      \item \defi{deterministic}, iff the next state of the \trefi[agentenv]{environment} is completely
        determined by the current state and $a$'s \trefi[agent-math]{action}, else \defi{stochastic}.
      \item \defi{episodic}, iff $a$'s experience is divided into atomic \defis{episode},
        where it perceives and then performs a single
        \trefi[agent-math]{action}. Crucially the next episode does not depend on previous
        ones. Non-\trefi{episodic} \trefis[agentenv]{environment} are called
        \defi{sequential}.
      \item \defi{dynamic}, iff the \trefi[agentenv]{environment} can change without an
        \trefi[agent-math]{action} performed by $a$, else \defi{static}. If the
        \trefi[agentenv]{environment} does not change but $a$'s performance measure does,
        we call $e$ \defi{semidynamic}.
      \item \defi{discrete}, iff the sets of $e$'s state and $a$'s
        \trefis[agent-math]{action} are countable, else \defi{continuous}.
      \item \defi{single-agent}, iff only $a$ acts on $e$; else \defi{multi-agent}\lec{when must we count parts of
          $e$ as agents?}
      \end{enumerate}
    \end{definition}
  \end{itemize}
\end{frame}

\begin{nomtext}
  Some examples will help us understand this better.
\end{nomtext} 

\begin{frame}[label=slide.envtypes-ex]
  \frametitle{Environment Types (Examples)}
  \begin{itemize}
  \item 
    \begin{example}
      Some \trefis[agentenv]{environment} classified: 
      \begin{center}
        \begin{tabular}{|@{\extracolsep\fill}|@{\quad}l@{\quad}|cccc@{\quad}|}\hline
          & {Solitaire} & {Backgammon} & {Internet shopping} & {Taxi} \\\hline
          \trefii{fully}{observable}   & Yes  & Yes  & No  & No \\
          \trefi{deterministic}  & Yes  & No  & Partly  & No \\
          \trefi{episodic}     & No  & No  & No  & No \\
          \trefi{static}           & Yes  & Semi  & Semi  & No \\
          \trefi{discrete}     & Yes  & Yes  & Yes  & No \\
          \trefi{single-agent} & Yes & No & Yes (except auctions) & No \\\hline
        \end{tabular}
      \end{center}
    \end{example}
  \item
    \begin{assertion}[type=observation]
      The real world is (of course) \trefii{partially}{observable}, \trefi{stochastic},
      \trefi{sequential}, \trefi{dynamic}, \trefi{continuous}, and 
      \trefi{multi-agent}\lec{worst case for AI}
    \end{assertion}
  \end{itemize}
\end{frame}

\begin{nomtext}
  In the \useSGvar{courseacronym} course we will work our way from the simpler
  environment types to the more general ones. Each environment type wil need its own agent
  types specialized to surviving and doing well in them.
\end{nomtext}
\end{module}
\end{document}

%  LocalWords:  envtypes semidynamic extracolsep
