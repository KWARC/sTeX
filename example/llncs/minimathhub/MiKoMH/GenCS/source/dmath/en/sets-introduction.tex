\documentclass[notes,mh]{mikoslides}
\libinput{preamble}
\begin{document}
\begin{module}[id=sets-introduction]
\gimport[smglom/sets]{set}

\begin{frame}
  \frametitle{Understanding Sets}
  \begin{itemize}
  \item Sets are one of the foundations of mathematics,
  \item and one of the most difficult concepts to get right axiomatically
  \item
    \begin{omtext}[id=cantor-set.def,title=Early Definition Attempt]
      \usemhmodule[mhrepos=MiKoMH/KWARC,dir=missing/en]{dates}

      A set is ``everything that can form a unity in the face of God''. \lec{Georg Cantor
        ($\born{1845}$, $\died{1918}$)}
    \end{omtext}
  \item For this course: no definition; just intuition\lec{naive set theory}
  \item To understand a set $S$, we need to determine, what is an element of $S$
    and what isn't.
  \item
    \begin{definition}[display=flow,for={set,setdots,nsete,nsetli,nsetui,setst,inset,ninset,minset,minsetdots,nminset,nminsetdots,hasprop},
      title=Representations of Sets]
      We can represent \drefis[set?set]{set} by 
      \begin{itemize}
      \item listing the elements within curly brackets: e.g. \notatiendum{$\set{a,b,c}$}
      \item describing the elements via a property: \notatiendum{$\setst{x}{\hasprop{x}P}$}
      \item stating element-hood (\notatiendum{$\inset{a}{S}$}) or not
        (\notatiendum{$\ninset{b}{S}$}).
      \end{itemize}
    \end{definition}
  \item 
    \begin{axiom}[id=set-comprehension.ax]
      Every set we can write down actually exists!\lec{Hidden Assumption}
    \end{axiom}
  \item
    \begin{omtext}[title=Warning]
      Learn to distinguish between objects and their representations!\lec{$\set{a,b,c}$
        and $\set{b,a,a,c}$ are different representations of the same set}
    \end{omtext}
  \end{itemize}
\end{frame}

\begin{note}
\begin{omtext}
  Indeed it is very difficult to define something as foundational as a set. We want sets
  to be collections of objects, and we want to be as unconstrained as possible as to what
  their elements can be. But what then to say about them? Cantor's intuition is one
  attempt to do this, but of course this is not how we want to define concepts in math.

  $\bsetst[stretchy]{A}A{\begin{array}{c}a\\b\\b\end{array}}$

  So instead of defining sets, we will directly work with representations of sets. For
  that we only have to agree on how we can write down sets. Note that with this practice,
  we introduce a hidden assumption: called \inlinedef{\defii{set}{comprehension},
    i.e. that every set we can write down actually exists}. We will see below that we
  cannot hold this assumption.
\end{omtext}
\end{note}
\end{module}

\end{document}
% LocalWords:  en setst insetFN ninset minset nminset def Georg stex symdef lec
% LocalWords: assocarg mixfixa nobrackets setdots ldots nsete defi
% LocalWords:  mixfixai nsetli mixfixii nsetui setstnl niprec minsetdots omtext
% LocalWords:  nminsetdots hasprop frametitle notatiendum inlinedef defii
% LocalWords:  gimport itemize
