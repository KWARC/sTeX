\documentclass[notes,mh]{mikoslides}
\libinput{preamble}
\begin{document}
\begin{module}[id=mathtalk]
\gimport[smglom/mv]{mv}

\begin{frame}[label=slide.mathtalk]
  \frametitle{Talking about Mathematics (MathTalk)}
  \begin{itemize}
 \item
    \begin{definition}[id=math-vernacular.def]
      Mathematicians use a stylized language that
      \begin{itemize}
      \item uses formulae to represent mathematical objects, e.g.
      \inlineex{\guse[smglom/calculus]{definiteintegral}$\defint01{\realpower{x}{\realdivide32}}{x}$}
     \item uses \defiis{math}{idiom} for special situations\lec{e.g. \nlex{iff},
          \nlex{hence}, \nlex{let\ldots be\ldots, then\ldots}}
      \item classifies statements by role\lec{e.g. Definition, Lemma, Theorem, Proof, Example} 
      \end{itemize}
      We call this language \defii{mathematical}{vernacular}.
    \end{definition}
  \item
    \begin{definition}[id=mathtalk.def,
      for={conj,disj,negate,foral,foralS,exis,exisS,biimpl,imply,uexis,uexisS,nexis,nexisS}]
      Abbreviations for Mathematical statements in \defi{MathTalk}
      \begin{itemize}
      \item $\conjFN$ and $\disjFN$ are common notations for \nlex{and} and \nlex{or}
      \item \nlex{not} is in mathematical statements often denoted with $\neg$
      \item $\foral{x}{P}$ ($\foralS{x}SP$) stands for \nlex{condition $P$ holds for all
          $x$ (in $S$)}
      \item $\exis{x}{P}$ ($\exisS{x}SP$) stands for \nlex{there exists an $x$ (in $S$)
          such that proposition $P$ holds}
      \item $\nexis{x}{P}$ ($\nexisS{x}SP$) stands for \nlex{there exists no $x$ (in $S$)
          such that proposition $P$ holds}
      \item $\uexis{x}{P}$ ($\uexisS{x}SP$) stands for \nlex{there exists one and only one
          $x$ (in $S$) such that proposition $P$ holds}
      \item \defi{iff} as abbreviation for \nlex{if and only if}, symbolized by
        \nlex{$\biimplFN$}
      \item the symbol \nlex{$\implyFN$} is used a as shortcut for \nlex{implies}
      \end{itemize}
    \end{definition}
  \item
    \begin{omtext}[title=Observation,id=mathtalk-statements-formulae] 
      With these abbreviations we can use formulae for statements.
   \end{omtext}
  \item 
    \begin{example}[id=implies-reads,for=mathtalk-statements-formulae]
      $\foral{x}{\exis{y}{\biimpl{x=y}{\negate{x\ne y}}}}$ reads
      \begin{quote}
        ``For all $x$, there is a $y$, such that $x=y$, iff (if and only if) it is not the
        case that $x\ne y$.''
      \end{quote}
    \end{example}
  \end{itemize}
\end{frame}
\end{module}


\end{document}
% LocalWords:  en mathtalk disjFN disj conjFN conj implyFN biimplFN biimpl 
% LocalWords:  foralFN foral exisFN exis uexis negateFN MathTalk stex symdef
% LocalWords:  assocarg mixfixa Rightarrow Leftrightarrow allcdot forall excdot
% LocalWords:  abbrdef hskip.1ex bvars mixfixai niprec niprec nexis nexisFN lec
% LocalWords:  frametitle twindef nlex ldots ldots ldots omtext assoc bargs
% LocalWords:  inlineex usevocabdefiniteintegral01 defii
% LocalWords:  highschool
