\documentclass[notes,mh]{mikoslides}
\libinput{preamble}
\begin{document}
\begin{module}[id=nat-basic]
 \importmhmodule[mhrepos=MiKoMH/KWARC,dir=missing/en]{dates}
  \symdef[name=sslash]{sslashFN}{/}
  \symdef{sslash}[1]{\mixfixi[p=1,pi=0]{}{#1}\sslashFN{}}
  \symdef{szero}{\;}

  \begin{frame}
  \frametitle{Something very basic:}
  \begin{itemize} 
  \item Numbers are symbolic representations of numeric quantities.
  \item There are many ways to represent numbers\lec{more on this later}
  \item let's take the simplest one\lec{about 8,000 to 10,000 years old}
    \cmhgraphics[width=10cm]{dmath/PIC/Ishango_bone}
  \item we count by making marks on some surface.
  \item For instance $\LXMID{four-unary}{\sslash{\sslash{\sslash{\sslash\szero}}}}$
    stands for the number four\lec{be it in 4 apples, or 4 worms}
\newpage
  \item Let us look at the way we construct numbers a little more algorithmically,
  \item
    \begin{definition}[display=flow,id=unary-nats-def-by-rules.def,for={szero,sslash}]
      these representations are those that can be created by the following two rules.
      \begin{description}
      \item[$o$-rule] consider '$\szero$' as an empty space.
      \item[$s$-rule] given a row of marks or an empty space, make another $\sslashFN$
        mark at the right end of the row.
      \end{description}
    \end{definition}
  \item
    \begin{example}[id=ex-four-unary,for=nat-basic]
      For $\sslash{\sslash{\sslash{\sslash\szero}}}$, Apply the $o$-rule once and then the $s$-rule four
      times.
    \end{example}
  \item
    \begin{definition}[id=unary-natural-numbers.def]
      we call these representations \defiii{unary}{natural}{numbers}.
    \end{definition}
  \end{itemize}
\end{frame}


\begin{note}
  \begin{omtext}
    In addition to manipulating normal objects directly linked to their daily survival,
    humans also invented the manipulation of place-holders or symbols. A {\emph{symbol}}
    represents an object or a set of objects in an abstract way. The earliest examples for
    symbols are the cave paintings showing iconic silhouettes of animals like the famous
    ones of Cro-Magnon.  The invention of symbols is not only an artistic, pleasurable
    ``waste of time'' for mankind, but it had tremendous consequences. There is
    archaeological evidence that in ancient times, namely at least some 8000 to 10000
    years ago, men started to use tally bones for counting. This means that the symbol
    ``bone with marks'' was used to represent numbers. The important aspect is that this
    bone is a symbol that is completely detached from its original down to earth meaning,
    most likely of being a tool or a waste product from a meal. Instead it stands for a
    universal concept that can be applied to arbitrary objects.
  \end{omtext}

  \begin{omtext}
    Instead of using bones, the slash $\sslashFN$ is a more convenient symbol, but it is
    manipulated in the same way as in the most ancient times of mankind. The $o$-rule
    allows us to start with a blank slate or an empty container like a bowl. The $s$- or
    successor-rule allows to put an additional bone into a bowl with bones, respectively,
    to append a slash to a sequence of slashes. For instance $\LXMRef{four-unary}$ stands
   for the number four --- be it 4 apples, or 4 worms. This representation is constructed
    by applying the $o$-rule once and then the $s$-rule four times.
  \end{omtext}
\end{note}
\end{module}



\end{document}
% LocalWords:  en highschool nat sslashFN sslash szero sslashx Cro Magnon stex
% LocalWords:  Giuseppesymdef mixfixi PeanoAxiom mathbf frametitle
% LocalWords:  lec mycgraphics dmath unary-nats-def-by-rules.def defiii omtext
% LocalWords:  emph defi peano-axioms.def peano-axioms Peano inlinedef twintoo
% LocalWords:  trefi defii itemize
