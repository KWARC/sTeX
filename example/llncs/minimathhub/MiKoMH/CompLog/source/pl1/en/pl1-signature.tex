\documentclass[notes,mh]{mikoslides}
\libinput{preamble}
\begin{document}
\begin{module}[id=pl1-signature]
\importmhmodule[mhrepos=MiKoMH/GenCS,dir=dmath/en]{sets-operations}
\importmhmodule[mhrepos=MiKoMH/GenCS,dir=pl0/en]{pl0-syntax}

\symdef{PREDLOG}{\operatorname{PL}^1} 

\symdef{typeind}{\iota}
\abbrdef{indVars}{\TVars\typeind}
\abbrdef{indSig}{\TSignature\typeind}

\symdef{SignaturePart}[1]{\Signature^{#1}}
\symdef{connectives}{\SignaturePart\typebool}
\symdef{SignaturePartArity}[2]{\SignaturePart{#1}_{#2}}
\symdef{functionSymbolsArity}[1]{\SignaturePart{f}_{#1}}
\abbrdef{constantSymbols}{\SignaturePart{f}_0}
\abbrdef{predicateSymbolsArity}[1]{\SignaturePart{p}_{#1}}
\abbrdef{SkolemFunctionsArity}[1]{\SignaturePart{sk}_{#1}}
\abbrdef{functionSymbols}{\SignaturePart{f}}
\abbrdef{predicateSymbols}{\SignaturePart{p}}
\abbrdef{SkolemFunctions}{\SignaturePart{sk}}

\begin{nomtext}
  The first step of defining a formal language is to specify the alphabet, here the
  first-order signatures and their components.
\end{nomtext}

\begin{frame}
  \frametitle{$\PREDLOG$ Syntax (Signature and Variables)}
  \begin{itemize}
  \item 
    \begin{definition}[for=PREDLOG,id=predlog.def]
      \Defii{first-order}{logic} ($\PREDLOG$), is a formal logical system extensively used
      in mathematics, philosophy, linguistics, and computer science. It combines
      propositional logic with the ability to quantify over individuals.
    \end{definition}
  \item 
    \begin{omtext}
      $\PREDLOG$ talks about two kinds of objects:\lec{so we have two kinds of symbols}
      \begin{itemize}
      \item \inlinedef{\defiis{truth}{value}; sometimes annotated by type
          $\typebool$}\lec{like in $\PropLog$}
      \item \inlinedef{\defis{individual}; sometimes annotated by type
          $\typeind$}\lec{numbers, foxes, Pok\'emon,\ldots}
      \end{itemize}
    \end{omtext}
  \item
    \begin{definition}[id=fo-signature.def,
      for={connectives,functionSymbolArity,functionSymbol,predicateSymbolsArity,predicateSymbols,SkolemFunctionsArity,SkolemFunctions
        constantSymbols}]
      A \defii{first-order}{signature} consists of \lec{all disjoint; $\inset{k}\NaturalNumbers$}
      \begin{itemize}
      \item\strut \defis{connective}:
        $\connectives=\set{\trueconst,\falseconst,\negationOp,\disjunctionOp,\conjunctionOp,
          \implicationOp,\equivalenceOp,\ldots}$\lec{functions
          on truth values}
      \item\strut\defiis{function}{constant}:
        $\functionSymbolsArity{k}=\setdots{f,g,h}$\lec{functions on individuals}
      \item\strut \defiis{predicate}{constant}:
        $\predicateSymbolsArity{k}=\setdots{p,q,r}$ \lec{relations
          among inds.}
      \item\strut (\defiis{Skolem}{constant}:
        $\SkolemFunctionsArity{k}=\set{\ulivar{f}k1,\ulivar{f}k2,\ldots}$)\lec{witness
          constructors; countably $\infty$}
      \item We take $\indSig$ to be all of these together:
        $\defeq\indSig{\union{\functionSymbols,\predicateSymbols,\SkolemFunctions}}$,
        where
        $\defeq{\SignaturePart{*}}{\munionCollection{k}\NaturalNumbers{\SignaturePartArity{*}k}}$
        and define $\defeq\Signature{\union{\indSig,\connectives}}$.
      \end{itemize}
    \end{definition}
  \item 
    \begin{definition}[id=fologic.def,for=indVars,display=flow]
      We assume a set of \defiis{individual}{variable}:
      $\indVars=\setdots{\livar{X}\typeind,\livar{Y}\typeind,\livar{Z}\typeind,\ulivar{X}1\typeind,\ulivar{X}2\typeind}$
      \lec{countably $\infty$}
   \end{definition}
  \end{itemize}
\end{frame}

\begin{nomtext}
  We make the deliberate, but non-standard design choice here to include Skolem constants
  into the signature from the start. These are used in inference systems to give names to
  objects and construct witnesses. Other than the fact that they are usually introduced by
  need, they work exactly like regular constants, which makes the inclusion rather
  painless. As we can never predict how many Skolem constants we are going to need, we
  give ourselves countably infinitely many for every arity. Our supply of individual
  variables is countably infinite for the same reason.
\end{nomtext}
\end{module}


\end{document}
% LocalWords:  pl typebool typeind propVars indVars FOSignature SignaturePart sk
% LocalWords:  SignaturePartArity functionSymbolsArity predicateSymbolsArity
% LocalWords:  SkolemFunctionsArity functionSymbols predicateSymbols wffbool
% LocalWords:  SkolemFunctions wffind wffall Pok emondmath symdef
% LocalWords:  abbrdef foterms mathit wff foTermsS foprops foPropsS gimport
% LocalWords:  fowff PREDLOG forall sdot hskip.1ex assocarg allcdot mixfixai
% LocalWords:  excdot frametitle lec inlinedef defii defi ldots trueconst inds
% LocalWords:  fo-signature.def falseconst setdots ulivar ulivar infty defeq
% LocalWords:  munionCollection fologic.def fo-terms.def sseteq lethan sneg
% LocalWords:  fo-propositions.def fo-connectives.def limpl siff simpl simpl
% LocalWords:  fo-exquant.def fo-atoms.def adefii naturalnumbers TVars itemize
% LocalWords:  TSignature omtext
