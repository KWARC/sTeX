\documentclass{llncs}
\usepackage[mh,mathhub=/Users/kohlhase/localmh/MathHub]{smglom}
\usepackage{stex-logo}
\usepackage[show]{ed}
\usepackage{hyperref}
\usepackage[capitalize]{cleveref} 
\title{Integrating \protect\sTeX into llncs.sty}
\author{Michael Kohlhase\orcidID{0000-0002-9859-6337} \and Dennis M\"uller\orcidID{0000-0002-4482-4912} \and Jan Frederik Schaefer}
\institute{Computer Science, FAU  Erlangen-N\"urnberg}

\usepackage[hyperref=auto,style=alphabetic,isbn=false,backend=bibtex]{biblatex}
\addbibresource{kwarcpubs.bib}
\addbibresource{extpubs.bib}
\addbibresource{kwarccrossrefs.bib}
\addbibresource{extcrossrefs.bib}
\renewcommand*{\bibfont}{\small}

\begin{document}
\maketitle
\begin{abstract}
  Paper shows how to integrate \sTeX stuff into regular papers, e.g. via the
  \texttt{llncs.cls} and get them through submission systems like EasyChair.
\end{abstract}

\section{Introduction}\label{sec:intro}

One of the useful new features of \sTeX is that we can mix in its semantic annotation
functionality into arbitrary document classes. This paper is an example of how this can be
done. Indeed it uses the \textsf{llncs} class as we would submit it to a conference with
LNCS or LNAI proceedings.

In \cref{sec:meat} show how we can use the \sTeX functionality to write an \sTeX module.
This presupposes that the author has cloned the necessary MathHub archives from
\url{https://gl.mathhub.info}, say under the local path
\url{/Users/kohlhase/localmh/MathHub}. One of the past hindrances was that it was
virtually impossible to submit a paper that includes MathHub archives via a conference
tool like EasyChair.org. Therefore the \sTeX distribution contains a python script
\texttt{mhlocalize} that copies a snapshot of the \sTeX files from the MathHub archives
into the current directory, so they can be zipped for submission with the paper sources. 


\section{The Meat of the Matter: An \sTeX module in a Paper}\label{sec:meat}
Here we make use of the \sTeX functionality: We initialize the preamble with
\begin{footnotesize}
\begin{verbatim}
\documentclass{llncs}
\usepackage[mh,mathhub=/Users/kohlhase/localmh/MathHub]{smglom}
\end{verbatim}
  The first line contains the \texttt{llncs} document class and the second calls the
  \texttt{smglom} package from the \sTeX distribution. We give it the \texttt{mathhub}
  option that allows to specify the system path to the local MathHub
  archives.

  This is sufficient solitary authoring. For collaborative authoring, the approach of
  supplying the respective \sTeX packages with a \texttt{path=} option will not work,
  since the path to the local MathHub home is system-dependent and can therefore not be
  committed. There are two ways of fixing this.  One is to have an unversioned file
  \verb|lmh.sty| with contents
\begin{verbatim}
\def\mathhub@path{/path/to/mathhub} 
\end{verbatim}
  and add a \verb|\usepackage{lmh}| before the \verb|\usepackage[mh]{smglom}|.  For the
  other see \cref{sec:easychair}.

  Either way, somewhere in the paper we can just write an \sTeX module, here one about the
  expected utility of an action in a partially observable environment.

\begin{verbatim}
\begin{module}[id=exputil]
\importmhmodule[mhrepos=MiKoMH/AI,dir=rational-decisions/en]{Ramsey-thm}
\importmhmodule[mhrepos=MiKoMH/AI,dir=probabilistic-reasoning/en]{condprob}
\gimport[smglom/probability]{expectation}
\symdef{expectedUtilityOp}{EU}
\symdef{expectedUtility}[1]{\prefix\expectedUtilityOp{#1}}
\symdef{expectedUtilityGiven}[2]{\prefix\expectedUtilityOp{#1|#2}}
\symdef{agactres}[1]{R_{#1}}

\begin{definition}
The \trefii{expected}{utility} $\expectedUtilityGiven{a}{\mathbf{e}}$ of an
\trefi[agent-math]{action} $a$ (given evidence $\mathbf{e}$) is
\[\fundefeq{a,\mathbf{e}}{\expectedUtilityGiven{a}{\mathbf{e}}}
    {\SumInColl{s}\Omega
    {\realtimes[cdot]{\CondProb{\agactres{a}=s}{a,\mathbf{e}},\utilityof{s}}}}
\]
where $\agactres{a}$  is a \trefii[unconditional-prob]{random}{variable} whose 
values are the results of performing $a$ in the current state.
\end{definition}
\end{module}
\end{verbatim}
\end{footnotesize}

The first line introduces a module, the next three lines make available the semantic
macros introduced by two modules in the \sTeX-based AI lecture hosted at
\url{https://gl.mathhub.info/MiKoMH/AI} and another from the SMGloM glossary archive at
\url{https://gl.mathhub.info/smglom/probability}. The next three lines introduce semantic
macros we use in the definition itself. With all the semantic macros (included and
directly provided), we can mark up the formulae and technical terms in the usual \sTeX
way. This gives the following result:\ednote{there is still a problem here, the
  \texttt{definition} environment does not produce a good label.}
\begin{quote}
\begin{module}[id=exputil]
\importmhmodule[mhrepos=MiKoMH/AI,dir=rational-decisions/en]{Ramsey-thm}
\importmhmodule[mhrepos=MiKoMH/AI,dir=probabilistic-reasoning/en]{condprob}
\gimport[smglom/probability]{expectation}
\symdef{expectedUtilityOp}{EU}
\symdef{expectedUtility}[1]{\prefix\expectedUtilityOp{#1}}
\symdef{expectedUtilityGiven}[2]{\prefix\expectedUtilityOp{#1|#2}}
\symdef{agactres}[1]{R_{#1}}

\begin{definition}
The \trefii{expected}{utility} $\expectedUtilityGiven{a}{\mathbf{e}}$ of an
\trefi[agent-math]{action} $a$ (given evidence $\mathbf{e}$) is
\[\fundefeq{a,\mathbf{e}}{\expectedUtilityGiven{a}{\mathbf{e}}}
    {\SumInColl{s}\Omega
    {\realtimes[cdot]{\CondProb{\agactres{a}=s}{a,\mathbf{e}},\utilityof{s}}}}
\]
where $\agactres{a}$  is a \trefii[unconditional-prob]{random}{variable} whose 
values are the results of performing $a$ in the current state.
\end{definition}
\end{module}
\end{quote}
Note that we are taking

\section{Localizing the Papers}\label{sec:easychair}

To make the paper sources into a self-contained folder, the \sTeX tools distribution at
\url{https://github.com/sLaTeX/lmhtools} provides the python script \texttt{localize} that
copies (the necessary parts) of MathHub archives into the \texttt{lmh} subdir in the
current directory.
\begin{verbatim}
python3 localize --localmh /path/to/MathHub paper.tex
\end{verbatim}
extracts all archive inclusions from the \sTeX document with main file \verb|paper.tex|
and recursively copies (or updates) all used files into the local \verb|lmh| directory
respecting the archive structure. This has two applications:
\begin{enumerate}
\item \textbf{Collaborative Authoring in GIT}: Instead of requiring all co-authors have
  the MathHub archives cloned, you can commit the \texttt{localize}d snapshot and use
  that. This has the advantage that the late-binding problem inherent in using central
  archives is mitigated as all co-authors use the same \texttt{lmh} folder. And it is
  sufficient that one co-author has the MathHub archives cloned, and \texttt{localize}s
  it.
\item \textbf{Submission via a Conference or Journal Submission system}: For submission,
  we need a self-contained zip file which we can now create by \texttt{localize}ing and
  \texttt{zip}ping.
\end{enumerate}

\section{Conclusion}\label{sec:conc}
With the new functionality developed in 2020, mixing \sTeX into generic {\LaTeX} document
classes is quite feasible. This considerably lowers the entry barrier to using \sTeX from
the state where you had to use the \sTeX document classes.

One of the advantages of \sTeX is the availability of well-thought out modules with
semantic macros for many areas of mathematics, which can be included directly from a local
clone of the archives. For collaborative authoring and paper submission we add a tool for
localizing the MathHub archives.

\printbibliography
\end{document}

%  LocalWords:  llncs.cls llncs mhlocalize mathhub lmh.sty lmh exputil Ramsey-thm tbw
%  LocalWords:  condprob agactres fundefeq utilityof sec:easychair sec:conc
