\documentclass{bluenote}
\usepackage[show]{ed}
\usepackage[hyperref=auto,style=alphabetic,isbn=false]{biblatex}
\usepackage{bibtweaks,shortvrb}
\usepackage{amstext}
\addbibresource{kwarcpubs.bib}
\addbibresource{extpubs.bib}
\addbibresource{kwarccrossrefs.bib}
\addbibresource{extcrossrefs.bib}
\usepackage{stex-logo}
\usepackage{xspace}
\usepackage{lstomdoc}
\usepackage{lststex}\lstset{language=[sTeX]TeX}

\title{Proposed changes on the \emph{dennis}-branch in \protect\sTeX}
\author{Dennis Müller\\Computer Science, FAU Erlangen-N\"urnberg}

\def\smglom{\textsf{SMGloM}\xspace}
\def\omdoc{\textsf{OMDoc}\xspace}
\def\openmath{\textsf{OpenMath}\xspace}
\def\latexml{{\LaTeX}ML\xspace}
\def\lmh{\textsf{lmh}\xspace}
\blueProject{\sTeX}\blueURI{http://github.com/sLaTeX/sTeX}
\def\lstomdoc{\lstinline[language={[1.3]OMDoc},mathescape]}
\def\nlex#1{``\emph{#1}''}
\def\meta#1{\ensuremath{\langle\kern-.2em\langle\text{#1}\rangle\kern-.2em\rangle}}
\MakeShortVerb{\|}

\def\done#1{\textcolor{green}{#1}}
\def\designquestion#1{\textcolor{blue}{#1}}
\def\dangerous#1{\textcolor{red}{#1}}

\begin{document}
\maketitle
\begin{abstract}
  Bla.
\end{abstract}

\section{Guiding Principles}

\begin{itemize}
  \item Simplify macro syntax as much as possible
  \item[$\Rightarrow$] Ideally, have every functionality be governed by a single macro,
    e.g. only |\begin{module}[...]| instead of |module|, |modnl|, |modsig|, |mhmodsig|,
      |mhmodnl|, etc. -- analogously for |\importmodule|, |\usemodule|, etc.
  \item get rid of e.g. |mh|-options - \sTeX should be able to figure out if we are
      in a mathhub/smglom repository on its own and resolve paths accordingly
  \item[$\Rightarrow$] as a user, include a single package that provides everything needed
      and works with \emph{and} without MathHub, smglom, etc.
  \item macros corresponding to content with a URI should carry their respective URIs
      and be accessible \emph{via} their URI, so that overloaded macros/names can
      be systematically dealt with.
  \item[$\Rightarrow$] massively simplifies bindings (I hope) by having \TeX\ itself do the
      brunt of the work.
  \item \done{No external tools required (localpaths, sms)}
\end{itemize}

\section{Modules}

\begin{itemize}
  \item |\begin{module}| picks a namespace and an id (if not explicitly provided) $\Rightarrow$
    \sTeX knows the full URI of this module.
  \item The namespace is either provided explicitly (via |[ns=http://mathhub.info/example]|), or
    take from a |MANIFEST.MF| if in a current repository, or computed from the current file, e.g.
    |file:///foo/bar/|, if the current file is |/foo/bar.tex|.
  \item The name of of the module is either provided explicitly (via |[id=foo]|), or
    by default enumerated (|module0|, |module1|, etc.)
  \item |\begin{module}[ns=http://mathhub.info/example, id=foo]| will create
    a new macro |http://mathhub.info/example?foo| and a shorter, accessible macro
    \designquestion{(maybe, but probably not, \texttt{\detokenize{\foo}} because of name clashes with symbols in the module)}
    that both expand to |\invoke@module{http://mathhub.info/example?foo}|, which will
    allow for e.g. accessing symbols in the module, in order to disambiguate e.g.
    |\natarith{plus}| and |\intarith{plus}|.
  \item |\begin{module}| sets |\this@module| to |\module@defs@http://mathhub.info/example?foo|.
\end{itemize}

variants that need to be dealt with:

\begin{itemize}
  \item |modnl|/|mhmodnl| - we can replace |\begin{modnl}{title}{lang}| by
    |\begin{module}[title=...,lang=...]|. Analogously for |mhmodnl|.

      \designquestion{In the smglom, we might want to consider removing the module-environments entirely,
      since the same information is already contained in filename + MANIFEST.MF.}
  \item |modsig|/|mhmodsig| - |mhmodsig| isn't even used anywhere.
      |modsig| could be replaced by e.g. |\begin{module}[title=...,lang=sig]|
      and \designquestion{analogously to |modnl| \emph{could maybe} be removed in the 
      smglom entirely}.
\end{itemize}

\section{Symbols/Notations}

Assume we're in module |http://mathhub.info/example?foo|.

\begin{itemize}
  \item deprecate |\symdef|,|\symvariant| and |\symi| and variants, and replace them by:
  \item |\symdecl[id=foo]{bar}| -- declares a new symbol with URI |http://mathhub.info/example?foo?foo|,
    creates macros |\http://mathhub.info/example?foo?foo| and |\bar| that expand to\\
    |\invoke@symbol{http://mathhub.info/example?foo?foo}|.

    If |id| is not provided, the name of the symbol is the macro name, e.g. |bar|.

  \item \designquestion{Design question: Types for \texttt{\detokenize{\symdecl}}?
    Separate macro or key in the [...] part? Should \sTeX do anything with it?
    Definientia, too? e.g. via \texttt{\detokenize{\abbrdef}}?}
  \item |\notation[lang=...,arity=...,variant=...,arg=...,prec=...]{foo}[n]{...}| declares
    a new notation for symbol |foo|. |lang|, |arity| and |variant| are keys for different
    notation ``types'', that can ultimately be used via e.g. |\foo[lang=en,arity=2,variant=op]|.
    |\foo[op]| is shorthand for |\foo[variant=op]|.

    |arg| is a sequence of |i|s and |a|s, where |i| is a simple argument and |a| a
    flexary/associative argument. If |arg| is not given, the optional |[n]| is used
    instead (i.e. then |arg| is |i|$^n$). If neither is given, but the |arity| keyword
    is set, then the arity is used. If neither is given, the arity is $0$.
    \designquestion{Probably we'll also need something like |b| (in addition to |a| and 
    |i|) for ``bound'' arguments (i.e. bound variables), and maybe others as well?
    Should \sTeX do anything with |b| arguments, or only LaTeXML? Maybe explicitly mark 
    macros that are ``allowed'' as ``head symbols'' in |b|-arguments, 
    such as \texttt{\detokenize{\setin}}?}

    |prec| is a string of numbers $p_{sym}$|;|$p_1$|x|...|x|$p_m$, where $p_{sym}$ is
    the precedence of the symbol (upwards) and the $p_i$ are the precedences of the
    individual arguments (downwards)(see below). The default precedence is $0$ except if
    the macro has arity $0$, in which case it is |-\infprec|.

    |\notation| takes an additional argument for each |a| in its |arg| for (the infix-notation
    of) an associative argument.

    \designquestion{Design question:} Currently (as in |\symvariant|) the |foo| argument stands for the \emph{macro name}
    of the symbol that will be given a notation. This is fine in most cases, but not very semantic and
    weird whenever |id| and macro name of a symbol differ, or |\foo| has been redefined as something
    entirely different. I propose that instead, |foo| should \emph{either} be a macro
    that ultimately expands to |\invoke@symbol{URI}|, in which case |URI| is the uri
    of the symbol (this would preserve the current syntax), \emph{or} |foo| is the name 
    of a symbol in the \emph{current} module, or |foo| is a full URI of a symbol in
    the same or a different module. This URI could externally be accessible via 
    |\invoke@module|, e.g. in |\notation[variant=foo]{\intarith{?plus}}| (which
    technically wouldn't give the URI of |?plus|, but rather ultimately expand to
    |\invoke@symbol{...?intarith?plus}|, which is also covered).

  \item precedences/bracketing: An argument $n$ with precedence |p|$_n$ in a notation is 
  wrapped in a |\notation@argpec{p|$_n$|}{...}|, the \emph{whole} notation is wrapped in a
  |\notation@symprec{|$p_{sym}$|}{...}|. E.g. |\notation[prec=50;20x20]{plus}{#1 + #2}| would
  actually have notation\\ 
  |\notation@symprec{50}{\notation@argpec{20}{#1} + \notation@argpec{20}{#2}}|.
  Associative arguments are wrapped in a |\notation@assoc|, e.g.
  |\notation[prec=50;20]{plus}{#1}{+}| results in 
  |\notation@symprec{50}{\notation@argpec{20}{\notation@assoc{+}{#1}}}|
  \item The initial ``downwards'' precedence $p$ is |-\infprec|. |\notation@symprec{n}{...}|
    compares $n$ with $p$. If $n\leq p$ (and $p\neq$|-\infprec|), brackets are inserted.
    |\notation@argprec{n}{...}| sets the downwards precedence to $n$. This subsumes
    |\mixfix|-variants, |\prefix|, |\suffix|, etc.
  \item For bracketing, the values of |\notation@lparen| and |\notation@rparen| are used.
    |\withbrackets{a}{b}{...}| temporarily changes those to |a| and |b| 
    (|\notation@symprec| changes them back afterwards, so that the changed ones are only used
    at the specific point |\withbrackets| is used).

    \designquestion{Design question:} |\notation| probably needs a key 
    |[withbrackets={a,b}]|,
    because the |\withbrackets|-macro needs to be \emph{outside} of the 
    |\notation@symprec|-macro, which |\notation| wraps around the \emph{whole} notation.

    Brackets are prefixed with |\left|/|\right| in |display| mode only.
\end{itemize}

Construction sites:
\begin{itemize}
  \item |\setnotation[key=value]| globally/locally sets e.g. |lang=de| for all notations.
    
    \designquestion{Tricky:} what to do if a symbol doesn't have a |lang=de| notation?
    How should that interact with explicitly provided notation variants, \emph{both}
    other e.g. languages (I suggest explicitly provided variants override those of a
    |\setnotation|) \emph{and} others (e.g. |\foo[variant=op]|) if the \emph{combination}
    (e.g. |\foo[variant=op,lang=de]|) doesn't exist?

  \item |\symi| and friends should be deprecated by
    |\symdecl|. For that, I'd need to figure
    out what the exact difference is between |\symii{a}{b}| and |\symi{a-b}| and |\symi{a b}|,
    and rethink |\trefi|-variants analogously.
  \item |\vardecl| should be like symdecl, but have exactly one notation 
    \designquestion{(I guess?)}
    and expand to something like |\invoke@variable{n}| (which gets LaTeXMLified to an |OMV|!)
    rather than having a full URI.

    \designquestion{Needs design: Local/global variables? variables that are theory parameters?
    Universally/existantially bound? Types?}

  \item |\symdecl| should be allowed outside of a |module|-environment, in which case
    e.g. the file\emph{path} (or |ns:|-field in the |MANIFEST.MF| + subfolder in |source|)
    could provide the namespace and the file\emph{name} the module name.
    Since this is what e.g. smglom consistently does (and to some extent MiKoMH as well?),
    it seems like we should make that the default, which would allow getting rid of the
    boilerplate in smglom files.
\end{itemize}

\section{\textbackslash importmodule/\textbackslash usemodule/\textbackslash inputref and variants}

\begin{itemize}
  \item Currently all these commands take care of setting |\this@module|, |\mh@currentrepos|, etc.
    As a result, we get \emph{some} trouble, e.g. with \sTeX macros in TOC lines. It seems
    to me that every module should be in charge of its own location, based on folder/archive-MANIFEST/file,
    rather than setting and resetting them on |\includemodule|.

  \item Again, lots of variants that should be unified, e.g. 
    one could use |\includemodule[mhrepo=A/B]| instead of |\gimport| or |\mhinclude| or what else is around there...
  \item |mh|-package variants currently require |\mh@currentrepos| to be set to something
    containg a |/|, which is awful.
  \item All these macros should copy files to a ``local cache'' (in the PWD probably?), which e.g. 
    can be submitted to git(labs)/Springer/arxiv etc. so that the document folder is always self-contained. 
    Consequently, all these macros should check a local cache if e.g. the |MATHHUB| system variable
    is not set, or the required repo doesn't exist, or a requested module file can't be found
    for whatever reason.

    This ``cache'' should only be a fallback to make submission/collaboration easier, 
    never the \emph{preferred} source for a module. This is something that a package
    option could turn on/off.
\end{itemize}


\end{document}

%%% Local Variables: 
%%% mode: latex
%%% TeX-master: t
%%% End: 

%  LocalWords:  maketitle smglom IanJucKoh sps14 dmesmgm14 latexml exp symdef arith mrow
%  LocalWords:  lstlisting csymbol csymbol ci ci ci ci mrow msub mn mn msub wp setst bvar
%  LocalWords:  powerset ensuremath bvar mtext mtext dmsmglom14 lstomdoc mathescape om es
%  LocalWords:  protoype msubsup msubsup expr omdoc nlex symvariant verbdef lst newmods
%  LocalWords:  defiii slgroup notatiendum trefi defarg defhead printbibliography nd-new
%  LocalWords:  nd-classical vd-classical plus.proto classical-nd ifmmode verbvariant von
%  LocalWords:  modsig modnl und ggT noindent stex@lang sTeXselectlanguage ednote miko gr
%  LocalWords:  foreignlanguage smultiling.dtx ldots sar Ein gro ses ist ein viel ser als
%  LocalWords:  kleines slg1 slg nl-slg nd-slg slgroup.proto cr
