\documentclass{bluenote}
\usepackage[show]{ed}
\usepackage[hyperref=auto,style=alphabetic,isbn=false]{biblatex}
\usepackage{bibtweaks}
\addbibresource{kwarcpubs.bib}
\addbibresource{extpubs.bib}
\addbibresource{kwarccrossrefs.bib}
\addbibresource{extcrossrefs.bib}
\usepackage{stex-logo}
\usepackage{xspace}
\usepackage{lstomdoc}
\usepackage{lststex}\lstset{language=[sTeX]TeX}

\title{Rethinking Modules and Semantic Macros in \protect\sTeX}
\author{Michael Kohlhase\\Computer Science, Jacobs University.de}

\def\smglom{\textsf{SMGloM}\xspace}
\def\omdoc{\textsf{OMDoc}\xspace}
\def\latexml{{\LaTeX}ML\xspace}

\blueProject{\sTeX}\blueURI{http://github.com/KWARC/sTeX}

\begin{document}
\maketitle
\begin{abstract}
  In this note, we document the state of rethinking the \sTeX infrastructure in terms of
  the \smglom.
\end{abstract}
\section{Introduction}
We have been using \sTeX as the encoding for the Semantic Multilingual Glossary of
Mathematics (\smglom; see~\cite{IanJucKoh:sps14}). The \smglom data model has been taxing
the representational capabilities of \sTeX with respect to multilingual support and
verbalization definitions; see~\cite{Kohlhase:dmesmgm14}, which we assume as background
reading for this note.

\section{Mixed Presentation/Content Markup}
Currently, \sTeX produces content markup in the OpenMath encoding. But often \sTeX
formulae often contain bits of presentational {\LaTeX}, which \latexml has to convert into
OpenMath heuristically, which often leads to non-optimal results. Therefore we want to
rethink the representation of formulae, instead of insisting on homogeneous content markup
in OpenMath, we switch to MathML allow mixed presentation/content MathML, which conforms
much more closely to user input (preserving presentational bits) and postpones full
semantification to later stages of processing. Let us make an example: consider the
formula $(a+b)^n$ encoded das \lstinline|\exp{a+b}n|, where we have a semantic macro
\lstinline|\exp| defined by \lstinline|\symdef{exp}[2]{#1^{#2}}| in module
\lstinline|arith| Then we should create
\begin{lstlisting}[language=MathML]
<math>
  <apply>
     <csymbol cd="arith">exp</csymbol>
     <mrow><ci>a</ci><mo>+</mo><ci>b</ci></mrow>
     <ci>n</ci>
  </apply>
</math>
\end{lstlisting}
Note that MathML does indeed allow to freely mix content and presentation MathML, here we
have an application produced by the semantic macro \lstinline|\exp| applied to the
presentational $a+b$, where $a$ and $b$ are ``content identifiers''. 

A side effect of the switch to MathML is that complex variable names are much nicer in
MathML: $x_5$ is just 
\begin{lstlisting}[language=MathML]
<ci name="x5"><msub><mi>x</mi><mn>5</mn></msub></ci>
\end{lstlisting}

\section{\protect\sTeX Module Signatures}

(monolingual) \sTeX had the intuition that the symbol definitions (\lstinline|\symdef| and
\lstinline|symvariant|) are interspersed with the text and we generate \sTeX module
signatures (SMS \lstinline|*.sms| files) from the \sTeX files. The SMS duplicate
``formal'' information from the ``narrative'' \sTeX files. In the \smglom, we extend this
idea by making the the SMS primary objects\footnote{Thanks to Deyan Ginev for realizing
  this.} that contain the language-independent part of the formal structure conveyed by
the \sTeX documents and there may be multiple narrative ``language bindings'' that are
translations of each other -- and as we do not want to duplicate the formal parts, those
are inherited from the SMS rather than written down in the language binding itself. So instead of 
\begin{lstlisting}[caption=Old-Style \protect\sTeX,label=lst:oldmods]
\begin{module}[id=foo]
\symdef{bar}{BAR}
\begin{definition}[for=bar]
  A \defiii{big}{array}{raster} ($\bar$) is a\ldots
\end{definition}
\end{module}
\end{lstlisting}

we now advocate the divided style in Listing~\ref{lst:newmods}\ednote{MK: the names of the
  environments are still very much in the air. ``modsig'' I rather like, but ``langbind''
  is terrible}. There the \lstinline|modsig| environment works exactly like the old
\lstinline|module| environment, only that the \lstinline|id| attribute has moved into the
required argument -- anonymous module signatures do not make sense. The
\lstinline|langbind| environment takes two arguments the first is the name of the module
signature it provides language bindings for and the second the ISO 639 language specifier
of the content language. We add the \lstinline|primary| key to the optional argument of
\lstinline|langbind|, which can specify the primary language binding (the one the others
translate from; and which serves as the reference in case of translation conflicts).

\begin{lstlisting}[caption=New-Style \protect\sTeX,label=lst:newmods]
\begin{modsig}{foo}
\symdef{bar}{BAR}
\end{modsig}

\begin{langbind}[creators=miko,primary]{foo}{en}
\begin{definition}[for=bar]
  A \defiii{big}{array}{raster} ($\bar$) is a\ldots
\end{definition}
\end{langbind}
\end{lstlisting}
We retain the old \lstinline|module| environment as an intermediate stage (during the ) 

\section{Conclusion}
\printbibliography
\end{document}

%%% Local Variables: 
%%% mode: latex
%%% TeX-master: t
%%% End: 
