\documentclass{bluenote}
\usepackage[show]{ed}
\usepackage[hyperref=auto,style=alphabetic,isbn=false]{biblatex}
\usepackage{bibtweaks}
\usepackage{amstext}
\addbibresource{kwarcpubs.bib}
\addbibresource{extpubs.bib}
\addbibresource{kwarccrossrefs.bib}
\addbibresource{extcrossrefs.bib}
\usepackage{stex-logo}
\usepackage{xspace}
\usepackage{lstomdoc}
\usepackage{lststex}\lstset{language=[sTeX]TeX}

\title{Rethinking Modules and Semantic Macros in \protect\sTeX}
\author{Michael Kohlhase\\Computer Science, Jacobs University.de}

\def\smglom{\textsf{SMGloM}\xspace}
\def\omdoc{\textsf{OMDoc}\xspace}
\def\latexml{{\LaTeX}ML\xspace}
\def\lmh{\textsf{lmh}\xspace}
\blueProject{\sTeX}\blueURI{http://github.com/KWARC/sTeX}
\def\lstomdoc{\lstinline[language={[1.3]OMDoc}]}
\def\nlex#1{``\emph{#1}''}
\begin{document}
\maketitle
\begin{abstract}
  In this note, we document the state of rethinking the \sTeX infrastructure in terms of
  the \smglom.
\end{abstract}
\section{Introduction}
We have been using \sTeX as the encoding for the Semantic Multilingual Glossary of
Mathematics (\smglom; see~\cite{IanJucKoh:sps14}). The \smglom data model has been taxing
the representational capabilities of \sTeX with respect to multilingual support and
verbalization definitions; see~\cite {Kohlhase:dmesmgm14}, which we assume as background
reading for this note.

\section{Mixed Presentation/Content Markup}
Currently, \sTeX produces content markup in the OpenMath encoding. But often \sTeX
formulae often contain bits of presentational {\LaTeX}, which \latexml has to convert into
OpenMath heuristically, which often leads to non-optimal results. Therefore we want to
rethink the representation of formulae, instead of insisting on homogeneous content markup
in OpenMath, we switch to MathML allow mixed presentation/content MathML, which conforms
much more closely to user input (preserving presentational bits) and postpones full
semantification to later stages of processing. Let us make an example: consider the
formula $(a+b)^n$ encoded das \lstinline|\exp{a+b}n|, where we have a semantic macro
\lstinline|\exp| defined by \lstinline|\symdef{exp}[2]{#1^{#2}}| in module
\lstinline|arith| Then we should create
\begin{lstlisting}[language=MathML]
<math>
  <apply>
     <csymbol cd="arith">exp</csymbol>
     <mrow><ci>a</ci><mo>+</mo><ci>b</ci></mrow>
     <ci>n</ci>
  </apply>
</math>
\end{lstlisting}
Note that MathML does indeed allow to freely mix content and presentation MathML, here we
have an application produced by the semantic macro \lstinline|\exp| applied to the
presentational $a+b$, where $a$ and $b$ are ``content identifiers''. 

A side effect of the switch to MathML is that complex variable names are much nicer in
MathML: $x_5$ is just 
\begin{lstlisting}[language=MathML]
<ci name="x5"><msub><mi>x</mi><mn>5</mn></msub></ci>
\end{lstlisting}

Finally, there is another effect of the switch to MathML: we finally have a good
representation of formulae with text in them, e.g. the set 
\[\{O\in\wp(X)\mid O \text{ is the union of open balls}\}\]%|
which we can encode as 
\begin{lstlisting}
\setst{O}{\inset{O}{\powerset{X}}}{\text{\ensuremath{O} is the union of open balls}}
\end{lstlisting}
given suitable semantic macros \lstinline|\setst|, \lstinline|\inset|, and
\lstinline|\powerset|. This should generate the mixed representation
\begin{lstlisting}[language=MathML]
<bind>
  <apply>
    <csymbol cd="sets">setst</csymbol>
    <apply><csymbol cd="sets">powerset</csymbol></apply>
  </apply>
  <bvar><ci>O</ci></bvar>
  <mtext>
\end{lstlisting}


\section{\protect\sTeX Module Signatures}

(monolingual) \sTeX had the intuition that the symbol definitions (\lstinline|\symdef| and
\lstinline|\symvariant|) are interspersed with the text and we generate \sTeX module
signatures (SMS \lstinline|*.sms| files) from the \sTeX files. The SMS duplicate
``formal'' information from the ``narrative'' \sTeX files. In the \smglom, we extend this
idea by making the the SMS primary objects\footnote{Thanks to Deyan Ginev for realizing
  this.} that contain the language-independent part of the formal structure conveyed by
the \sTeX documents and there may be multiple narrative ``language bindings'' that are
translations of each other -- and as we do not want to duplicate the formal parts, those
are inherited from the SMS rather than written down in the language binding itself. So instead of 
\begin{lstlisting}[caption=Old-Style \protect\sTeX,label=lst:oldmods]
\begin{module}[id=foo]
\symdef{bar}{BAR}
\begin{definition}[for=bar]
  A \defiii{big}{array}{raster} ($\bar$) is a\ldots
\end{definition}
\end{module}
\end{lstlisting}

we now advocate the divided style in Listing~\ref{lst:newmods}\ednote{MK: the names of the
  environments are still very much in the air. ``modsig'' I rather like, but ``modnl''
  is terrible}. There the \lstinline|modsig| environment works exactly like the old
\lstinline|module| environment, only that the \lstinline|id| attribute has moved into the
required argument -- anonymous module signatures do not make sense. The
\lstinline|modnl| environment takes two arguments the first is the name of the module
signature it provides language bindings for and the second the ISO 639 language specifier
of the content language. We add the \lstinline|primary| key to the optional argument of
\lstinline|modnl|, which can specify the primary language binding (the one the others
translate from; and which serves as the reference in case of translation conflicts).

\begin{lstlisting}[caption=New-Style \protect\sTeX,label=lst:newmods]
\usepackage[english,ngerman]{multiling}
\begin{modsig}{foo}
\symdef{bar}{BAR}
\end{modsig}

\begin{modnl}[creators=miko,primary]{foo}{en}
\begin{definition}
  A \defiii[bar]{big}{array}{raster} ($\bar$) is a\ldots
\end{definition}
\end{modnl}

\begin{modnl}[creators=miko]{foo}{de}
\begin{definition}
  Ein \defiii[bar]{gro"ses}{Feld}{Raster} ($\bar$) is a\ldots
\end{definition}
\end{modnl}
\end{lstlisting}
We retain the old \lstinline|module| environment as an intermediate stage. It is still
useful for monolingual texts. Note that for files with a module, we still have to extract
\lstinline|*.sms| files. It is not completely clear yet, how to adapt the workflows. We
clearly need a \lmh or editor command that transfers an old-style module into a new-style
signature/binding combo to prepare it for multilingual treatment.

\section{Verbalization Definitions}

Currently, \sTeX only supports notation definitions for symbols, but we also need
verbalization definitions for flexiformal mathematics; see~\cite{Kohlhase:dmesmgm14} for a
description of the concept and background on their use and~\cite[section
5]{Kohlhase:dmsmglom14} for first ideas towards an \sTeX encoding. We will extend the
latter here.

The first thing to understand is that \lstinline|\symdef| does two things in the \latexml
workflow. It creates a \lstomdoc|symbol| element and a
\lstomdoc|notation| element. In our new infrastructure, both go into the module
signature. For verbalization definitions the situation is different. We want the
\lstomdoc|symbol| element in the module signature and the verbalization definition in the
language bindings.

For verbalization definitions in \omdoc we want to reuse the \lstomdoc|notation| element,
thus it seems normal to use the \lstinline|\symdef| macro as well. In the situation of
Listing~\ref{lst:newmods}, a verbalization definition for \lstinline|bar| as the English
phrase \nlex{big array raster} could be encoded as something like
\begin{lstlisting}
\symvariant{bar}{lang:en}{\text{big array raster}}
\end{lstlisting}
Note that we already have a symbol \lstinline|bar| generated by the \lstinline|\symdef|,
so we have to use the \lstinline|\symvariant| macro for this, if there were no prior
\lstinline|\symdef|, we would have to use a \lstinline|\symdef|. To hide this choice from
the user we hould probably have a wrapper macro
\begin{lstlisting}
\verbdef[name=bar]{en}{big array raster}
\end{lstlisting}
But in most situations, an explicit language binding is unnecessary, since we have the
definiendum markup. In the situation of Listing~\ref{lst:newmods}, we have a symbol
\lstinline|bar| generated by the \lstinline|\symdef| and a definiendum for the symbol
\lstinline|bar| marked up by the \lstinline|\defiii| macro -- see~\cite{Kohlhase:smms:svn}
for details on \lstinline|\def*|. Note that the optional argument of \lstinline|\defiii|
is used to specify the symbol name, here \lstinline|bar| here. We could let \latexml let
generate the equivalent of a \lstinline|verbdef| as above implicitly, freeing the user
from writing down specifications twice. 

But let us also look at a more interesting symbol: the ``special linear group'' already
discussed in~\cite{Kohlhase:dmsmglom14}. Here the \sTeX verbalization definition would be 
\begin{lstlisting}
\verbdef[name=slgroup]{SLGroup}[2]{special linear group of order #1 over #2}
\end{lstlisting}
Here we have a problem with retrieving this from the definition without additional
markup. A normal definition would have the form
\begin{lstlisting}
\begin{definition}
  The \defiii[slgroup]{special}{linear}{group} \notatiendum{$\SLgroup{n}{F}$}
  of degree $n$ over a \trefi[field]{field} $F$ is ...
\end{definition}
\end{lstlisting}
In particular, the definiendum is discontiguous and usually only the ``head'' is
explicitly emphasized by boldface font. In this situation, a ``continuation markup might
help -- just exploring the syntax here:
\begin{lstlisting}
\begin{definition}
  The \defiii[slgroup]{special}{linear}{group} \notatiendum{$\SLgroup{n}{F}$}
  \defc[slgroup]{of degree \defarg[1]{$n$}} 
  \defc[slgroup]{over \defarg[2]{a \trefi[field]{field} $F$}} is ...
\end{definition}
\end{lstlisting}
Here the \lstinline|\defc| macro continues the definiendum started with the
\lstinline|\defiii| -- we specify which one with the symbol name in the optional argument
and the embedded \lstinline|\defarg| macro excapes out of that and marks its argument as
an argument specifier. I am not sure that this is better than just adding the explicit
verbalization definition above. But maybe the inline markup gives us more structure.

An alternative would be to have a long definiendum markup and use \lstinline|\notatiendum|
to escape out of it. Something like 
\begin{lstlisting}
\begin{definition}
  The \definiendum[slgroup]{special linear group \notatiendum{$\SLgroup{n}{F}$}
  of degree \defarg[1]{$n$} over \defarg[2]{a \trefi[field]{field} $F$}} is ...
\end{definition}
\end{lstlisting}
This implies less markup work. But do we lose structure here? If we have optional
arguments (and here both are), we would like to associate \nlex{of order} with the first
argument and \nlex{over} with the second. So maybe something like 
\begin{lstlisting}
\begin{definition}
  The \definiendum[slgroup]{\defhead{special linear group} 
    \notatiendum{$\SLgroup{n}{F}$}
    \defarg[1,opt]{of degree \arg{$n$}} 
    \defarg[2,opt]{over \arg{a \trefi[field]{field} $F$}}} is ...
\end{definition}
\end{lstlisting}
is more useful. That would allow us to account for all the elision forms.

But that could also be done with the explicit verbalization definition
\begin{lstlisting}
\verbdef[name=slgroup]{SLGroup}[2]{[special linear group] [of order #1] [over #2]}
\end{lstlisting}
where \lstinline|[| and \lstinline|]| group the ellision groups. But maywe we also want to
use curly braces instead of them. We have to see what works best. 


\section{Conclusion}
We have described a set of new functionalities for \sTeX and specified some aspects of
them. Now, they need to be implemented and tested. 

\printbibliography
\end{document}

%%% Local Variables: 
%%% mode: latex
%%% TeX-master: t
%%% End: 
