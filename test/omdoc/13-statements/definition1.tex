\documentclass[minimal]{omdoc}
\usepackage[defindex]{statements}
\begin{document}
\begin{module}[name=foo]
  \begin{definition}[id=test.one]
    This \defi{concept} is old.

    This \defi[name=conc]{one} too.
  \end{definition}

  \begin{definition}
    This \defii{new}{concept} is defined in terms of the existing \trefi{concept}.

    This \defii[name=newconc]{new}{one} is defined in terms of the existing \trefi[?conc]{one}.
  \end{definition}

  \begin{definition}
    This \defiii{very}{new}{concept} is defined in terms of the \trefi{concept} and the 
    \trefii{new}{concerpt}. 

    This \defiii[name=vnconc]{very}{new}{one} is defined in terms of the old
    \trefi[?conc]{one} and the \trefii[?newconc]{new}{one}.
  \end{definition}

  \begin{definition}
    This \defiv{absolutely}{completely}{new}{concept} is defined in terms of the
    \trefiii{very}{new}{concept}.
    
    This \defiv[name=?acnconc]{absolutely}{completely}{new}{one} is defined in terms of the
    \trefiii[?vnconc]{very}{new}{one}.
  \end{definition}
  We can also reference the \trefiv{absolutely}{completely}{new}{concept} directly or the
 \trefiv[?acnconc]{absolutely}{completely}{new}{one} by name.
\end{module}
\end{document}
%%% Local Variables:
%%% mode: latex
%%% TeX-master: t
%%% End:
