% \iffalse meta-comment
% An Infrastructure for marking up Metadata in LaTeX documents
% Copyright (c) 2009 Michael Kohlhase, all rights reserved
%               this file is released under the
%               LaTeX Project Public License (LPPL)
% The original of this file is in the public repository at 
% http://github.com/KWARC/sTeX/
% \fi
% 
% \iffalse
%<*package>
\NeedsTeXFormat{LaTeX2e}[1999/12/01]
\ProvidesPackage{metakeys}[2012/10/26 v0.9 Framework for Metadata Keys]
%</package>
%<*driver>
\documentclass{ltxdoc}
\usepackage{stex-logo,url,array,float}
\usepackage[show]{ed}
\usepackage{ctangit}
\usepackage[hyperref=auto,style=alphabetic]{biblatex}
\addbibresource{kwarcpubs.bib}
\addbibresource{extpubs.bib}
\addbibresource{kwarccrossrefs.bib}
\addbibresource{extcrossrefs.bib}
\usepackage{hyperref}
%\makeindex
\floatstyle{boxed}
\newfloat{exfig}{thp}{lop}
\floatname{exfig}{Example}
\def\tracissue#1{\cite{sTeX:online}, \hyperlink{http://trac.kwarc.info/sTeX/ticket/#1}{issue #1}}
\usepackage[showmeta]{metakeys}
\begin{document}\DocInput{metakeys.dtx}\end{document}
%</driver>
% \fi
%
%\iffalse\CheckSum{176}\fi
%
% \changes{v0.1}{2009/12/14}{First version}
% \changes{v0.8}{2010/06/18}{This is almost done}
% \changes{v0.9}{2010/09/02}{make sure that showkeys is always initialized}
% 
% \GetFileInfo{metakeys.sty}
% 
% \MakeShortVerb{\|}
% \def\scsys#1{{{\sc #1}}\index{#1@{\sc #1}}}
% \def\latexml{\scsys{LaTeXML}}
% \def\omdoc{OMDoc}
% \def\omdocv#1{\omdoc{#1}}
%
% \title{{\texttt{metakeys.sty}}: A generic framework for extensible Metadata in {\LaTeX}\thanks{Version {\fileversion} (last revised
%        {\filedate})}}
%    \author{Michael Kohlhase\\
%            Jacobs University, Bremen\\
%            \url{http://kwarc.info/kohlhase}}
% \maketitle
% \begin{abstract}
%   The |metakeys| package is part of the {\sTeX} collection, a version of {\TeX/\LaTeX} that
%   allows to markup {\TeX/\LaTeX} documents semantically without leaving the document
%   format, essentially turning {\TeX/\LaTeX} into a document format for mathematical
%   knowledge management (MKM).
%
%   This package supplies the infrastructure for extending {\sTeX} macros with {\omdoc}
%   metadata. This package is mainly intended for authors of {\sTeX} extension packages.
% \end{abstract}
%
% \setcounter{tocdepth}{2}\tableofcontents\newpage
%
% \section{The User Interface}\label{sec:user}
%
% Many of the {\stex} macros and environments take an optional first argument which uses
% key/value pairs to specify metadata relations of the marked up objects. The |metakeys|
% package supplies the infrastructure managing these key/value pairs. It also forms the
% basis for the |rdfmeta| package which allows to use these for flexible, user-extensible
% metadata relations (see~\ctancite{Kohlhase:rdfmeta} for details).
%
% \subsection{Package Options}\label{sec:user:options}
% 
% The |metakeys| package takes a single option: \DescribeMacro{showmeta}|showmeta|. If
% this is set, then the metadata keys defined by the |\addmetakey| are shown
% (see~\ref{sec:user:showmeta})
% 
% \subsection{Adding Metadata Keys to Commands}\label{sec:user:keys}
%
% Key/value pairs in \stex are organized in \textbf{key groups}: every {\stex} macro and
% environment that takes a key/value argument has an associated key group, and only keys
% that are registered in this group can be utilized. The |metakeys| package supplies the
% \DescribeMacro{\addmetakey}|\addmetakey| macro to add a new key to a key group: If
% \meta{group} is the name of a key group \meta{key} is a metadata keyword name, then
% \begin{quote}
% |\addmetakey[|\meta{default}|]{|\meta{group}|}{|\meta{key}|}[|\meta{dval}|]|
% \end{quote}
% registers \meta{key} in the metadata group \meta{group}, with an optional values
% \meta{default} and \meta{dval} for \meta{key}. \meta{default} is the default value for
% \meta{key}, if it is not specified, and \meta{dval} is the value \meta{key} gets, if
% \meta{key} is given without specifying a value. These two defaults are often used as
% \begin{quote}
% |\addmetakey[false]{|\meta{group}|}{|\meta{key}|}[true]|
% \end{quote}
% Then, the value of \meta{key} is |false| if \meta{key} is not given and |true|, if
% \meta{key} is specified without value. This is often the best way if we want to use
% \meta{key} as an indicator to have a feature of name \meta{key} (we can test that with
% |\ifx\|\meta{group}|@|\meta{key}|\@true|, if we prepared the macro |\def\@true{true}|
% earlier).
% 
% The keys registered for a metadata group can be used for defining macros with a
% key/value arguments via the \DescribeMacro{\metasetkeys}|\metasetkeys| macro, see for
% instance the the definition in Figure~\ref{fig:foo}. This macro is used exactly like the
% |\setkeys| macro from the |keyval| package~\cite{Carlisle:tkp99}, but integrates custom
% initialization and draft display functionality. This usage is mostly for package
% designers. There is another: If a macro or environment cannot be extended by an optional
% argument, e.g. because anther package already does so (e.g. the |document| environment
% is extended -- by redefining it -- by various packages, which causes problems), the
% |\metasetkeys| macro can be used directly. 
%
% The \DescribeMacro{\addmetalistkey}|\addmetalistkey| macro is a variant of |\addmetakey|
% that adds a list-valued metadata key. The |\addmetalistkey{foo}{val}| in
% Figure~\ref{fig:foo} would allows to use multiple occurrences of the |val| keys in the
% metadata argument of |\foo|, the values of the |val| keys are collected as a
% comma-separated list in the token register |\foo@vals|. Note that the |val| key can also
% deal with comma-separated lists for convenience.
%
% With these definitions in a used package\footnote{\makeatletter Recall that the
% \texttt{@} character is only allowed in packages, where comma-separated lists can be
% iterated over e.g. by the \texttt{\textbackslash{@}for} macro.}  an invocation of
% \begin{quote}
% |\foo[type=bar,id=f4711,val=4,val=7,val={1,1}]|
% \end{quote}
% is formatted to
% \begin{quote}
% \addmetakey{foo}{id}\addmetakey{foo}{type}\addmetalistkey{foo}{val}\makeatletter
% \newcommand\foo[1][]{\metasetkeys{foo}{#1} I have seen a \emph{foo} of type \texttt{\foo@type}
% with identifier \texttt{\foo@id} and values
% \let\@join=\relax\def\@thejoin{, and }
% \@for\@I:=\foo@vals\do{\@join\@I\let\@join=\@thejoin}!}\makeatother 
% \foo[type=bar,id=f4711,val=4,val=7,val=1,val=1]
% \end{quote}
% 
% \begin{exfig}[ht]
% \begin{verbatim}
% \addmetakey{foo}{id}
% \addmetakey{foo}{type}
% \addmetakey[yes]{foo}{visible}
% \addmetalistkey{foo}{val}
% \def\@yes{yes}
% \newcommand\foo[1][]{\metasetkeys{foo}{#1}
% \ifx\foo@visible\@yes % testing for visibility
% I have seen a \emph{foo} of type \texttt{\foo@type} with identifier 
% \texttt\foo@id and values \texttt\foo@vals.
% \let\@join=\relax\def\@thejoin{, and }
% \@for\@I:=\foo@vals\do{\@join\@I\let\@join=\@thejoin}!
% \fi}
% \end{verbatim}
% \vspace*{-2em}
% \caption{Defining a macro with metadata}
% \label{fig:foo}
% \end{exfig}
%
% \subsection{Showing Metadata Keys/Values}\label{sec:user:showmeta}
%
% If the |showmeta| package option is set, the |metakeys| package sets an internal switch that
% shows the values of all keys specified with the |\addmetakey| macro. The default behavior
% is to write the key/value pairs into the margin as \meta{key}|:|\meta{value}. Package
% designers can customize this behavior by redefining the |\metakeys@show@key| and
% |\metakeys@show@keys| macro.
%
% \DescribeMacro{\metakeys@show@key}|\metakeys@show@key{|\meta{key}|}{|\meta{value}|}| shows the a
% single key value pair, and
% \DescribeMacro{\metakeys@show@keys}|\metakeys@show@keys{|\meta{group}|}{|\meta{keys}|}| shows the
% a list of keys metadata, by default we disregard the \meta{group} and show \meta{keys}
% in a marginpar.
% 
% For keys that should not be shown in this manner, the |\addmetakey| macro has a variant
% \DescribeMacro{\addmetakey*}|\addmetakey*|. Its behavior is exactly the same, only that
% it keeps the key from being shown by the |showmeta| option.
%
% Note that setting the |showmeta| option will enable metadata presentation on the whole
% document. But sometimes we want to disable that, e.g. inside figures, where |\marginpar|
% is not allowed. Therefore the |metakeys| package provides the
% \DescribeMacro{\hidemetakeys}|\hidemetakeys| macro that reverses this. The
% \DescribeMacro{\showmetakeys}|\showmetakeys| macro re-enables metadata presentation.
%
%
% \section{Limitations}\label{sec:limitations}
% 
% In this section we document known limitations. If you want to help alleviate them,
% please feel free to contact the package author. Some of them are currently discussed in
% the \sTeX GitHub repository~\cite{sTeX:github:on}. 
% \begin{compactenum}
% \item none reported yet
% \end{compactenum}
% 
% \StopEventually{\newpage\PrintChanges\printbibliography}
% 
% \section{The Implementation}\label{sec:impl}
% 
% The |metakeys| package generates two files: the {\LaTeX} package (all the code between
% {\textsf{$\langle$*package$\rangle$}} and {\textsf{$\langle$/package$\rangle$}}) and the
% {\latexml} bindings (between {\textsf{$\langle$*ltxml$\rangle$ and
% $\langle$/ltxml$\rangle$}}). We keep the corresponding code fragments together, since
% the documentation applies to both of them and to prevent them from getting out of sync.
%
% The general preamble for \latexml:
%    \begin{macrocode}
%<*ltxml>
# -*- CPERL -*-
package LaTeXML::Package::Pool;
use strict;
use LaTeXML::Package;
%</ltxml>
%    \end{macrocode}
%
% \subsection{Package Options}\label{sec:impl:options}
%
% We declare some switches which will modify the behavior according to the package
% options. Generally, an option |xxx| will just set the appropriate switches to true
% (otherwise they stay false). First we have the general options
%    \begin{macrocode}
%<*package>
\newif\ifmetakeys@showmeta\metakeys@showmetafalse
\DeclareOption{showmeta}{\metakeys@showmetatrue}
\DeclareOption*{}
\ProcessOptions 
%</package>
%<*ltxml> 
DeclareOption('showmeta', '');
DeclareOption(undef, '');
ProcessOptions();
%</ltxml>
%    \end{macrocode}
% 
% We build on the |keyval| package which we first need to load. For {\latexml}, we also
% initialize the package inclusions.
%    \begin{macrocode}
%<*package>
\RequirePackage{keyval}[1997/11/10]
\RequirePackage{etoolbox}
%</package>
%<*ltxml>
RequirePackage('keyval');
%</ltxml>
%    \end{macrocode}
%
% \subsection{Adding Metadata Keys}\label{sec:impl:keys}
%
% \begin{macro}{\addmetakey}
%   The |\addmetakey| macro looks at the next character and invokes helper macros accordingly.
%    \begin{macrocode}
%<*package>
\newcommand\addmetakey{\@ifstar\addmetakey@star\addmetakey@nostar}%
%</package>
%    \end{macrocode}
% \end{macro}
%
% \begin{macro}{\addmetakey@star}
%   |\addmetakey@star| takes care of the starred form of |\addmetakey|. An invocation of
%   |\addmetakey@star{|\meta{default}|}{|\meta{group}|}{|\meta{key}|}| macro first extends the
%   |\metakeys@clear@|\meta{group}|@keys| macro then defines the key \meta{key} with the
%   |\define@key| macro from the |keyval| package. This stores the key value given in the
%   local macro |\|\meta{group}|@|\meta{key}.
%    \begin{macrocode}
%<*package>
\newcommand\addmetakey@star[3][]{%
  \@ifnextchar[{%
    \addmetakey@star@aux[#1]{#2}{#3}%
  }{%
    \addmetakey@star@aux[#1]{#2}{#3}[]%
  }%
}%
\def\addmetakey@star@aux[#1]#2#3[#4]{%
  \metakeys@ext@clear@keys{#2}{#3}{#1}%
  \metakeys@initialize@showkeys{#2}%
  \define@key{#2}{#3}[#4]{%
    \csxdef{#2@#3}{##1}%
  }%
}%
%    \end{macrocode}
% \end{macro}
%
% \begin{macro}{\addmetakey@nostar}
%   |\addmetakey@nostar| takes care of the no-starred form of |\addmetakey| by first extending the
%   |\metakeys@|\meta{group}|@showkeys| macro which contains those keys that should be
%   shown and then calling |\addmetakey@star|.
%    \begin{macrocode}
\newcommand\addmetakey@nostar[3][]{%
  \metakeys@ext@showkeys{#2}{#3}%
  \addmetakey@star[#1]{#2}{#3}%
}%
%</package>
%    \end{macrocode}
% \end{macro}
%
% \begin{macro}{\metasetkeys}
% The |\metasetkeys{|\meta{group}|}| clears/presets the key of \meta{group} via
% |\clear@|\meta{group}|@keys|, (if the |showmeta| option is set) shows them, and
% then sets the keys via |keyval|s |\setkeys| command.
%    \begin{macrocode}
%<*package>
\newcommand\metasetkeys[2]{%
  \@nameuse{clear@#1@keys}%
  \setkeys{#1}{#2}%
  \ifmetakeys@showmeta%
    \edef\@@keys{\@nameuse{#1@showkeys}}%
    \metakeys@show@keys{#1}{%
      \@for\@I:=\@@keys\do{%
        \metakeys@show@keyval{#1}{\@I}%
      }%
    }%
  \fi%
}%
%</package>
%    \end{macrocode}
% \end{macro}
%
% \begin{macro}{\metakeys@ext@clear@keys}
%   |\metakeys@ext@clear@keys{|\meta{group}|}{|\meta{key}|}{|\meta{default}|}| extends (or
%   sets up if this is the first |\addmetakey| for \meta{group}) the
%   |\clear@|\meta{group}|@keys| macro to set the default value \meta{default} for
%   \meta{key}. The |\clear@|\meta{group}|@keys| macro is used in the generic
%   |\metasetkeys| macro below. The variant |\@metakeys@ext@clear@keys| is provided for
%   use in the |sref| package.
%    \begin{macrocode}
%<*package>
\newrobustcmd\metakeys@ext@clear@keys[3]{%
  \@metakeys@ext@clear@keys{#1}{#1@#2}{#3}%
}%
\newrobustcmd\@metakeys@ext@clear@keys[3]{%
  \@ifundefined{clear@#1@keys}{%
	\csgdef{clear@#1@keys}{\csgdef{#2}{#3}}%
  }%
  {\expandafter\gappto\csname clear@#1@keys\endcsname{\csgdef{#2}{#3}}}%
}%
%</package>
%    \end{macrocode}
% \end{macro}
%
% \begin{macro}{\addmetalistkey}
%    \begin{macrocode}
%<*package>
\newrobustcmd\addmetalistkey{%
  \@ifstar\addmetalistkey@star\addmetalistkey@nostar%
}%
\newrobustcmd\addmetalistkey@star[3][]{%
  \metakeys@ext@clear@keys{#2}{#3}{#1}%
  \metakeys@initialize@showkeys{#2}%
  \csgdef{#2@#3s}{}%
  \define@key{#2}{#3}[#1]{%
    \ifcsempty{#2@#3s}{%
      \csgdef{#2@#3s}{##1}%
    }{%
      \csxdef{#2@#3s}{\csuse{#2@#3s},##1}%
    }%
  }%
}%
\newrobustcmd\addmetalistkey@nostar[3][]{%
  \metakeys@ext@showkeys{#2}{#3}%
  \addmetalistkey@star[#1]{#2}{#3}%
}%
%</package>
%    \end{macrocode}
% \end{macro}
%
% \subsection{Showing Metadata Keys/Values}\label{sec:impl:showmeta}
%
% \begin{macro}{\metakeys@initialize@showkeys}
%   |\metakeys@initialize@showkeys{|\meta{group}|}| sets up the |\|\meta{group}|@showkeys|
%   macro which is is used to store the keys to be shown of the metadata in in the generic
%   |\setmetakeys| macro below.
%    \begin{macrocode}
%<*package>
\newrobustcmd\metakeys@initialize@showkeys[1]{%
  \@ifundefined{#1@showkeys}{%
    \csdef{#1@showkeys}{}%
  }{}%
}%
%    \end{macrocode}
% \end{macro}
%
% \begin{macro}{\metakeys@ext@showkeys}
%   |\metakeys@ext@showkeys{|\meta{group}|}{|\meta{key}|}| extends (or sets up) the
%   |\|\meta{group}|@showkeys| macro which is is used to store the keys to be shown of the
%   metadata in in the generic |\setmetakeys| macro below.
%    \begin{macrocode}
\newrobustcmd\metakeys@ext@showkeys[2]{%
  \@ifundefined{#1@showkeys}{%
    \csdef{#1@showkeys}{#2}%
  }{%
    \csedef{#1@showkeys}{\csuse{#1@showkeys},#2}%
  }%
}%
%    \end{macrocode}
% \end{macro}
%
% \begin{macro}{\metakeys@show@key}
%   |\metakeys@show@key{|\meta{key}|}{|\meta{value}|}| shows the a single key value pair, as a
%   default we just write \meta{key}|:|\meta{value}.
%    \begin{macrocode}
\newrobustcmd\@metakeys@show@key[2]{\metakeys@show@key{#2}{#1}}%
\newrobustcmd\metakeys@show@key[2]{%
  \edef\@test{#2}%
  \ifx\@test\@empty\else #1:#2\quad\fi%
}%
%    \end{macrocode}
% \end{macro}
%
% \begin{macro}{\metakeys@show@keys}
%   |\metakeys@show@keys{|\meta{group}|}{|\meta{keys}|}| shows the metadata, by default we
%   disregard the \meta{group} and show \meta{keys} in a marginpar. 
%    \begin{macrocode}
\newrobustcmd\metakeys@show@keys[2]{\marginpar{{\scriptsize #2}}}%
%    \end{macrocode}
% \end{macro}
%
% \begin{macro}{\metakeys@show@keyval}
%   |\metakeys@show@keyval{|\meta{group}|}|{|\meta{key}|} shows the key/value pair of a given key \meta{key}.
%    \begin{macrocode}
\newrobustcmd\metakeys@show@keyval[2]{%
  \expandafter\@metakeys@show@key\csname #1@#2\endcsname{#2}%
}%
%</package>
%    \end{macrocode}
% \end{macro}
%
% \begin{macro}{\showmetakeys}
%    \begin{macrocode}
%<*package>
\newrobustcmd\showmetakeys{\metakeys@showmetatrue}%
%</package>
%<*ltxml> 
DefConstructor('\showmetakeys','');
%</ltxml>
%    \end{macrocode}
% \end{macro}
%
% \begin{macro}{\hidemetakeys}
%    \begin{macrocode}
%<*package>
\newrobustcmd\hidemetakeys{\metakeys@showmetafalse}%
%</package>
%<*ltxml> 
DefConstructor('\hidemetakeys','');
%</ltxml>
%    \end{macrocode}
% \end{macro}
%
% \subsection{Using better defaults than empty}
% 
% \begin{macro}{\addmetakeynew}
%   |\addmetakeynew| is an experimental version of |\addmetakey| which gives
%   |\omd@unspecified| as an optional argument, so that it is used as the default value
%   here and then test for it in |\omfidus|. But unfortunately, this does not work yet.
%    \begin{macrocode}
%<*package>
\newrobustcmd\addmetakeynew[3][]{%
  \metakeys@ext@clear@keys{#2}{#3}{#1}%
  \define@key{#2}{#3}{%
    \csgdef{#2@#3}{##1}%
  }%
}%
%    \end{macrocode}
% \end{macro}
%
% 
% \begin{macro}{\metakeys@unspecified}
%   An internal macro for unspecified values. It is used to initialize keys.\ednote{MK:
%   we could probably embed an package error or warning in here}
%    \begin{macrocode}
\newrobustcmd\metakeys@unspecified{an metakeys-defined key left unspecified}%
%    \end{macrocode}
% \end{macro}
%
% \begin{macro}{\metakeysifus}
%   This just tests for equality of the first arg with |\metakeys@unspecified|
%    \begin{macrocode}
\newrobustcmd\metakeysifus[4]{%
  \message{testing #1@#2=\csname#1@#2\endcsname}%
  \expandafter\ifx\csname #1@#2\endcsname\metakeys@unspecified{#3}\else{#4}\fi%
}%
%</package>
%    \end{macrocode}
% \end{macro}
%
% \subsection{Finale}
%
% Finally, we need to terminate the file with a success mark for perl.
%    \begin{macrocode}
%<ltxml>1;
%    \end{macrocode}
% \Finale
\endinput
% \iffalse
%%% Local Variables: 
%%% mode: doctex
%%% TeX-master: t
%%% End: 
% \fi


% LocalWords:  iffalse kohlhase Thu metakeys metakeys.dtx scsys sc sc latexml omdoc foo
% LocalWords:  omdocv texttt fileversion maketitle setcounter tocdepth newpage filedate
% LocalWords:  tableofcontents ednote showmeta NeedsTeXFormat stex rdfmeta ctancite dval
% LocalWords:  makeatletter newcommand emph makeatother exfig vspace impl ltxml textbf
% LocalWords:  printbibliography keyval expandafter gdef csname ifx tkp99 sref
% LocalWords:  endcsname setkeys ifundefined omfidus metakeysifus addmetakey foo@vals
% LocalWords:  doctex metasetkeys metasetkeys marginpar hidemetakeys textsf omd rangle
% LocalWords:  hidemetakeys showmetakeys showmetakeys langle textsf langle addmetalistkey
% LocalWords:  newif ifmetakeys showmetafalse showmetatrue ifstar showkeys addmetalistkey
% LocalWords:  nameuse setmetakeys addmetakeynew textbackslash compactenum rangle xdef
%  LocalWords:  addmetakey@nostar addmetalistkey@nostar scriptsize endinput
