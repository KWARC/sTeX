% \iffalse meta-comment
% A LaTeX Class and Package for OMDoc Text Fragments
% Copyright (c) 2010 Michael Kohlhase, all rights reserved
%               this file is released under the
%               LaTeX Project Public License (LPPL)
%
% The original of this file is in the public repository at 
% http://github.com/KWARC/sTeX/
% \fi
%   
% \iffalse
%<package>\NeedsTeXFormat{LaTeX2e}[1999/12/01]
%<package>\ProvidesPackage{omtext}[2012/11/06 v1.0 OMDoc Text Fragments]
%
%<*driver>
\documentclass{ltxdoc}
\usepackage{url,array,float}
\usepackage{omtext,omdoc}
\usepackage[show]{ed}
\usepackage[hyperref=auto,style=alphabetic]{biblatex}
\addbibresource{kwarcpubs.bib}
\addbibresource{extpubs.bib}
\addbibresource{kwarccrossrefs.bib}
\addbibresource{extcrossrefs.bib}
\usepackage{stex-logo}
\usepackage{../ctangit}
\usepackage{hyperref}
\makeindex
\floatstyle{boxed}
\newfloat{exfig}{thp}{lop}
\floatname{exfig}{Example}
\def\tracissue#1{\cite{sTeX:online}, \hyperlink{http://trac.kwarc.info/sTeX/ticket/#1}{issue #1}}
\begin{document}\DocInput{omtext.dtx}\end{document}
%</driver>
% \fi
% 
%\iffalse\CheckSum{392}\fi
% 
% \changes{v0.4}{2008/07/22}{added index markup}
% \changes{v0.4}{2008/09/29}{augmenting the index macros with optional values}
% \changes{v0.6}{2009/07/08}{removing {\texttt{ttin}} {\texttt{{emin}}} and friends, 
%   they were almost never used.}
% \changes{v0.6}{2009/07/08}{renmanig {\texttt{myindex}} to {\texttt{omdoc\@ index}},
%   {\texttt{twin}} to {\texttt{\@ twin}}, and {\texttt{atwin}} to {\texttt{\@ atwin}} to
%   make them packge-local} 
% \changes{v0.7}{2010/02/11}{changing blockquote to sblockquote and inlinequote similarly}
% \changes{v0.9}{2010/05/25}{separated out from \texttt{omdoc.dtx}}
% \changes{v1.0}{2010/06/18}{fixing typos}
% 
% \GetFileInfo{omtext.sty}
% 
% \MakeShortVerb{\|}
%
% \def\omdoc{OMDoc}
% \def\latexml{{\LaTeX}ML}
% \title{{\texttt{omtext}}: Semantic Markup for Mathematical Text Fragments in {\LaTeX}\thanks{Version {\fileversion} (last revised
%        {\filedate})}}
%    \author{Michael Kohlhase\\
%            Jacobs University, Bremen\\
%            \url{http://kwarc.info/kohlhase}}
% \maketitle
%
% \begin{abstract}
%   The |omtext| package is part of the {\sTeX} collection, a version of {\TeX/\LaTeX} that
%   allows to markup {\TeX/\LaTeX} documents semantically without leaving the document
%   format, essentially turning {\TeX/\LaTeX} into a document format for mathematical
%   knowledge management (MKM).
%
%   This package supplies an infrastructure for writing {\omdoc} text fragments in
%   {\LaTeX}.  
% \end{abstract}
%
%\newpage\tableofcontents\newpage
% 
%\begin{omgroup}[id=sec:STR]{Introduction}
%
%  The |omtext| package supplies macros and environment that allow to mark up mathematical
%  texts in {\sTeX}, a version of {\TeX/\LaTeX} that allows to markup {\TeX/\LaTeX}
%  documents semantically without leaving the document format, essentially turning
%  {\TeX/\LaTeX} into a document format for mathematical knowledge management (MKM). The
%  package supports direct translation to the {\omdoc} format~\cite{Kohlhase:OMDoc1.2}
% \end{omgroup}
% 
% \begin{omgroup}[id=sec:user]{The User Interface}
% 
% \begin{omgroup}[id=sec:user:options]{Package Options}
% 
%   The |omtext| package takes a single option: \DescribeMacro{showmeta}|showmeta|. If
%   this is set, then the metadata keys are shown (see~\cite{Kohlhase:metakeys:ctan} for
%   details and customization options).
% \end{omgroup}
% 
% \begin{omgroup}[id=sec:user:omtext]{Mathematical Text}
% 
%   The \DescribeEnv{omtext}|omtext| environment is used for any text fragment that has a
%   contribution to a text that needs to be marked up. It can have a title, which can be
%   specified via the \DescribeMacro{title=}|title| key. Often it is also helpful to
%   annotate the \DescribeMacro{type=}|type| key. The standard relations from rhetorical
%   structure theory |abstract|, |introduction|, |conclusion|, |thesis|, |comment|,
%   |antithesis|, |elaboration|, |motivation|, |evidence|, |transition|, | note|, |annote|
%   are recommended as values. Note that some of them are unary relations like
%   |introduction|, which calls for a target. In this case, a target using the
%   \DescribeMacro{for=}|for| key should be specified. The |transition| relation is
%   special in that it is binary (a ``transition between two statements''), so
%   additionally, a source should be specified using the \DescribeMacro{from=}|from| key.
% 
%   Note that the values of the |title| and |type| keys are often displayed in the
%   text. This can be turned off by setting the \DescribeMacro{display=}|display| key to
%   the value |flow|. Sometimes we want to specify that a text is a continuation of
%   another, this can be done by giving the identifier of this in the
%   \DescribeMacro{continues=}|continues| key.
%
%   Finally, there is a set of keys that pertain to the mathematical formulae in the
%   text. The \DescribeMacro{functions=}|functions| key allows to specify a list of
%   identifiers that are to be interpreted as functions in the generate content
%   markup. The \DescribeMacro{theory=}|theory| specifies a module
%   (see~\cite{KohAmb:smmssl:svn}) that is to be pre-loaded in this one\ednote{this is not
%   implemented yet.} Finally, \DescribeMacro{verbalizes=}|verbalizes| specifies a (more)
%   formal statement (see~\cite{Kohlhase:smms:svn}) that this text verbalizes or
%   paraphrases.\ednote{MK:specify the form of the reference.}
% \end{omgroup}
% 
% \begin{omgroup}[id=sec:user:phrase]{Phrase-Level Markup}
%
%   \DescribeMacro{\phrase} The |phrase| macro allows to mark up phrases with semantic
%   information. It takes an optional |KeyVal| argument with the keys
%   \DescribeMacro{verbalizes=}|verbalizes| and \DescribeMacro{type=}|type| as above and
%   \DescribeMacro{style}|style,| \DescribeMacro{class}|class|,
%   \DescribeMacro{index}|index| that are disregarded in the {\LaTeX}, but copied into the
%   generated content markup.
% 
%   We use the \DescribeMacro{\nlex}|\nlex{|\meta{phrase}|}| for marking up phrases that
%   serve as natural language examples and \DescribeMacro{\nlcex}|\nlcex{|\meta{phrase}|}|
%   for counter-examples (utterances that are not acceptable for some reason). In natural
%   language examples, we sometimes use ``co-rereference markers'' to specify the
%   resolution of anaphora and the like. We use the
%   \DescribeMacro{\coreft}|\coreft{|\meta{phrase}|}{|\meta{mark}|}| to mark up the
%   ``target'' of a co-reference and analogously \DescribeMacro{\corefs}|\corefs| for
%   coreference source -- e.g. for an anaphoric reference. The usage is the following:
%   \begin{verbatim}
%   \nlex{If \coreft{a farmer}1 owns \coreft{a donkey}2, 
%            \corefs{he}2 beats \corefs{it}2.}
%   \end{verbatim}
%   is formatted to 
%   \begin{quote}
%     \nlex{If \coreft{a farmer}1 owns \coreft{a donkey}2, \corefs{he}2 beats \corefs{it}2.}
%   \end{quote}
%   
%   \DescribeMacro{\sinlinequote} The |sinlinequote| macro allows to mark up quotes inline
%   and attribute them. The quote itself is given as the argument, possibly preceded by
%   the a specification of the source in a an optional argument. For instance, we would
%   quote Hamlet with
% \begin{verbatim}
% \sinlinequote[Hamlet, \cite{Shak:1603:Hamlet}]{To be or not to be}
% \end{verbatim}
% which would appear as \sinlinequote[Hamlet, (Shakespeare 1603)]{To be or not to be} in
% the text. The style in which inline quotations appear in the text can be adapted by
% specializing the macros \DescribeMacro{\@sinlinequote}|\@sinlinequote| --- for quotations
% without source and \DescribeMacro{\@@sinlinequote}|\@@sinlinequote| --- for quotations with
% source.
% \end{omgroup}
% 
% \begin{omgroup}[id=sec:user:block]{Block-Level Markup}
%
%   \DescribeEnv{sblockquote} The |sblockquote| environment is the big brother of the
%   |\sinlinequote| macro. It also takes an optional argument to specify the source. Here
%   the four internal macros \DescribeMacro{\begin@sblockquote}|\begin@sblockquote| to
%       \DescribeMacro{\end@@sblockquote}|\end@@sblockquote| are used for styling and can be
%   adapted by package integrators. Here a quote of Hamlet would marked up as
% \begin{verbatim}
% \begin{sblockquote}[Hamlet, \cite{Shak:1603:Hamlet}]\obeylines
%   To be, or not to be: that is the question:
%   Whether 'tis nobler in the mind to suffer
% \end{sblockquote}
% \end{verbatim}
% and would render as 
% \begin{sblockquote}[Hamlet, (Shakespeare 1603)]\obeylines
%   To be, or not to be: that is the question:
%   Whether 'tis nobler in the mind to suffer
% \end{sblockquote}
% 
% \DescribeMacro{\lec}The |\lec| macro takes one argument and sets it as a comment at the
% end of the line, making sure that if the content is too long it is pushed into a new
% line. We use it internally for placing the of source of the |sblockquote| environment
% above.  The actual appearance of the line end comment is determined by the
% \DescribeMacro{\@@lec}|\@@lec| macro, which can be customized in the document class.
% \end{omgroup}
% 
% \begin{omgroup}[id=sec:user:index]{Index Markup}
%
%   The |omtext| package provides some extensions for the well-known indexing macros of
%   {\LaTeX}. The main reason for introducing these macros is that index markup in
%   {\omdoc} wraps the indexed terms rather than just marking the spot for
%   cross-referencing. Furthermore the index commands only indexes words unless
%   the\DescribeMacro{noindex} |noindex| option is set in the |\usepackage|. The |omtext|
%   package and class make the usual |\index| macro undefined\ednote{implement this and
%     issue the respective error message}.
% 
%   The \DescribeMacro{\indextoo}|\indextoo| macro renders a word and marks it for the
%   index. Sometimes, we want to index a slightly different form of the word, e.g. for
%   non-standard plurals: while |\indextoo{word}s| works fine, we cannot use this for the
%   word ``datum'', which has the plural ``data''. For this we have the macro
%   \DescribeMacro{\indexalt}|\indexalt|, which takes another argument for the displayed
%   text, allowing us to use |\indexalt{data}{datum}|, which prints ``data'' but puts
%   ``datum'' into the index.
% 
%   The second set of macros adds an infrastructure for two-word compounds. Take for
%   instance the compound ``OMDoc document'', which we usually want to add into the index
%   under ``OMDoc'' and ``document''. \DescribeMacro{\twintoo}|\twintoo{OMDoc}{document}|
%   is a variant of |\indextoo| that will do just this. Again, we have a version that
%   prints a variant: This is useful for situations like this the one in
%   Figure~\ref{fig:index-markup}:
% \begin{exfig}
% \begin{verbatim}
% We call group \twinalt{Abelian}{Abelian}{group}, iff \ldots
% \end{verbatim}
% \vspace*{-1em}will result in the following
% \begin{quote}
%   We call group \twinalt{Abelian}{Abelian}{group}, iff \ldots
% \end{quote}
% and put ``Abelian Group'' into the index.
% \caption{Index markup}\label{fig:index-markup}
% \end{exfig}
% 
% The third set of macros does the same for two-word compounds with adjectives,
% e.g. ``wonderful OMDoc
% document''. \DescribeMacro{\atwintoo}|\atwin{wonderful}{OMdoc}{document}| will make the
% necessary index entries under ``wonderful'' and ``document''. Again, we have a variant
% \DescribeMacro{\atwinalt}|\atwinalt| whose first argument is the alternative text.
%
% All index macros take an optional first argument that is used for ordering the
% respective entries in the index.
% \end{omgroup}
%
% \begin{omgroup}{Support for \textsf{MathHub}}\label{sec:user:mathhub}
% 
% Much of the \sTeX content is hosed on \textsf{MathHub} (\url{http://MathHub.info}), a
% portal and archive for flexiformal mathematics. \textsf{MathHub} offers GIT repositories
% (public and private escrow) for mathematical documentation projects, online and offline
% authoring and document development infrastructure, and a rich, interactive reading
% interface. The |modules| package supports repository-sensitive operations on
% \textsf{MathHub}.
% 
% Note that \textsf{MathHub} has two-level repository names of the form
% \meta{group}|/|\meta{repo}, where \meta{group} is a \textsf{MathHub}-unique repository
% group and \meta{repo} a repository name that is \meta{group}-unique. The file and
% directory structure of a repository is arbitrary -- except that it starts with the
% directory |source| because they are Math Archives in the sense
% of~\cite{HorIacJuc:cscpnrr11}. But this structure can be hidden from the \sTeX author
% with \textsf{MathHub}-enabled versions of the |modules| macros.
% 
% The \DescribeMacro{\mhcgraphics}|\mhcgraphics| macro is a variant of |\mycgraphics| with
% repository support. Instead of writing
% \begin{verbatim}
% \defpath{MathHub}{/user/foo/lmh/MathHub}
% \mycgraphics{\MathHub{fooMH/bar/source/baz/foobar}}
% \end{verbatim}
% we can simply write (assuming that |\MathHub| is defined as above)
% \begin{verbatim}
% \mhcgraphics[fooMH/bar]{baz/foobar}
% \end{verbatim}
% Note that the |\mhcgraphics| form is more semantic, which allows more advanced document
% management features in \textsf{MathHub}.
% 
% If |baz/foobar| is the ``current module'', i.e. if we are on the \textsf{MathHub} path
% \ldots|MathHub/fooMH/bar|\ldots, then stating the repository in the first optional
% argument is redundant, so we can just use
% \begin{verbatim}
% \mhcgraphics{baz/foobar}
% \end{verbatim}
% Of course, neither {\LaTeX} nor \latexml know about the repositories when they are
% called from a file system, so we can use the |\mhcurrentrepos| macro from the |modules|
% package to tell them. But this is only needed to initialize the infrastructure in the
% driver file. In particular, we do not need to set it in in each module, since the
% |\importmhmodule| macro sets the current repository automatically.
% 
% \paragraph{Caveat} if you want to use the \textsf{MathHub} support macros (let's call
% them mh-variants), then every time a module is imported or a document fragment is
% included from another repos, the mh-variant |\importmhmodule| must be used, so that the
% ``current repository'' is set accordingly. To be exact, we only need to use mh-variants,
% if the imported module or included document fragment use mh-variants.
% \end{omgroup}
% \end{omgroup}
% 
% \begin{omgroup}{Limitations}\label{sec:limitations}
% 
% In this section we document known limitations. If you want to help alleviate them,
% please feel free to contact the package author. Some of them are currently discussed in
% the \sTeX TRAC~\cite{sTeX:online}. 
% \begin{compactenum}
% \item none reported yet
% \end{compactenum}
% \end{omgroup}
% 
% \StopEventually{\newpage\PrintIndex\newpage\PrintChanges\printbibliography}\newpage
%
% \begin{omgroup}[id=sec:impl]{Implementation}
%
% The |omtext| package generates two files: the {\LaTeX} package (all the code between
% {\textsf{$\langle$*package$\rangle$}} and {\textsf{$\langle$/package$\rangle$}}) and the
% {\latexml} bindings (between {\textsf{$\langle$*ltxml$\rangle$ and
%     $\langle$/ltxml$\rangle$}}). We keep the corresponding code fragments together,
% since the documentation applies to both of them and to prevent them from getting out of
% sync.
%
% \begin{omgroup}[id=sec:impl:options]{Package Options}
% 
% The initial setup for {\latexml}: 
%
%    \begin{macrocode}
%<*ltxml>
package LaTeXML::Package::Pool;
use strict;
use LaTeXML::Package;
use LaTeXML::Util::Pathname;
%</ltxml>
%    \end{macrocode}
%
% We declare some switches which will modify the behavior according to the package
% options. Generally, an option |xxx| will just set the appropriate switches to true
% (otherwise they stay false).\ednote{need an implementation for {\latexml}}
%
%    \begin{macrocode}
%<*package>
\DeclareOption{showmeta}{\PassOptionsToPackage{\CurrentOption}{metakeys}}
\newif\ifindex\indextrue
\DeclareOption{noindex}{\indexfalse}
\ProcessOptions
\ifindex\makeindex\fi
%</package>
%<*ltxml>
DeclareOption('noindex','');
%</ltxml>
%    \end{macrocode}
%
% Then we need to set up the packages by requiring the |sref| package to be loaded.
%
%    \begin{macrocode}
%<*package>
\RequirePackage{sref}
\RequirePackage{xspace}
\RequirePackage{modules}
\RequirePackage{comment}
\RequirePackage{mdframed}
%</package>
%<*ltxml>
RequirePackage('sref');
RequirePackage('xspace');
RequirePackage('modules');
RequirePackage('lxRDFa');
%</ltxml>
%    \end{macrocode}
% \end{omgroup}
% 
% \begin{omgroup}[id=sec:impl:metadata]{Metadata}
% 
%   All the {\omdoc} elements allow to specify metadata in them, which is modeled by the
%   |omdoc:metadata| element. Since the content of this element is precisely controlled
%   by the Schema, we can afford to auto-open and auto-close it. Thus metadata elements
%   from various sources will just be included into one |omdoc:metadata| element, even if
%   they are supplied by different {\sTeX} bindings. Also we add numbering and location
%   facilities.
%
%    \begin{macrocode}
%<*ltxml> 
Tag('omdoc:metadata',afterOpen=>\&numberIt,afterClose=>\&locateIt,autoClose=>1,autoOpen=>1); 
%</ltxml>
%    \end{macrocode}
% 
% the |itemize|, |description|, and |enumerate| environments generate |omdoc:li|,
% |omdoc:di| with |autoclose| inside a CMP. This behavior will be overwritten later, so we
% remember that we are in a |CMP| by assigning |_LastSeenCMP|.
% 
%    \begin{macrocode}
%<*ltxml>
Tag('omdoc:CMP', afterOpen => sub {AssignValue('_LastSeenCMP', $_[1], 'global');return;});#$
%</ltxml>
%    \end{macrocode}
%
% the |itemize|, |description|, and |enumerate| environments originally introduced in the
% |omtext| package do double duty in OMDoc, outside a |CMP| they are transformed into a 
%   |<omgroup layout='itemize|description|enumerate'>|, where the text after the macros
%   |\item| come to be the children. If that is only text, then it is enclosed in an
%   |<omtext><CMP>|, otherwise it is left as it is. The optional argument of the |\item|
%   is transformed into the |<metadata><dc:title>| of the generated |\item| element.
%    \begin{macrocode}
%<*ltxml>
DefParameterType('IfBeginFollows', sub {
		   my ($gullet) = @_;
		   $gullet->skipSpaces;
                   my $next = $gullet->readToken;
                   $gullet->unread($next);
                   $next = ToString($next);
                   #Hm, falling back to regexp handling, the $gullet->ifNext approach didn't work properly
                   return 1 unless ($next=~/^\\begin/);
                   return;
                 },
		 reversion=>'', optional=>1);
%</ltxml>
%    \end{macrocode}
% \end{omgroup}
% 
% \begin{omgroup}[id=sec:impl:mtxt]{Mathematical Text}
%
%   We define the actions that are undertaken, when the keys are encountered. The first
%   set just records metadata; this is very simple via the |\addmetakey|
%   infrastructure~\ctancite{Kohlhase:metakeys}. Note that we allow math in the |title|
%   field, so we do not declare it to be |Semiverbatim| (indeed not at all, which allows
%   it by default).
%
%    \begin{macrocode}
%<*package>
\srefaddidkey{omtext}
\addmetakey[]{omtext}{functions}
\addmetakey*{omtext}{display}
\addmetakey{omtext}{for}
\addmetakey{omtext}{from}
\addmetakey{omtext}{type}
\addmetakey*{omtext}{title}
\addmetakey*{omtext}{start}
\addmetakey{omtext}{theory}
\addmetakey{omtext}{continues}
\addmetakey{omtext}{verbalizes}
\addmetakey{omtext}{subject}
%</package>
%<*ltxml>
DefKeyVal('omtext','functions','CommaList');
DefKeyVal('omtext','display','Semiverbatim');
DefKeyVal('omtext','for','Semiverbatim');
DefKeyVal('omtext','from','Semiverbatim');
DefKeyVal('omtext','type','Semiverbatim');
DefKeyVal('omtext','title','Plain'); #Math mode in titles.
DefKeyVal('omtext','start','Plain'); #Math mode in start phrases
DefKeyVal('omtext','theory','Semiverbatim');
DefKeyVal('omtext','continues','Semiverbatim');
DefKeyVal('omtext','verbalizes','Semiverbatim');
%</ltxml>
%    \end{macrocode}
% The next keys handle module loading (see~\ctancite{KohAmb:smmssl}). 
%    \begin{macrocode}
% \ednote{need to implement these in LaTeXML, I wonder whether there is a general
% mechanism like numberit.}\ednote{this needs to be rethought in the light of
% |\usemodule|. It is probably obsolete. Is this used? Is this documented?}
%<*package>
\define@key{omtext}{require}{\requiremodules{#1}{sms}}
\define@key{omtext}{module}{\message{module: #1}\importmodule{#1}\def\omtext@theory{#1}}
%</package>
%<*ltxml>
%</ltxml>
%    \end{macrocode}
%
% \begin{macro}{\st@flow}
% We define this macro, so that we can test whether the |display| key has the value |flow|
%    \begin{macrocode}
%<*package>
\def\st@flow{flow}
%</package>
%    \end{macrocode}
% \end{macro}
%
% \begin{environment}{omtext}
%   The |omtext| environment is different, it does not have a keyword that marks
%   it. Instead, it can have a title, which is used in a similar way. We redefine the
%   |\lec| macro so the trailing |\par| does not get into the way.
%    \begin{macrocode}
%<*package>
\def\omtext@pre@skip{\smallskip}
\def\omtext@post@skip{}
\providecommand{\stDMemph}[1]{\textbf{#1}}
\newenvironment{omtext}[1][]{\bgroup\metasetkeys{omtext}{#1}\sref@label@id{this paragraph}%
\def\lec##1{\@lec{##1}}%
\ifx\omtext@display\st@flow\else\omtext@pre@skip\par\noindent%
\ifx\omtext@title\@empty%
\ifx\omtext@start\@empty\else\stDMemph{\omtext@start}\xspace\fi%
\else\stDMemph{\omtext@title}:\xspace%
\ifx\omtext@start\@empty\else\omtext@start\xspace\fi%
\fi% omtext@title empty
\fi% omtext@display=flow
\ignorespaces}
{\egroup\omtext@post@skip}
%</package>
%<*ltxml>
DefEnvironment('{omtext} OptionalKeyVals:omtext',
		  "<omdoc:omtext "
		     . "?&GetKeyVal(#1,'id')(xml:id='&GetKeyVal(#1,'id')')() "
		     . "?&GetKeyVal(#1,'type')(type='&GetKeyVal(#1,'type')')() "
		     . "?&GetKeyVal(#1,'for')(for='&GetKeyVal(#1,'for')')() "
	             . "?&GetKeyVal(#1,'from')(from='&GetKeyVal(#1,'from')')()>"
		  . "?&GetKeyVal(#1,'title')(<dc:title>&GetKeyVal(#1,'title')</dc:title>)()"
		  .     "?&GetKeyVal(#1,'start')(<ltx:text class='startemph'>&GetKeyVal(#1,'start')</ltx:text>)()"
                  .     "#body"
                 ."</omdoc:omtext>");
%</ltxml>
%    \end{macrocode}
% \end{environment}
% \end{omgroup}
%
% \begin{omgroup}[id=sec:impl:phrase]{Phrase-level Markup}
% 
% \begin{macro}{\phrase}
%    For the moment, we do disregard the most of the keys
%    \begin{macrocode}
%<*package>
\srefaddidkey{phrase}
\addmetakey{phrase}{style}
\addmetakey{phrase}{class}
\addmetakey{phrase}{index}
\addmetakey{phrase}{verbalizes}
\addmetakey{phrase}{type}
\addmetakey{phrase}{only}
\newcommand\phrase[2][]{\metasetkeys{phrase}{#1}%
\ifx\prhase@only\@empty\only<\phrase@only>{#2}\else #2\fi}
%</package>
%<*ltxml>
DefKeyVal('phrase','id','Semiverbatim');
DefKeyVal('phrase','style','Semiverbatim');
DefKeyVal('phrase','class','Semiverbatim');
DefKeyVal('phrase','index','Semiverbatim');
DefKeyVal('phrase','verbalizes','Semiverbatim');
DefKeyVal('phrase','type','Semiverbatim');
DefKeyVal('phrase','only','Semiverbatim');
DefConstructor('\phrase OptionalKeyVals:phrase {}',
	       "<ltx:text %&GetKeyVals(#1) ?&GetKeyVal(#1,'only')(rel='beamer:only' content='&GetKeyVal(#1,'only')')()>#2</ltx:text>");
%</ltxml>
%    \end{macrocode}
% \end{macro}
%
% \begin{macro}{\coref*}
%    \begin{macrocode}
%<*package>
\providecommand\textsubscript[1]{\ensuremath{_{#1}}}
\newcommand\corefs[2]{#1\textsubscript{#2}}
\newcommand\coreft[2]{#1\textsuperscript{#2}}
%</package>
%<*ltxml>
DefConstructor('\corefs{}',
  "<ltx:text class='coref-source' stex:index='#2'>#1</ltx:text>");
DefConstructor('\coreft{}',
  "<ltx:text class='coref-target' stex:index='#2'>#1</ltx:text>");
%</ltxml>
%    \end{macrocode}
% \end{macro}
%
% \begin{macro}{\n*lex}
%    \begin{macrocode}
%<*package>
\newcommand\nlex[1]{\green{\sl{#1}}}
\newcommand\nlcex[1]{*\green{\sl{#1}}}
%</package>
%<*ltxml>
DefConstructor('\nlex{}',"<ltx:text class='nlex'>#1</ltx:text>");
DefConstructor('\nlcex{}',"<ltx:text class='nlcex'>#1</ltx:text>");
%</ltxml>
%    \end{macrocode}
% \end{macro}
% 
% \begin{macro}{sinlinequote}
%    \begin{macrocode}
%<*package>
\def\@sinlinequote#1{``{\sl{#1}}''}
\def\@@sinlinequote#1#2{\@sinlinequote{#2}~#1}
\newcommand\sinlinequote[2][]
{\def\@opt{#1}\ifx\@opt\@empty\@sinlinequote{#2}\else\@@sinlinequote\@opt{#2}\fi}
%</package>
%<*ltxml>
DefConstructor('\sinlinequote [] {}',
              "<ltx:quote type='inlinequote'>"
               . "?#1(<dc:source>#1</dc:source>\n)()"
               . "#2"
            . "</ltx:quote>");
%</ltxml>
%    \end{macrocode}
% \end{macro}
% \end{omgroup}
%
% \begin{omgroup}[id=sec:impl:block]{Block-Level Markup}
%
% \begin{environment}{sblockquote}
%    \begin{macrocode}
%<*package>
\def\begin@sblockquote{\begin{quote}\sl}
\def\end@sblockquote{\end{quote}}
\def\begin@@sblockquote#1{\begin@sblockquote}
\def\end@@sblockquote#1{\def\@@lec##1{{\rm ##1}}\@lec{#1}\end@sblockquote}
\newenvironment{sblockquote}[1][]
  {\def\@opt{#1}\ifx\@opt\@empty\begin@sblockquote\else\begin@@sblockquote\@opt\fi}
  {\ifx\@opt\@empty\end@sblockquote\else\end@@sblockquote\@opt\fi}
%</package>
%<*ltxml>
DefEnvironment('{sblockquote} []',
  "<ltx:quote>?#1(<ltx:note role='source'>#1</ltx:note>)()#body</ltx:quote>");
%</ltxml>
%    \end{macrocode}
% \end{environment}
%
% \begin{environment}{sboxquote}
%    \begin{macrocode}
%<*package>
\newenvironment{sboxquote}[1][]
{\begin{mdframed}[leftmargin=1cm,rightmargin=1cm]}
{\end{mdframed}}
%</package>
%<*ltxml>
DefEnvironment('{sboxquote} []',
  "<ltx:quote class='boxed'>?#1(<ltx:note role='source'>#1</ltx:note>)()#body</ltx:quote>");
%</ltxml>
%    \end{macrocode}
% \end{environment}
% 
% The line end comment macro makes sure that it will not be forced on the next line unless
% necessary.
% \begin{macro}{\lec}
%   The actual appearance of the line end comment is determined by the |\@@lec| macro,
%   which can be customized in the document class. The basic one here is provided so that
%   it is not missing.
%    \begin{macrocode}
%<*package>
\providecommand{\@@lec}[1]{(#1)}
\def\@lec#1{\strut\hfil\strut\null\nobreak\hfill\hbox{\@@lec{#1}}}
\def\lec#1{\@lec{#1}\par}
%</package>
%<*ltxml>
DefConstructor('\lec{}',
   "\n<omdoc:note type='line-end-comment'>#1</omdoc:note>");
%</ltxml>
%    \end{macrocode}
% \end{macro}
%
% \begin{macro}{\my*graphics}
%   We set up a special treatment for including graphics to respect the intended {\omdoc}
%   document structure. The main work is done in the transformation stylesheet though.
%    \begin{macrocode}
%<ltxml>RawTeX('
%<*ltxml|package>
\newcommand\mygraphics[2][]{\includegraphics[#1]{#2}}
\newcommand\mycgraphics[2][]{\begin{center}\mygraphics[#1]{#2}\end{center}}
\newcommand\mybgraphics[2][]{\fbox{\mygraphics[#1]{#2}}}
\newcommand\mycbgraphics[2][]{\begin{center}\fbox{\mygraphics[#1]{#2}}\end{center}}
%</ltxml|package>
%<ltxml>');
%    \end{macrocode}
% \end{macro}
% \end{omgroup}
% 
% \begin{omgroup}[id=sec:impl:index]{Index Markup}
% \begin{macro}{\omdoc@index}
%   this is the main internal indexing command. It makes sure that the modules necessary
%   for interpreting the math in the index entries are loaded. If the |loadmodules| key is
%   given, we import all the imported modules. If the |at| key is given, then we use that
%   for sorting in the index. 
%    \begin{macrocode}
%<*package>
\addmetakey{omdoc@index}{at}
\addmetakey[false]{omdoc@index}{loadmodules}[true]
\newcommand\@@abei[2]{\AtBeginEnvironment{theindex}{\@requiremodules{#1}{#2}}}
\newcommand\omdoc@index[2][]{\ifindex%
\metasetkeys{omdoc@index}{#1}%
\@bsphack\begingroup\@sanitize%
\ifx\omdoc@index@loadmodules\@true%
\@for\@I:=\imported@modules\do{%
\expandafter\@@abei\csname\@I @cd@file@base\endcsname{tex}}%
\protected@write\@indexfile{}{\string\indexentry%
{\ifx\omdoc@index@at\@empty\else\omdoc@index@at @\fi%
{\string\importmodules{\imported@modules}%
#2}}{\thepage}}%
\else%
\protected@write\@indexfile{}{\string\indexentry%
{\ifx\omdoc@index@at\@empty\else\omdoc@index@at @\fi#2}{\thepage}}%
\fi% loadmodules
\endgroup\@esphack\fi}%ifindex
%    \end{macrocode}
% \end{macro}
% 
% Now, we make two interface macros that make use of this: 
%
% \begin{macro}{\indexalt}
%    \begin{macrocode}
\newcommand\indexalt[3][]{{#2}\omdoc@index[#1]{#3}}               % word in text and index
%</package>
%<*ltxml>
DefConstructor('\indexalt[]{}{}',
	       "<omdoc:idx>"
	      .  "<omdoc:idt>#2</omdoc:idt>"
	      .  "<omdoc:ide ?#1(sort-by='#1')()>"
	      .    "<omdoc:idp>#3</omdoc:idp>"
	      .  "</omdoc:ide>"
	      ."</omdoc:idx>");
%</ltxml>
%    \end{macrocode}
% \end{macro}
%
% \begin{macro}{\indextoo}
%    \begin{macrocode}
%<*package>
\newcommand\indextoo[2][]{{#2}\omdoc@index[#1]{#2}}               % word in text and index
%</package>
%<*ltxml>
DefConstructor('\indextoo[]{}',
	       "<omdoc:idx>"
	      .  "<omdoc:idt>#2</omdoc:idt>"
	      .  "<omdoc:ide ?#1(sort-by='#1')()>"
	      .    "<omdoc:idp>#2</omdoc:idp>"
	      .  "</omdoc:ide>"
	      ."</omdoc:idx>");
%</ltxml>
%    \end{macrocode}
% \end{macro}
%
% \begin{macro}{\@twin}
% this puts two-compound words into the index in various permutations
%    \begin{macrocode}
%<*package>
\newcommand\@twin[3][]{\indexalt[#1]{#2 #3}{#2!#3}\omdoc@index[#1]{#3!#2}}
%    \end{macrocode}
% \end{macro}
%
% And again we have two interface macros building on this 
%
% \begin{macro}{\twinalt}
%    \begin{macrocode}
\newcommand\twinalt[4][]{#2\@twin[#1]{#3}{#4}}
%</package>
%<*ltxml>
DefConstructor('\twinalt[]{}{}{}',
	       "<omdoc:idx>"
	      .  "<omdoc:idt>#2</omdoc:idt>"
	      .  "<omdoc:ide ?#1(sort-by='#1')()>"
	      .    "<omdoc:idp>#2</omdoc:idp>"
	      .    "<omdoc:idp>#3</omdoc:idp>"
	      .  "</omdoc:ide>"
	      ."</omdoc:idx>");
%</ltxml>
%    \end{macrocode}
% \end{macro}
%
% \begin{macro}{\twinalt}
%    \begin{macrocode}
%<*package>
\newcommand\twintoo[3][]{{#2 #3}\@twin[#1]{#2}{#3}} % and use the word compound too
%</package>
%<*ltxml>
DefConstructor('\twintoo[]{}{}',
	       "<omdoc:idx>"
	      .  "<omdoc:idt>#2 #3</omdoc:idt>"
	      .  "<omdoc:ide ?#1(sort-by='#1')()>"
	      .    "<omdoc:idp>#2</omdoc:idp>"
	      .    "<omdoc:idp>#3</omdoc:idp>"
	      .  "</omdoc:ide>"
	      ."</omdoc:idx>");
%</ltxml>
%    \end{macrocode}
% \end{macro}
% 
% \begin{macro}{\@atwin}
%   this puts adjectivized two-compound words into the index in various
%   permutations\ednote{what to do with the optional argument here and below?}
%    \begin{macrocode}
%<*package>
\newcommand\@atwin[4][]{\indexalt[#1]{#2 #3 #4}{#2!#3!#4}\omdoc@index[#1]{#3!#2 (#4)}}
%    \end{macrocode}
% \end{macro}
% 
% and the two interface macros for this case: 
% \begin{macro}{\@atwinalt}
%    \begin{macrocode}
\newcommand\atwinalt[5][]{#2\@atwin[#1]{#3}{#4}{#4}}
%</package>
%<*ltxml>
DefConstructor('\atwinalt[]{}{}{}{}',
	       "<omdoc:idx>"
	      .  "<omdoc:idt>#2</omdoc:idt>"
	      .  "<omdoc:ide ?#1(sort-by='#1')()>"
	      .    "<omdoc:idp>#2</omdoc:idp>"
	      .    "<omdoc:idp>#3</omdoc:idp>"
	      .    "<omdoc:idp>#4</omdoc:idp>"
	      .  "</omdoc:ide>"
	      ."</omdoc:idx>");
%</ltxml>
%    \end{macrocode}
% \end{macro}
% 
% \begin{macro}{\atwintoo}
%    \begin{macrocode}
%<*package>
\newcommand\atwintoo[4][]{{#2 #3 #4}\@atwin[#1]{#2}{#3}{#4}}         % and use it too
%</package>
%<*ltxml>
DefConstructor('\atwintoo[]{}{}{}',
	       "<omdoc:idx>"
	      .  "<omdoc:idt>#2 #3</omdoc:idt>"
	      .  "<omdoc:ide ?#1(sort-by='#1')()>"
	      .    "<omdoc:idp>#2</omdoc:idp>"
	      .    "<omdoc:idp>#3</omdoc:idp>"
	      .    "<omdoc:idp>#4</omdoc:idp>"
	      .  "</omdoc:ide>"
	      ."</omdoc:idx>");
%</ltxml>
%    \end{macrocode}
% \end{macro}
% \end{omgroup}
% 
% \begin{omgroup}{{\LaTeX} Commands we interpret differently}
%
% The first think we have to take care of are the paragraphs, we want to generate {\omdoc}
% that uses the |ltx:p| element for paragraphs inside |CMP|s. For that we have modified
% the DTD only to allowed |ltx:p| elements in |omdoc:CMP| (in particular no text). Then
% we instruct the |\par| macro to close a |ltx:p| element if possible. The next
% |ltx:p| element is then opened automatically, since we make |ltx:p| and |omdoc:CMP|
% autoclose and autoopen.
%
%    \begin{macrocode}
%<*ltxml>
Tag('omdoc:CMP', autoClose=>1, autoOpen=>1);
Tag('omdoc:omtext', autoClose=>1, autoOpen=>1);
Tag('ltx:p', autoClose=>1, autoOpen=>1);
%</ltxml>
%    \end{macrocode}
% the rest of the reinterpretations is quite simple, we either disregard presentational
% markup or we re-interpret it in terms of {\omdoc}.\ednote{MK: we should probably let
% LaTeXML deal with these and allow more text in the omdoc+ltxml.xsl}
%    \begin{macrocode}
%<*package>
\def\omspace#1{\hspace*{#1}}
%</package>
%<*ltxml>
DefConstructor('\footnote[]{}',
       "<omdoc:note type='foot' ?#1(mark='#1')>#2</omdoc:note>");
DefConstructor('\footnotemark[]',"");
DefConstructor('\footnotetext[]{}',
	       "<omdoc:note class='foot' ?#1(mark='#1')>#2</omdoc:note>");
%</ltxml>
%    \end{macrocode}
% \end{omgroup}
% 
% \begin{omgroup}[id=sec:impl:ids]{Providing IDs for {\omdoc} Elements}
% 
%   To provide default identifiers, we tag all {\omdoc} elements that allow |xml:id|
%   attributes by executing the |numberIt| procedure below. Furthermore, we use the
%   |locateIt| procedure to give source links.
% 
%    \begin{macrocode}
%<*ltxml>
Tag('omdoc:omtext',afterOpen=>\&numberIt,afterClose=>\&locateIt);
Tag('omdoc:omgroup',afterOpen=>\&numberIt,afterClose=>\&locateIt);
Tag('omdoc:CMP',afterOpen=>\&numberIt,afterClose=>\&locateIt);
Tag('omdoc:idx',afterOpen=>\&numberIt,afterClose=>\&locateIt);
Tag('omdoc:ide',afterOpen=>\&numberIt,afterClose=>\&locateIt);
Tag('omdoc:idt',afterOpen=>\&numberIt,afterClose=>\&locateIt);
Tag('omdoc:note',afterOpen=>\&numberIt,afterClose=>\&locateIt);
Tag('omdoc:metadata',afterOpen=>\&numberIt,afterClose=>\&locateIt);
Tag('omdoc:meta',afterOpen=>\&numberIt,afterClose=>\&locateIt);
Tag('omdoc:resource',afterOpen=>\&numberIt,afterClose=>\&locateIt);
Tag('omdoc:recurse',afterOpen=>\&numberIt,afterClose=>\&locateIt);
Tag('omdoc:imports',afterOpen=>\&numberIt,afterClose=>\&locateIt);
Tag('omdoc:theory',afterOpen=>\&numberIt,afterClose=>\&locateIt);
Tag('omdoc:ignore',afterOpen=>\&numberIt,afterClose=>\&locateIt);
Tag('omdoc:ref',afterOpen=>\&numberIt,afterClose=>\&locateIt);
%</ltxml>
%    \end{macrocode}
% We also have to number some {\latexml} tags, so that we do not get into trouble with the
% \omdoc tags inside them.
%    \begin{macrocode}
%<*ltxml>
Tag('ltx:p',afterOpen=>\&numberIt,afterClose=>\&locateIt);
Tag('ltx:tabular',afterOpen=>\&numberIt,afterClose=>\&locateIt);
Tag('ltx:thead',afterOpen=>\&numberIt,afterClose=>\&locateIt);
Tag('ltx:td',afterOpen=>\&numberIt,afterClose=>\&locateIt);
Tag('ltx:tr',afterOpen=>\&numberIt,afterClose=>\&locateIt);
Tag('ltx:caption',afterOpen=>\&numberIt,afterClose=>\&locateIt);
Tag('ltx:Math',afterOpen=>\&numberIt,afterClose=>\&locateIt);
%</ltxml>
%    \end{macrocode}
% The |numberIt| procedure gets the prefix from first parent with an |xml:id| attribute and then
% extends it with a label that reflects the number of preceding siblings, provided that
% there is not already an identifier. Additionally, it estimates an XPointer position in the original document
% of the command sequence which produced the tag.
% The |locateIt| subroutine is a sibling of |numberIt| as it is required as an |afterClose| handle for tags
% produced by {\LaTeX} environments, as opposed to commands. |locateIt| estimates an XPointer end position of
% the LaTeX environment, allowing to meaningfully locate the entire environment at the source.
%    \begin{macrocode}
%<*ltxml>
sub numberIt {
  my($document,$node,$whatsit)=@_;
  my(@parents)=$document->findnodes('ancestor::*[@xml:id]',$node);
  my $prefix= (@parents ? $parents[$#parents]->getAttribute('xml:id')."." : '');
  my(@siblings)=$document->findnodes('preceding-sibling::*[@xml:id]',$node);
  my $n = scalar(@siblings)+1;
  my $id = ($node -> getAttribute('xml:id'));
  my $localname = $node->localname;
  $node->setAttribute('xml:id'=>$prefix."$localname$n") unless $id;
  my $about = $node -> getAttribute('about');
  $node->setAttribute('about'=>'#'.$node->getAttribute('xml:id')) unless $about;
  #Also, provide locators:
  my $locator = $whatsit->getProperty('locator');
  #Need to inherit locators if missing:
  $locator = (@parents ? $parents[$#parents]->getAttribute('stex:srcref') : '') unless $locator;
  if ($locator) {
    # There is a BUG with namespace declarations (or am I using the API wrongly??) which
    # does not recognize the stex namespace. Hence, I need to redeclare it...
    my $parent=$document->getNode;
    if(! defined $parent->lookupNamespacePrefix("http://kwarc.info/ns/sTeX"))
      { # namespace not already declared?
        $document->getDocument->documentElement->setNamespace("http://kwarc.info/ns/sTeX","stex",0); 
      }
    $node->setAttribute('stex:srcref'=>$locator);
  }return;}

sub locateIt {
  my($document,$node,$whatsit)=@_;
  #Estimate trailer and locator:
  my $locator = $node->getAttribute('stex:srcref');
  return unless $locator; # Nothing to do here...
  my $trailer = $whatsit->getProperty('trailer');
  $trailer = $trailer->getLocator if $trailer;
  $trailer = $locator unless $trailer; # bootstrap
  # TODO: Both should be local, or both remote, any mixture or undefinedness will produce garbage
  my $file_path = LookupValue('SOURCEFILE');
  my $baselocal = LookupValue('BASELOCAL');
  # Hmm, we only care about relative paths, so let's just do a URL->pathname map
  $file_path=~s/^\w+\:\/// if $file_path;
  $baselocal=~s/^\w+\:\/// if $baselocal;
  if ($file_path && $baselocal && ($locator =~ s/^([^\#]+)\#/\#/)) {
    my $relative_path = pathname_relative($file_path,$baselocal);
    $locator = $relative_path.$locator;
  }
  if ($locator =~ /^(.+from=\d+;\d+)/) {
    my $from = $1;
    if ($trailer =~ /(,to=\d+;\d+.+)$/) {
      my $to = $1;
      $locator = $from.$to;
    } else { Error("stex","locator",undef, "Trailer is garbled, expect nonsense in stex:srcref ..."); }
  } else { Error("stex","locator",undef, "Locator \"$locator\" is garbled, expect nonsense in stex:srcref ..."); }
  my $parent = $document->getNode;
  if(! defined $parent->lookupNamespacePrefix("http://kwarc.info/ns/sTeX"))
    { # namespace not already declared?
      $document->getDocument->documentElement->setNamespace("http://kwarc.info/ns/sTeX","stex",0);
    }
  $node->setAttribute('stex:srcref' => $locator);
  return;
}
%</ltxml>#$
%    \end{macrocode}
% \end{omgroup}
% 
% \begin{omgroup}{Support for \textsf{MathHub}}\label{sec:user:mathhub}
% 
% \begin{macro}{\mh*graphics}
%   Use the current value of |\mh@currentrepos| or the value of the |mhrepos| key if it is
%   given in |\my*graphics|.
% 
%    \begin{macrocode}
%<*package>
\addmetakey{Gin}{mhrepos}
\newcommand\mhgraphics[2][]{\metasetkeys{Gin}{#1}%
\edef\mh@@repos{\mh@currentrepos}%
\ifx\Gin@mhrepos\@empty\mygraphics[#1]{\MathHub{\mh@currentrepos/source/#2}}%
\else\mygraphics[#1]{\MathHub{\Gin@mhrepos/source/#2}}\fi
\def\Gin@mhrepos{}\mhcurrentrepos\mh@@repos}
\newcommand\mhcgraphics[2][]{\begin{center}\mhgraphics[#1]{#2}\end{center}}
\newcommand\mhbgraphics[2][]{\fbox{\mhgraphics[#1]{#2}}}
\newcommand\mhcbgraphics[2][]{\begin{center}\fbox{\mhgraphics[#1]{#2}}\end{center}}
%</package>
%<*ltxml>
sub mhgraphics {
  my ($gullet,$keyval,$arg2) = @_;
  my $repo_path;
  if ($keyval) {
    $repo_path = ToString(GetKeyVal($keyval,'mhrepos')); }
  if (! $repo_path) {
    $repo_path = ToString(Digest(T_CS('\mh@currentrepos'))); }
  else {
    $keyval->setValue('mhrepos',undef); }
  my $mathhub_base = ToString(Digest('\MathHub{}'));
  my $finalpath = $mathhub_base.$repo_path.'/source/'.ToString($arg2);
  return Invocation(T_CS('\@includegraphicx'), $keyval, T_OTHER($finalpath)); }#$
DefKeyVal('Gin','mhrepos','Semiverbatim');
DefMacro('\mhgraphics OptionalKeyVals:Gin {}', \&mhgraphics);
DefMacro('\mhcgraphics []{}','\begin{center}\mhgraphics[#1]{#2}\end{center}');
DefMacro('\mhbgraphics []{}','\fbox{\mhgraphics[#1]{#2}}');
%</ltxml>
%    \end{macrocode}
% \end{macro}
% \end{omgroup}
% 
% \begin{omgroup}{Finale}
%
% We need to terminate the file with a success mark for perl.
%    \begin{macrocode}
%<ltxml>1;
%    \end{macrocode}
% \end{omgroup}
%\end{omgroup}
% \Finale
\endinput
% \iffalse
%%% Local Variables: 
%%% mode: doctex
%%% TeX-master: t
%%% End: 
% \fi
% LocalWords:  GPL structuresharing STR omdoc dtx stex CPERL LoadClass url dc filedate sl 
% LocalWords:  RequirePackage RegisterNamespace namespace xsl DocType ltxml dtd annote mh
% LocalWords:  ltx DefEnvironment beforeDigest AssignValue inPreamble getGullet indexalt
% LocalWords:  afterDigest keyval omgroup DefKeyVal Semiverbatim KeyVal regexp indexalt
% LocalWords:  OptionalKeyVals DefParameterType IfBeginFollows skipSpaces CMP rangle rel
% LocalWords:  ifNext DefMacro needwrapper unlist DefConstructor omtext bgroup rangle baz
% LocalWords:  useCMPItemizations RefStepItemCounter egroup beginItemize li di makeindex
% LocalWords:  beforeDigestEnd dt autoclose ul ol dl env showignores srcref def st@flow
% LocalWords:  afterOpen LastSeenCMP autoClose DefCMPEnvironment proto ToString KeyVals
% LocalWords:  addAttribute nlex nlcex omdocColorMacro args tok MergeFont qw rm XPointer
% LocalWords:  TokenizeInternal toString isMath foreach maybeCloseElement id'd endinput
% LocalWords:  autoOpen minipage footnotesize scriptsize numberIt whatsit href HorIacJuc
% LocalWords:  getAttribute setAttribute OMDoc RelaxNGSchema noindex xml lec sc cscpnrr11
% LocalWords:  Subsubsection useDefaultItemizations refundefinedtrue sblockquote foobar
% LocalWords:  DefCMPConstructor sinlinequote idx idt ide idp emph  extrefs sref mhrepos
% LocalWords:  flushleft flushright DeclareOption PassOptions undef cls iffalse mathhub
% LocalWords:  ProcessOptions subparagraph ignoresfalse ignorestrue raisebox tr finalpath
% LocalWords:  texorpdfstring latexml texttt fileversion maketitle newpage iff mh-variant
% LocalWords:  tableofcontents newpage ednote obeylines usepackage indextoo Cwd coreft
% LocalWords:  indextoo twintoo twintoo exfig vspace twinalt ldots ttin emin importmodule
% LocalWords:  renmanig myindex atwin atwin packge-local blockquote inlinequote coreft
% LocalWords:  atwintoo atwinalt atwinalt printbibliography impl cwd newif ifx mhgraphics
% LocalWords:  ifindex indextrue indexfalse srefaddidkey smallskip showmeta importmodules
% LocalWords:  providecommand stDMemph textbf newenvironment hfil showmeta NeedsTeXFormat
% LocalWords:  noindent ignorespaces newcommand nobreak hfill hbox mygraphics mhcgraphics
% LocalWords:  includegraphics mycgraphics mybgraphics fbox adjectivized hspace corefs
% LocalWords:  printindex jobname.ind jobname.ind omspace footnotemark thead mhcgraphics
% LocalWords:  footnotetext findnodes doctex textsf langle textsf langle funval corefs
% LocalWords:  metakeys funsymbs addmetakey metasetkeys startemph textsl textit mdframed
% LocalWords:  compactenum ifundefined localname localname xspace ctancite mhcurrentrepos
% LocalWords:  KohAmb smmssl requiremodules prhase bsphack begingroup thepage mh-variants
% LocalWords:  indexentry endgroup esphack SOURCEFILE baselocal BASELOCAL importmhmodule
%  LocalWords:  mh@currentrepos mhbgraphics co-rerefence usemodule coref textsubscript
%  LocalWords:  ensuremath textsuperscript sboxquote leftmargin rightmargin mycbgraphics
%  LocalWords:  mhcbgraphics

% \endinput
% Local Variables:
% mode: doctex
% TeX-master: t
% End:
