\documentclass{article}
\usepackage{smglom}
\begin{document}
\setpstlabelstyle{prefix=Pr,delimiter=-,postfix=\checkmark}
\begin{sproof}[id=simple-proof,for=sum-over-odds]
	{We prove that $\sum_{i=1}^n{2i-1}=n^{2}$ by induction over $n$}
	\begin{spfcases}{For the induction we have to consider the following cases:}
		\begin{spfcase}{$n=1$}
		    \begin{spfstep}[display=flow] then we compute $1=1^2$\end{spfstep}
		\end{spfcase}
		\begin{spfcase}{$n=2$}
			\begin{sproofcomment}[display=flow]
		        This case is not really necessary, but we do it for the fun of it (and to get more intuition).
			\end{sproofcomment}
			\begin{spfstep}[display=flow]
				We compute $1+3=2^{2}=4$
			\end{spfstep}
		\end{spfcase}
		\begin{spfcase}{$n>1$}
			\begin{spfstep}[type=hypothesis,id=ind-hyp]
				Now, we assume that the assertion is true for a certain $k\geq 1$, i.e. $\sum_{i=1}^k{(2i-1)}=k^{2}$.
			\end{spfstep}
			\begin{sproofcomment}
				We have to show that we can derive the assertion for $n=k+1$ from this assumption, i.e.  $\sum_{i=1}^{k+1}{(2i-1)}=(k+1)^{2}$.
			\end{sproofcomment}
			\begin{spfstep}[id=splitit]
				We obtain $\sum_{i=1}^{k+1}{(2i-1)}=\sum_{i=1}^k{(2i-1)}+2(k+1)-1$
				\begin{justification}[method=arith:split-sum]
					by splitting the sum
			    \end{justification}
			\end{spfstep}
			\begin{spfstep}[id=byindhyp]
				Thus we have $\sum_{i=1}^{k+1}{(2i-1)}=k^2+2k+1$
				\begin{justification}[method=fertilize]
					by \premise[ind-hyp]{inductive hypothesis}.
				\end{justification}
			\end{spfstep}
			\begin{spfstep}[type=conclusion]
				We can \begin{justification}[method=simplify-eq]
				simplify the {\justarg[rhs]{right-hand side}}
				\end{justification} to $(k+1)^2$, which proves the assertion.
			\end{spfstep}
		\end{spfcase}
		\begin{spfstep}[type=conclusion]
			We have considered all the cases, so we have proven the assertion.
		\end{spfstep}
	\end{spfcases}
\end{sproof}
\end{document}