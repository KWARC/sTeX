% \iffalse meta-comment 
% An Infrastructure for Structural Markup for Proofs
% Copyright (C) 2004-2011 Michael Kohlhase, all rights reserved
%               this file is released under the
%               LaTeX Project Public License (LPPL)
%
% The original of this file is in the public repository at 
% http://github.com/KWARC/sTeX/
% \fi
% 
% \iffalse
%<package>\NeedsTeXFormat{LaTeX2e}[1999/12/01]
%<package>\ProvidesPackage{sproof}[2013/09/29 v1.1 Semantic Markup for Proofs]
%
%<*driver>
\documentclass{ltxdoc}
\usepackage{url,array,stex,float,moreverb}
\usepackage[show]{ed}
\usepackage[hyperref=auto,style=alphabetic]{biblatex}
\addbibresource{kwarcpubs.bib}
\addbibresource{extpubs.bib}
\addbibresource{kwarccrossrefs.bib}
\addbibresource{extcrossrefs.bib}
\usepackage{../ctangit}
\usepackage{hyperref}
\makeindex
\floatstyle{boxed}
\newfloat{exfig}{thp}{lop}
\floatname{exfig}{Example}
\def\tracissue#1{\cite{sTeX:online}, \hyperlink{http://trac.kwarc.info/sTeX/ticket/#1}{issue #1}}
\begin{document}\DocInput{sproof.dtx}\end{document}
%</driver>
% \fi
% 
% \iffalse\CheckSum{356}\fi
%
% \changes{v0.9}{2005/06/14}{First Version with Documentation}
% \changes{v0.9a}{2005/07/01}{Completed Documentation}
% \changes{v0.9b}{2005/08/06}{Complete functionality and Updated Documentation}
% \changes{v0.9c}{2006/01/13}{more packaging}
% \changes{v0.9d}{2006/10/31}{made sproof.dtx independent of statements.dtx}
% \changes{v0.9d}{2006/10/31}{revamped the proof end mark management}
% \changes{v0.9e}{2008/10/11}{taking type seriously}
% \changes{v0.9f}{2009/12/14}{changing to omd metadata framework}
% \changes{v0.9f}{2008/12/14}{first steps to sref}
% \changes{v1.0}{2011/01/23}{making proof step labels stylable}
% \changes{v1.1}{2013/09/29}{subproof now more like proof (takes comment arg)}
% 
% \GetFileInfo{sproof.sty}
% 
% \MakeShortVerb{\|}
%\def\scsys#1{{{\sc #1}}\index{#1@{\sc #1}}}
% \def\xml{\scsys{Xml}}
% \def\mathml{\scsys{MathML}}
% \def\omdoc{\scsys{OMDoc}}
% \def\openmath{\scsys{OpenMath}}
% \def\latexml{\scsys{LaTeXML}}
% \def\perl{\scsys{Perl}}
% \def\cmathml{Content-{\sc MathML}\index{Content {\sc MathML}}\index{MathML@{\sc MathML}!content}}
% \def\activemath{\scsys{ActiveMath}}
% \def\twin#1#2{\index{#1!#2}\index{#2!#1}}
% \def\twintoo#1#2{{#1 #2}\twin{#1}{#2}}
% \def\atwin#1#2#3{\index{#1!#2!#3}\index{#3!#2 (#1)}}
% \def\atwintoo#1#2#3{{#1 #2 #3}\atwin{#1}{#2}{#3}}
% \title{{\texttt{sproof.sty}}: Structural Markup for Proofs\thanks{Version {\fileversion} (last revised
%        {\filedate})}}
%    \author{Michael Kohlhase\\
%            Jacobs University, Bremen\\
%            \url{http://kwarc.info/kohlhase}}
% \maketitle
%
% \begin{abstract}
%   The |sproof| package is part of the {\stex} collection, a version of {\TeX/\LaTeX} that
%   allows to markup {\TeX/\LaTeX} documents semantically without leaving the document
%   format, essentially turning {\TeX/\LaTeX} into a document format for mathematical
%   knowledge management (MKM).
%
%   This package supplies macros and environment that allow to annotate the structure of
%   mathematical proofs in {\stex} files. This structure can be used by MKM systems for
%   added-value services, either directly from the {\sTeX} sources, or after translation.
% \end{abstract}
%
% \newpage\tableofcontents\newpage
%
%\section{Introduction}\label{sec:sproof}
%
% The |sproof| ({\twintoo{semantic}{proofs}}) package supplies macros and environment that
% allow to annotate the structure of mathematical proofs in {\stex} files. This structure
% can be used by MKM systems for added-value services, either directly from the {\sTeX}
% sources, or after translation. Even though it is part of the {\stex} collection, it can
% be used independently, like it's sister package |statements|.
%
% {\stex} is a version of {\TeX/\LaTeX} that allows to markup {\TeX/\LaTeX} documents
% semantically without leaving the document format, essentially turning {\TeX/\LaTeX} into
% a document format for mathematical knowledge management (MKM).
%
% \begin{exfig}\scriptsize
% \begin{verbatim}
% \begin{sproof}[id=simple-proof,for=sum-over-odds]
%    {We prove that $\sum_{i=1}^n{2i-1}=n^{2}$ by induction over $n$}
%   \begin{spfcases}{For the induction we have to consider the following cases:}
%    \begin{spfcase}{$n=1$}
%     \begin{spfstep}[display=flow] then we compute $1=1^2$\end{spfstep}
%    \end{spfcase}
%    \begin{spfcase}{$n=2$}
%       \begin{sproofcomment}[display=flow]
%         This case is not really necessary, but we do it for the
%         fun of it (and to get more intuition).
%       \end{sproofcomment}
%       \begin{spfstep}[display=flow] We compute $1+3=2^{2}=4$.\end{spfstep}
%    \end{spfcase}
%    \begin{spfcase}{$n>1$}
%       \begin{spfstep}[type=assumption,id=ind-hyp]
%         Now, we assume that the assertion is true for a certain $k\geq 1$,
%         i.e. $\sum_{i=1}^k{(2i-1)}=k^{2}$.
%       \end{spfstep}
%       \begin{sproofcomment}
%         We have to show that we can derive the assertion for $n=k+1$ from
%         this assumption, i.e. $\sum_{i=1}^{k+1}{(2i-1)}=(k+1)^{2}$.
%       \end{sproofcomment}
%       \begin{spfstep}
%         We obtain $\sum_{i=1}^{k+1}{2i-1}=\sum_{i=1}^k{2i-1}+2(k+1)-1$
%         \begin{justification}[method=arith:split-sum]
%           by splitting the sum.
%         \end{justification}
%       \end{spfstep}
%       \begin{spfstep}
%         Thus we have $\sum_{i=1}^{k+1}{(2i-1)}=k^2+2k+1$
%         \begin{justification}[method=fertilize]
%           by inductive hypothesis.
%         \end{justification}
%       \end{spfstep}
%       \begin{spfstep}[type=conclusion]
%         We can \begin{justification}[method=simplify]simplify\end{justification} 
%         the right-hand side to ${k+1}^2$, which proves the assertion.
%       \end{spfstep}
%    \end{spfcase}
%     \begin{spfstep}[type=conclusion]
%       We have considered all the cases, so we have proven the assertion.
%     \end{spfstep}
%   \end{spfcases}
% \end{sproof}
% \end{verbatim}
% \vspace*{-.5cm}
% \caption{A very explicit proof, marked up semantically}\label{fig:proof:src}
% \end{exfig}
%
% We will go over the general intuition by way of our running example (see
% Figure~\ref{fig:proof:src} for the source and Figure~\ref{fig:proof:result} for the
% formatted result).\ednote{talk a bit more about proofs and their structure,... maybe
%   copy from OMDoc spec. }
%
% \section{The User Interface}
% 
% \subsection{Package Options}\label{sec:user:options}
% 
% The |sproof| package takes a single option: \DescribeMacro{showmeta}|showmeta|. If
% this is set, then the metadata keys are shown (see~\cite{Kohlhase:metakeys:ctan} for details
% and customization options).
%
% \subsection{Proofs and Proof steps}
%
% \DescribeEnv{sproof} The |proof| environment is the main container for proofs. It takes
% an optional |KeyVal| argument that allows to specify the |id| (identifier) and |for|
% (for which assertion is this a proof) keys. The regular argument of the |proof|
% environment contains an introductory comment, that may be used to announce the proof
% style. The |proof| environment contains a sequence of |\step|, |proofcomment|, and
% |pfcases| environments that are used to markup the proof steps. The |proof| environment
% has a variant |Proof|, which does not use the proof end marker. This is convenient, if a
% proof ends in a case distinction, which brings it's own proof end marker with it.
% \DescribeEnv{sProof} The |Proof| environment is a variant of |proof| that does not mark
% the end of a proof with a little box; presumably, since one of the subproofs already has
% one and then a box supplied by the outer proof would generate an otherwise empty line.
% \DescribeMacro{\spfidea} The |\spfidea| macro allows to give a one-paragraph
% description of the proof idea.
% 
% For one-line proof sketches, we use the \DescribeMacro{spfsketch}|\spfsketch| macro,
% which takes the |KeyVal| argument as |sproof| and another one: a natural language text
% that sketches the proof.
%
% \DescribeEnv{spfstep} Regular proof steps are marked up with the |step| environment, which
% takes an optional |KeyVal| argument for annotations. A proof step usually contains a
% local assertion (the text of the step) together with some kind of evidence that this can
% be derived from already established assertions.
%
% Note that both |\premise| and |\justarg| can be used with an empty second argument to
% mark up premises and arguments that are not explicitly mentioned in the text.
%
% \begin{exfig}
% \begin{sproof}[id=simple-proof,for=sum-over-odds]
%   {We prove that $\sum_{i=1}^n{2i-1}=n^{2}$ by induction over $n$}
%   \begin{spfcases}{For the induction we have to consider the following cases:}
%     \begin{spfcase}{$n=1$}
%       \begin{spfstep}[display=flow] then we compute $1=1^2$\end{spfstep}
%     \end{spfcase}
%     \begin{spfcase}{$n=2$}
%       \begin{sproofcomment}[display=flow]
%          This case is not really necessary, but we do it for the fun
%          of it (and to get more intuition).
%       \end{sproofcomment}
%       \begin{spfstep}[display=flow]
%          We compute $1+3=2^{2}=4$
%       \end{spfstep}
%     \end{spfcase}
%     \begin{spfcase}{$n>1$}
%       \begin{spfstep}[type=hypothesis,id=ind-hyp]
%         Now, we assume that the assertion is true for a certain $k\geq 1$, i.e.
%         $\sum_{i=1}^k{(2i-1)}=k^{2}$.
%       \end{spfstep}
%       \begin{sproofcomment}
%         We have to show that we can derive the assertion for $n=k+1$ from this
%         assumption, i.e.  $\sum_{i=1}^{k+1}{(2i-1)}=(k+1)^{2}$.
%       \end{sproofcomment}
%       \begin{spfstep}[id=splitit]
%         We obtain $\sum_{i=1}^{k+1}{(2i-1)}=\sum_{i=1}^k{(2i-1)}+2(k+1)-1$
%        \begin{justification}[method=arith:split-sum]
%          by splitting the sum
%        \end{justification}
%      \end{spfstep}
%      \begin{spfstep}[id=byindhyp]
%        Thus we have $\sum_{i=1}^{k+1}{(2i-1)}=k^2+2k+1$
%        \begin{justification}[method=fertilize]
%          by \premise[ind-hyp]{inductive hypothesis}.
%        \end{justification}
%      \end{spfstep}
%      \begin{spfstep}[type=conclusion]
%        We can \begin{justification}[method=simplify-eq]
%          simplify the {\justarg[rhs]{right-hand side}}
%        \end{justification} to $(k+1)^2$, which proves the assertion.
%      \end{spfstep}
%    \end{spfcase}
%    \begin{spfstep}[type=conclusion]
%      We have considered all the cases, so we have proven the assertion.
%    \end{spfstep}
%   \end{spfcases}
% \end{sproof}
% \caption{The formatted result of the proof in Figure~\ref{fig:proof:src}}\label{fig:proof:result}
% \end{exfig}
%
% \subsection{Justifications}
% 
% \DescribeEnv{justification} This evidence is marked up with the |justification|
% environment in the |sproof| package. This environment totally invisible to the formatted
% result; it wraps the text in the proof step that corresponds to the evidence. The
% environment takes an optional |KeyVal| argument, which can have the |method| key, whose
% value is the name of a proof method (this will only need to mean something to the
% application that consumes the semantic annotations). Furthermore, the justification can
% contain ``premises'' (specifications to assertions that were used justify the step) and
% ``arguments'' (other information taken into account by the proof method).
%
% \DescribeMacro{\premise} The |\premise| macro allows to mark up part of the text as
% reference to an assertion that is used in the argumentation. In the example in
% Figure~\ref{fig:proof:src} we have used the |\premise| macro to identify the inductive
% hypothesis.
%
% \DescribeMacro{\justarg} The |\justarg| macro is very similar to |\premise| with the
% difference that it is used to mark up arguments to the proof method. Therefore the
% content of the first argument is interpreted as a mathematical object rather than as an
% identifier as in the case of |\premise|. In our example, we specified that the
% simplification should take place on the right hand side of the equation. Other examples
% include proof methods that instantiate. Here we would indicate the substituted object in
% a |\justarg| macro.
%
% \subsection{Proof Structure}
% 
% \DescribeEnv{subproof} The |pfcases| environment is used to mark up a subproof. This
% environment takes an optional |KeyVal| argument for semantic annotations and a second
% argument that allows to specify an introductory comment (just like in the |proof|
% environment). The \DescribeMacro{method}|method| key can be used to give the name of the
% proof method executed to make this subproof.
% 
% \DescribeEnv{spfcases} The |pfcases| environment is used to mark up a proof by
% cases. Technically it is a variant of the |subproof| where the |method| is
% |by-cases|. Its contents are |spfcase| environments that mark up the cases one by one.
%
% \DescribeEnv{spfcase} The content of a |pfcases| environment are a sequence of case
% proofs marked up in the |pfcase| environment, which takes an optional |KeyVal| argument
% for semantic annotations. The second argument is used to specify the the description of
% the case under consideration. The content of a |pfcase| environment is the same as that
% of a |proof|, i.e. |step|s, |proofcomment|s, and |pfcases|
% environments. \DescribeMacro{\spfcasesketch}|\spfcasesketch| is a variant of the |spfcase|
% environment that takes the same arguments, but instead of the |spfstep|s in the body
% uses a third argument for a proof sketch.
%
% \DescribeEnv{sproofcomment} The |proofcomment| environment is much like a |step|, only
% that it does not have an object-level assertion of its own. Rather than asserting some
% fact that is relevant for the proof, it is used to explain where the proof is going,
% what we are attempting to to, or what we have achieved so far. As such, it cannot be the
% target of a |\premise|.
% 
% \subsection{Proof End Markers}
% 
% Traditionally, the end of a mathematical proof is marked with a little box at the end of
% the last line of the proof (if there is space and on the end of the next line if there
% isn't), like so:\sproofend
% 
% The |sproof| package provides the \DescribeMacro{\sproofend}|\sproofend| macro for
% this. If a different symbol for the proof end is to be used (e.g. {\sl{q.e.d}}), then
% this can be obtained by specifying it using the
% \DescribeMacro{\sProofEndSymbol}|\sProofEndSymbol| configuration macro (e.g. by specifying
% |\sProofEndSymbol{q.e.d}|).
% 
% Some of the proof structuring macros above will insert proof end symbols for sub-proofs,
% in most cases, this is desirable to make the proof structure explicit, but sometimes
% this wastes space (especially, if a proof ends in a case analysis which will supply its
% own proof end marker). To suppress it locally, just set |proofend={}| in them or use use
% |\sProofEndSymbol{}|.
% 
% \subsection{Configuration of the Presentation}\label{sec:user:conf}
%
% Finally, we provide configuration hooks in Figure~\ref{fig:hooks} for the keywords in
% proofs. These are mainly intended for package authors building on |statements|, e.g. for
% multi-language support.\ednote{we might want to develop an extension
% \texttt{sproof-babel} in the future.}.
%\begin{figure}[ht]\centering
% \begin{tabular}{|lll|}\hline
% Environment & configuration macro & value\\\hline\hline
% \texttt{sproof} & \texttt{\textbackslash spf@proof@kw} & \makeatletter\spf@proof@kw\\\hline
% \texttt{sketchproof} & \texttt{\textbackslash spf@sketchproof@kw} & \makeatletter\spf@proofsketch@kw\\\hline
% \end{tabular}
% \caption{Configuration Hooks for Semantic Proof Markup}\label{fig:hooks}
% \end{figure}
% The proof step labels can be customized via the
% \DescribeMacro{\pstlabelstyle}|\pstlabelstyle| macro: |\pstlabelstyle{|\meta{style}|}|
% sets the style; see Figure~\ref{fig:pstlabel} for an overview of styles. Package writers
% can add additional styles by adding a macro |\pst@make@label@|\meta{style} that takes
% two arguments: a comma-separated list of ordinals that make up the prefix and the current
% ordinal. Note that comma-separated lists can be conveniently iterated over by the
% {\LaTeX} |\@for|\ldots|:=|\ldots|\do{|\ldots|}| macro; see Figure~\ref{fig:pstlabel} for
% examples.
%
%\begin{figure}[ht]\centering\makeatletter\small
% \begin{tabular}{|lll|}\hline
% style & example& configuration macro\\\hline\hline
% \texttt{long} &  \pst@make@label@long{0,8,1}{5} & \verb|\def\pst@make@label@long#1#2{\@for\@I:=#1\do{\@I.}#2}|\\\hline
% \texttt{angles} & \pst@make@label@angles{0,8,1}{5} & \verb|\def\pst@make@label@angles#1#2|\\
% &&\quad\verb|{\ensuremath{\@for\@I:=#1\do{\rangle}}#2}|\\\hline
% \texttt{short} & \pst@make@label@short{0,8,1}{5} & \verb|\def\pst@make@label@short#1#2{#2}|\\\hline
% \texttt{empty} & \pst@make@label@empty{0,8,1}{5} & \verb|\def\pst@make@label@empty#1#2{}|\\\hline
% \end{tabular} 
% \caption{Configuration Proof Step Label Styles}\label{fig:pstlabel}
% \end{figure}
%
% \section{Limitations}\label{sec:limitations}
% 
% In this section we document known limitations. If you want to help alleviate them,
% please feel free to contact the package author. Some of them are currently discussed in
% the TRAC. 
% \begin{compactenum}
% \item The numbering scheme of proofs cannot be changed. It is more geared for teaching
%   proof structures (the author's main use case) and not for writing papers.\lec{reported
%   by Tobias Pfeiffer; see \tracissue{1658}} (fixed)
% \item currently proof steps are formatted by the {\LaTeX} |description| environment. We
%   would like to configure this, e.g. to use the |inparaenum| environment for more
%   condensed proofs. I am just not sure what the best user interface would be I can
%   imagine redefining an internal environment |spf@proofstep@list| or adding a key
%   |prooflistenv| to the |proof| environment that allows to specify the environment
%   directly. Maybe we should do both. 
% \end{compactenum}
% 
% 
% \StopEventually{\newpage\PrintIndex\newpage\PrintChanges\printbibliography}
% \newpage
%
% \section{The Implementation} 
% 
% The |sproof| package generates to files: the {\LaTeX} package (all the code between
% {\textsf{$\langle$*package$\rangle$}} and {\textsf{$\langle$/package$\rangle$}}) and the
% {\latexml} bindings (between {\textsf{$\langle$*ltxml$\rangle$}} and
% {\textsf{$\langle$/ltxml$\rangle$}}). We keep the corresponding code fragments together,
% since the documentation applies to both of them and to prevent them from getting out of
% sync.
%
% We first set up the Perl Packages for {\latexml}
%
%    \begin{macrocode}
%<*ltxml>
# -*- CPERL -*-
package LaTeXML::Package::Pool;
use strict;
use LaTeXML::Package;
RequirePackage('sref');
%</ltxml>
%    \end{macrocode}
%
% \subsection{Package Options}\label{sec:impl:options}
%
% We declare some switches which will modify the behavior according to the package
% options. Generally, an option |xxx| will just set the appropriate switches to true
% (otherwise they stay false).\ednote{need an implementation for {\latexml}}
%
%    \begin{macrocode}
%<*package>
\DeclareOption{showmeta}{\PassOptionsToPackage{\CurrentOption}{metakeys}}
\ProcessOptions
%</package>
%<*ltxml>
DeclareOption('showmeta', sub
{PassOptions('metakeys','sty',ToString(Digest(T_CS('\CurrentOption')))); });
ProcessOptions();
%</ltxml>
%    \end{macrocode}
%
% Then we make sure that the |sref| package is loaded~\ctancite{Kohlhase:sref}.
%    \begin{macrocode}
%<*package>
\RequirePackage{sref}
%</package>
%    \end{macrocode}
%
% \subsection{Proofs}\label{sec:impl:proofs}
% 
% We first define some keys for the |proof| environment.
%    \begin{macrocode}
%<*package>
\srefaddidkey{spf}
\addmetakey*{spf}{display}
\addmetakey{spf}{for}
\addmetakey{spf}{from}
\addmetakey*[\sproof@box]{spf}{proofend}
\addmetakey{spf}{type}
\addmetakey*{spf}{title}
\addmetakey{spf}{continues}
\addmetakey{spf}{functions}
\addmetakey{spf}{method}
%</package>
%    \end{macrocode}
%
% \begin{macro}{\spf@flow}
% We define this macro, so that we can test whether the |display| key has the value |flow|
%    \begin{macrocode}
%<*package>
\def\spf@flow{flow}
%</package>
%    \end{macrocode}
% \end{macro}
%
% For proofs, we will have to have deeply nested structures of enumerated list-like
% environments. However, {\LaTeX} only allows |enumerate| environments up to nesting depth
% 4 and general list environments up to listing depth 6. This is not enough for us.
% Therefore we have decided to go along the route proposed by Leslie Lamport to use a
% single top-level list with dotted sequences of numbers to identify the position in the
% proof tree. Unfortunately, we could not use his |pf.sty| package directly, since it does
% not do automatic numbering, and we have to add keyword arguments all over the place, to
% accomodate semantic information.
%
% \begin{environment}{pst@with@label}
%   This environment manages\footnote{This gets the labeling right but only works 8 levels
%   deep} the path labeling of the proof steps in the description environment of the
%   outermost |proof| environment. The argument is the label prefix up to now; which we
%   cache in |\pst@label| (we need evaluate it first, since are in the right place
%   now!). Then we increment the proof depth which is stored in |\count10| (lower counters
%   are used by {\TeX} for page numbering) and initialize the next level counter
%   |\count\count10| with 1. In the end call for this environment, we just decrease the
%   proof depth counter by 1 again.
%    \begin{macrocode}
%<*package>
\newenvironment{pst@with@label}[1]%
{\edef\pst@label{#1}\advance\count10 by 1\count\count10=1}
{\advance\count10 by -1}
%    \end{macrocode}
% \end{environment}
%
% \begin{macro}{\the@pst@label}
% |\the@pst@label| evaluates to the current step label.
%    \begin{macrocode}
\def\the@pst@label{\pst@make@label\pst@label{\number\count\count10}}
%    \end{macrocode}
%\end{macro}
%
% \begin{macro}{\pstlabelstyle}
% |\pstlabelstyle| just sets the |\pst@make@label| macro according to the style. 
%    \begin{macrocode}
\def\pst@make@label@long#1#2{\@for\@I:=#1\do{\@I.}#2}
\def\pst@make@label@angles#1#2{\ensuremath{\@for\@I:=#1\do{\rangle}}#2}
\def\pst@make@label@short#1#2{#2}
\def\pst@make@label@empty#1#2{}
\def\pstlabelstyle#1{\def\pst@make@label{\@nameuse{pst@make@label@#1}}}
\pstlabelstyle{long}
%    \end{macrocode}
%\end{macro}
%
% \begin{macro}{\next@pst@label}
% |\next@pst@label| increments the step label at the current level.
%    \begin{macrocode}
\def\next@pst@label{\global\advance\count\count10 by 1}
%    \end{macrocode}
%\end{macro}
%
%\begin{macro}{\sproofend}
%    This macro places a little box at the end of the line if there is space, or at the
%    end of the next line if there isn't
%    \begin{macrocode}
\def\sproof@box{\hbox{\vrule\vbox{\hrule width 6 pt\vskip 6pt\hrule}\vrule}}
\def\spf@proofend{\sproof@box}
\def\sproofend{\ifx\spf@proofend\@empty\else\hfil\null\nobreak\hfill\spf@proofend\par\smallskip\fi}
\def\sProofEndSymbol#1{\def\sproof@box{#1}}
%</package>
%<*ltxml>
DefConstructor('\sproofend',"");
DefConstructor('\sProofEndSymbol{}','');
%</ltxml>
%    \end{macrocode}
% \end{macro}
%
% \begin{macro}{spf@*@kw}
%    \begin{macrocode}
%<*package>
\def\spf@proofsketch@kw{Proof Sketch}
\def\spf@proof@kw{Proof}
\def\spf@step@kw{Step}
%</package>
%    \end{macrocode}
% \end{macro}
%
% For the other languages, we set up triggers
%    \begin{macrocode}
%<*package>
\AfterBabelLanguage{ngerman}{\input{sproof-ngerman.ldf}}
%</package>
%    \end{macrocode}
%
% \begin{macro}{spfsketch}
%    \begin{macrocode}
%<*package>
\newcommand\spfsketch[2][]{\metasetkeys{spf}{#1}\sref@target%
\ifx\spf@display\spf@flow\else%
{\stDMemph{\ifx\spf@type\@empty\spf@proofsketch@kw\else\spf@type\fi}:}\fi{ #2}%
\sref@label@id{this \ifx\spf@type\@empty\spf@proofsketch@kw\else\spf@type\fi}\sproofend}
%</package>
%<*ltxml>
DefConstructor('\spfsketch OptionalKeyVals:pf{}',
	       "<omdoc:proof "
               .  "?&defined(&GetKeyVal(#1,'id'))(xml:id='&GetKeyVal(#1,'id')')()>\n"
	       . "?#2(<omdoc:omtext><omdoc:CMP>#2\n)()"
               . "</omdoc:proof>\n");
%</ltxml>
%    \end{macrocode}
% \end{macro}
%
% \begin{macro}{spfeq}
%   This is very similar to |\spfsketch|, but uses a computation array\ednote{This should
%   really be more like a tabular with an ensuremath in it. or invoke text on the last
%   column}\ednote{document above}
%    \begin{macrocode}
%<*package>
\newenvironment{spfeq}[2][]%
{\metasetkeys{spf}{#1}\sref@target%
\ifx\spf@display\spf@flow\else%
{\stDMemph{\ifx\spf@type\@empty\spf@proof@kw\else\spf@type\fi}:} #2%
\fi% display=flow
\begin{displaymath}\begin{array}{rcll}}
{\end{array}\end{displaymath}}
%</package>
%<*ltxml>
RawTeX('
\newenvironment{spfeq}[2][]%
{\begin{sproof}[#1]{#2}\begin{displaymath}\begin{array}{rcll}}
{\end{array}\end{displaymath}\end{sproof}}
');
%</ltxml>
%    \end{macrocode}
% \end{macro}
%
% \begin{environment}{sproof}
%    In this environment, we initialize the proof depth counter |\count10| to 10, and set
%    up the description environment that will take the proof steps. At the end of the
%    proof, we position the proof end into the last line.
%    \begin{macrocode}
%<*package>
\newenvironment{spf@proof}[2][]{\metasetkeys{spf}{#1}\sref@target%
\count10=10%
\par\noindent%
\ifx\spf@display\spf@flow\else%
\stDMemph{\ifx\spf@type\@empty\spf@proof@kw\else\spf@type\fi}:\fi{ #2}%
\sref@label@id{this \ifx\spf@type\@empty\spf@proof@kw\else\spf@type\fi}%
\def\pst@label{}\newcount\pst@count% initialize the labeling mechanism
\begin{description}\begin{pst@with@label}{P}}
{\end{pst@with@label}\end{description}}
\newenvironment{sproof}[2][]{\begin{spf@proof}[#1]{#2}}{\sproofend\end{spf@proof}}
\newenvironment{sProof}[2][]{\begin{spf@proof}[#1]{#2}}{\end{spf@proof}}
%</package>
%<*ltxml>
DefEnvironment('{sproof} OptionalKeyVals:pf{}',
	       "<omdoc:proof "
               . "?&GetKeyVal(#1,'for')(for='&hash_wrapper(&GetKeyVal(#1,'for'))')()"
               .  "?&defined(&GetKeyVal(#1,'id'))(xml:id='&GetKeyVal(#1,'id')')()>\n"
	       . "?#2(<omdoc:omtext>"
              .       "<omdoc:CMP>#2</omdoc:CMP>"
               .     "</omdoc:omtext>\n)()"
               . "#body"
             . "</omdoc:proof>\n");
DefMacro('\sProof','\sproof');
DefMacro('\endsProof','\endsproof');
%</ltxml>
%    \end{macrocode}
% \end{environment}
% 
% \begin{environment}{spfidea}
%    \begin{macrocode}
%<*package>
\newcommand\spfidea[2][]{\metasetkeys{spf}{#1}%
\stDMemph{\ifx\spf@type\@empty{Proof Idea}\else\spf@type\fi:} #2\sproofend}
%</package>
%<*ltxml>
DefConstructor('\spfidea OptionalKeyVals:pf {}',
	       "<omdoc:proof "
             .      "?&defined(&GetKeyVal(#1,'id'))(xml:id='&GetKeyVal(#1,'id')')()"
             . "?&GetKeyVal(#1,'for')(for='&hash_wrapper(&GetKeyVal(#1,'for'))')()>\n"
	     .   "<omdoc:omtext><omdoc:CMP>#2</omdoc:omtext>\n"
	     . "</omdoc:proof>\n");
%</ltxml>
%    \end{macrocode}
% \end{environment}
%
% The next two environments (proof steps) and comments, are mostly semantical, they take
% |KeyVal| arguments that specify their semantic role. In draft mode, they read these
% values and show them. If the surrounding proof had |display=flow|, then no new |\item| is
% generated, otherwise it is. In any case, the proof step number (at the current level) is
% incremented.
% \begin{environment}{spfstep}
% \ednote{MK: labeling of steps does not work yet.}
%    \begin{macrocode}
%<*package>
\newenvironment{spfstep}[1][]{\metasetkeys{spf}{#1}%
\ifx\spf@display\spf@flow\else\item[\the@pst@label]\fi%
\ifx\spf@title\@empty\else{(\stDMemph{\spf@title})}\fi%
\sref@label@id{\pst@label}\ignorespaces}%
{\next@pst@label\ignorespaces}
%</package>
%<*ltxml>
DefEnvironment('{spfstep} OptionalKeyVals:pf',
             "<omdoc:derive "
                 .   "?&defined(&GetKeyVal(#1,'id'))(xml:id='&GetKeyVal(#1,'id')')()>"
        .  "<omdoc:CMP>#body</omdoc:derive>\n",
         beforeConstruct=>sub {
          $_[0]->maybeCloseElement('omdoc:CMP');
         });#$
%</ltxml>
%    \end{macrocode}
% \end{environment}
%
% \begin{environment}{sproofcomment}
%    \begin{macrocode}
%<*package>
\newenvironment{sproofcomment}[1][]{\metasetkeys{spf}{#1}%
\ifx\spf@display\spf@flow\else\item[\the@pst@label]\fi}
{\next@pst@label}
%</package>
%<*ltxml>
DefEnvironment('{sproofcomment} OptionalKeyVals:pf',
	       "<omdoc:omtext "
               .   "?&defined(&GetKeyVal(#1,'id'))(xml:id='&GetKeyVal(#1,'id')')()>"
	       . "<omdoc:CMP>#body</omdoc:CMP>"
	     . "</omdoc:omtext>");
%</ltxml>
%    \end{macrocode}
% \end{environment}
%
% The next two environments also take a |KeyVal| argument, but also a regular one, which
% contains a start text. Both environments start a new numbered proof level.
%
% \begin{environment}{subproof}
%   In the |subproof| environment, a new (lower-level) proof environment is started.
%    \begin{macrocode}
%<*package>
\newenvironment{subproof}[2][]{\metasetkeys{spf}{#1}%
\def\@test{#2}\ifx\@test\empty\else%
\ifx\spf@display\spf@flow {#2}\else\item[\the@pst@label]{#2} \fi\fi%
\begin{pst@with@label}{\pst@label,\number\count\count10}}
{\end{pst@with@label}\next@pst@label}
%</package>
%<*ltxml>
DefEnvironment('{subproof} OptionalKeyVals:pf {}',
	       "<omdoc:derive "
              .     "?&defined(&GetKeyVal(#1,'id'))(xml:id='&GetKeyVal(#1,'id')')()>\n"
	      .  "<omdoc:CMP>#2</omdoc:CMP>\n"
	      .  "<omdoc:method ?&defined(&GetKeyVal(#1,'method'))(xref='&GetKeyVal(#1,'method')')>"
              .    "<omdoc:proof>#body</omdoc:proof>"
              .  "</omdoc:method>"
	     . "</omdoc:derive>\n");
%</ltxml>
%    \end{macrocode}
% \end{environment}
%
% \begin{environment}{spfcases}
%   In the |pfcases| environment, the start text is displayed as the first comment of the
%   proof.
%    \begin{macrocode}
%<*package>
\newenvironment{spfcases}[2][]{\def\@test{#1}%
\ifx\@test\empty\begin{subproof}[method=by-cases]{#2}%
\else\begin{subproof}[#1,method=by-cases]{#2}\fi}
{\end{subproof}}
%</package>
%<*ltxml>
DefEnvironment('{spfcases} OptionalKeyVals:pf {}',
	       "<omdoc:derive "
              .     "?&defined(&GetKeyVal(#1,'id'))(xml:id='&GetKeyVal(#1,'id')')()>\n"
	      .  "<omdoc:CMP>#2</omdoc:CMP>\n"
	      .  "<omdoc:method ?&defined(&GetKeyVal(#1,'method'))(xref='&GetKeyVal(#1,'method')')>"
              .    "#body"
              .  "</omdoc:method>"
	     . "</omdoc:derive>\n");
%</ltxml>
%    \end{macrocode}
% \end{environment}
%
% \begin{environment}{spfcase}
%    In the |pfcase| environment, the start text is displayed specification of the case
%    after the |\item|
%    \begin{macrocode}
%<*package>
\newenvironment{spfcase}[2][]{\metasetkeys{spf}{#1}%
\ifx\spf@display\spf@flow\else\item[\the@pst@label]\fi%
\def\@test{#2}\ifx\@test\@empty\else{\stDMemph{#2}: }\fi% need blank here
\begin{pst@with@label}{\pst@label,\number\count\count10}}
{\ifx\spf@display\spf@flow\else\sproofend\fi\end{pst@with@label}\next@pst@label}
%</package>
%<*ltxml>
DefEnvironment('{spfcase} OptionalKeyVals:pf{}',
	       "<omdoc:proof "
               .  "?&defined(&GetKeyVal(#1,'id'))(xml:id='&GetKeyVal(#1,'id')')()>\n"
	       .  "?#2(<omdoc:omtext>"
              .        "<omdoc:CMP>#2</omdoc:CMP>"
               .      "</omdoc:omtext>\n)()"
	       . "#body"
	     . "</omdoc:proof>\n");
%</ltxml>
%    \end{macrocode}
% \end{environment}
%
% \begin{environment}{spfcase}
%    similar to |spfcase|, takes a third argument. 
%    \begin{macrocode}
%<*package>
\newcommand\spfcasesketch[3][]{\metasetkeys{spf}{#1}%
\ifx\spf@display\spf@flow\else\item[\the@pst@label]\fi%
\def\@test{#2}\ifx\@test\@empty\else{\stDMemph{#2}: }\fi#3%
\next@pst@label}
%</package>
%<*ltxml>
DefConstructor('\spfcasesketch OptionalKeyVals:pf{}{}',
	       "<omdoc:proof "
               .  "?&defined(&GetKeyVal(#1,'id'))(xml:id='&GetKeyVal(#1,'id')')()>\n"
	       .  "?#2(<omdoc:omtext>"
              .        "<omdoc:CMP>#2</omdoc:CMP>"
               .      "</omdoc:omtext>\n)()"
	       . "#3"
	     . "</omdoc:proof>\n");
%</ltxml>
%    \end{macrocode}
% \end{environment}
%
% \subsection{Justifications}
%
% We define the actions that are undertaken, when the keys for justifications are
% encountered. Here this is very simple, we just define an internal macro with the value,
% so that we can use it later.
%    \begin{macrocode}
%<*package>
\srefaddidkey{just}
\addmetakey{just}{method}
\addmetakey{just}{premises}
\addmetakey{just}{args}
%</package>
%<*ltxml>
DefKeyVal('just','id','Semiverbatim');
DefKeyVal('just','method','Semiverbatim');
DefKeyVal('just','premises','Semiverbatim');
DefKeyVal('just','args','Semiverbatim');
%</ltxml>
%    \end{macrocode}
%
% The next three environments and macros are purely semantic, so we ignore the keyval
% arguments for now and only display the content.\ednote{need to do something about the
% premise in draft mode.}
%
% \begin{environment}{justification}
%    \begin{macrocode}
%<*package>
\newenvironment{justification}[1][]{}{}
%</package>
%<*ltxml>
sub extractBodyText {
  my ($box, $remove) = @_;
  my $str = '';
  my @boxes = $box->unlist;
  foreach my $b(@boxes) {
    my $s = '';
    if ($b =~ /LaTeXML::Whatsit/) {
      my $body = $b->getBody;
      $s = $body ? extractBodyText($body, $remove) : '';
    } elsif ($b =~ /LaTeXML::Box/) {
      $s = $b->toString || '';
      @{$b}[0] = '' if $remove; }
    $str .= $s; }
  $str =~ s/\s+/ /g;
  $str; }

DefEnvironment('{justification} OptionalKeyVals:just', sub {
		 my ($doc, $keys, %props) = @_;
		 my $text = extractBodyText($props{body}, 1);
		 my $node = LookupValue('_LastSeenCMP');
		 #$node->appendText($text) if $node;
		 my $method = $keys ? $keys->getValue('method') : undef;
		 $doc->openElement("omdoc:method", $method ? (xref => $method) : ());
		 $doc->absorb($props{body}) if $props{body};
		 $doc->closeElement("omdoc:method");
		 return; });
%</ltxml>
%    \end{macrocode}
% \end{environment}
%
% \begin{macro}{\premise}
%    \begin{macrocode}
%<*package>
\newcommand\premise[2][]{#2}
%</package>
%<*ltxml>
DefMacro('\premise[]{}', sub {
	   my ($xref, $text) = ($_[1], $_[2]);
	   my @res = (T_CS('\premise@content'));
	   push(@res, T_OTHER('['), $xref->unlist, T_OTHER(']')) if $xref;
	   push(@res, T_SPACE, $text->unlist) if $text;
	   @res; });
DefConstructor('\premise@content[]', 
               "<omdoc:premise xref='#1'/>");
%</ltxml>
%    \end{macrocode}
% \end{macro}
%
% \begin{macro}{\justarg}
% the |\justarg| macro is purely semantic, so we ignore the keyval arguments for now and
% only display the content.
%    \begin{macrocode}
%<*package>
\newcommand\justarg[2][]{#2}
%</package>
%<*ltxml>
DefMacro('\justarg[]{}', sub { (($_[1] ? $_[1]->unlist : ()),
				T_SPACE, $_[2]->unlist, T_SPACE); });
Tag('omdoc:derive', afterClose=>sub {
      my ($doc, $node) = @_;
      my @children = grep($_->nodeType == XML_ELEMENT_NODE, $node->childNodes);
      my $firstCMP = undef;
      foreach my $child(@children) {
        next unless ($child->localname || '') eq 'CMP';
        if ($child->hasChildNodes()) {
          next unless $#{$child->childNodes} == 0;
          next unless $child->firstChild->nodeType == XML_TEXT_NODE; } 

        if ($firstCMP) {
          $firstCMP->appendText($child->textContent);
          $node->removeChild($child);
        } else { $firstCMP = $child; }
      }
    });#$
%</ltxml>
%    \end{macrocode}
% \end{macro}
%
% \subsection{Providing IDs for {\omdoc} Elements}\label{sec:impl:ids}
%
% To provide default identifiers, we tag all {\omdoc}
% elements that allow |xml:id| attributes by executing the |numberIt| procedure from |omdoc.sty.ltxml|.
% 
%    \begin{macrocode}
%<*ltxml>
Tag('omdoc:proof',afterOpen=>\&numberIt,afterClose=>\&locateIt);
Tag('omdoc:derive',afterOpen=>\&numberIt,afterClose=>\&locateIt);
Tag('omdoc:method',afterOpen=>\&numberIt,afterClose=>\&locateIt);
Tag('omdoc:premise',afterOpen=>\&numberIt,afterClose=>\&locateIt);
Tag('omdoc:derive',afterOpen=>\&numberIt,afterClose=>\&locateIt);
%</ltxml>
%    \end{macrocode}
% 
% \section{Finale}
%
% Finally, we need to terminate the file with a success mark for perl.
%    \begin{macrocode}
%<ltxml>1;
%    \end{macrocode}
% \Finale
\endinput

%%% Local Variables: 
%%% mode: doctex
%%% TeX-master: t
%%% End: 
% LocalWords:  GPL structuresharing STR sproof dtx CPERL keyval methodfalse env
% LocalWords:  methodtrue envtrue medhodtrue DefKeyVal Semiverbatim omdoc args
% LocalWords:  DefEnvironment OptionalKeyVals KeyVal omtext DefConstructor str
% LocalWords:  proofidea KeyVal pfstep DefCMPEnvironment KeyVal proofcomment eq
% LocalWords:  KeyVal pfcases KeyVal pfcase KeyVal extractBodyText unlist elsif
% LocalWords:  foreach getBody toString str str str LookupValue LastSeenCMP Thu
% LocalWords:  appendText getValue undef openElement closeElement DefMacro omd
% LocalWords:  afterClose nodeType childNodes firstCMP localname hasChildNodes
% LocalWords:  firstChild textContent removeChild iffalse kohlhase sref scsys
% LocalWords:  sproofs.sty sc sc mathml openmath latexml cmathml activemath geq
% LocalWords:  twintoo atwin atwintoo texttt fileversion maketitle stex newpage
% LocalWords:  tableofcontents newpage exfig scriptsize vspace ednote spfidea
% LocalWords:  spfidea spfsketch spfsketch spfstep justarg spfcases spfcase rhs
% LocalWords:  sproofcomment ind-hyp splitit arith byindhyp sproofend proofend
% LocalWords:  printbibliography textsf langle textsf langle ltxml ctancite spf
% LocalWords:  srefaddidkey pf.sty newenvironment hbox vrule vbox ifx showmeta
% LocalWords:  hrule vskip hrule vrule hfil nobreak hfill smallskip newcommand
% LocalWords:  stDMemph newcount endsproof xref doctex showmeta hline lec ldots
% LocalWords:  textbackslash makeatletter sketchproof compactenum tracissue
% LocalWords:  metakeys addmetakey metasetkeys stylable pstlabelstyle pstlabel
% LocalWords:  pstlabelstyle pstlabelstyle ldots ldots ensuremath inparaenum
% LocalWords:  nameuse prooflistenv spfcasesketch spfcasesketch spfeq rcll
% LocalWords:  displaymath noindent ignorespaces
