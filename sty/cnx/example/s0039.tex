%%%%%%%%%%%%%%%%%%%%%%%%%%%%%%%%%%%%%%%%%%%%%%%%%%%%%%%%%%%%%%%%%%%%%%%%%%%%%%%%%%%%%%%%%%%%%%
% An example for the LaTeX to CNXML translation
% $URL: svn://kwarc.faculty.iu-bremen.de/kohlhase/kwarc/projects/content/cnx/xsl/latexml.xsl$ 
% $Revision: 242 $; last modified by $Author: kohlhase $ 
% $Date: 2005-01-14 15:22:03 +0100 (Fri, 14 Jan 2005) $
% Copyright (c) 2005 Michael Kohlhase, released under the Gnu Public License 
%%%%%%%%%%%%%%%%%%%%%%%%%%%%%%%%%%%%%%%%%%%%%%%%%%%%%%%%%%%%%%%%%%%%%%%%%%%%%%%%%%%%%%%%%%%%%%

\documentclass[letterpaper]{cnx}
\usepackage{cmathml}%,amsfonts}

\begin{document}
\begin{cnxmodule}[name=Fourier Series,id=m0039]
\begin{metadata}[version=2.19,created=2000/07/21,revised=2004/08/17 22:07:27.213 GMT-5]
\begin{authorlist}
  \cnxauthor[id=dhj,firstname=Don,surname=Johnson,email=dhj@rice.edu]
\end{authorlist}
\begin{maintainerlist}
  \maintainer[id=rha,firstname=Roy,surname=Ha,email=rha@rice.edu]
  \maintainer[id=dhj,firstname=Don,surname=Johnson,email=dhj@rice.edu]
  \maintainer[id=bfite,firstname=Benjamin,surname=Fite,email=bfite@rice.edu]
\end{maintainerlist}
\begin{keywordlist}
\keyword{Euler}
\keyword{Fourier coefficients}
\keyword{Fourier series}
\keyword{frequency}
\keyword{Gauss}
\keyword{orthogonality}
\keyword{sinusoid}
\keyword{square wave}
\end{keywordlist}
\begin{cnxabstract}
  Signals can be composed by a superposition of an infinite number
 of sine and cosine functions.  The coefficients of the superposition
 depend on the signal being represented and are equivalent to knowing
 the function itself.
\end{cnxabstract}
\end{metadata}


\begin{ccontent}
\begin{cpara}[id=p01]
      In 
      \cnxn[document=m0008,target=swsuper,strength=9]{Signal Decomposition}
      we have shown that we could express the square wave as a superposition
      of pulses. This superposition does not generalize well to other periodic signals:
      How can a superposition of pulses equal a smooth signal like a sinusoid?  Because of
      the importance of sinusoids to linear systems, you might wonder whether they could
      be added together to represent a large number of periodic signals. You would be
      right and in good company as well.
      \link[src=http://www-groups.dcs.st-and.ac.uk/~history/Mathematicians/Euler.html]{Euler}
      and 
      \link[src=http://www-groups.dcs.st-and.ac.uk/~history/Mathematicians/Guass.html]{Gauss}
      in particular worried about this problem, and 
      \link[src=http://www-groups.dcs.st-and.ac.uk/~history/Mathematicians/Fourier.html]{Fourier}
      got the credit even though tough mathematical issues were not settled until
      later. They worked on what is now known as the Fourier series.
    \end{cpara}
    
    \begin{cpara}[id=p02] 
      Let $s(t)$ have period $T$.  We want to show that periodic signals, even those that
      have constant-valued segments like a square wave, can be expressed as sum of
      \term{harmonically} related sine waves.

      \begin{cequation}[id=sine]
        \Ceq{s(t)}
            {\Cplus{a_0,\CsumLimits{k}1\infty{(a_k\cos({2\pi kt\over T}))}, 
                        \CsumLimits{k}1\infty{(b_k\sin({2\pi kt\over T}))}}}
      \end{cequation}

      The family of functions called \term{basis functions}
      $\Cset{\Ccos{\Cquotient{\Ctimes{2,\Cpi,k,t}}T},
        \Csin{\Cquotient{\Ctimes{2,\Cpi,k,t}}T}}$ form the foundation of the Fourier
      series. No matter what the periodic signal might be, these functions are always
      present and form the representation's building blocks. They do depend on the
      signal's period $T$ and are indexed by $k$ The frequency of each term is
      $\Cquotient{k}T$.  For $k=0$, the frequency is zero and the corresponding term $a_0$
      is a constant. The basic frequency $\Cquotient{1}T$ is called the \term{fundamental
        frequency} because all other terms have frequencies that are integer multiples of
      it. These higher frequency terms are called \term{harmonics}: The term at frequency
      $\Cquotient{1}T$ is the fundamental and the first harmonic, $\Cquotient{2}T$ the second
      harmonic, etc. Thus, larger values of the series index correspond to higher
      harmonic-frequency sinusoids. The \term{Fourier coefficients}, $a_k$ and $b_k$,
      depend on the signal's waveform. Because frequency is linked to index, the
      coefficients implicitly depend on frequency.

      \begin{cnote}[type=Key point] 
	Assuming we know the period, knowing the Fourier coefficients is equivalent to
        knowing the signal.
      \end{cnote}
    \end{cpara}

    \begin{cexample}[id=squarewave]
      \begin{cpara}[id=sqwaveex1]
	Finding the Fourier series coefficients for the square wave is very
        simple. $sq_T(t)$.
      \end{cpara}
    \end{cexample}

    \begin{cexample}[id=squarewave-two]
      \begin{itemize}
	\item Finding the Fourier series coefficients for the square wave is very
        simple. $sq_T(t)$.
        \begin{eqnarray*}
          A&=&B\\
          C& = & D
        \end{eqnarray*}
      \end{itemize}
    \end{cexample}
    
    \begin{definition}[id=antipodef,term=antipodal]
      \begin{cmeaning}
        Signals $s_1(t)$ and $s_2(t)$ are antipodal if
        $\CforallCond{t}{\Cin{t}{\Cccinterval{0}T}}{s_2(t)=s_1(t)}$
    \end{cmeaning}
  \end{definition}
  \end{ccontent}
\end{cnxmodule}
\end{document}
