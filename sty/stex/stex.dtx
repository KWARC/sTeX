% \iffalse meta-comment
% The sTeX packages all at once
% Copyright (c) 2019 Michael Kohlhase, all rights reserved
%               this file is released under the
%               LaTeX Project Public License (LPPL)
%
% The original of this file is in the public repository at 
% http://github.com/sLaTeX/sTeX/
% \fi
%   
% \iffalse
%<package|logo>\NeedsTeXFormat{LaTeX2e}[1999/12/01]
%<package>\ProvidesPackage{stex}[2019/03/20 v1.0 Semantic Markup]
%<logo>\ProvidesPackage{stex-logo}[2019/03/20 v1.0 sTeX Logo]
%<base>\ProvidesPackage{stex-base}[2019/11/23 v1.0 sTeX Basics]
%
%<*driver>
\documentclass{ltxdoc}
\usepackage[utf8]{inputenc}
\usepackage{url,float,xspace,tikz}
\usepackage[show]{ed}
\usepackage[hyperref=auto,style=alphabetic]{biblatex}
\addbibresource{../../lib/bib/kwarcpubs.bib}
\addbibresource{../../lib/bib/extpubs.bib}
\addbibresource{../../lib/bib/kwarccrossrefs.bib}
\addbibresource{../../lib/bib/extcrossrefs.bib}
\usepackage{stex-logo}
\usepackage{ctangit}
\usepackage[colorlinks]{hyperref}
\usepackage{paralist}
\makeindex
\def\latexml{\hbox{{\LaTeX}ML}\xspace}
\floatstyle{boxed}
\newfloat{exfig}{thp}{lop}
\floatname{exfig}{Example}
\def\xml{XML\xspace}
\def\xslt{XSLT\xspace}
\def\mathml{MathML\xspace}
\def\omdoc{OMDoc\xspace}
\def\smglom{SMGloM\xspace}
\begin{document}\DocInput{stex.dtx}\end{document}
%</driver>
% \fi
% 
% \iffalse\CheckSum{36}\fi
%
% \changes{v1.0}{2015/11/19}{self-documenting package}
% \changes{v1.0}{2015/12/17}{Package dependency figure}
% \changes{v1.0}{2016/04/06}{documenting all \protect\sTeX packages}
% \changes{v1.0}{2019/11/23}{introducing \textbf{stex-base.sty} to be loaded in all \sTeX packages}
% 
% \MakeShortVerb{\|}
% 
% \title{Semantic Markup in {\TeX/\LaTeX}}
% \author{Michael Kohlhase\\
% FAU Erlangen-N\"urnberg\\
% \url{http://kwarc.info/kohlhase}}
% \maketitle
%
% \begin{abstract}
%   We present a collection of {\TeX} macro packages that allow to markup {\TeX/\LaTeX}
%   documents semantically without leaving the document format, essentially turning
%   {\TeX/\LaTeX} into a document format for mathematical knowledge management (MKM).
% \end{abstract}
%
% The functionality of \sTeX is distributed over many packages and classes, see the
% original \href{https://kwarc.info/kohlhase/papers/mcs08-stex.pdf}{\sTeX
% article}~\cite{Kohlhase:ulsmf08} for an overview and motivation.
% \begin{small}
%   \begin{compactitem}
% \item Content Markup of Mathematical Formulae in {\TeX/\LaTeX}
%   \begin{compactitem}
%   \item
%   \href{https://github.com/sLaTeX/sTeX/raw/master/sty/cmath/cmath.pdf}{|cmath|:
%   Content Markup for Mathematical Formulae}
% \item
%   \href{https://github.com/sLaTeX/sTeX/raw/master/sty/presentation/presentation.pdf}{|presentation|:
%   Flexible Notations for Semantic Macros}
%   \end{compactitem}
% \item Mathematical Statements
%   \begin{compactitem}
%   \item
%   \href{https://github.com/sLaTeX/sTeX/raw/master/sty/omtext/omtext.pdf}{|omtext|: Mathematical Text}
%   \item
%   \href{https://github.com/sLaTeX/sTeX/raw/master/sty/statements/statements.pdf}{|statements|: Semantic Markup for Mathematical Statements}
%   \item
%   \href{https://github.com/sLaTeX/sTeX/raw/master/sty/sproof/sproof.pdf}{|sproof|:
%   Semantic Markup for Proofs}
%   \item
%   \href{https://github.com/sLaTeX/sTeX/raw/master/sty/problem/problem.pdf}{|problem|: \emph{Semantic Markup for Problems and Exercises}}
%   \end{compactitem}
% \item Context Markup
%   \begin{compactitem}
%   \item
%   \href{https://github.com/sLaTeX/sTeX/raw/master/sty/modules/modules.pdf}{|modules|: \emph{Object-Oriented Mathematics}}
%   \item
%   \href{https://github.com/sLaTeX/sTeX/raw/master/sty/smultiling/smultiling.pdf}{|smultiling|: \emph{Support for Multilingual Mathematics}}
%   \item
%   \href{https://github.com/sLaTeX/sTeX/raw/master/sty/structview/structview.pdf}{|structview|: \emph{Advanced Inheritance and Interpretation in Mathematics}}
%   \end{compactitem}
% \item Mathematical Document Classes
%   \begin{compactitem}
%   \item
%   \href{https://github.com/sLaTeX/sTeX/raw/master/sty/omdoc/omdoc.pdf}{|omdoc|:  \emph{Open Mathematical Documents}}
%   \item
%   \href{https://github.com/sLaTeX/sTeX/raw/master/sty/smglom.sty/smglom.pdf}{|smglom|: \emph{Semantic Multilingual Glossary of Mathematics}}
%   \item
%   \href{https://github.com/sLaTeX/sTeX/raw/master/sty/mikoslides.sty/mikoslides}{|mikoslides|: \emph{Slides and Lectures for \sTeX}}
%   \item
%   \href{https://github.com/sLaTeX/sTeX/raw/master/sty/hwexam.sty/hwexam.pdf}{|hwexam|: \emph{Content Markup for Homeworks and Exams}}
%   \end{compactitem}
% \item Metadata
%   \begin{compactitem}
%   \item \href{https://github.com/sLaTeX/sTeX/raw/master/sty/dcm/dcm.pdf}{|dcm|:
%   Dublin Core Metadata}
%   \item
%   \href{https://github.com/sLaTeX/sTeX/raw/master/sty/workaddress/workaddress.pdf}{|workaddress|: \emph{FOAF-like Metadata (persons/affiliations)}}
%   \end{compactitem}
% \item Auxiliary Packages
%   \begin{compactitem}
%   \item
%   \href{https://github.com/sLaTeX/sTeX/raw/master/sty/metakeys/metakeys.pdf}{|metakeys|: \emph{Extended Key/Value Management}}
%   \item
%   \href{https://github.com/sLaTeX/sTeX/raw/master/sty/pathsuris/pathsuris.pdf}{|pathsuris|: \emph{Absolute Path Handling for \sTeX}}
%   \item \href{https://github.com/sLaTeX/sTeX/raw/master/sty/sref/sref.pdf}{|sref|:
%   \emph{Semantic References in \sTeX}}
%   \item
%   \href{https://github.com/sLaTeX/sTeX/raw/master/sty/tikzinput/tikzinput.pdf}{|tikzinput|: \emph{Diagrams as Standalone Images}}
%   \item
%   \href{https://github.com/sLaTeX/sTeX/raw/master/sty/mathhub/mathhub.pdf}{|mathhub|: \emph{Support for \sTeX on MathHub.info}}
%   \end{compactitem}
% \end{compactitem}
% \end{small}
% 
% \newpage
% \setcounter{tocdepth}{2}\tableofcontents\newpage
% \newpage
% 
% \section{Introduction}
%
% The last few years have seen the emergence of various content-oriented {\xml}-based,
% content-oriented markup languages for mathematics on the web, e.g.
% OpenMath~\cite{BusCapCar:2oms04}, content MathML~\cite{CarIon:MathML03}, or our own
% {\omdoc}~\cite{Kohlhase:OMDoc1.2}. These representation languages for mathematics, that
% make the structure of the mathematical knowledge in a document explicit enough that
% machines can operate on it. Other examples of content-oriented formats for mathematics
% include the various logic-based languages found in automated reasoning tools
% (see~\cite{RobVor:hoar01} for an overview), program specification languages (see
% e.g.~\cite{Bergstra:as89}).
%
% The promise if these content-oriented approaches is that various tasks involved in ``doing
% mathematics'' (e.g. search, navigation, cross-referencing, quality control, user-adaptive
% presentation, proving, simulation) can be machine-supported, and thus the working
% mathematician is relieved to do what humans can still do infinitely better than machines:
% The creative part of mathematics --- inventing interesting mathematical objects,
% conjecturing about their properties and coming up with creative ideas for proving these
% conjectures. However, before these promises can be delivered upon (there is even a
% conference series~\cite{MKM-IG-Meetings:online} studying ``Mathematical Knowledge
% Management (MKM)''), large bodies of mathematical knowledge have to be converted into
% content form.
%
% Even though {\mathml} is viewed by most as the coming standard for representing
% mathematics on the web and in scientific publications, it has not not fully taken off in
% practice. One of the reasons for that may be that the technical communities that need
% high-quality methods for publishing mathematics already have an established method which
% yields excellent results: the {\TeX/\LaTeX} system: and a large part of mathematical
% knowledge is prepared in the form of {\TeX}/{\LaTeX} documents.
%
% {\TeX}~\cite{Knuth:ttb84} is a document presentation format that combines complex
% page-description primitives with a powerful macro-expansion facility, which is utilized
% in {\LaTeX} (essentially a set of {\TeX} macro packages, see~\cite{Lamport:ladps94}) to
% achieve more content-oriented markup that can be adapted to particular tastes via
% specialized document styles. It is safe to say that {\LaTeX} largely restricts content
% markup to the document structure\footnote{supplying macros e.g. for sections,
% paragraphs, theorems, definitions, etc.}, and graphics, leaving the user with the
% presentational {\TeX} primitives for mathematical formulae. Therefore, even though
% {\LaTeX} goes a great step into the direction of an MKM format, it is not, as it lacks
% infrastructure for marking up the functional structure of formulae and mathematical
% statements, and their dependence on and contribution to the mathematical context.
%
% \subsection{The {\xml} vs. {\TeX/\LaTeX} Formats and Workflows}
%
% {\mathml} is an {\xml}-based markup format for mathematical formulae, it is standardized
% by the World Wide Web Consortium in {\cite{CarIon:MathML03}}, and is supported by the
% major browsers. The {\mathml} format comes in two integrated components: presentation
% {\mathml} presentation MathML and content {\mathml} content MathML. The former provides
% a comprehensive set of layout primitives for presenting the visual appearance of
% mathematical formulae, and the second one the functional/logical structure of the
% conveyed mathematical objects. For all practical concerns, presentation {\mathml} is
% equivalent to the math mode of {\TeX}. The text mode facilitates of {\TeX} (and the
% multitude of {\LaTeX} classes) are relegated to other {\xml} formats, which embed
% {\mathml}.
% 
% The programming language constructs of {\TeX} (i.e. the macro definition
% facilities\footnote{We count the parser manipulation facilities of {\TeX}, e.g. category
%   code changes into the programming facilities as well, these are of course impossible for
%   {\mathml}, since it is bound to {\xml} syntax.}) are relegated to the {\xml}
% programming languages that can be used to develop language extensions. 
% transformation language {\xslt}~\cite{Deach:exls99,Kay:xpr00} or proper {\xml}-enabled
% The {\xml}-based syntax and the separation of the presentational-, functional- and
% programming/extensibility concerns in {\mathml} has some distinct advantages over the
% integrated approach in {\TeX/\LaTeX} on the services side: {\mathml} gives us better
% \begin{itemize}
% \item integration with web-based publishing,
% \item accessibility to disabled persons, e.g. (well-written) {\mathml} contains enough
%   structural information to supports screen readers.
% \item reusability, searchabiliby and integration with mathematical software systems
%   (e.g. copy-and-paste to computer algebra systems), and
% \item validation and plausibility checking.
% \end{itemize}
% 
% On the other hand, {\TeX/\LaTeX}/s adaptable syntax and tightly integrated programming
% features within has distinct advantages on the authoring side:
%  
% \begin{itemize}
% \item The {\TeX/\LaTeX} syntax is much more compact than {\mathml}, and if needed, the
%   community develops {\LaTeX} packages that supply new functionality in with a succinct
%   and intuitive syntax.
% \item The user can define ad-hoc abbreviations and bind them to new control sequences to
%   structure the source code.
% \item The {\TeX/\LaTeX} community has a vast collection of language extensions and best
%   practice examples for every conceivable publication purpose and an established and very
%   active developer community that supports these.
% \item There is a host of software systems centered around the {\TeX/\LaTeX} language that
%   make authoring content easier: many editors have special modes for {\LaTeX}, there are
%   spelling/style/grammar checkers, transformers to other markup formats, etc.
% \end{itemize}
%
% In other words, the technical community is is heavily invested in the whole
% {\index*{workflow}}, and technical know-how about the format permeates the
% community. Since all of this would need to be re-established for a {\mathml}-based
% workflow, the technical community is slow to take up {\mathml} over {\TeX/\LaTeX}, even
% in light of the advantages detailed above.
% 
% \subsection{A {\LaTeX}-based Workflow for {\xml}-based Mathematical Documents}
% 
% An elegant way of sidestepping most of the problems inherent in transitioning from a
% {\LaTeX}-based to an {\xml}-based workflow is to combine both and take advantage of the
% respective advantages.
% 
% The key ingredient in this approach is a system that can transform {\TeX\LaTeX} documents
% to their corresponding {\xml}-based counterparts. That way, {\xml}-documents can be
% authored and prototyped in the {\LaTeX} workflow, and transformed to {\xml} for
% publication and added-value services, combining the two workflows. 
% 
% There are various attempts to solve the {\TeX/\LaTeX} to {\xml} transformation problem
% (see ~\cite{StaGinDav:maacl09} for an overview); the most mature is probably Bruce
% Miller's \latexml system~\cite{Miller:latexml:online}. It consists of two parts: a
% re-implementation of the {\TeX} {\index*{analyzer}} with all of it's intricacies, and a
% extensible {\xml} emitter (the component that assembles the output of the parser). Since
% the {\LaTeX} style files are (ultimately) programmed in {\TeX}, the {\TeX} analyzer can
% handle all {\TeX} extensions, including all of {\LaTeX}. Thus the \latexml parser can
% handle all of {\TeX/\LaTeX}, if the emitter is extensible, which is guaranteed by the
% \latexml binding language: To transform a {\TeX/\LaTeX} document to a given {\xml}
% format, all {\TeX} extensions\footnote{i.e. all macros, environments, and syntax
%   extensions used int the source document} must have ``\latexml
% bindings''\index{LaTeXML}{binding}, i.e. a directive to the \latexml emitter that
% specifies the target representation in {\xml}.
% 
% \subsection{Generating \protect\omdoc from \protect\sTeX}
% 
% The \sTeX packages (see Section~\ref{sec:packages}) provide functionalities for marking
% up the functional structure of mathematical documents, so that the {\LaTeX} sources
% contain enough information that can be exported to the \omdoc format (\underline{O}pen
% \underline{M}athematical \underline{Doc}uments; see~\cite{Kohlhase:OMDoc1.2}). For the
% actual transformation, we use a \latexml plugin~\cite{LaTeXMLsTeX:github:on} that
% provides the \latexml bindings for the \sTeX packages.
%
% \subsection{Conclusion}\label{sec:concl}
%
% The {\stex} collection provides a set of semantic macros that extends the familiar and
% time-tried {\LaTeX} workflow in academics until the last step of Internet publication of
% the material. For instance, an {\smglom} module can be authored and maintained in
% {\LaTeX} using a simple text editor, a process most academics in technical subjects are
% well familiar with. Only in a last publishing step (which is fully automatic) does it get
% transformed into the {\xml} world, which is unfamiliar to most academics. 
%
% Thus, {\stex} can serve as a conceptual interface between the document author and MKM
% systems: Technically, the semantically preloaded {\LaTeX} documents are transformed into
% the (usually {\xml}-based) MKM representation formats, but conceptually, the ability to
% semantically annotate the source document is sufficient.
% 
% The {\stex} macro packages have been validated together with a case
% study~\cite{Kohlhase04:stex}, where we semantically preload the course materials for a
% two-semester course in Computer Science at Jacobs University Bremen and transform them to
% the {\omdoc} MKM format.
%
% \subsection{Licensing, Download and Setup}\label{sec:setup}
% 
% The {\stex} packages are licensed under the {\LaTeX} Project Public License~\cite{LPPL},
% which basically means that they can be downloaded, used, copied, and even modified by
% anyone under a set of simple conditions (e.g. if you modify you have to distribute under a
% different name). 
%
% \subsubsection{The \protect\sTeX Distribution}
%
% The {\stex} packages and classes are available from the Comprehensive {\TeX} Archive
% Network (CTAN~\cite{CTAN:on}) and are part of the primary {\TeX/\LaTeX} distributions
% (e.g. TeXlive~\cite{TeXLive:on} and MikTeX~\cite{MiKTeX:on}). The development version is
% on GitHub~\cite{sTeX:github:on}, it can cloned or forked from the repository URL
% \begin{center}
%   \url{https://github.com/sLaTeX/sTeX.git}
% \end{center}
% \textbf{Warning}: the \sTeX is under continuous development and CTAN updates are as 
% infrequent as they are tedious, so the GIT version is almost always preferable. 
% \iffalse
% It is usually a good idea to enlarge the internal memory allocation of the \TeX/\LaTeX executables. This can be done by
% adding the following configurations in |texmf.cnf| (or changing them, if they
% already exist). Note that you will probably need |sudo| to do this. 
% \begin{footnotesize}\sf
% \begin{verbatim}
% max_in_open = 50        % simultaneous input files and error insertions, 
% param_size = 20000      % simultaneous macro parameters, also applies to MP
% nest_size = 1000        % simultaneous semantic levels (e.g., groups)
% stack_size = 10000      % simultaneous input sources
% main_memory = 12000000
% \end{verbatim}
% \end{footnotesize}
% After that, you have to run the 
% \begin{verbatim}
% sudo fmtutil-sys --all
% \end{verbatim}
% 
% With this installation using \sTeX is as painless as using \LaTeX, just make sure the
% \sTeX distribution is where the |latex| can find it and run |pdflatex| over the main
% file.
% \fi
%
% \subsubsection{The \protect\sTeX Plugin for \protect\latexml}
% 
% For the \omdoc transformation of \sTeX documents we use a \latexml plugin that provides
% the \latexml bindings for the \sTeX packages. For installation and setup follow the
% instructions at~\cite{LaTeXMLsTeX:github:on}
% 
%\newpage
% \section{The Packages of the \protect\stex Collection}\label{sec:packages}
%
% In the following, we will shortly preview the packages and classes in the {\stex}
% collection. They all provide part of the solution of representing semantic structure in
% the {\TeX/\LaTeX} workflow. We will group them by the conceptual level they
% address. Figure~\ref{fig:packages} gives an overview. 
%
% \subsection{The \protect\sTeX Distribution}\label{pkg:stex}
% 
% The |stex| package provides |stex.sty| that just loads all packages below and passes
% around the package options accordingly and |stex-logo.sty| that provides the macros
% \DescribeMacro{\sTeX}|\sTeX| and \DescribeMacro{\stex}|\stex| that typeset the \sTeX
% logo.
%
% \begin{figure}[ht]\centering
%   \begin{tikzpicture}[xscale=1.3]\tt
%   \node(stex-base) at (0,-1) {stex-base};
%   \node (metakeys) at (0,0) {metakeys};
%   \node (pathsuris) at (-2,0) {pathsuris};
%   \node (presentation) at (2,0) {presentation};
%
%   \node (sref) at (0,1) {sref};
%   \node (cmath) at (2.5,1) {cmath};
%
%   \node (modules) at (0,2) {modules};
%   \node (omdoc) at (1.5,2)  {omdoc};
%   \node (sproof) at (3,2)  {sproof};
%
%   \node (wa) at (-2,2) {workaddress};
%   \node (omtext) at (0,3) {omtext};
%   \node (structview) at (-2,4)  {structview};
%
%   \node (dcm) at (-2,3) {dcm};
%   \node (statements) at (0,4) {statements};
%   \node (stex-logo) at (3.5,4)  {stex-logo};
%
%   \node (problem) at (2.5,5) {problem};
%   \node (tikzinput) at (4.5,5) {tikzinput};
%   \node (stex) at (0,5) {stex};
%   \node (smultiling) at (-2,5) {smultiling};
%
%   \node (smglomsty) at (-2,6) {smglom.sty};
%   \node (mikoslidessty) at (.5,6) {mikoslides.sty};
%   \node (hwexamsty) at (3.5,6) {hwexam.sty};
%
%   \node (smglomcls) at (-2,7) {smglom.cls};
%   \node (mikoslidescls) at (.5,7) {mikoslides.cls};
%   \node (hwexamcls) at (4.5,7) {hwexam.cls};
%   \node (omdoccls) at (2.5,6.5) {omdoc.cls};
%
%   \draw[->] (stex-logo) to[bend left=50] (stex-base);
%   \draw[->] (tikzinput) to[bend left=50] (stex-base);
%   \draw[->] (metakeys) -- (stex-base);
%   \draw[->] (pathsuris) -- (stex-base);
%   \draw[->] (presentation) -- (stex-base);
%   \draw[->] (sref) -- (metakeys);
%   \draw[->] (cmath) -- (presentation);
%   \draw[->] (dcm) -- (sref);
%   \draw[->] (wa) -- (sref);
%   \draw[->] (modules) -- (sref);
%   \draw[->] (modules) -- (pathsuris);
%   \draw[->] (omdoc) -- (sref);
%   \draw[->] (omdoc) to[bend left=20] (pathsuris);
%   \draw[->] (sproof) -- (sref);
%   \draw[->] (dcm) -- (wa);
%   \draw[->] (omtext) -- (modules);
%   \draw[->] (statements) -- (omtext);
%   \draw[->] (stex) -- (statements);
%   \draw[->] (stex) -- (dcm);
%   \draw[->] (stex) to[bend left=5] (sproof);
%   \draw[->] (stex) to[bend left=5] (structview);
%   \draw[->] (structview) -- (modules);
%   \draw[->] (stex) to[bend left=15] (cmath);
%   \draw[->] (stex) to[bend left=20] (omdoc);
%   \draw[->] (stex) --  (stex-logo);
%   \draw[->] (problem) -- (omtext);
%   \draw[->] (smglomsty) -- (smultiling);
%   \draw[->] (smultiling) -- (structview);
%   \draw[->] (smglomsty) -- (statements);
%   \draw[->] (smglomcls) -- (smglomsty);
%   \draw[->] (smglomcls) -- (stex);
%   \draw[->] (mikoslidescls) -- (mikoslidessty);
%   \draw[->] (mikoslidescls) -- (smglomsty);
%   \draw[->] (mikoslidessty) -- (tikzinput);
%   \draw[->] (mikoslidessty) -- (stex);
%   \draw[->] (mikoslidessty) -- (smglomsty);
%   \draw[->] (hwexamcls) -- (hwexamsty);
%   \draw[->] (hwexamsty) -- (problem);
%   \draw[->] (omdoccls) to[bend right=20] (omdoc);
%
%   \draw[->] (hwexamcls) -- (tikzinput);
%   \draw[->] (hwexamcls) --   (omdoccls);
%   \draw[->] (mikoslidescls) to[bend left=10] (omdoccls);
%   \draw[->] (smglomcls) to[bend right=10] (omdoccls);
% \end{tikzpicture}
% \caption{The \protect\sTeX packages and their dependencies.}\label{fig:packages}
% \end{figure}
%
% \subsection{Content Markup of Mathematical Formulae in {\TeX/\LaTeX}}
%
% \subsubsection{\texttt{cmath}: Building Content Math Representations}\label{pkg:cmath}
% 
% The |cmath| package (see~\ctancite{Kohlhase:cmath}) supplies an interface for building
% content math representations. It gives special macros for marking up variables,
% applications and bindings. It supports the transformation into both
% OpenMath~\cite{BusCapCar:2oms04} and content MathML~\cite{CarIon:MathML03}.
%
% \subsubsection{{\texttt{presentation}}: Flexible Notations for Semantic
% Macros}\label{pkg:presentation}
%
% The |presentation| package (see~\ctancite{Kohlhase:ipsmsl}) supplies an infrastructure
% that allows to specify the presentation of semantic macros, including preference-based
% bracket elision. This allows to markup the functional structure of mathematical formulae
% without having to lose high-quality human-oriented presentation in {\LaTeX}. Moreover,
% the notation definitions can be used by MKM systems for added-value services, either
% directly from the \sTeX sources, or after translation.
%
% \subsection{Mathematical Statements}
%
% \subsubsection{\texttt{omtext}: Mathematical Text}\label{pkg:omtext}
% The |omtext| package (see\ctancite{Kohlhase:smmtf}) supplies an infrastructure for
% writing mathematical text fragments in \sTeX. The basic |omtext| environment can be
% given various metadata that are relevant for content-oriented mathematics. |omtext| is
% specialized to various specially layouted text forms like quotes and such. Furthermore,
% the |omtext| package provides content-oriented index markup.
% 
% \subsubsection{{\texttt{statements}}: Definition, Theorem, Example and the like}\label{pkg:statements}
% 
% The |statements| package (see\ctancite{Kohlhase:smms}) provides semantic markup
% facilities for mathematical statements like Theorems, Lemmata, Axioms, Definitions,
% etc. in {\stex} files. This structure can be used by MKM systems for added-value services,
% either directly from the \sTeX sources, or after translation.
%
% \subsubsection{{\texttt{sproof}}: sTeX Proofs}\label{pkg:sproof}
%
% The |sproof| package (see~\ctancite{Kohlhase:smp}) supplies macros and environment
% that allow to annotate the structure of mathematical proofs in {\stex} files. This
% structure can be used by MKM systems for added-value services, either directly from the
% \sTeX sources, or after translation.
% 
% \subsubsection{{\texttt{problem}}: Problems and Exercises}\label{pkg:problem}
%
% The |problem| package (see~\ctancite{Kohlhase:problem}) supplies macros and environment
% that allow to annotate the structure of mathematical and exercises in {\stex}
% files. These structures can be used for homework assignments and exams; see
% section~\ref{pkg:hwexam}.
% 
% \subsection{Context Markup for Mathematics}
%
% \subsubsection{{\texttt{modules}}: Extending Content Macros for Mathematical\
% Notation}\label{pkg:modules}
% 
% The |modules| package (see~\ctancite{KohAmb:smmssl}) supplies a definition
% mechanism for semantic macros and a non-standard scoping construct for them, which is
% oriented at the semantic dependency relation rather than the document structure. This
% structure can be used by MKM systems for added-value services, either directly from the
% \sTeX sources, or after translation. A side effect of this is that we have an
% ``object-oriented'' inheritance mechanism for semantic macros: the semantic macros for
% the mathematical objects described in a module come with the module itself.
% 
% \subsubsection{\texttt{smultiling}: Multilingual Mathematical
% Modules}\label{pkg:smultiling}
% 
% In multilingual settings, i.e. where we have multiple \sTeX documents that are
% translations of each other, it is better to separate the module signature from the
% descriptive document. The |smultiling| package (see~\ctancite{KohGin:smss}) provides the
% necessary infrastructure. 
%
% \subsubsection{\texttt{structview}: Structures and Views}\label{pkg:structview}
%
% OMDoc/MMT structures add advanced inheritance and interpretation methods for
% mathematical knowledge management -- see~\cite{RabKoh:WSMSML13} for a motivation and the
% grisly details. The |structview| package (see~\ctancite{Kohlhase:svs}) provides the
% necessary infrastructure. 
%
% \subsection{Mathematical Document Classes}\label{sec:mathdoc}
%
% The \sTeX distribution provides four top-level document classes for various types of
% mathematical documents. In all cases, the functionality of the \meta{class}|.cls| is
% mainly provided by \meta{class}|.sty|, allowing the \LaTeX document class
% \meta{class}|.cls| to be minimal, only setting up the required packages and calling
% corresponding package \meta{class}|.sty|.
% 
% \subsubsection{OMDoc Documents}\label{pkg:omdoc}
%
% The |omdoc| class and package (see \ctancite{Kohlhase:smomdl}) provides an
% infrastructure that allows to markup up generic mathematical documents with {\omdoc}
% structures. It provides the |omgroup| environment for reusable document fragments.
%
% \subsubsection{\texttt{hwexam}: Homeworks and Exams}\label{pkg:hwexam}
%
% The \texttt{hwexam} package~\ctancite{Kohlhase:hwexam} infrastructure for marking up
% homework assignments, and exams. The content markup strategy employed in \sTeX allows to
% specify -- and profit from -- administrative metadata such as time and point
% counts. This package relies on the |problem| package~\ctancite{Kohlhase:problem} which
% provides markup for problems, hints, and solutions.
% 
% \subsubsection{\texttt{smglom}: Mathematical Glossary}\label{pkg:smglom}
%
% The |smglom| package provides a document class for the Semantic Multilingual Glossary of
% Mathematics. This bakes in special file/module  conventions that allow the concise and
% convenient markup of multilingual glossary modules.
% 
% \subsubsection{\texttt{mikoslides}: Slides and Course Notes}\label{pkg:mikoslides}
%
% The |mikoslides| package provides a document class from which we can generate both
% course slides -- via the |beamer| class -- and course notes -- via the |omdoc| class --
% in a transparent way.
% 
% \subsection{Metadata}\label{sec:metadata}
%
% The metadata packages allow embedding linked-data-style annotations into texts. They are
% tied to well-known ontologies. 
% 
% \subsubsection{\texttt{dcm}: Dublin Core Metadata}\label{pkg:dcm}
% The |dcm| package (see \ctancite{Kohlhase:dcm}) provides keys and
% macros for annotating Dublin Core Metadata~\cite{DublinCore:on}.
%
% \subsubsection{\texttt{workaddress}: Markup for FOAF Metadata}\label{pkg:workaddress}
% The |workaddress| package (see \ctancite{Kohlhase:workaddress}) provides keys and macros
% for annotating metadata from a part of the FOAF (Friend-of-a-Friend; persons, their
% relations, and affiliations) ontology~\cite{FOAF:on}.
%
% \subsection{Support for MathHub}\label{sec:MathHub}
% 
% The |mathhub| package provides the supplementary packages |mikoslides-mh|,
% |modules-mh.sty|, |omtext-mh.sty|, |problem-mh.sty|, |smultiling-mh.sty|,
% |structview-mh.sty|, and |tikzinput-mh.sty| with variants of the user-visible macros
% that are adapted to the MathHub system -- see Section~\ref{sec:mathhub} for details.
%
% \subsection{Auxiliary Packages}\label{sec:au}
% 
% \subsubsection{\texttt{metakeys}: An extended key/value Interface for \protect\sTeX}\label{pkg:metakeys}
% |metakeys| is a purely syntactic utility package, see~\ctancite{Kohlhase:metakeys}, it should
% eventually, probably be replaced by something more standard and maintained.
%
% \subsubsection{\texttt{pathsuris}: Managing Relative/Absolute File Paths}\label{pkg:pathsuris}
%
% Experience shows that distributing \sTeX content into small {\LaTeX} files makes content
% much more reusable. The price we pay is that we have to manage many file names from
% which a document is composed. Practical experience shows that system-internally, we want
% to use absolute file URIs (as canonical file names) to get around redundant file loading
% which would lead to exponential blowups. The |pathsURIs| package (see~\ctancite{ZhaKoh:pus})
% provides the necessary infrastructure. 
%
% \subsubsection{\texttt{sref}: Semantic References}\label{pkg:sref}
% 
% \sTeX uses its own (semantic) infrastructure for internal (and cross-document!)
% references: the |sref| package (see~\ctancite{Kohlhase:sref}). It should eventually,
% probably be replaced by -- or at least based on -- something more standard and
% maintained.
%
% \subsubsection{\texttt{tikzinput}: External TIKZ Pictures as Standalone Images}\label{pkg:tikzinput}
% 
% |tikz| is a diagram package for {\LaTeX}, which can be used for mixed (i.e. with \sTeX
% islands) diagrams in mathematical documents. However \latexml treatment of |tikz| is
% still experimental, so we use the |tikzinput| package (see~\ctancite{KohGin:sitd}) to
% externalize the tikz material into standalone files, which can be easily re-used and
% converted to e.g. SVG independently and then input as regular images. 
%
% \newpage
% \section{Workflows and Best Practices}\label{sec:workflows}
%
% \subsection{The ``Little Modules'' Approach}
% 
% One of the key advantages of semantic markup with \sTeX is that the \sTeX sources are
% highly reusable by the ``object-oriented'' inheritance model induced by \sTeX
% modules. It turned out to be useful to divide \sTeX documents into three kinds of files:
% \begin{enumerate}
% \item \textbf{module files}: files that essentially contain a collection of \sTeX
%   modules~\ctancite{KohAmb:smmssl} -- usually a single one whose module name coincides
%   the file name base.
% \item \textbf{fragment files}: files that contain a group of input references to module-
%   or fragment files -- usually one group deep for flexibility, transition text, and
%   additional remarks.
% \item \textbf{driver files} that set up the document class, contain the preambles, and
%   input reference fragment files. 
% \end{enumerate}
%   
% These correspond to the \sTeX documents, but can reuse and share \sTeX fragments and
% modules. Figure~\ref{fig:reuse} shows a situation, where we have two courses given over
% multiple years, which results in five course notes documents given by driver files, wich
% share quite a few components. As drivers and fragment files are mostly content-free --
% they only contribute document structure, this lets all documents contribute from the
% development of the modules.
% 
% \begin{figure}[ht]\centering
% \begin{tikzpicture}\footnotesize
%   \tikzstyle{course} = [draw,rounded corners,inner sep=0pt]
%   \tikzstyle{topic} = [draw,inner sep=2pt]
%
%   \node at (-1,0) {modules};
%   \node at (-1,1) {fragments};
%   \node at (-1,2) {drivers};
%
%   \node[topic] (strings) at (1,0) {strings};
%   \node[topic] (pref) at (2.5,0) {\begin{tabular}{c}prefix\\codes\end{tabular}};
%   \node[topic] (codes) at (1.5,1) {codes};
%
%   \node[topic] (xml) at (5.3,0) {DAG};
%   \node[topic] (uc) at (4,0) {Trees};
%   \node[topic] (html) at (4.5,1) {GraphTheo};
%  
%   \node[topic] (docbook) at (8,0) {NatDed};
%   \node[topic] (dita) at (6.8,0) {FOL};
%   \node[topic] (man) at (7.5,1) {Logic};
%  
%   \node[course] (gencs) at (3,2) {\begin{tabular}{c}GenCS\\2011\end{tabular}};
%   \node[course] (gencs10) at (1.5,2) {\begin{tabular}{c}GenCS\\2010\end{tabular}};
%   \node[course] (gencs12) at (4.5,2) {\begin{tabular}{c}GenCS\\2012\end{tabular}};
%   \node (gdots) at (3,1) {\ldots};
%   \node[course] (tdm) at (6.5,2) {\begin{tabular}{c}AdvCS\\2011\end{tabular}};
%   \node[course] (tdm12) at (8,2) {\begin{tabular}{c}AdvCS\\2012\end{tabular}};
%   \node (tdots) at (6,1) {\ldots};
%  
%   \draw (gencs) -- (strings) -- (codes);
%   \draw (gencs) -- (html) -- (uc);
%   \draw (tdm) -- (man) -- (docbook);
%   \draw (tdm) -- (html) -- (xml);
%   \draw (codes) -- (pref);
%   \draw (man) -- (dita);
%   \draw (gencs) -- (gdots);
%   \draw (tdm) -- (tdots);
%   \draw (gencs12) -- (codes);
%   \draw (gencs10) -- (codes);
%   \draw (gencs12) -- (gdots);
%   \draw (gencs10) -- (gdots);
%   \draw (gencs12) -- (html);
%   \draw (gencs10) -- (html);
%   \draw (tdm12) -- (html);
%   \draw (tdm12) -- (tdots);
%   \draw (tdm12) -- (man);
% \end{tikzpicture}
% \caption{Reuse of Fragments and Modules in a Course Notes Setting}\label{fig:reuse}
% \end{figure}
% 
% Another advantage of the ``little modules approach'' is that modules can be developed
% separately. Indeed all modules of a given subject share common pre- and post-ambles
% which can be developed in special files -- usually named |pre.tex|, |post.tex|, and
% |preamble.tex| (the latter is included in |pre.tex|). Given such a setup, the call to
% |pdflatex| can be suitably adapted to handle the pre/postfixes.
%
% \subsection{Basic Utilities \& Makefiles}\label{sec:utilities}
% 
% The \sTeX distribution contains three basic command line utilities to manage \sTeX
% documents in the |bin| directory of the distribution.
% \begin{description}
% \item[\texttt{filedate} and \texttt{checksum}] that help keeping the metadata of the
%   self-documenting {\LaTeX} packages in the \stex distribution up to date.
% \item[\texttt{installFonts.sh}] that installs the fonts necessary for Chinese \sTeX
%   documents.
% \end{description}
% These are supplemented by a set of UNIX |Makefiles| in the |lib/make| directory. The way
% to use them is to include them into a |Makefile| in the directory and then run one of
% the targets |pdf| and |mpdf| to make the PDF versions of the drivers and modules and
% |omdoc| and |mods| to generate \omdoc.
% 
% \subsection{MathHub: a Portal for Active Mathematical Documents}\label{sec:mathhub}
% 
% MathHub (\url{http://mathhub.info} see~\cite{IanJucKoh:sdm14}) is a portal for Active
% Mathematical Documents -- documents that are made context-aware and interactive by
% semantic annotations. \sTeX is one of the main input formats for informal active
% documents. MathHub supports \sTeX documents in three ways:
% \begin{enumerate}
% \item MathHub offers free/open hosting in document repositories for (mathematical) \sTeX
%   document collections.
% \item the backend system supports the large-scale change- and error-management for \sTeX
%   documents in the ``little modules'' paradigm.
% \item the front-end displays interactive (HTML5) documents generated from the \sTeX
%   sources (via \omdoc).
% \end{enumerate}
% The MathHub system is probably the best way of developing and hosting larger \sTeX
% document collections. It offers two authoring workflows an online authoring workflow via
% a direct web interface~\cite{MathHub:oa:on} or casual users and an offline authoring
% workflow that we describe next.
% 
% \StopEventually{\newpage\PrintIndex\newpage\PrintChanges\newpage\printbibliography}\newpage
%
% \newpage
% \section{The Implementation}\label{sec:implementation}
%
% \subsection{Package Options}\label{sec:impl:options}
% 
% The first step is to declare (a few) package options that handle whether certain
% information is printed or not. They all come with their own conditionals that are set by
% the options.
%
%    \begin{macrocode}
%<*package>
\DeclareOption*{\PassOptionsToPackage{\CurrentOption}{statements}
  \PassOptionsToPackage{\CurrentOption}{structview}
  \PassOptionsToPackage{\CurrentOption}{sproofs}
  \PassOptionsToPackage{\CurrentOption}{omdoc}
  \PassOptionsToPackage{\CurrentOption}{cmath}
  \PassOptionsToPackage{\CurrentOption}{dcm}}
\ProcessOptions
%    \end{macrocode}
%
% Then we make sure that the necessary packages are loaded (in the right versions).
%    \begin{macrocode}
\RequirePackage{stex-logo}
\RequirePackage{omdoc}
\RequirePackage{statements}
\RequirePackage{structview}
\RequirePackage{sproof}
\RequirePackage{cmath}
\RequirePackage{dcm}
%</package>
%    \end{macrocode}
%
% \subsection{The \protect\sTeX Logo}\label{sec:impl:logo}
%
%    \begin{macrocode}
%<*logo>
\RequirePackage{stex-base}
\RequirePackage{xspace}
\protected\def\stex{%
  \@ifundefined{texorpdfstring}%
  {\let\texorpdfstring\@firstoftwo}%
  {}%
  \texorpdfstring{\raisebox{-.5ex}S\kern-.5ex\TeX}{sTeX}\xspace%
}
\def\sTeX{\stex}
%</logo>
%    \end{macrocode}
%
% \subsection{The \protect\sTeX Basics}\label{sec:impl:basics}
%
% Basic setup stuff
%
% We define a conditional |\if@latexml| to determine whether we are running \latexml or
% |pdflatex|. We set it to |false| here, and in the \latexml binding we set it to true. 
%    \begin{macrocode}
%<*base>
\ifcsname if@latexml\endcsname\else
  \expandafter\newif\csname if@latexml\endcsname\@latexmlfalse
\fi
%</base>
%    \end{macrocode}
% \Finale
\endinput
% \iffalse
%%% Local Variables: 
%%% mode: doctex
%%% TeX-master: t
%%% End: 
% \fi
% LocalWords:  GPL structuresharing STR dtx pts keyval xcomment CPERL DefKeyVal iffalse 
% LocalWords:  RequirePackage Semiverbatim DefEnvironment OptionalKeyVals soln texttt baz
% LocalWords:  exnote DefConstructor inclprob NeedsTeXFormat omd.sty textbackslash exfig
% LocalWords:  stopsolution fileversion filedate maketitle setcounter tocdepth newpage uc
% LocalWords:  tableofcontents showmeta showmeta solutionstrue usepackage minipage hrule
% LocalWords:  linewidth elefants.prob Elefants smallskip noindent textbf startsolutions
% LocalWords:  startsolutions stopsolutions stopsolutions includeproblem includeproblem
% LocalWords:  textsf HorIacJuc cscpnrr11 includemhproblem includemhproblem importmodule
% LocalWords:  importmhmodule foobar ldots latexml mhcurrentrepos mh-variants mh-variant
% LocalWords:  compactenum langle rangle langle rangle ltxml metakeys newif ifexnotes rm
% LocalWords:  exnotesfalse exnotestrue ifhints hintsfalse hintstrue ifsolutions ifpts
% LocalWords:  solutionsfalse ptsfalse ptstrue ifmin minfalse mintrue ifboxed boxedfalse
% LocalWords:  boxedtrue sref mdframed marginpar prob srefaddidkey addmetakey refnum kv
% LocalWords:  newcounter ifx thesection theproblem hfill newenvironment metasetkeys ltx
% LocalWords:  stepcounter currentsectionlevel xspace ignorespaces surroundwithmdframed
% LocalWords:  omdoc autoopen autoclose solvedinminutes kvi qw vals newcommand exhint pgk
% LocalWords:  specialcomment excludecomment mhrepos xref marginpar addtocounter doctex
% LocalWords:  mh@currentrepos endinput stex Bergstra mathml ttb84 utilized ladps94 xslt
%  LocalWords:  specialized standardized Deach exls99 xpr00 compactitem searchabiliby tdm
%  LocalWords:  StaGinDav maacl09 analyzer athematical uments concl smglom subsubsection
%  LocalWords:  TeXlive TeXLive texmf.cnf pdflatex ednote cmath cmath ctancite ipsmsl lmh
%  LocalWords:  centering tikzpicture xscale  rdfmeta rdfmeta sproof sproof
%  LocalWords:  workaddress omtext omtext tikzinput tikzinput smglomsty mikoslidessty
%  LocalWords:  mikoslides.sty hwexamsty hwexam.sty smglomcls smglom.cls mikoslidescls
%  LocalWords:  mikoslides.cls hwexamcls hwexam.cls omdoccls omdoc.cls smms KohAmb smmssl
%  LocalWords:  omdocdoc.sty hwexam hwexam mikoslides mikoslides mathhub modules-mh.sty
%  LocalWords:  omtext-mh.sty problem-mh.sty smultiling-mh.sty structview-mh.sty docbook
%  LocalWords:  tikzinput-mh.sty pathsuris pathsuris tikzstyle NatDed gencs gdots tdots
%  LocalWords:  printbibliography sproofs texorpdfstring raisebox IanJucKoh sdm14 localmh
%  LocalWords:  gitsc gitsc Kohlhase:ulsmf08 smultiling RobVor:hoar01 Kohlhase:smmtf tikz
%  LocalWords:  Deach:exls99,Kay:xpr00 layouted pkg:statements pkg:modules KohGin:smss
%  LocalWords:  pkg:structview Kohlhase:svs sec:mathdoc Kohlhase:smomdl ZhaKoh:pus mpdf
%  LocalWords:  externalize
