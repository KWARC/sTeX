% \iffalse meta-comment
% A LaTeX Class and Package for Semantic Cross-Referencing
% Copyright (c) 2009 Michael Kohlhase, all rights reserved
%               this file is released under the
%               LaTeX Project Public License (LPPL)
%
% The original of this file is in the public repository at 
% http://github.com/KWARC/sTeX/
% \fi
%   
% \iffalse
%<*package>
\NeedsTeXFormat{LaTeX2e}[1999/12/01]
\ProvidesPackage{sref}[2013/10/15 v1.1 Semantic Cross-References]
%</package>
%<*driver>
\documentclass{ltxdoc}
\usepackage{url,array,float}
\usepackage[show]{ed}
\usepackage[hyperref=auto,style=alphabetic]{biblatex}
\addbibresource{kwarcpubs.bib}
\addbibresource{extpubs.bib}
\addbibresource{kwarccrossrefs.bib}
\addbibresource{extcrossrefs.bib}
\usepackage{stex-logo,sref}
\usepackage{../ctangit}
\usepackage{hyperref}
\makeindex
\floatstyle{boxed}
\newfloat{exfig}{thp}{lop}
\floatname{exfig}{Example}
\newcommand\tracissue[1]{\cite{sTeX:online}, \hyperlink{http://trac.kwarc.info/sTeX/ticket/#1}{issue #1}}
\begin{document}\DocInput{sref.dtx}\end{document}
%</driver>
% \fi
% 
%\iffalse\CheckSum{315}\fi
% 
% \changes{v0.5}{2008/03/28}{split off from omdoc.sty}
% \changes{v0.6}{2009/12/07}{introducing sequation}
% \changes{v0.9}{2009/07/14}{using omd for metadata}
% \changes{v1.0}{2011/12/20}{adding \texttt{\textbackslash sreflabel}}
% \changes{v1.1}{2013/10/15}{adding fallback argument to \texttt{\textbackslash sref}}
% 
% \GetFileInfo{sref.sty}
% 
% \MakeShortVerb{\|}
%
% \def\omdoc{OMDoc}
% \def\latexml{{\LaTeX}ML}
% \title{{\texttt{sref.sty}}: Semantic Cross-Referencing in {\LaTeX}\thanks{Version {\fileversion} (last revised
%        {\filedate})}}
%    \author{Michael Kohlhase\\
%            Jacobs University, Bremen\\
%            \url{http://kwarc.info/kohlhase}}
% \maketitle
%
% \begin{abstract}
%   The |sref| package is part of the {\sTeX} collection, a version of {\TeX/\LaTeX} that
%   allows to markup {\TeX/\LaTeX} documents semantically without leaving the document
%   format, essentially turning {\TeX/\LaTeX} into a document format for mathematical
%   knowledge management (MKM).
%
%   The |sref| package supplies an for semantic cross-referencing over multiple documents. 
% \end{abstract}
%
%\tableofcontents\newpage
% 
% \section{Introduction}\label{sec:intro}
%
% The automatic computation of cross-references is one of the traditional strong points of
% {\LaTeX}. However, cross-referencing is limited to labels in the current document
% only. Cross-referencing between multiple documents in a jointly developed document
% collection is not easy to achieve in the {\LaTeX} processing model, which reads files
% sequentially and lacks a path concept. 
%
% The |sref| package is mainly aimed at package developers. It supplies the internal
% macros that can be used to make document structuring elements cross-referencable. The
% general strategy here is to equip the document structuring macros with an
% \DescribeMacro{id}|id| key, so that the author can specify meaningful ones, but to let
% the transformation give default ones if the author did not. The value of the |id| key
% can also be used for cross-referencing like the |\label|/|\ref| mechanism in
% {\LaTeX}. We implement an independent referencing mechanism, since the referencing model
% is geared more towards referencing text fragments than text fragment labels like section
% numbers. Therefore we let the referenced fragments define the reference text much like
% the |\autoref| macro from |\hpyerref|.
%
% \section{The User Interface}\label{sec:user}
% 
% This package is currently mainly meaningful in the context of the {\sTeX} collection,
% since all cross-referencable macros and environments must be extended to know about
% their referencing mechanism. We explain the user interface in \sref{sucrossref}. To port
% the functionality to other {\LaTeX} classes and packages, they have to be upgraded via
% the API in \sref{supackage-authors}.
%
% \subsection{Package Options}\label{sec:user:options}
% 
% The |sref| package has the \DescribeMacro{extrefs}|extrefs| package option, which can be
% set to activate multi-file support (see \sref{suinterdoc}). If the
% \DescribeMacro{showmeta}|showmeta| is set, then the metadata keys are shown
% (see~\cite{Kohlhase:metakeys:ctan} for details and customization options).
% 
% \subsection{Cross-Referencing}\sreflabel{Section}{sucrossref}
%
% The \DescribeMacro{\sref}|\sref{|\meta{id}|}| macro is the main cross-referencing macro,
% see Figure~\ref{fig:mysection} for an example. Depending on the whether macro or
% environment marking up the respective document fragment carries the key/value pair
% |id=|\meta{id} the cross-reference will expand to ``Section 2.1'' or ``this remark'',
% both carrying hyper-references. The |\sref| macro takes an optional key/value argument
% that allows to customize its behavior: The |linktext| key can be used to specify a link
% text that overrides the auto-generated one and the |fallback| allows to give the
% fallback text when the label \meta{id} is not defined.
% 
% \begin{exfig}
% \begin{verbatim}
% \mysection[id=foo]{#2}
% ... \sref{foo} ...
% ... \sref[linktext=this section]{foo} ...
% ... \sref[fallback=above]{foo} ...
% \end{verbatim}
% \vspace*{-1.5em}
% \caption{Semantic Crossreferencing}\label{fig:mysection}
% \end{exfig}
% 
% The \DescribeMacro{\sreflabel}|\sreflabel|\footnote{It would have been more natural to
% name the macro \texttt{slabel}, but this is overwritten by other packages without
% warning.} macro is a variant to the |\label| macro provided by {\LaTeX} proper. It takes
% two arguments, the first one is a classification (used in |\sref|) and the second one
% the identifier.
%
% The
% \DescribeMacro{\srefs}|\srefl{|\meta{id\textsuperscript{1}}|}{|\meta{id\textsuperscript{2}}|}|
% is a variant it |\sref|, only that it allows to reference two semantic objects and
% expands to ``\meta{reference\textsuperscript{1}} and
% \meta{reference\textsuperscript{2}}''. \DescribeMacro{\srefl}|\srefl{|\meta{id\textsuperscript{1}}|}{|\meta{id\textsuperscript{n}}|}|
% is similar, but for ranges; it expands to ``\meta{reference\textsuperscript{1}} to
% \meta{reference\textsuperscript{n}}''. Its use should be restricted to cases, where the
% types of objects references are homogenous.
% 
% Finally, there is a variant \DescribeMacro{\spageref}|\spageref| that only outputs the
% page number of the referenced object. It can be used in cases where no hyper-referencing
% is present. It uses the macro \DescribeMacro{\sref@page@label}|\sref@page@label| for
% styling the page reference. Redefining this will allows to customize this. The default
% setting is
% \begin{verbatim}
% \newcommand\sref@page@label[1]{p.~{#1}}
% \end{verbatim}
%
% \subsection{An API for Package Authors}\sreflabel{Section}{supackage-authors}
%
% To make use of the |sref| package, the package must define the document structuring
% infrastructure using the |sref| internal macros. The {\sTeX} packages already does this,
% so we make an example here for a slightly upgraded sectioning command in
% Figure~\ref{fig:mysection:setup}. The first three lines define the keys for the keyval
% attribute of the |\mysection| command using the infrastructure supplied by the |omd|
% package~\ctancite{Kohlhase:metakeys} (remember the |\RequirePackage{metakeys}|).  The first two just
% initialize the keys to save the key values in internal macros, and the |\metasetkeys|
% activates the keys when reading the keyval argument. The
% \DescribeMacro{\srefaddidkey}|\srefaddidkey| macro is a variant of |\addmetakey| macro
% supplied by the |sref| package that sets up the keys to set the
% \DescribeMacro{\sref@id}|\sref@id| register for later use by the |sref|
% infrastructure. Note that the |\srefaddidkey| macro uses the |prefix| key to
% systematically construct prefixed identifiers. This can be useful in particular for
% sectioning commands. 
% 
% \begin{exfig}
% \begin{verbatim}
% \addmetakey{sec}{short}
% \addmetakey[black]{sec}{color}
% \srefaddidkey[prefix=sec.]{sec}
% \newcommand\mysection[2][]{\metasetkeys{#1}\sref@target\color{\sec@color}
% \section[\sec@short]{#2}\sref@label@id{Section \thesection}}
% \end{verbatim}
% \vspace*{-1.5em}
% \caption{A slightly upgraded sectioning command}\label{fig:mysection:setup}
% \end{exfig}
% 
% In this situation, the |\mysection| macro processes the optional argument with
% |\metasetkeys| and then sets the color of the section. The
% \DescribeMacro{\sref@target}|\sref@target| sets up the hyper-target for the |hyperref|
% package to use. Then we use the regular |\section| command, and we use the
% \DescribeMacro{\sref@label@id}|\sref@label@id| macro to define the label that the
% |\sref| macro will use for cross-referencing.
% 
% Note that the use of the straight use of the label ``Section'', which will be written
% into the auxiliary files is bad practice since it is not configurable. It would be much
% better to make it configurable via a presentation macro like |\my@section@label| in
% Figure~\ref{fig:mysection:configurable}. Then translators or even the user could redefine
% the |\my@section@label| to adapt them to their needs.
%
% \begin{exfig}
% \begin{verbatim}
% \newcommand\my@section@label[1]{Section~{#1}}
% \newcommand\mysection[2][]{\metasetkeys{#1}\sref@target\color{\sec@color}
% \section[\sec@short]{#2}\sref@label@id{\my@section@label\thesection}}
% \end{verbatim}
% \vspace*{-1.5em}
% \caption{A Sectioning Command with Configurable Label}\label{fig:mysection:configurable}
% \end{exfig}
% 
% \subsection{Inter-Document Cross-Referencing}\sreflabel{Section}{suinterdoc}
%
% |sref.sty| provides inter-document cross-referencing. The use case is simple: we want to
% have a document collection (e.g. a book with conference proceedings), but also want to
% be able format the individual documents separately, and still have meaningful
% cross-references. To show off the possibilities, let us assume that we have a book with
% two separate papers, which we put into separate directories |idc| and |scr| to minimize
% interference between the authors Jane Doe and John Bull. To achieve this, we would set
% up paper driver files |main.tex| like the one in Figure~\ref{fig:paper-driver} in the
% two directories. These use the \DescribeMacro{\makeextrefs}|\makeextrefs| macro, which
% causes the |sref| package to generate a {\emph{external references file}}
% |main.refs|. Note that the |\makeextrefs| macro reads the previous |main.refs| file so
% that forward-referencing is possible (in the pass after a reference was labeled).
% 
% \begin{exfig}
% \begin{verbatim}
% \documentclass{article}
% \usepackage[extrefs]{sref}
% \makeextrefs{idc}
% \inputrefs{scr}{../scr/main}
% \extrefstyle{scr}{\cite[\protect{\theextref}]{Doe09}}
% \title{Inter-Document Crossreferencing}
% \author{John Bull\\...}
% \begin{document}\maketitle\documentclass{omdoc}
\usepackage{amssymb}
\usepackage{alltt}
\usepackage{hyperref}
\usepackage{listings}
\usepackage[hyperref=auto,style=alphabetic]{biblatex}
\addbibresource{../../lib/bib/kwarcpubs.bib}
\addbibresource{../../lib/bib/extpubs.bib}
\addbibresource{../../lib/bib/kwarccrossrefs.bib}
\addbibresource{../../lib/bib/extcrossrefs.bib}

\def\omdoc{OMDoc}
\def\latexml{LaTeXML}
\defpath{backmods}{../background}
%% defining the author metadata
\WAperson[id=miko,
           affiliation=JUB,
           url=http://kwarc.info/kohlhase]
           {Michael Kohlhase}
\WAinstitution[id=JUB,
               url=http://jacobs-university.de,
               streetaddress={Campus Ring 1},
               townzip={28759 Bremen},
               countryshort=D,
               country=Germany,
               type=University,
               acronym=JACU,
               shortname=Jacobs Univ.]
               {Jacobs University Bremen}

\begin{document}
 
% metadata and title page
% \begin{DCmetadata}[maketitle]
%   \DCMcreators{miko}         
%   \DCMrights{Copyright (c) 2009 Michael Kohlhase}
%   \DCMtitle{An example of semantic Markup in {\sTeX}}
%   \DCMabstract{In this note we give an example of semantic markup in {\sTeX}: 
%     Continuous and differentiable functions are introduced using real numbers, sets and
%     functions as an assumed background.}
% \end{DCmetadata}

\inputref{intro}

\begin{omgroup}[id=sec.math]{Mathematical Content}
  \begin{omgroup}{Calculus}
    We present some standard mathematical definitions, here from calculus.
  \inputref{continuous}
  \inputref{differentiable}
\end{omgroup}

\begin{omgroup}[id=sec.math]{A Theory Graph for Elementary Algebra}
  Here we show an example for more advanced theory graph manipulations, in particular
  imports via morphisms.

\begin{module}[id=magma]
  \importmodule[load=\backmods{functions}]{functions}
  \symdef{magbase}{G}
  \symdef[name=magmaop]{magmaopOp}{\circ}
  \symdef{magmaop}[2]{\infix\magmaopOp{#1}{#2}}
  \begin{definition}[id=magma.def]
    A \defi{magma} is a structure $\tup{\magbase,\magmaopOp}$, such that $\magbase$ is
    closed under the operation $\fun\magmaopOp{\cart{\magbase,\magbase}}\magbase$.
 \end{definition}
\end{module}

\begin{module}[id=semigroup]
  \importmodule{magma}
  \begin{definition}[id=semigroup.def]
    A \trefi[magma]{magma} $\tup{\magbase,\magmaopOp}$, is called a \defi{semigroup}, iff
    $\magmaopOp$ is associative.
  \end{definition}
\end{module}

\begin{module}[id=monoid]
  \importmodule{semigroup}
  \symdef{monneut}{e}
  \symdef{noneut}[1]{#1^*}
  \begin{definition}[id=monoid.def]
    A \defi{monoid} is a structure $\tup{\magbase,\magmaopOp,\monneut}$, such that
    $\tup{\magbase,\magmaopOp}$ is a \trefi[semigroup]{semigroup} and $\monneut$ is a
    \defii{neutral}{element}, i.e. that $\magmaop{x}\monneut=x$ for all $\inset{x}\magbase$.
 \end{definition}

 \begin{definition}[id=noneut.def]
   In a monoid $\tup{\magbase,\magmaopOp,\monneut}$, we use denote the set
   $\setst{\inset{x}S}{x\ne\monneut}$ with $\noneut{S}$.
 \end{definition}
\end{module}

\begin{module}[id=group]
  \importmodule{monoid}
  \symdef{ginvOp}{i}
  \symdef{ginv}[1]{\prefix\ginvOp{#1}}
  \begin{definition}[id=group.def]
    A \defi{group} is a structure $\tup{\magbase,\magmaopOp,\monneut,\ginvOp}$, such that
    $\tup{\magbase,\magmaopOp,\monneut}$ is a \trefi[monoid]{monoid} and $\ginvOp$ acts as
    a \defi{inverse}, i.e. that $\magmaop{x}{\ginv{x}}=\monneut$ for all
    $\inset{x}\magbase$.
  \end{definition}
\end{module}

\begin{module}[id=cgroup]
\importmodule{group}
\begin{definition}[id=cgroup.def]
  We call a \trefi[group]{group} $\tup{\magbase,\magmaopOp,\monneut,\ginvOp}$ a
  \defii{commutative}{group}, iff $\magmaopOp$ is commutative.
\end{definition}
\end{module}

\begin{module}[id=ring]
\symdef{rbase}{R}
\symdef[name=rtimes]{rtimesOp}{\cdot}
\symdef{rtimes}[2]{\infix\rtimesOp{#1}{#2}}
\symdef{rone}{1}
\begin{sstructure}{mul}{monoid}
  \tassign{magbase}{rbase}
  \fassign{a,b}{\magmaop{a}b}{\rtimes{a}b}
  \tassign{monunit}{rone}
\end{sstructure}
\symdef[name=rplus]{rplusOp}{+}
\symdef{rplus}[2]{\infix\rplusOp{#1}{#2}}
\symdef{rzero}{0}
\symdef[name=rminus]{rminusOp}{-}
\symdef{rminus}[2]{\infix\rminusOp{#1}{#2}}
\begin{sstructure}{add}{cgroup}
  \fassign{a,b}{\magmaop{a}b}{\rplus{a}b}
  \tassign{monunit}{rzero}
  \tassign{cginvOp}{\rminus}
\end{sstructure}

\begin{definition}
  A \defi{ring} is a structure $\tup{\rbase,\rplusOp,\rzero,\rtimesOp,\rone,\rminusOp}$,
  such that $\tup{\noneut\rbase,\rtimesOp,\rone}$ is a monoid and
  $\tup{\rbase,\rplusOp,\rzero,\rminusOp}$ is a commutative group.
\end{definition}
\end{module}
\end{omgroup}
\end{omgroup}

\begin{omgroup}[id=concl]{Conclusion}
  In this note we have given an example of standard mathematical markup and shown how a a
  {\sTeX} collection can be set up for automation.
\end{omgroup}
\printbibliography
\end{document}

%%% Local Variables: 
%%% mode: LaTeX
%%% TeX-master: t
%%% End: 

% LocalWords:  miko Makefiles tex contfuncs modf sms pdflatex latexml Makefile
% LocalWords:  latexmlpost omdoc STEXDIR BUTFILES DIRS
\end{document}
% \end{verbatim}
% \vspace*{-2em}
% \caption{A document driver {\texttt{idc/main.tex}} for a paper}\label{fig:paper-driver}
% \end{exfig}
% 
% The external references file can be read by other documents; in
% Figure~\ref{fig:paper-driver}, we read the references file of Jane Doe's paper via the
% \DescribeMacro{\inputrefs}|\inputrefs| macro. This allows John Bull to use\footnote{Note
%   that the external references file is updated every time {\LaTeX} is run, so that
%   references may be off by one version.} references like
% \DescribeMacro{\extref}|\extref{scr}{foo}| to reference document fragments in Jane Doe's
% paper she has labeled with the {\emph{reference prefix}} |\sreflabel{foo}| (assuming that
% she has added |\makeextrefs{scr}| in the preamble of her paper). Note that just as the
% |\sref| macro |\extref| takes an optional first argument that allows to specify the link
% text. Here, John Bull uses the \DescribeMacro{\extrefstyle}|\extrefstyle| macro to
% specify how the external references are to be formatted, in this case he decided to use a
% {\LaTeX} citation. Generally, first argument of the |\extrefstyle| macro is the
% reference prefix which should be configured, and the second is the format, where the
% \DescribeMacro{\theextref}|\theextref| macro expands to the cross-reference. In this
% case, John chose to use a bib{\TeX} citation (he has an entry |Doe09| in his database)
% for the reference to the external paper.
% 
% As the content of the respective paper is input from a file |paper.tex| in the
% individual papers, we can re-use these in the book. To do this we set up a book driver
% file like the one in Figure~\ref{fig:book-driver}. This one does not use the |extrefs|
% option, so the references are written to the |.aux| file. Furthermore |\extref| is
% redefined to act like |\sref| disregarding the first required argument. Thus all
% references work like they should.
% \begin{exfig}
% \begin{verbatim}
% \documentclass{book}
% \usepackage{sref}
% \title{Cross-Referencing in {\LaTeX}}
% \author{Elder Reseacher}
% \begin{document}
% \maketitle
% \chapter{Semantic Crossreferencing (Jane Doe, ...)}
% \documentclass{omdoc}
\usepackage{amssymb}
\usepackage{alltt}
\usepackage{hyperref}
\usepackage{listings}
\usepackage[hyperref=auto,style=alphabetic]{biblatex}
\addbibresource{../../lib/bib/kwarcpubs.bib}
\addbibresource{../../lib/bib/extpubs.bib}
\addbibresource{../../lib/bib/kwarccrossrefs.bib}
\addbibresource{../../lib/bib/extcrossrefs.bib}

\def\omdoc{OMDoc}
\def\latexml{LaTeXML}
\defpath{backmods}{../background}
%% defining the author metadata
\WAperson[id=miko,
           affiliation=JUB,
           url=http://kwarc.info/kohlhase]
           {Michael Kohlhase}
\WAinstitution[id=JUB,
               url=http://jacobs-university.de,
               streetaddress={Campus Ring 1},
               townzip={28759 Bremen},
               countryshort=D,
               country=Germany,
               type=University,
               acronym=JACU,
               shortname=Jacobs Univ.]
               {Jacobs University Bremen}

\begin{document}
 
% metadata and title page
% \begin{DCmetadata}[maketitle]
%   \DCMcreators{miko}         
%   \DCMrights{Copyright (c) 2009 Michael Kohlhase}
%   \DCMtitle{An example of semantic Markup in {\sTeX}}
%   \DCMabstract{In this note we give an example of semantic markup in {\sTeX}: 
%     Continuous and differentiable functions are introduced using real numbers, sets and
%     functions as an assumed background.}
% \end{DCmetadata}

\inputref{intro}

\begin{omgroup}[id=sec.math]{Mathematical Content}
  \begin{omgroup}{Calculus}
    We present some standard mathematical definitions, here from calculus.
  \inputref{continuous}
  \inputref{differentiable}
\end{omgroup}

\begin{omgroup}[id=sec.math]{A Theory Graph for Elementary Algebra}
  Here we show an example for more advanced theory graph manipulations, in particular
  imports via morphisms.

\begin{module}[id=magma]
  \importmodule[load=\backmods{functions}]{functions}
  \symdef{magbase}{G}
  \symdef[name=magmaop]{magmaopOp}{\circ}
  \symdef{magmaop}[2]{\infix\magmaopOp{#1}{#2}}
  \begin{definition}[id=magma.def]
    A \defi{magma} is a structure $\tup{\magbase,\magmaopOp}$, such that $\magbase$ is
    closed under the operation $\fun\magmaopOp{\cart{\magbase,\magbase}}\magbase$.
 \end{definition}
\end{module}

\begin{module}[id=semigroup]
  \importmodule{magma}
  \begin{definition}[id=semigroup.def]
    A \trefi[magma]{magma} $\tup{\magbase,\magmaopOp}$, is called a \defi{semigroup}, iff
    $\magmaopOp$ is associative.
  \end{definition}
\end{module}

\begin{module}[id=monoid]
  \importmodule{semigroup}
  \symdef{monneut}{e}
  \symdef{noneut}[1]{#1^*}
  \begin{definition}[id=monoid.def]
    A \defi{monoid} is a structure $\tup{\magbase,\magmaopOp,\monneut}$, such that
    $\tup{\magbase,\magmaopOp}$ is a \trefi[semigroup]{semigroup} and $\monneut$ is a
    \defii{neutral}{element}, i.e. that $\magmaop{x}\monneut=x$ for all $\inset{x}\magbase$.
 \end{definition}

 \begin{definition}[id=noneut.def]
   In a monoid $\tup{\magbase,\magmaopOp,\monneut}$, we use denote the set
   $\setst{\inset{x}S}{x\ne\monneut}$ with $\noneut{S}$.
 \end{definition}
\end{module}

\begin{module}[id=group]
  \importmodule{monoid}
  \symdef{ginvOp}{i}
  \symdef{ginv}[1]{\prefix\ginvOp{#1}}
  \begin{definition}[id=group.def]
    A \defi{group} is a structure $\tup{\magbase,\magmaopOp,\monneut,\ginvOp}$, such that
    $\tup{\magbase,\magmaopOp,\monneut}$ is a \trefi[monoid]{monoid} and $\ginvOp$ acts as
    a \defi{inverse}, i.e. that $\magmaop{x}{\ginv{x}}=\monneut$ for all
    $\inset{x}\magbase$.
  \end{definition}
\end{module}

\begin{module}[id=cgroup]
\importmodule{group}
\begin{definition}[id=cgroup.def]
  We call a \trefi[group]{group} $\tup{\magbase,\magmaopOp,\monneut,\ginvOp}$ a
  \defii{commutative}{group}, iff $\magmaopOp$ is commutative.
\end{definition}
\end{module}

\begin{module}[id=ring]
\symdef{rbase}{R}
\symdef[name=rtimes]{rtimesOp}{\cdot}
\symdef{rtimes}[2]{\infix\rtimesOp{#1}{#2}}
\symdef{rone}{1}
\begin{sstructure}{mul}{monoid}
  \tassign{magbase}{rbase}
  \fassign{a,b}{\magmaop{a}b}{\rtimes{a}b}
  \tassign{monunit}{rone}
\end{sstructure}
\symdef[name=rplus]{rplusOp}{+}
\symdef{rplus}[2]{\infix\rplusOp{#1}{#2}}
\symdef{rzero}{0}
\symdef[name=rminus]{rminusOp}{-}
\symdef{rminus}[2]{\infix\rminusOp{#1}{#2}}
\begin{sstructure}{add}{cgroup}
  \fassign{a,b}{\magmaop{a}b}{\rplus{a}b}
  \tassign{monunit}{rzero}
  \tassign{cginvOp}{\rminus}
\end{sstructure}

\begin{definition}
  A \defi{ring} is a structure $\tup{\rbase,\rplusOp,\rzero,\rtimesOp,\rone,\rminusOp}$,
  such that $\tup{\noneut\rbase,\rtimesOp,\rone}$ is a monoid and
  $\tup{\rbase,\rplusOp,\rzero,\rminusOp}$ is a commutative group.
\end{definition}
\end{module}
\end{omgroup}
\end{omgroup}

\begin{omgroup}[id=concl]{Conclusion}
  In this note we have given an example of standard mathematical markup and shown how a a
  {\sTeX} collection can be set up for automation.
\end{omgroup}
\printbibliography
\end{document}

%%% Local Variables: 
%%% mode: LaTeX
%%% TeX-master: t
%%% End: 

% LocalWords:  miko Makefiles tex contfuncs modf sms pdflatex latexml Makefile
% LocalWords:  latexmlpost omdoc STEXDIR BUTFILES DIRS
\newpage
% ...
% \chapter{Inter-Document Crossreferencing (John Bull, ...)}
% \documentclass{omdoc}
\usepackage{amssymb}
\usepackage{alltt}
\usepackage{hyperref}
\usepackage{listings}
\usepackage[hyperref=auto,style=alphabetic]{biblatex}
\addbibresource{../../lib/bib/kwarcpubs.bib}
\addbibresource{../../lib/bib/extpubs.bib}
\addbibresource{../../lib/bib/kwarccrossrefs.bib}
\addbibresource{../../lib/bib/extcrossrefs.bib}

\def\omdoc{OMDoc}
\def\latexml{LaTeXML}
\defpath{backmods}{../background}
%% defining the author metadata
\WAperson[id=miko,
           affiliation=JUB,
           url=http://kwarc.info/kohlhase]
           {Michael Kohlhase}
\WAinstitution[id=JUB,
               url=http://jacobs-university.de,
               streetaddress={Campus Ring 1},
               townzip={28759 Bremen},
               countryshort=D,
               country=Germany,
               type=University,
               acronym=JACU,
               shortname=Jacobs Univ.]
               {Jacobs University Bremen}

\begin{document}
 
% metadata and title page
% \begin{DCmetadata}[maketitle]
%   \DCMcreators{miko}         
%   \DCMrights{Copyright (c) 2009 Michael Kohlhase}
%   \DCMtitle{An example of semantic Markup in {\sTeX}}
%   \DCMabstract{In this note we give an example of semantic markup in {\sTeX}: 
%     Continuous and differentiable functions are introduced using real numbers, sets and
%     functions as an assumed background.}
% \end{DCmetadata}

\inputref{intro}

\begin{omgroup}[id=sec.math]{Mathematical Content}
  \begin{omgroup}{Calculus}
    We present some standard mathematical definitions, here from calculus.
  \inputref{continuous}
  \inputref{differentiable}
\end{omgroup}

\begin{omgroup}[id=sec.math]{A Theory Graph for Elementary Algebra}
  Here we show an example for more advanced theory graph manipulations, in particular
  imports via morphisms.

\begin{module}[id=magma]
  \importmodule[load=\backmods{functions}]{functions}
  \symdef{magbase}{G}
  \symdef[name=magmaop]{magmaopOp}{\circ}
  \symdef{magmaop}[2]{\infix\magmaopOp{#1}{#2}}
  \begin{definition}[id=magma.def]
    A \defi{magma} is a structure $\tup{\magbase,\magmaopOp}$, such that $\magbase$ is
    closed under the operation $\fun\magmaopOp{\cart{\magbase,\magbase}}\magbase$.
 \end{definition}
\end{module}

\begin{module}[id=semigroup]
  \importmodule{magma}
  \begin{definition}[id=semigroup.def]
    A \trefi[magma]{magma} $\tup{\magbase,\magmaopOp}$, is called a \defi{semigroup}, iff
    $\magmaopOp$ is associative.
  \end{definition}
\end{module}

\begin{module}[id=monoid]
  \importmodule{semigroup}
  \symdef{monneut}{e}
  \symdef{noneut}[1]{#1^*}
  \begin{definition}[id=monoid.def]
    A \defi{monoid} is a structure $\tup{\magbase,\magmaopOp,\monneut}$, such that
    $\tup{\magbase,\magmaopOp}$ is a \trefi[semigroup]{semigroup} and $\monneut$ is a
    \defii{neutral}{element}, i.e. that $\magmaop{x}\monneut=x$ for all $\inset{x}\magbase$.
 \end{definition}

 \begin{definition}[id=noneut.def]
   In a monoid $\tup{\magbase,\magmaopOp,\monneut}$, we use denote the set
   $\setst{\inset{x}S}{x\ne\monneut}$ with $\noneut{S}$.
 \end{definition}
\end{module}

\begin{module}[id=group]
  \importmodule{monoid}
  \symdef{ginvOp}{i}
  \symdef{ginv}[1]{\prefix\ginvOp{#1}}
  \begin{definition}[id=group.def]
    A \defi{group} is a structure $\tup{\magbase,\magmaopOp,\monneut,\ginvOp}$, such that
    $\tup{\magbase,\magmaopOp,\monneut}$ is a \trefi[monoid]{monoid} and $\ginvOp$ acts as
    a \defi{inverse}, i.e. that $\magmaop{x}{\ginv{x}}=\monneut$ for all
    $\inset{x}\magbase$.
  \end{definition}
\end{module}

\begin{module}[id=cgroup]
\importmodule{group}
\begin{definition}[id=cgroup.def]
  We call a \trefi[group]{group} $\tup{\magbase,\magmaopOp,\monneut,\ginvOp}$ a
  \defii{commutative}{group}, iff $\magmaopOp$ is commutative.
\end{definition}
\end{module}

\begin{module}[id=ring]
\symdef{rbase}{R}
\symdef[name=rtimes]{rtimesOp}{\cdot}
\symdef{rtimes}[2]{\infix\rtimesOp{#1}{#2}}
\symdef{rone}{1}
\begin{sstructure}{mul}{monoid}
  \tassign{magbase}{rbase}
  \fassign{a,b}{\magmaop{a}b}{\rtimes{a}b}
  \tassign{monunit}{rone}
\end{sstructure}
\symdef[name=rplus]{rplusOp}{+}
\symdef{rplus}[2]{\infix\rplusOp{#1}{#2}}
\symdef{rzero}{0}
\symdef[name=rminus]{rminusOp}{-}
\symdef{rminus}[2]{\infix\rminusOp{#1}{#2}}
\begin{sstructure}{add}{cgroup}
  \fassign{a,b}{\magmaop{a}b}{\rplus{a}b}
  \tassign{monunit}{rzero}
  \tassign{cginvOp}{\rminus}
\end{sstructure}

\begin{definition}
  A \defi{ring} is a structure $\tup{\rbase,\rplusOp,\rzero,\rtimesOp,\rone,\rminusOp}$,
  such that $\tup{\noneut\rbase,\rtimesOp,\rone}$ is a monoid and
  $\tup{\rbase,\rplusOp,\rzero,\rminusOp}$ is a commutative group.
\end{definition}
\end{module}
\end{omgroup}
\end{omgroup}

\begin{omgroup}[id=concl]{Conclusion}
  In this note we have given an example of standard mathematical markup and shown how a a
  {\sTeX} collection can be set up for automation.
\end{omgroup}
\printbibliography
\end{document}

%%% Local Variables: 
%%% mode: LaTeX
%%% TeX-master: t
%%% End: 

% LocalWords:  miko Makefiles tex contfuncs modf sms pdflatex latexml Makefile
% LocalWords:  latexmlpost omdoc STEXDIR BUTFILES DIRS
\newpage
% \end{document}
% \end{verbatim}
% \vspace*{-2em}
% \caption{A document driver for the book assembling the papers}\label{fig:book-driver}
% \end{exfig}
% 
% This example has been carried through (without the separation of chapters in to
% subdirectories) in the files accompanying the source distribution of the |sref|
% package. They are used for testing the package.
% 
% \subsection{Semantic Versions of Commonly used Referencing Commands}\label{sec:user:common}
% 
% The |sref| package defines semantically referencable versions of commonly used {\LaTeX}
% environments and command sequences.\footnote{This section will be extended by need, so
% if you miss some semantic environment, please contact the package author, or (better)
% file an issue at~\cite{sTeX:online})}
% 
% The \DescribeEnv{sequation}|sequation| environment takes an optional key/value argument
% that allows to specify an identifier and unifies the behavior of the |equation| (if an
% |id| key is given) and |displaymath| (else) environments. So the markup
% \begin{exfig}
% \begin{verbatim}
% A semantic equation with id
% \begin{sequation}[id=foo]
%   e^{mc}=-1
% \end{sequation}
% and another one without id
% \begin{sequation}
%   e^{mc}=-1
% \end{sequation}
% now, we reference the first equation: \sref{foo}
% \end{verbatim}
% \vspace*{-1.5em}\hrule
% yields the result:\hrule
% A semantic equation with id
% \begin{sequation}[id=foo]
%   e^{mc}=-1
% \end{sequation}
% and another one without id
% \begin{sequation}
%   e^{mc}=-1 
% \end{sequation}
% now, we reference the first equation: \sref{foo}
% \caption{Semantic Equation}\label{fig:sequation}
% \end{exfig}
% 
% \subsection{Semantic Citations}\label{sec:user:citations}
%
% \withcite{Patashnik:b88}{bib\TeX} and \withcite{Lehmann:biblatex10}{bib\LaTeX} provide a
% semi-semantic way of referencing literature. If we look at the 
% \begin{withcitation}{LasSwi:rdf99}
%   current practice of citing from an RDF standpoint~\citeit which views links as
%   subject/predicate/object triples,
% \end{withcitation}
% then the treatment of the predicate and object are semantic, but the subject is hinted
% at by mere juxtaposition in the text. The |sref| package helps out here via the
% \DescribeMacro{\withcite} macro for short subjects (in the second argument) that are
% postfixed by the citation (key in the first argument). For instance the occurrence at
% the beginning of this paragraph was created by
% \begin{verbatim}
% \withcite{Patashnik:b88}{bib\TeX} 
% \end{verbatim}
% The general case is covered by the \DescribeEnv{withcitation}|withcitation| environment
% for long subjects. In the latter, the citation can be placed by the
% \DescribeMacro{\citeit}. For instance, the second sentence was marked up as 
% \begin{verbatim}
% If we look at the 
% \begin{withcitation}{LasSwi:rdf99}
%   current practice of citing from an RDF standpoint which views links as
%   subject/predicate/object triples,
% \end{withcitation}
% then the treatment of the predicate ...
% \end{verbatim}
% The advantage of this treatment is that the meaning of the reference is fully marked up
% and can be taken advantage of in the \omdoc transformation, from which RDF triples can
% then be harvested for a linked open data treatment. 
%
% \section{Limitations}\label{sec:limitations}
% 
% In this section we document known limitations. If you want to help alleviate them,
% please feel free to contact the package author. Some of them are currently discussed in
% the \sTeX TRAC~\cite{sTeX:online}. 
% \begin{compactenum}
% \item currently only the |\sref| macro has a fallback argument. The others |\srefs| and
%   |\srefl| and their external variants should also have them, but I am not clear what
%   the adequate invocation pattern would be.
% \end{compactenum}
% 
% \StopEventually{\newpage\PrintIndex\newpage\PrintChanges\printbibliography}
% \newpage
%
% \section{Implementation}\label{sec:impl}
%
% The |sref| package generates two files: the {\LaTeX} package (all the code between
% {\textsf{$\langle$*package$\rangle$}} and {\textsf{$\langle$/package$\rangle$}}) and the
% {\latexml} bindings (between {\textsf{$\langle$*ltxml$\rangle$ and
%     $\langle$/ltxml$\rangle$}}). We keep the corresponding code fragments together,
% since the documentation applies to both of them and to prevent them from getting out of
% sync.
%
% We first set up header information for the {\latexml} binding file. 
%    \begin{macrocode}
%<*ltxml>
package LaTeXML::Package::Pool;
use strict;
use LaTeXML::Package;
%</ltxml>
%    \end{macrocode}
% 
% \subsection{Package Options}\label{sec:impl:options}
%
% We declare some switches which will modify the behavior according to the package
% options. Generally, an option |xxx| will just set the appropriate switches to true
% (otherwise they stay false).\ednote{need an implementation for {\latexml}}
%
%    \begin{macrocode}
%<*package>
\DeclareOption{showmeta}{\PassOptionsToPackage{\CurrentOption}{metakeys}}
\newif\ifextrefs\extrefsfalse
\DeclareOption{extrefs}{\extrefstrue}
\ProcessOptions
%</package>
%<*ltxml>
DeclareOption('extrefs','');
%</ltxml>
%    \end{macrocode}
%
% Then we need to set up the packages by requiring the |metakeys| package~\ctancite{Kohlhase:metakeys}
% to be loaded (in the right version).
%
%    \begin{macrocode}
%<*package>
\RequirePackage{metakeys}
\RequirePackage{xspace}
\RequirePackage{etoolbox}
%</package>
%<*ltxml>  
RequirePackage('metakeys');
%</ltxml> 
%    \end{macrocode}
% 
% \subsection{Crossreferencing}\label{sec:impl:crossref}
% 
% The following user-level macros just use the |\sref@hlink| macros in various ways for
% internal referencing.\ednote{they need implementation in LaTeXML, the ones here only are
% stubs to make the error messages shut up.}
% 
% \begin{macro}{\sref}
%    \begin{macrocode}
%<*package>
\addmetakey{sref}{linktext}
\addmetakey{sref}{fallback}
\newrobustcmd\sref[2][]{\metasetkeys{sref}{#1}%
\@ifundefined{sref@part}%
{\sref@hlink[\sref@linktext]{#2}{\sref@fallback}}%
{\sref@hlink[\sref@linktext]{\sref@part @#2}{\sref@fallback}}\xspace}
%</package>
%<*ltxml>
sub withhash {'#';}
DefConstructor('\sref[]{}[]',
 "<omdoc:oref href='&withhash()#2'/>");
%</ltxml>
%    \end{macrocode}
% \end{macro}
%
% \begin{macro}{\srefs}
%    \begin{macrocode}
%<*package>
\newrobustcmd\srefs[3][]{%
\def\@test{#1}\ifx\@test\@empty\sref{#2} and \sref{#3}\else #1\fi}
%</package>
%<*ltxml>
DefConstructor('\srefs[]{}',
 "<omdoc:oref href='&withhash()#2'/>");
%</ltxml>
%    \end{macrocode}
% \end{macro}
%
% \begin{macro}{\srefl}
%    \begin{macrocode}
%<*package>
\newcommand\srefl[3][]{%
\def\@test{#1}\ifx\@test\@empty\sref{#2} to \sref{#3}\else #1\fi}
%</package>
%<*ltxml>
DefConstructor('\srefl[]{}',
 "<omdoc:oref href='&withhash()#2'/>");
%</ltxml>
%    \end{macrocode}
% \end{macro}
% \ednote{it is not clear what we want in the LaTeXML implementation of spageref}
% \begin{macro}{\spageref}
%    \begin{macrocode}
%<*package>
\newcommand\spageref[1]{%
\@ifundefined{sref@part}{\sref@pageref{#1}}{\sref@pageref{\sref@part @#1}}}
%</package>
%<*ltxml>
DefConstructor('\spageref{}',
 "<omdoc:oref href='&withhash()#1'/>");
%</ltxml>
%    \end{macrocode}
% \end{macro}
%
% \subsection{An API for Package Authors}\label{sec:impl:package-authors}
%
% We find out whether the |hyperref| package is loaded, since we may want to use it for
% cross-references, for which we set up some internal macros that gracefully degrade if
% |hyperref| is not loaded. 
% 
% \begin{macro}{\sref@*@ifh}
%    \begin{macrocode}
%<*package>
\newif\ifhref\hreffalse
\AtBeginDocument{\@ifpackageloaded{hyperref}{\hreftrue}{\hreffalse}}
\newrobustcmd\sref@href@ifh[2]{\ifhref\href{#1}{#2}\else#2\fi}
\newrobustcmd\sref@hlink@ifh[2]{\ifhref\hyperlink{#1}{#2}\else#2\fi}
\newrobustcmd\sref@target@ifh[2]{\ifhref\hypertarget{#1}{#2}\else#2\fi}
%    \end{macrocode}
% \end{macro}
% 
% Then we provide some macros for {\sTeX}-specific crossreferencing 
%
% \begin{macro}{\sref@target}
%   The next macro uses this and makes an target from the current |sref@id| declared by
%   a |id| key.
%    \begin{macrocode}
\def\sref@target%
{\ifx\sref@id\@empty\else%
\edef\@target{sref@\@ifundefined{sref@part}{}{\sref@part @}\sref@id @target}
\sref@target@ifh\@target{}\fi}
%    \end{macrocode}
% \end{macro}
%
% The next two macros are used for setting labels, it is mainly used for enabling forward
% references, to do this, it is written into \meta{jobname}|.aux| or \meta{jobname}|.refs|.
% 
% \begin{macro}{\@sref@def}
%   This macro stores the value of its last argument in a custom macro for reference.
%    \begin{macrocode}
\newcommand\@sref@def[3]{\expandafter\gdef\csname sref@#1@#2\endcsname{#3}}
%    \end{macrocode}
% \end{macro}
%
% The next step is to set up a file to which the references are written, this is normally
% the |.aux| file, but if the |extref| option is set, we have to use an |.ref| file. 
%    \begin{macrocode}
\ifextrefs\newwrite\refs@file\else\def\refs@file{\@auxout}\fi
%    \end{macrocode}
%
% \begin{macro}{\sref@def}
%    This macro writes an |\@sref@def| command to the current aux file and also executes it. 
%    \begin{macrocode}
\newcommand\sref@def[3]{%\@sref@def{#1}{#2}{#3}%
\protected@write\refs@file{}{\string\@sref@def{#1}{#2}{#3}}}
%    \end{macrocode}
% \end{macro}
%
% \begin{macro}{\srefaddidkey}
%   |\srefaddidkey[|\meta{keyval}|]{|\meta{group}|}| extends the metadata keys of the
%   group \meta{group} with an |id| key. In the optional key/value pairs in \meta{keyval}
%   the |prefix| key can be used to specify a prefix. Note that the |id| key defined by
%   |\srefaddidkey[|\meta{keyval}|]{|\meta{group}|}| not only defines |\sref@id|, which is
%   used for referencing by the |sref| package, but also |\|\meta{group}|@id|, which is
%   used for showing metadata via the |showmeta| option of the |metakeys| package.
%    \begin{macrocode}
\addmetakey{srefaddidkey}{prefix}
\newcommand\srefaddidkey[2][]{\metasetkeys{srefaddidkey}{#1}%
\@metakeys@ext@clear@keys{#2}{sref@id}{}% id cannot have a default
\metakeys@ext@clear@keys{#2}{id}{}%
\metakeys@ext@showkeys{#2}{id}%
\define@key{#2}{id}{\edef\sref@id{\srefaddidkey@prefix ##1}%
\expandafter\edef\csname #2@id\endcsname{\srefaddidkey@prefix ##1}}}
%</package>
%    \end{macrocode}
% \end{macro}
%
% \subsection{Inter-Document Crossreferencing}\label{sec:impl:interdoc}
%
% \begin{macro}{\makeextrefs}
%    \begin{macrocode}
%<*package>
\newcommand\makeextrefs[1]{\gdef\sref@part{#1}%
\makeatletter
\IfFileExists{\jobname.refs}{\input{\jobname.refs}}{}%
\immediate\openout\refs@file=\jobname.refs
\makeatother}
%</package>
%<ltxml>DefConstructor('\makeextrefs{}','');
%    \end{macrocode}
% \end{macro}
%
% \begin{macro}{\sref@label}
%   The |\sref@label| macro writes a label definition to the auxfile. 
%    \begin{macrocode}
%<*package>
\newcommand\sref@label[2]{%
\sref@def{\@ifundefined{sref@part}{}{\sref@part @}#2}{page}{\thepage}%
\sref@def{\@ifundefined{sref@part}{}{\sref@part @}#2}{label}{#1}}
%</package>
%    \end{macrocode}
% \end{macro}
%
% \begin{macro}{\sreflabel}
%   The |\sreflabel| macro is a semantic version of |\label|, it combines the categorization
%   given in the first argument with {\LaTeX}'s |\@currentlabel|.
%    \begin{macrocode}
%<*package>
\newcommand\sreflabel[2]{\sref@label{#1 \@currentlabel}{#2}}
%</package>
%    \end{macrocode}
% \end{macro}
%
% \begin{macro}{\sref@label@id}
%   The |\sref@label@id| writes a label definition for the current |\sref@id| if it is
%   defined.
%    \begin{macrocode}
%<*package>
\newcommand\sref@label@id[1]{\ifx\sref@id\@empty\else\sref@label{#1}{\sref@id}\fi}
%</package>
%    \end{macrocode}
% \end{macro}
%
% Finally we come to the user visible macro |\sref| which is used for
% referencing.\ednote{The {\latexml} does not take into account the optional argument
%   yet.}
%
% \begin{macro}{\sref@hlink}
%   |\sref@hlink[|\meta{alt}|]{|\meta{label}|}{|\meta{fallback}|}| creates an error
%   message if the target specified by \meta{label} is not defined (but uses
%   \meta{fallback} if provided), and otherwise generates a hyperlinked reference whose
%   link text is \meta{alt} (if the optional argument is given) and the label generated by
%   object specified by \meta{label} otherwise.
%    \begin{macrocode}
%<*package>
\newrobustcmd\sref@hlink[3][]{\edef\@linktext{#1}\edef\@fallback{#3}%
\@ifundefined{sref@#2@label}% 
{\ifx\@fallback\@empty% warn about undefined links and show a substitute
\protect\G@refundefinedtrue\@latex@warning{reference #2 undefined}%
\ifx\@linktext\@empty?#2?\else\@linktext\fi%
\else%
\protect\G@refundefinedtrue\@latex@warning{using fallback for undefined reference #2}%
\@fallback%
\fi}%
{\edef\@link{\@nameuse{sref@#2@label}}% retrieve the reference label
\ifx\@linktext\@empty\sref@hlink@ifh{sref@#2@target}{\@link}%
\else\sref@hlink@ifh{sref@#2@target}{\@linktext}\fi}}
%</package> 
%    \end{macrocode}
% \end{macro}
%
% \begin{macro}{\sref@page@label}
%   This macro styles a page reference.
%    \begin{macrocode}
%<*package>
\newcommand\sref@page@label[1]{p.~{#1}}
%</package> 
%    \end{macrocode}
% \end{macro}
%
% \begin{macro}{\sref@pageref}
%   The next macro creates an error message if the target is not defined, and otherwise
%   generates a page reference.
%    \begin{macrocode}
%<*package>
\newcommand\sref@pageref[1]{\@ifundefined{sref@#1@page}%
{\protect\G@refundefinedtrue\@latex@warning{reference #1 undefined}\sref@page@label{??}}%
{\sref@hlink@ifh{sref@#1@target}{\sref@page@label{\@nameuse{sref@#1@page}}}}}
%</package> 
%    \end{macrocode}
% \end{macro}
%
% \begin{macro}{\sref@href}
%   The next macro creates an error message if the target is not defined, and otherwise
%   generates a hyperlinked reference.
%    \begin{macrocode}
%<*package>
\newcommand\sref@href[3][]{\edef\@linktext{#1}%
\@ifundefined{sref@#2@label}%
{\protect\G@refundefinedtrue\@latex@warning{reference #2 undefined}??}%
{\@ifundefined{sref@#3@URI}%
{\protect\G@refundefinedtrue\@latex@warning{external refs of type  #3 undefined}??}%
{\edef\@uri{\@nameuse{sref@#3@URI}.pdf\#sref@#2@target}
\edef\@label{\ifx\@linktext\@empty\@nameuse{sref@#2@label}\else\@linktext\fi}
\sref@href@ifh\@uri\@label}}}
%</package> 
%    \end{macrocode}
% \end{macro}
%
% \begin{macro}{\extref}
%   The next macros use |\sref@href| with the respective prefix for external referencing
%   if external references are used as indicated by the |extrefs| option; otherwise it
%   disregards the first required macro and uses internal referencing.\ednote{This needs
%     to be implemented on the LaTeXML side.}
%    \begin{macrocode}
%<*package>
\ifextrefs
\newcommand\extref[3][]{\def\theextref{\sref@href[#1]{#2@#3}{#2}}%
\csname doextref@#2\endcsname}
\else
\newcommand\extref[3][]{\sref[#1]{#3}}
\fi
%</package>
%<*ltxml>
DefConstructor('\extref[]{}{}',
  "<omdoc:oref href='#2@#3'/>");
DefConstructor('\theextref','');
%</ltxml>
%    \end{macrocode}
% \end{macro}
%
% \begin{macro}{\extpageref}
%   The next macros use |\sref@pagref| with the respective prefix for external referencing
%   if external references are used as indicated by the |extrefs| option; otherwise it
%   disregards the first required macro and uses internal referencing.\ednote{This needs
%   to be implemented on the LaTeXML side.}
%    \begin{macrocode}
%<*package>
\ifextrefs
\newcommand\extpageref[3][]{\def\theextref{\sref@pageref{#2@#3}}%
\csname doextpageref@#2\endcsname}
\else
\newcommand\extpageref[3][]{\spageref{#3}}
\fi
%</package>
%<*ltxml>
DefConstructor('\extpageref[]{}{}',
  "<omdoc:oref href='#2@#3'/>");
DefConstructor('\theextref','');
%</ltxml>
%    \end{macrocode}
% \end{macro}
%
% \begin{macro}{\extrefstyle}
%   This user macro defines an internal macro that is used for internal styling; for
%   instance |\extrefstyle{foo}{\theextref in bar}| defines the macro |\doextref@foo|
%   which evaluates to \meta{the reference} |in bar|. This is used in the |\extref| macro.
%    \begin{macrocode}
%<*package>
\newcommand\extrefstyle[2]{\expandafter\gdef\csname doextref@#1\endcsname{#2}}
%</package>
%<*ltxml>
DefConstructor('\extrefstyle{}{}',"");
%</ltxml>
%    \end{macrocode}
% \end{macro}
%
% \begin{macro}{\extpagerefstyle}
%   This is analogous to |\extrefstyle|
%    \begin{macrocode}
%<*package>
\newcommand\extpagerefstyle[2]{\expandafter\gdef\csname doextpageref@#1\endcsname{#2}}
%</package>
%<*ltxml>
DefConstructor('\extrefstyle{}{}',"");
%</ltxml>
%    \end{macrocode}
% \end{macro}
%
% \begin{macro}{\inputrefs}
%   If the external references file exists, it is read (under the protection of
%   |\makeatother|) otherwise an error message is displayed.
%    \begin{macrocode}
%<*package>
\newcommand\inputrefs[2]{%
\@namedef{sref@#1@URI}{#2}%
\extrefstyle{#1}{\theextref}\extpagerefstyle{#1}{\theextref}%
\makeatletter%
\IfFileExists{#2.refs}{\message{Reading external references: #2.refs}\input{#2.refs}}
             {\PackageError{sref}{Reference file #2.refs does not exist}
                                 {Maybe you have to run LaTeX on #2.tex first}}
\makeatother}
%</package>
%<*ltxml>
DefConstructor('\inputrefs{}{}','');
%</ltxml>
%    \end{macrocode}
% \end{macro}
% 
% \subsection{Semantic Versions of Commonly used Referencing Commands}\label{sec:impl:common}
% 
% \begin{environment}{sequation}
%    \begin{macrocode}
%<*package>
\srefaddidkey{sequation}
\def\sref@sequation@heading{equation}
\newenvironment{sequation}[1][]{\metasetkeys{sequation}{#1}% 
\ifx\sref@id\@empty\begin{displaymath}\else% no id, using equation*
\begin{equation}\sref@target\sref@label@id{\sref@sequation@heading~(\theequation)}\fi}
{\ifx\sref@id\@empty\end{displaymath}\else\end{equation}\fi}
%</package>
%<*ltxml>
DefEnvironment('{sequation} OptionalKeyVals',
	       "<ltx:equation "
             .     "?&GetKeyVal(#1,'id')(xml:id='&GetKeyVal(#1,'id')' "
             .     "refnum='#refnum')(xml:id='#id')>"
	     .   "<ltx:Math mode='display'>"
	     .     "<ltx:XMath>#body</ltx:XMath>"
	     .   "</ltx:Math>"
	     . "</ltx:equation>",
	       mode=>'display_math',
	       properties=> sub { RefStepCounter('equation') },
	       locked=>1);
%</ltxml>
%    \end{macrocode}
% \end{environment}
%
% \begin{environment}{seqnarray}
%    \begin{macrocode}
%<*package>
\newenvironment{seqnarray}[1][]%
{\metasetkeys{sequation}{#1}\begin{eqnarray*}\sref@target%
\sref@label@id{\sref@sequation@heading~(\theequation)}}
{\end{eqnarray*}}
%</package>
%<*ltxml>
DefMacro('\seqnarray OptionalKeyVals','\begin{eqnarray*}');
DefMacro('\endseqnarray','\end{eqnarray*}');
%</ltxml>
%    \end{macrocode}
% \end{environment}
%
% \subsection{Semantic Citations}\label{sec:impl:citations}
%
% \begin{macro}{withcite}
%    \begin{macrocode}
%<*package>
\newcommand\withcite[2]{#2~\cite{#1}}
%</package>
%<*ltxml>
DefMacro('\withcite{}{}','\begin{withcitation}{#1}#2\citeit\end{withcitation}');
%</ltxml>
%    \end{macrocode}
% \end{macro}
% 
% \begin{environment}{withcitation}
%    \begin{macrocode}
%<*package>
\newenvironment{withcitation}[1]{\def\citeit{\cite{#1}}}{}
%</package>
%<*ltxml>
DefConstructor('\citeit',"<omdoc:citation/> ",
afterConstruct => sub {
  my ($document,$whatsit) = @_;
  # LibXML acrobatics, since we can't talk about the xml:id prior to construction's end
  # (and please do correct me if this is inaccurate)
  my $node = $document->getNode;
  my ($citenode) = $document->findnodes('preceding-sibling::omdoc:citation',$node);
 my ($phrase_parent) = $document->findnodes('ancestor::ltx:text[@xml:id]',$node);
  return unless (defined $phrase_parent) && (defined $citenode);
  my $id = $phrase_parent->getAttribute('xml:id');
  my $refs = $phrase_parent->getAttribute('citeit-refs');
  $phrase_parent->removeAttribute('citeit-refs');
  $citenode->setAttribute('for',$id);
  $citenode->setAttribute('refs',$refs);
});#$
DefEnvironment('{withcitation}{}',
   "<ltx:text citeit-refs='#1'>#body</ltx:text>");
%</ltxml>
%    \end{macrocode}
% \end{environment}
% 
% \subsection{Finale}\label{sec:common}
%
% Finally, we need to terminate the file with a success mark for perl.
%    \begin{macrocode}
%<*ltxml>
1;
%</ltxml>
%    \end{macrocode}
% \Finale
\endinput
% \iffalse
%%% Local Variables: 
%%% mode: doctex
%%% TeX-master: t
%%% End: 
% \fi

% LocalWords:  GPL structuresharing STR omdoc dtx stex CPERL LoadClass url dc sreflabel
% LocalWords:  RequirePackage RegisterNamespace namespace xsl DocType ltxml dtd filedate
% LocalWords:  ltx DefEnvironment beforeDigest AssignValue inPreamble getGullet slabel
% LocalWords:  afterDigest keyval omgroup DefKeyVal Semiverbatim KeyVal iffalse rangle
% LocalWords:  OptionalKeyVals DefParameterType IfBeginFollows skipSpaces CMP sucrossref
% LocalWords:  ifNext DefMacro needwrapper unlist DefConstructor omtext bgroup suinterdoc
% LocalWords:  useCMPItemizations RefStepItemCounter egroup beginItemize li di rangle
% LocalWords:  beforeDigestEnd useTheoryItemizations dt autoclose ul ol dl env xspace
% LocalWords:  afterOpen LastSeenCMP autoClose proto ToString textsuperscript withhash
% LocalWords:  addAttribute nlex nlcex omdocColorMacro args tok MergeFont qw oref XMath
% LocalWords:  TokenizeInternal toString isMath foreach maybeCloseElement id'd extpageref
% LocalWords:  autoOpen minipage footnotesize scriptsize numberIt whatsit href citenode
% LocalWords:  getAttribute setAttribute OMDoc RelaxNGSchema noindex xml lec hypertarget
% LocalWords:  Subsubsection useDefaultItemizations refundefinedtrue blockquote findnodes
% LocalWords:  DefCMPConstructor inlinequote idx idt ide idp emph  sref extrefs endinput
% LocalWords:  flushleft flushright DeclareOption hyperref LaTeX tex pdf hlink
% LocalWords:  sequation omd def texorpdfstring raisebox latexml texttt newpage
% LocalWords:  fileversion maketitle tableofcontents autoref hpyerref mysection
% LocalWords:  exfig vspacesrefaddidkey srefaddidkey scr vspace withcite LasSwi
% LocalWords:  makeextrefs inputrefs extref foo Patashnik withcite biblatex10 sref@pagref
% LocalWords:  extrefstyle theextref theextref displaymath hrule impl ednote sref@ifundef
% LocalWords:  printbibliography newif ifextrefs extrefsfalse extrefstrue srefs
% LocalWords:  Crossreferencing newcommand ifundefined ifx srefl xref ifh gdef
% LocalWords:  ifhref hreffalse ifpackageloaded hreftrue expandafter jobname doextpageref
% LocalWords:  jobname csname endcsname newwrite auxout makeatletter openout
% LocalWords:  makeatother auxfile nameuse doextref namedef newenvironment NeedsTeXFormat
% LocalWords:  theequation refnum refnum seqnarray eqnarray endseqnarray doctex
% LocalWords:  textsuperscript spageref spageref thepage pageref ctancite rdf99
% LocalWords:  cross-referencable showmeta showmeta metakeys metasetkeys textsf
% LocalWords:  addmetakey referencable langle textsf langle withcitation citeit
% LocalWords:  compactenum textbackslash supackage-authors extpagerefstyle
