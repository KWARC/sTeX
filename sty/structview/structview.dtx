% \iffalse meta-comment
% An Infrastructure for Structures and Views
% Copyright (c) 2019 Michael Kohlhase, all rights reserved
%               this file is released under the
%               LaTeX Project Public License (LPPL)
%
% The original of this file is in the public repository at 
% http://github.com/sLaTeX/sTeX/
% \fi
%  
% \iffalse
%<package>\NeedsTeXFormat{LaTeX2e}[1999/12/01]
%<package>\ProvidesPackage{structview}[2019/03/20 v0.9 Structures and Views]
%
%<*driver>
\documentclass{ltxdoc}
\usepackage[utf8]{inputenc}
\usepackage{stex-logo,structview}
\usepackage{url,array,float,textcomp}
\usepackage[show]{ed}
\usepackage[hyperref=auto,style=alphabetic]{biblatex}
\addbibresource{kwarcpubs.bib}
\addbibresource{extpubs.bib}
\addbibresource{kwarccrossrefs.bib}
\addbibresource{extcrossrefs.bib}
\usepackage{listings}
\usepackage{amsfonts}
\usepackage{ctangit}
\usepackage{xspace}
\usepackage{hyperref}
\usepackage{tikz}
\usetikzlibrary{mmt}
\makeindex
\floatstyle{boxed}
\newfloat{exfig}{thp}{lop}
\floatname{exfig}{Example}
\def\githubissue#1{\cite{sTeX:github:on},\hyperlink{https://github.com/sLaTeX/sTeX/issues/#1}{issue #1}}
\def\gray#1{\textcolor{gray}{#1}}
\def\map#1#2{#1\mapsto #2}
\begin{document}
\RecordChanges
\DocInput{structview.dtx}
\end{document} 
%</driver>
% \fi
% 
% \iffalse\CheckSum{926}\fi
%
% \changes{v0.9}{2015/11/05}{Moved out of the \texttt{modules} package}
% \changes{v0.9}{2016/06/11}{Deprecated \texttt{viewsketch}}
% 
% \GetFileInfo{modules.sty}
% \MakeShortVerb{\|}
%\def\scsys#1{{{\sc #1}}\index{#1@{\sc #1}}\xspace}
% \def\xml{\scsys{Xml}}
% \def\mathml{\scsys{MathML}}
% \def\omdoc{\scsys{OMDoc}}
% \def\openmath{\scsys{OpenMath}}
% \def\latexml{\scsys{LaTeXML}}
% \def\perl{\scsys{Perl}}
% \def\cmathml{Content-{\sc MathML}\index{Content {\sc MathML}}\index{MathML@{\sc MathML}!content}}
% \def\activemath{\scsys{ActiveMath}}
% \def\twin#1#2{\index{#1!#2}\index{#2!#1}}
% \def\twintoo#1#2{{#1 #2}\twin{#1}{#2}}
% \def\atwin#1#2#3{\index{#1!#2!#3}\index{#3!#2 (#1)}}
% \def\atwintoo#1#2#3{{#1 #2 #3}\atwin{#1}{#2}{#3}}
% \def\cT{\mathcal{T}}\def\cD{\mathcal{D}}
% \def\cn#1{\mathsf{#1}}
% \title{{\texttt{structview.sty}}: Structures and Views in {\stex}\thanks{Version {\fileversion} (last revised
%        {\filedate})}}
%    \author{Michael Kohlhase\\
%    FAU Erlangen-N\"urnberg\\
%    \url{http://kwarc.info/kohlhase}}
% \maketitle
%
% \begin{abstract}
%   The |structview| package is part of the {\stex} collection, a version of {\TeX/\LaTeX}
%   that allows to markup {\TeX/\LaTeX} documents semantically without leaving the
%   document format, essentially turning {\TeX/\LaTeX} into a document format for
%   mathematical knowledge management (MKM).
%
%   This package supplies infrastructure for \omdoc structures and views: complex semantic
%   relations between modules/theories.
% \end{abstract}
%
% \setcounter{tocdepth}{2}\tableofcontents\newpage
%
% \section{Introduction}\label{sec:intro}
% 
% Structures and views constitute ways of defining and relating theories in a theory graph
% that considerably extend the ``object-oriented inheritance'' constituted by the imports
% relation given by the \sTeX |module| package.
% 
% Structures are like imports, only that they allow to define new theories via inheritance
% with renaming.  Views relate pre-existing theories and model conceptual refinements,
% framing, and implementation relations, again via a mapping between the languages defined
% by the source and target theories; we call these mappings \textbf{theory morphisms}.
% 
% For details about theory morphisms we refer to~\cite{RabKoh:WSMSML13}, but hope to make
% the underlying concepts clear with examples.
%
%\begin{figure}[ht]\centering
% \begin{tikzpicture}[scale=.9]\footnotesize
% \node[thy] (Mon) at (0,1.2) {$\mmtthy{monoid}{\gray{\cn{op},\;\cn{unit}}}{}$};
% \node[thy] (CGr) at (0,3) {$\mmtthy{cgp}{\gray{\cn{mon},\;\cn{inv}}}{}$};
% \node[thy] (Ring) at (-3.5,3) {$\mmtthy{ring}{\gray{\cn{add}},\;\gray{\cn{mul}}}{}$};
% \draw[struct](Mon) -- node[left,near end]{$\cn{cgp}?\cn{mon}$} (CGr);
% \draw[struct](CGr) -- node[below]{$\cn{ring}?\cn{add}$}(Ring);
% \draw[struct](Mon) -- node[left]{$\cn{ring}?\cn{mul}$} (Ring);
% \node[thy] (Int) at (3,3) {$\mmtthy{integers}{\gray{+,0,-,\cdot,1}}{}$};
% \node[draw] (v2) at (1.5,4.8) {
%  $\begin{array}{l}
%    v_2\colon \\
%    \gray{\left\{\begin{array}{c}
%                   \map{\cn{mon}/\cn{op}}{+} \\
%                   \map{\cn{mon}/\cn{unit}}{0}
%                 \end{array}
%     \right\}}
%     \;\mathrm{or}\; \gray{\map{\cn{mon}}{v_1}} \\
%    \gray{\map{\cn{inv}}{-}}
%   \end{array}$};
% \node[draw] at (2.5,1.4) {
%   $\begin{array}{l}
%     v_1\colon\\
%     \gray{\map{\cn{op}}{+}} \\
%     \gray{\map{\cn{unit}}{0}}
%    \end{array}$};
% \draw[view](Mon) -- (Int);
% \draw[view](CGr) -- node[below]{$v_2$} (Int);
% \draw[struct] (-4,4.5) -- +(1,0);
% \draw[view] (-4,5) -- +(1,0);
% \node[right] at (-3,4.5) {import};
% \node[right] at (-3,5) {view};
% \draw[view] (Ring) to[out=17,in=163] node[above,very near start] {$v_3$} (Int);
% \draw[view] (Mon) to[loop left] node[left,draw,solid,outer sep=3pt,thin]{$v_4\colon\map{x\,\cn{op}\,y}{y\,\cn{op}\,x}$}(Mon);
% \end{tikzpicture}
% \caption{A Theory Graph with Structures and Views}\label{fig:structview}
% \end{figure}
% \ednote{explain the contribution of structures and views to theory graphs and
% synchronize with Figure~\ref{fig:ring}.}
% 
% \section{The User Interface}
%
% The main contributions of the |modules| package are the |module| environment, which
% allows for lexical scoping of semantic macros with inheritance and the |\symdef| macro
% for declaration of semantic macros that underly the |module| scoping.
% 
% \subsection{Package Options}\label{sec:options}
%
% The \DescribeMacro{mh}|mh| option turns on MathHub support.
% 
% \subsection{Theory Morphisms}\label{sec:user:assignments}
% 
% A theory morphism is a mapping between the languages of its source and target
% theory. This can be described mathematically using all the structures in the \sTeX
% distribution. However, in many situations, the language transformation of a morphism can
% be given in form of \textbf{assignments} that map symbols of the source theory to
% expressions of the target theory.
% 
% There are three kinds assignments:\ednote{MK: we need better macros here.}
% \begin{description}
% \item[symbol assignments] via
%   \DescribeMacro{\vassign}|\vassign{|\meta{sym}|}{|\meta{exp}|}|, which maps a symbol
%   \meta{sym} from source theory an expression \meta{exp} in the target theory.
% \item[function assignments] via
%   \DescribeMacro{\fassign}|\fassign{|\meta{bvars}|}{|\meta{pat}|}{|\meta{exp}|}|, is a
%   variant which maps a function symbol \meta{sym} by mapping a pattern expression
%   \meta{pat} (\meta{sym} applied to \meta{bvars}) to an expression \meta{exp} in the
%   target theory on bound variables \meta{bvars}.
% \item[term assignments] via
%   \DescribeMacro{\tassign}|\tassign{|\meta{sym}|}{|\meta{tname}|}|, another special
%   case, where the value is the symbol with name \meta{tname} in the target theory.
% \end{description}
% Figure~\ref{fig:ring} shows a concrete example\ednote{adapt when we fully understand
% this, and the implementation works.}
% 
% The assignments above can be seen as abbreviations for a simple, formal definitions,
% which define a symbol of the source theory by an expression in the target theory.
% 
% \subsection{Structures}\label{sec:user:structures}
% 
% Structures are specified by the \DescribeEnv{structure}|sstructure|\footnote{The old
% |importmodulevia| environment is now deprecated.} environment:
% \begin{center}
%   |\begin{sstructure}[|\meta{keys}|]{|\meta{name}|}{|\meta{sthy}|}|\meta{morph}|\end{sstructure}|
% \end{center}
% gives the structure the name \meta{name}, specifies the ``source theory'' via its
% identifier \meta{sthy}, and the morphism \meta{morph}. The |structure| environment takes
% the same keys as the |\importmodule| macro, which it generalizes. The morphism
% \meta{morph} in the body of the |structure| environment specifies the morphism
% (see~\ref{sec:user:assignments} above). In a structure, we take the target theory to be
% the current theory.
%
%\begin{exfig}
% \begin{verbatim}
% \begin{module}[id=ring]
% \symdef{rbase}{R}
% \symdef{rtimes}[2]{\infix\cdot{#1}{#2}}
% \symdef{rone}{1}
% \begin{sstructure}{mul}{monoid}
%   \tassign{magbase}{rbase}
%   \fassign{a,b}{\magmaop{a}b}{\rtimes{a}b}
%   \tassign{monunit}{rone}
% \end{sstructure}
% \symdef{rplus}[2]{\infix+{#1}{#2}}
% \symdef{rminus}[1]{\infix-{#1}{#2}}
% \begin{sstructure}{add}{cgroup}
%   \fassign{a,b}{\magmaop{a}b}{\rplus{a}b}
%   \tassign{monunit}{rzero}
%   \tassign{cginvOp}{\rminus}
% \end{sstructure}
% ...
% \end{module}
% \end{verbatim}
% \caption{A Module for Rings with inheritance from monoids and commutative groups}\label{fig:ring}
% \end{exfig}
%
% \subsection{Views}\label{sec:user:views}
% 
% A view is a mapping between modules, such that all model assumptions (axioms) of the
% source module are satisfied in the target module. For marking up views the |structview|
% package supplies the \DescribeEnv{view}|view| environment; see Figure~\ref{fig:view} for
% the \stex markup of view $v_1$ from Figure~\ref{fig:structview}. The |view| environment
% takes one optional key/value argument followed by two mandatory ones: the names of the
% source and target modules. The |view| environment takes the following keys: |id| for a
% name, |title| and |display| for visual presentation, |loadfrom|, |loadto|, and
% |ext|\ednote{MK: we probably need toext and fromext here, but this never came up yet.}
% for specifying the source files that supply the source and target modules, |creators|,
% |contributors|, |srccite| for document metadata, and |type|\ednote{????}.
%
%\begin{exfig}
% \begin{verbatim}
% \begin{view}{monoid}{integers}
%   \vassign{magbase}{base}
%   \fassign{a,b}{\magmaop{a}b}{\inttimes{a,b}}
%   \tassign{monunit}{\intzero}
%   \begin{assertion}
%     The Integers with addition form a monoid in the obvious way.
%   \end{assertion}
% \end{view}
% \end{verbatim}
% \caption{A view from monoids to integers}\label{fig:view}
% \end{exfig}
%
% \section{Limitations \& Extensions}\label{sec:limitations}
% 
% In this section we will discuss limitations and possible extensions of the |modules|
% package. Any contributions and extension ideas are welcome; please discuss ideas,
% requests, fixes, etc on the {\sTeX} TRAC~\cite{sTeX:github:on}.
% 
% \section{The Implementation}\label{sec:impl} 
%
% \subsection{Package Options}\label{sec:impl:options}
% 
% We declare some switches which will modify the behavior according to the package
% options. Generally, an option |xxx| will just set the appropriate switches to true
% (otherwise they stay false). The options we are not using, we pass on to the |sref|
% package we require next.
%    \begin{macrocode}
%<*package>
\newif\if@structview@mh@\@structview@mh@false
\DeclareOption{mh}{\@structview@mh@true
\PassOptionsToPackage{\CurrentOption}{modules}}
\DeclareOption*{\PassOptionsToPackage{\CurrentOption}{modules}}
\ProcessOptions
%    \end{macrocode}
%
% The next measure is to ensure that the |sref| and |xcomment| packages are loaded (in the
% right version). For {\latexml}, we also initialize the package inclusions.
%    \begin{macrocode}
\if@structview@mh@\RequirePackage{structview-mh}\fi
\RequirePackage{modules}
%    \end{macrocode}
%
% \subsection{Theory Morphisms by Assignments}\label{sec:impl:assignments}
% 
% \begin{environment}{\*assign}
%   \ednote{probably get rid of the optional argument}
%    \begin{macrocode}
\newrobustcmd\vassign[3][]{\ifmod@show\ensuremath{#2\mapsto #3}, \fi}%
\newrobustcmd\fassign[4][]{\ifmod@show \ensuremath{#3(#2)\mapsto #4}, \fi}%
\newrobustcmd\tassign[3][]{\ifmod@show \ensuremath{#2\mapsto} #3, \fi}%
%    \end{macrocode}
% \end{environment}
% 
% \subsection{Structures}\label{sec:impl:structures}
% 
% \begin{environment}{sstructure} 
%   The |structure| environment just calls |\importmodule|, but to get around the
%   group, we first define a local macro |\@@doit|, which does that and can be called with
%   an |\aftergroup| to escape the environment grouping introduced by |structure|.
%    \begin{macrocode}
\newenvironment{sstructure}[3][]{%
  \gdef\@@doit{\importmodule[#1]{#3}}%
  \ifmod@show\par\noindent importing module #3 via \@@doit\fi%
}{%
  \aftergroup\@@doit\ifmod@show end import\fi%
}%
%    \end{macrocode}
% \end{environment}
% 
% \begin{environment}{importmodulevia}
%   This is now deprecated, we give an error, but punt to |structure|.
%    \begin{macrocode}
\newenvironment{importmodulevia}[2][]%
{\PackageError{structview}%
  {The {importmodulevia} environment is deprecated}{use the {sstructure} instead!}%
  \begin{sstructure}[#1]{missing}{#2}}
{\end{sstructure}}
%    \end{macrocode}
% \end{environment}
% 
% \subsection{Views}\label{sec:impl:views}
% 
% We first prepare the ground by defining the keys for the |view| environment.
%    \begin{macrocode}
\srefaddidkey{view}
\addmetakey*{view}{title}
\addmetakey{view}{display}
\addmetakey{view}{loadfrom}
\addmetakey{view}{loadto}
\addmetakey{view}{creators}
\addmetakey{view}{contributors}
\addmetakey{view}{srccite}
\addmetakey{view}{type}
\addmetakey[sms]{view}{ext}
%    \end{macrocode}
%
% \begin{macro}{\view@heading}
%   Then we make a convenience macro for the view heading. This can be customized. 
%    \begin{macrocode} 
\ifdef{\thesection}{\newcounter{view}[section]}{\newcounter{view}}
\newrobustcmd\view@heading[4]{%
  \if@importing%
  \else%
    \stepcounter{view}%
    \edef\@display{#3}\edef\@title{#4}%
    \noindent%
      \ifx\@display\st@flow%
      \else%
        {\textbf{View} {\thesection.\theview} from \textsf{#1} to \textsf{#2}}%
        \sref@label@id{View \thesection.\theview}%
        \ifx\@title\@empty%
          \quad%
        \else%
          \quad(\@title)%
        \fi%
        \par\noindent%
      \fi%
      \ignorespaces%
  \fi%
}%ifmod@show
%    \end{macrocode}
% \end{macro}
%
% \begin{environment}{view}
%   The |view| environment relies on the |@view| environment (used also in the \sTeX
%   module signatures) for module bookkeeping and adds presentation (a heading and a box)
%   if the |showmods| option is set.
%    \begin{macrocode}
\newenvironment{view}[3][]{% keys, from, to
  \metasetkeys{view}{#1}%
  \sref@target%
  \begin{@view}{#2}{#3}%
  \view@heading{#2}{#3}{\view@display}{\view@title}%
}{%
  \end{@view}%
  \ignorespaces%
}%
\ifmod@show\surroundwithmdframed{view}\fi%
%    \end{macrocode}
% \end{environment}
%
% \begin{environment}{@view}
%   The |@view| does the actual bookkeeping at the module level.
%    \begin{macrocode}
\newenvironment{@view}[2]{%from, to
  \@importmodule[\view@loadfrom]{#1}{\view@ext}%
  \@importmodule[\view@loadto]{#2}{\view@ext}%
}{}%
%    \end{macrocode}
% \end{environment}
% 
% \begin{environment}{viewsketch}
%   The |viewsketch| environment is deprecated, we give an error
%    \begin{macrocode}
\newenvironment{viewsketch}[3][]%
{\PackageError{structview}%
  {The {viewsketch} environment is deprecated}{use the {view} instead!}%
  \begin{view}[#1]{#2}{#3}}
{\end{view}}
%    \end{macrocode}
% \end{environment}
%
% \begin{macro}{\obligation}
%   The |\obligation| element does not do anything yet on the latexml
%   side.\ednote{document above}
%    \begin{macrocode}
\newrobustcmd\obligation[3][]{%
  \if@importing%
  \else Axiom #2 is proven by \sref{#3}%
  \fi%
}%
%</package>
%    \end{macrocode}
% \end{macro}
%
% \Finale
\endinput
%%% Local Variables: 
%%% mode: doctex
%%% TeX-master: t
%%% End: 

%  LocalWords:  iffalse NeedsTeXFormat structview.dtx texttt scsys sc sc xspace mathml cn
%  LocalWords:  omdoc latexml cmathml activemath twintoo atwin atwintoo mathcal mathcal
%  LocalWords:  structview.sty stex maketitle newpage setcounter tocdepth tableofcontents
%  LocalWords:  newpage symdef importmodulevia thyid ednote compactdesc vassign vassign
%  LocalWords:  sym exp fassign fassign bvars tassign tassign tname source-tname exfig
%  LocalWords:  monoids impl behavior sref newif xcomment initialize structview-mh gdef
%  LocalWords:  aftergroup newenvironment ifmod@show noindent newrobustcmd ensuremath ifx
%  LocalWords:  mapsto srefaddidkey addmetakey srccite customized newcounter stepcounter
%  LocalWords:  textbf thesection theview textsf textsf ignorespaces metasetkeys doctex
%  LocalWords:  surroundwithmdframed mathsf RabKoh centering tikzpicture mmtthy CGr cgp
%  LocalWords:  inv cdot mathrm synchronize sstructure sthy loadfrom loadto toext fromext
