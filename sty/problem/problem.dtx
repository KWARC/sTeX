% \iffalse meta-comment
% An Infrastructure for Problems 
% Copyright (c) 2006-2008 Michael Kohlhase, all rights reserved
%               this file is released under the
%               LaTeX Project Public License (LPPL)
% The original of this file is in the public repository at 
% http://github.com/KWARC/sTeX/
% \fi
% 
% \iffalse
%<package>\NeedsTeXFormat{LaTeX2e}[1999/12/01]
%<package>\ProvidesPackage{problem}[2013/12/12 v1.1 Semantic Markup for Problems]
%
%<*driver>
\documentclass{ltxdoc}
\usepackage{url,float,latexml,xspace}
\usepackage[solutions,hints,notes]{problem}
\usepackage[show]{ed}
\usepackage[hyperref=auto,style=alphabetic]{biblatex}
\addbibresource{kwarcpubs.bib}
\addbibresource{extpubs.bib}
\addbibresource{kwarccrossrefs.bib}
\addbibresource{extcrossrefs.bib}
\usepackage{stex-logo}
\usepackage{../ctangit}
\usepackage{hyperref}
\makeindex
\def\latexml{\hbox{{\LaTeX}ML}\xspace}
\floatstyle{boxed}
\newfloat{exfig}{thp}{lop}
\floatname{exfig}{Example}
\def\tracissue#1{\cite{sTeX:online}, \hyperlink{http://trac.kwarc.info/sTeX/ticket/#1}{issue #1}}
\begin{document}\DocInput{problem.dtx}\end{document}
%</driver>
% \fi
% 
% \iffalse\CheckSum{358}\fi
%
% \changes{v0.9}{2006/09/18}{First Version with Documentation}
% \changes{v0.9a}{2007/01/02}{Renamed to \texttt{problem.sty}}
% \changes{v0.9c}{2010/01/03}{based on \texttt{omd.sty} now}
% \changes{v1.0}{2013/10/09}{adding \texttt{\textbackslash start/stopsolution}}
% \changes{v1.1}{2013/12/12}{adding MathHub support}
% 
% \GetFileInfo{problem.sty}
% 
% \MakeShortVerb{\|}
%
% \title{\texttt{problem.sty}: An Infrastructure for formatting Problems\thanks{Version {\fileversion} (last revised
%        {\filedate})}}
%    \author{Michael Kohlhase\\
%            Jacobs University, Bremen\\
%            \url{http://kwarc.info/kohlhase}}
% \maketitle
%
% \begin{abstract}
%   The |problem| package supplies an infrastructure that allows specify problems and to
%   reuse them efficiently in multiple environments.
% \end{abstract}
% \setcounter{tocdepth}{2}\tableofcontents\newpage
%
%\section{Introduction}\label{sec:intro}
%
% The |problem| package supplies an infrastructure that allows specify problem.  Problems
% are text fragments that come with auxiliary functions: hints, notes, and
% solutions\footnote{for the moment multiple choice problems are not supported, but may
%   well be in a future version}. Furthermore, we can specify how long the solution to a
% given problem is estimated to take and how many points will be awarded for a perfect
% solution.
% 
% Finally, the |problem| package facilitates the management of problems in small files,
% so that problems can be re-used in multiple environment. 
% 
% \section{The User Interface}\label{sec:ui}
%
% \subsection{Package Options}
% The |problem| package takes the options \DescribeMacro{solutions}|solutions| (should
% solutions be output?), \DescribeMacro{notes}|notes| (should the problem notes be
% presented?), \DescribeMacro{hints}|hints| (do we give the hints?),
% \DescribeMacro{pts}|pts| (do we display the points awarded for solving the problem?),
% \DescribeMacro{min}|min| (do we display the estimated minutes for problem soling). If
% theses are specified, then the corresponding auxiliary parts of the problems are output,
% otherwise, they remain invisible.
% 
% The \DescribeMacro{boxed}|boxed| option specifies that problems should be formatted in
% framed boxes so that they are more visible in the text. Finally, the
% \DescribeMacro{test}|test| option signifies that we are in a test situation, so this
% option does not show the solutions (of course), but leaves space for the students to
% solve them.
% 
% Finally, if the \DescribeMacro{showmeta}|showmeta| is set, then the metadata keys are
% shown (see~\cite{Kohlhase:metakeys:ctan} for details and customization options).
% 
% \subsection{Problems and Solutions}\label{sec:user:probsols}
%
% \DescribeEnv{problem} The main environment provided by the |problem| package is
% (surprise surprise) the |problem| environment. It is used to mark up problems and
% exercises. The environment takes an optional KeyVal argument with the keys
% \DescribeMacro{id}|id| as an identifier that can be reference later,
% \DescribeMacro{pts}|pts| for the points to be gained from this exercise in homework or
% quiz situations, \DescribeMacro{min}|min| for the estimated minutes needed to solve the
% problem, and finally \DescribeMacro{title}|title| for an informative title of the
% problem. For an example of a marked up problem see Figure~\ref{fig:problem} and the
% resulting markup see Figure~\ref{fig:result}. 
%
%\begin{exfig}
% \begin{verbatim}
% \usepackage[solutions,hints,pts,min]{problem}
% \begin{document}
%   \begin{problem}[id=elefants,pts=10,min=2,title=Fitting Elefants]
%     How many Elefants can you fit into a Volkswagen beetle?
% \begin{hint}
%   Think positively, this is simple!
% \end{hint}
% \begin{exnote}
%   Justify your answer
% \end{exnote}
% \begin{solution}[for=elefants,height=3cm]
%   Four, two in the front seats, and two in the back.
% \end{solution}
%   \end{problem}
% \end{document}
% \end{verbatim}
% \caption{A marked up Problem}\label{fig:problem}
% \end{exfig}
%
% \DescribeEnv{solution} The |solution| environment can be to specify a solution to a
% problem. If the \DescribeMacro{solutions}|solutions| option is set or |\solutionstrue|
% is set in the text, then the solution will be presented in the output. The |solution|
% environment takes an optional KeyVal argument with the keys \DescribeMacro{id}|id| for
% an identifier that can be reference \DescribeMacro{for}|for| to specify which problem
% this is a solution for, and \DescribeMacro{height}|height| that allows to specify the
% amount of space to be left in test situations (i.e. if the \DescribeMacro{test}|test|
% option is set in the |\usepackage| statement).
%
%\begin{exfig}
% \begin{minipage}{.9\linewidth}
% \begin{problem}[id=elefants.prob,title=Fitting Elefants]
%   How many Elefants can you fit into a Volkswagen beetle?
% \begin{hint}
%   Think positively, this is simple!
% \end{hint}
% \begin{exnote}
%   Justify your answer
% \end{exnote}
% \smallskip\noindent\hrule\textbf{Solution:}
%   Four, two in the front seats, and two in the back.
% \hrule
% \end{problem}
% \end{minipage}
% \caption{The Formatted Problem from Figure~\ref{fig:problem}}\label{fig:result}
% \end{exfig}
% 
% \DescribeEnv{hint}\DescribeEnv{note}, the |hint| and |exnote| environments can be used
% in a |problem| environment to give hints and to make notes that elaborate certain aspects
% of the problem.
%
% \subsection{Starting and Stopping Solutions}
% 
% Sometimes we would like to locally override the |solutions| option we have given to the
% package. To turn on solutions we use the
% \DescribeMacro{\startsolutions}|\startsolutions|, to turn them off,
% \DescribeMacro{\stopsolutions}|\stopsolutions|. These two can be used at any point in
% the documents. 
%
% \subsection{Including Problems}\label{sec:user:includeproblem}
% 
% The \DescribeMacro{\includeproblem}|\includeproblem| macro can be used to include a
% problem from another file. It takes an optional KeyVal argument and a second argument
% which is a path to the file containing the problem (the macro assumes that there is only
% one problem in the include file). The keys \DescribeMacro{title}|title|,
% \DescribeMacro{min}|min|, and \DescribeMacro{pts}|pts| specify the problem title, the
% estimated minutes for solving the problem and the points to be gained, and their values
% (if given) overwrite the ones specified in the |problem| environment in the included
% file.
% 
% \subsection{Reporting Metadata}\label{sec:user:reporting}
% 
% The sum of the points and estimated minutes (that we specified in the |pts| and |min|
% keys to the |problem| environment or the |\includeproblem| macro) to the log file and
% the screen after each run. This is useful in preparing exams, where we want to make sure
% that the students can indeed solve the problems in an allotted time period.
% 
% The |\min| and |\pts| macros allow to specify (i.e. to print to the margin) the
% distribution of time and reward to parts of a problem, if the |pts| and |pts| package
% options are set. This allows to give students hints about the estimated time and the
% points to be awarded.
%
% \subsection{Support for \textsf{MathHub}}\label{sec:user:mathhub}
% 
% Much of the \sTeX content is hosed on \textsf{MathHub} (\url{http://MathHub.info}), a
% portal and archive for flexiformal mathematics. \textsf{MathHub} offers GIT repositories
% (public and private escrow) for mathematical documentation projects, online and offline
% authoring and document development infrastructure, and a rich, interactive reading
% interface. The |modules| package supports repository-sensitive operations on
% \textsf{MathHub}. 
% 
% Note that \textsf{MathHub} has two-level repository names of the form
% \meta{group}|/|\meta{repo}, where \meta{group} is a \textsf{MathHub}-unique repository
% group and \meta{repo} a repository name that is \meta{group}-unique. The file and
% directory structure of a repository is arbitrary -- except that it starts with the
% directory |source| because they are Math Archives in the sense
% of~\cite{HorIacJuc:cscpnrr11}. But this structure can be hidden from the \sTeX author
% with \textsf{MathHub}-enabled versions of the |modules| macros.
% 
% The \DescribeMacro{\includemhproblem}|\includemhproblem| macro is a variant of
% |\importmodule| with repository support. Instead of writing 
% \begin{verbatim}
% \defpath{MathHub}{/user/foo/lmh/MathHub}
% \includeproblem[pts=7]{\MathHub{fooMH/bar/source/baz/foobar}}
% \end{verbatim}
% we can simply write (assuming that |\MathHub| is defined as above)
% \begin{verbatim}
% \includemhproblem[fooMH/bar]{baz/foobar}
% \end{verbatim}
% Note that the |\importmhmodule| form is more semantic, which allows more advanced
% document management features in \textsf{MathHub}. 
% 
% If |baz/foobar| is the ``current module'', i.e. if we are on the \textsf{MathHub} path
% \ldots|MathHub/fooMH/bar|\ldots, then stating the repository in the first optional
% argument is redundant, so we can just use
% \begin{verbatim}
% \includemhproblem{baz/foobar}
% \end{verbatim}
% Of course, neither {\LaTeX} nor \latexml know about the repositories when they are
% called from a file system, so we can use the |\mhcurrentrepos| macro from the |modules|
% package to tell them. But this is only needed to initialize the infrastructure in the
% driver file. In particular, we do not need to set it in in each module, since the
% |\importmhmodule| macro sets the current repository automatically.
% 
% \paragraph{Caveat} if you want to use the \textsf{MathHub} support macros (let's call
% them mh-variants), then every time a module is imported or a document fragment is
% included from another repos, the mh-variant |\importmhmodule| must be used, so that the
% ``current repository'' is set accordingly. To be exact, we only need to use mh-variants,
% if the imported module or included document fragment use mh-variants.
%
% \section{Limitations}\label{sec:limitations}
% 
% In this section we document known limitations. If you want to help alleviate them,
% please feel free to contact the package author. Some of them are currently discussed in
% the \sTeX GitHub repository~\cite{sTeX:github:on}. 
% \begin{compactenum}
% \item none reported yet
% \end{compactenum}
% 
% \StopEventually{\newpage\PrintChanges}
% \newpage
%
% \section{The Implementation}\label{sec:implementation}
%
% The |problem| package generates two files: the {\LaTeX} package (all the code between
% {\textsf{$\langle$*package$\rangle$}} and {\textsf{$\langle$/package$\rangle$}}) and the
% {\LaTeXML} bindings (between {\textsf{$\langle$*ltxml$\rangle$ and
%     $\langle$/ltxml$\rangle$}}). We keep the corresponding code fragments together,
% since the documentation applies to both of them and to prevent them from getting out of
% sync.
% 
% First the general setup for \latexml
%
%    \begin{macrocode}
%<*ltxml>
# -*- CPERL -*-
package LaTeXML::Package::Pool;
use strict;
use LaTeXML::Package;
%</ltxml>
%    \begin{macrocode}
%
% \subsection{Package Options}\label{sec:impl:options}
% 
% The first step is to declare (a few) package options that handle whether certain
% information is printed or not. They all come with their own conditionals that are set by
% the options.
%
%    \begin{macrocode}
%<*package>
\DeclareOption{showmeta}{\PassOptionsToPackage{\CurrentOption}{metakeys}}
\newif\ifexnotes\exnotesfalse
\DeclareOption{notes}{\exnotestrue}
\newif\ifhints\hintsfalse
\DeclareOption{hints}{\hintstrue}
\newif\ifsolutions\solutionsfalse
\DeclareOption{solutions}{\solutionstrue}
\newif\ifpts\ptsfalse
\DeclareOption{pts}{\ptstrue}
\newif\ifmin\minfalse
\DeclareOption{min}{\mintrue}
\newif\ifboxed\boxedfalse
\DeclareOption{boxed}{\boxedtrue}
\DeclareOption*{\PassOptionsToPackage{\CurrentOption}{sref}}
\ProcessOptions
%</package>
%    \end{macrocode}
%
% On the {\LaTeXML} side we only make sure that the switches are defined. Since {\LaTeXML} currently does not process package options, we have
% nothing to do.
%    \begin{macrocode}
%<*ltxml>
RawTeX('
\newif\ifexnotes\exnotesfalse
\newif\ifhints\hintsfalse
\newif\ifsolutions\solutionsfalse
\newif\ifpts\ptsfalse
\newif\ifmin\minfalse
\newif\ifboxed\boxedfalse
');
DeclareOption('notes', '');
DeclareOption('hints', '');
DeclareOption('solutions', '');
DeclareOption('pts', '');
DeclareOption('min', '');
DeclareOption('boxed', '');
DeclareOption(undef,sub {PassOptions('sref','sty',ToString(Digest(T_CS('\CurrentOption')))); });  
ProcessOptions();
%</ltxml>
%    \end{macrocode}
% Then we make sure that the necessary packages are loaded (in the right versions).
%    \begin{macrocode}
%<*package>
\RequirePackage{comment}
\RequirePackage{sref}
\RequirePackage{mdframed}
%</package>
%<*ltxml>
RequirePackage('sref');
%</ltxml>
%    \end{macrocode}
%
% Then we register the namespace of the requirements ontology
%    \begin{macrocode}
%<*ltxml>
RegisterNamespace('prob'=>"http://omdoc.org/ontology/problems#");
RegisterDocumentNamespace('prob'=>"http://omdoc.org/ontology/problems#");
%</ltxml>
%    \end{macrocode}
%
% \subsection{Problems and Solutions}\label{sec:impl:probsols}
% 
% We now prepare the KeyVal support for problems. The key macros just set appropriate
% internal macros.
%
%    \begin{macrocode}
%<*package>
\srefaddidkey[prefix=prob.]{problem}
\addmetakey{problem}{pts}
\addmetakey{problem}{min}
\addmetakey*{problem}{title}
\addmetakey{problem}{refnum}
%    \end{macrocode}
%
% Then we set up a counter for problems
%
%    \begin{macrocode}
\newcounter{problem}[section]
%    \end{macrocode}
%
% \begin{macro}{\prob@number}
%   We consolidate the problem number into a reusable internal macro
% \begin{macrocode}
\def\prob@number{\ifx\inclprob@refnum\@empty%
\ifx\problem@refnum\@empty\thesection.\theproblem\else\problem@refnum\fi%
\inclprob@refnum\fi}
%    \end{macrocode}
% \end{macro}
% 
% \begin{macro}{\prob@title}
%   We consolidate the problem title into a reusable internal macro as well. |\prob@title|
%   takes three arguments the first is the fallback when no title is given at all, the
%   second and third go around the title, if one is given.
% \begin{macrocode}
\newcommand\prob@title[3]{%
\ifx\inclprob@title\@empty% if there is no outside title
\ifx\problem@title\@empty{#1}\else{#2\problem@title{#3}}\fi
\else{#2}\inclprob@title{#3}\fi}% else show the outside title
%    \end{macrocode}
% \end{macro}
% 
% With these the problem header is a one-liner
%
% \begin{macro}{\prob@heading}
%   We consolidate the problem header line into a separate internal macro that can be
%   reused in various settings.
% \begin{macrocode}
\def\prob@heading{Problem \prob@number\prob@title{ }{ (}{)\strut\\}%
\sref@label@id{Problem \prob@number}}
%    \end{macrocode}
% \end{macro}
% 
% With this in place, we can now define the |problem| environment. It comes in two shapes,
% depending on whether we are in boxed mode or not. In both cases we increment the problem
% number and output the points and minutes (depending) on whether the respective options
% are set.
% \begin{environment}{problem}
%    \begin{macrocode}
\newenvironment{problem}[1][]{\metasetkeys{problem}{#1}\sref@target%
\stepcounter{problem}\record@problem%
\def\currentsectionlevel{problem\xspace}%
\def\Currentsectionlevel{Problem\xspace}%
\par\noindent\textbf\prob@heading\show@pts\show@min\rm\noindent\ignorespaces}
{\smallskip}
\ifboxed\surroundwithmdframed{problem}\fi
%</package>
%    \end{macrocode}
% \end{environment}
% 
% Note that we allow hints and solutions in the body of a |problem| environment so we have
% to allow the |omdoc:CMP| and |ltx:p| elements to autoopen and autoclose. 
%
%    \begin{macrocode}
%<*ltxml>
DefEnvironment('{problem} OptionalKeyVals:problem',
 "<omdoc:exercise ?&GetKeyVal(#1,'id')(xml:id='&GetKeyVal(#1,'id')')()>"
.    "?&GetKeyVal(#1,'title')(<dc:title>&GetKeyVal(#1,'title')</dc:title>)()"
 .    "?&GetKeyVal(#1,'min')("
   .       "<omdoc:meta property='prob:solvedinminutes' prob:dummy='for the namespace'>"
 .          "&GetKeyVal(#1,'min')"
   .       "</omdoc:meta>)()"
 .    "?&GetKeyVal(#1,'pts')("
   .       "<omdoc:meta property='prob:points' prob:dummy='for the namespace'>"
 .         "&GetKeyVal(#1,'pts')"
   .       "</omdoc:meta>)()"
   .   "#body"
   ."</omdoc:exercise>",
   afterDigest => sub {
     my ($stomach,$kv)=@_;
     my $kvi = LookupValue('inclprob');
     my @keys = qw(id title min pts);
     my @vals = $kvi && map($kvi->getValue($_), @keys);
     foreach my $i(0..$#vals) {
       $kv->setValue($keys[$i],$vals[$i]) if $vals[$i];
     }
     return;});#$
%</ltxml>
%    \end{macrocode}
%
% \begin{macro}{\record@problem}
%   This macro records information about the problems in the |*.aux| file. 
%    \begin{macrocode}
%<*package>
\def\record@problem{\protected@write\@auxout{}%
{\string\@problem{\prob@number}%
{\ifx\inclprob@pts\@empty\problem@pts\else\inclprob@pts\fi}%
{\ifx\inclprob@min\@empty\problem@min\else\inclprob@min\fi}}}
%</package>
%    \end{macrocode}
% \end{macro}
%
% \begin{macro}{\@problem}
%   This macro acts on a problem's record in the |*.aux| file. It does not have any
%   functionality here, but can be redefined elsewhere (e.g. in the |assignment|
%   package). 
%    \begin{macrocode}
%<*package>
\def\@problem#1#2#3{}
%</package>
%    \end{macrocode}
% \end{macro}
% 
% The \DescribeEnv{solution}|solution| environment is similar to the |problem|
% environment, only that it is independent of the boxed mode. It also has it's own keys
% that we need to define first.
% 
%    \begin{macrocode}
%<*package>
\srefaddidkey{soln}
\addmetakey{soln}{for}
\addmetakey{soln}{height}
\addmetakey{soln}{creators}
\addmetakey{soln}{contributors}
\addmetakey{soln}{srccite}
%    \begin{macrocode}
% the next step is to define a helper macro that does what is needed to start a solution. 
%    \begin{macrocode}
\newcommand\@startsolution[1][]{\metasetkeys{soln}{#1}%
\ifboxed\else\hrule\fi\smallskip\noindent{\bf Solution: }\begin{small}%
\def\currentsectionlevel{solution\xspace}%
\def\Currentsectionlevel{Solution\xspace}%
\ignorespaces}
%    \end{macrocode}
%
% \begin{macro}{\startsolutions}
% for the |\startsolutions| macro we use the |\specialcomment| macro from the |comment|
% package. Note that we use the |\@startsolution| macro in the start codes, that parses
% the optional argument. 
%    \begin{macrocode}
\newcommand\startsolutions{\specialcomment{solution}{\@startsolution}%
{\ifboxed\else\hrule\medskip\fi\end{small}}%
\ifboxed\surroundwithmdframed{solution}\fi}
%</package>
%<*ltxml>
DefConstructor('\startsolutions','');
%</ltxml>
%    \end{macrocode}
% \end{macro}
% 
% \begin{macro}{\stopsolutions}
%    \begin{macrocode}
%<*package>
\newcommand\stopsolutions{\excludecomment{solution}}
%</package>
%<*ltxml>
DefConstructor('\stopsolutions','');
%</ltxml>
%    \end{macrocode}
% \end{macro}
% 
% so it only remains to start/stop solutions depending on what option was specified.
%
%    \begin{macrocode}
%<*package>
\ifsolutions\startsolutions\else\stopsolutions\fi
%</package>
%    \end{macrocode}
%
% the LaTeXML binding for the |solutions| is straightforward.
%
%    \begin{macrocode}
%<*ltxml>
DefKeyVal('soln','id','Semiverbatim');
DefKeyVal('soln','height','Semiverbatim');
DefKeyVal('soln','for','Semiverbatim');
DefKeyVal('soln','creators','Semiverbatim'); 
DefKeyVal('soln','contributors','Semiverbatim'); 
DefEnvironment('{solution} OptionalKeyVals:soln',
       "<omdoc:solution ?&GetKeyVals(#1,'for')(for='&GetKeyVal(#1,'for')')()>"
	     .   "#body"
	     . "</omdoc:solution>");
%</ltxml>
%    \end{macrocode}
%
%    \begin{macrocode}
%<*package>
\ifexnotes
\newenvironment{exnote}[1][]%
{\par\smallskip\hrule\smallskip\noindent\textbf{Note: }\small}
{\smallskip\hrule}
\else%ifexnotes
\excludecomment{exnote}
\fi%ifexnotes
\ifhints
\newenvironment{hint}[1][]%
{\par\smallskip\hrule\smallskip\noindent\textbf{Hint: }\small}
{\smallskip\hrule}
\newenvironment{exhint}[1][]%
{\par\smallskip\hrule\smallskip\noindent\textbf{Hint: }\small}
{\smallskip\hrule}
\else%ifhints
\excludecomment{hint}
\excludecomment{exhint}
\fi%ifhints
%</package>
%<*ltxml>
DefEnvironment('{exnote}',"<omdoc:hint>#body</omdoc:hint>");
DefEnvironment('{hint}',"<omdoc:hint>#body</omdoc:hint>");
DefConstructor('\pts{}',"");
DefConstructor('\min{}',"");
%</ltxml>
%    \end{macrocode}
% 
% \subsection{Including Problems}\label{sec:impl:includeproblem}
%
% \begin{macro}{\includeproblem}
%   The |\includeproblem| command is essentially a glorified |\input| statement, it sets
%   some internal macros first that overwrite the local points. Importantly, it resets the
%   |inclprob| keys after the input. 
%    \begin{macrocode}
%<*package>
\addmetakey{inclprob}{pts}
\addmetakey{inclprob}{min}
\addmetakey*{inclprob}{title}
\addmetakey{inclprob}{refnum}
\addmetakey{inclprob}{mhrepos}
\clear@inclprob@keys%initially
\newcommand\includeproblem[2][]{\metasetkeys{inclprob}{#1}%
\input{#2}\clear@inclprob@keys}
%</package>
%<*ltxml>
DefKeyVal('prob','pts','Semiverbatim');
DefKeyVal('prob','min','Semiverbatim');
DefKeyVal('prob','title','Semiverbatim'); 
DefKeyVal('prob','refnum','Semiverbatim'); 
DefConstructor('\includeproblem OptionalKeyVals:prob Semiverbatim',
   "<omdoc:exercise tref='#2'>"
.  "?&GetKeyVal(#1,'title')(<dc:title>&GetKeyVal(#1,'title')</dc:title>)()"
.  "?&GetKeyVal(#1,'min')("
 .     "<omdoc:meta property='prob:solvedinminutes' prob:dummy='for the namespace'>"
.       "&GetKeyVal(#1,'min')"
 .     "</omdoc:meta>)()"
.  "?&GetKeyVal(#1,'pts')("
 .     "<omdoc:meta property='prob:points' prob:dummy='for the namespace'>"
.       "&GetKeyVal(#1,'pts')"
 .     "</omdoc:meta>)()"
 ."</omdoc:exercise>",
 afterDigest => sub{
   my ($stomach,$kv) = @_;
   AssignValue('inclprob',$kv) if $kv;
 });
%</ltxml>
%    \end{macrocode}
% \end{macro}
%
%    \begin{macrocode}
%<*ltxml>
Tag('omdoc:exercise',afterOpen=>\&numberIt);
Tag('omdoc:solution',afterOpen=>\&numberIt);
Tag('omdoc:hint',afterOpen=>\&numberIt);
%</ltxml>
%    \end{macrocode}
%
% \subsection{Reporting Metadata}
%
%    \begin{macrocode}
%<*package>
\def\pts#1{\ifpts\marginpar{#1 pt}\fi}
\def\min#1{\ifmin\marginpar{#1 min}\fi}
%</package>
%<*ltxml>
%</ltxml>
%    \end{macrocode}
%
%    \begin{macrocode}
%<*package>
\AtEndDocument{\ifpts\message{Total: \arabic{pts} points}\fi
\ifmin\message{Total: \arabic{min} minutes}\fi}
%</package>
%<*ltxml>
%</ltxml>
%    \end{macrocode}
%
% \begin{macro}{\show@pts}
%   The |\show@pts| shows the points: if no points are given from the outside and also no
%   points are given locally do nothing, else show and add. If there are outside points
%   then we show them in the margin.
%    \begin{macrocode}
%<*package>
\newcounter{pts}
\def\show@pts{\ifx\inclprob@pts\@empty% 
\ifx\problem@pts\@empty\else%
\ifpts\marginpar{\problem@pts pt\smallskip}\addtocounter{pts}{\problem@pts}\fi%
\fi\else% inclprob@pts nonempty
\ifpts\marginpar{\inclprob@pts pt\smallskip}\addtocounter{pts}{\inclprob@pts}\fi%
\fi}
%    \end{macrocode}
% \end{macro}
% and now the same for the minutes
% \begin{macro}{\show@min}
%    \begin{macrocode}
\newcounter{min}
\def\show@min{\ifx\inclprob@min\@empty%
\ifx\problem@min\@empty\else%
\ifmin\marginpar{\problem@min min}\addtocounter{min}{\problem@min}\fi%
\fi\else%
\ifmin\marginpar{\inclprob@min min}\addtocounter{min}{\inclprob@min}\fi
\fi}
%</package>
%    \end{macrocode}
% \end{macro}
% 
% \subsection{Support for \textsf{MathHub}}\label{sec:user:mathhub}
% 
% \begin{macro}{\includemhproblem}
%   The |\includemhproblem| saves the current value of |\mh@currentrepos| in a local macro
%   |\mh@@repos|, resets |\mh@currentrepos| to the new value if one is given in the
%   optional argument, and after importing resets |\mh@currentrepos| to the old value in
%   |\mh@@repos|.
%    \begin{macrocode}
%<*package>
\newcommand\includemhproblem[2][]{\metasetkeys{inclprob}{#1}%
\edef\mh@@repos{\mh@currentrepos}%
\ifx\inclprob@mhrepos\@empty\else\mhcurrentrepos\inclprob@mhrepos\fi%
\input{\MathHub{\mh@currentrepos/source/#2}}%
\mhcurrentrepos\mh@@repos\clear@inclprob@keys}
%</package>
%<*ltxml>
sub includemhproblem {
  my ($gullet,$keyval,$arg2) = @_;
  my $repo_path;
  if ($keyval) {
    $repo_path = ToString(GetKeyVal($keyval,'mhrepos')); }
  if (! $repo_path) {
    $repo_path = ToString(Digest(T_CS('\mh@currentrepos'))); }
  else {
    $keyval->setValue('mhrepos',undef); }
  my $mathhub_base = ToString(Digest('\MathHub{}'));
  my $finalpath = $mathhub_base.$repo_path.'/source/'.ToString($arg2);
  return Invocation(T_CS('\includeproblem'), $keyval, T_OTHER($finalpath)); }#$
DefKeyVal('inclprob','mhrepos','Semiverbatim');
DefMacro('\includemhproblem OptionalKeyVals:inclprob {}', \&includemhproblem);
%</ltxml>
%    \end{macrocode}
% \end{macro}
%
% \subsection{Providing IDs Elements}\label{sec:impl:ids}
%
% To provide default identifiers, we tag all elements that allow |xml:id| attributes by
% executing the |numberIt| procedure from |omdoc.sty.ltxml|.
% 
%    \begin{macrocode}
%<*ltxml>
Tag('omdoc:exercise',afterOpen=>\&numberIt,afterClose=>\&locateIt);
Tag('omdoc:solution',afterOpen=>\&numberIt,afterClose=>\&locateIt);
Tag('omdoc:hint',afterOpen=>\&numberIt,afterClose=>\&locateIt);
%</ltxml>
%    \end{macrocode}
%
% \subsection{Finale}
% Finally, we need to terminate the file with a success mark for perl.
%    \begin{macrocode}
%<ltxml>1;
%    \end{macrocode}
% \Finale
\endinput
% \iffalse
% LocalWords:  GPL structuresharing STR dtx pts keyval xcomment CPERL DefKeyVal iffalse
%%% Local Variables: 
%%% mode: doctex
%%% TeX-master: t
%%% End: 
% \fi
% LocalWords:  RequirePackage Semiverbatim DefEnvironment OptionalKeyVals soln texttt baz
% LocalWords:  exnote DefConstructor inclprob NeedsTeXFormat omd.sty textbackslash exfig
%  LocalWords:  stopsolution fileversion filedate maketitle setcounter tocdepth newpage
%  LocalWords:  tableofcontents showmeta showmeta solutionstrue usepackage minipage hrule
%  LocalWords:  linewidth elefants.prob Elefants smallskip noindent textbf startsolutions
%  LocalWords:  startsolutions stopsolutions stopsolutions includeproblem includeproblem
%  LocalWords:  textsf HorIacJuc cscpnrr11 includemhproblem includemhproblem importmodule
%  LocalWords:  importmhmodule foobar ldots latexml mhcurrentrepos mh-variants mh-variant
%  LocalWords:  compactenum langle rangle langle rangle ltxml metakeys newif ifexnotes rm
%  LocalWords:  exnotesfalse exnotestrue ifhints hintsfalse hintstrue ifsolutions ifpts
%  LocalWords:  solutionsfalse ptsfalse ptstrue ifmin minfalse mintrue ifboxed boxedfalse
%  LocalWords:  boxedtrue sref mdframed marginpar prob srefaddidkey addmetakey refnum kv
%  LocalWords:  newcounter ifx thesection theproblem hfill newenvironment metasetkeys ltx
%  LocalWords:  stepcounter currentsectionlevel xspace ignorespaces surroundwithmdframed
%  LocalWords:  omdoc autoopen autoclose solvedinminutes kvi qw vals newcommand exhint
%  LocalWords:  specialcomment excludecomment mhrepos xref marginpar addtocounter doctex
%  LocalWords:  mh@currentrepos endinput

