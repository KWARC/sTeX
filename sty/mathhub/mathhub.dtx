% \iffalse meta-comment
% MathHub support for sTeX
% Copyright (c) 2015 Michael Kohlhase, all rights reserved
%               this file is released under the
%               LaTeX Project Public License (LPPL)
%
% The original of this file is in the public repository at 
% http://github.com/KWARC/sTeX/
% \fi
%   
% \iffalse
%<*package>
\NeedsTeXFormat{LaTeX2e}[1999/12/01]
\ProvidesPackage{mathhub}[2015/10/27 v1.2 Semantic Cross-References]
%</package>
%<*driver>
\documentclass{ltxdoc}
\usepackage{url,array,float}
\usepackage[show]{ed}
\usepackage[hyperref=auto,style=alphabetic]{biblatex}
\addbibresource{kwarcpubs.bib}
\addbibresource{extpubs.bib}
\addbibresource{kwarccrossrefs.bib}
\addbibresource{extcrossrefs.bib}
\usepackage{stex-logo,mathhub}
\usepackage{../ctangit}
\usepackage{hyperref}
\makeindex
\floatstyle{boxed}
\newfloat{exfig}{thp}{lop}
\floatname{exfig}{Example}
\newrobustcmd\tracissue[1]{\cite{sTeX:online}, \hyperlink{http://trac.kwarc.info/sTeX/ticket/#1}{issue #1}}
\begin{document}\DocInput{mathhub.dtx}\end{document}
%</driver>
% \fi
% 
%\iffalse\CheckSum{315}\fi
% 
% \GetFileInfo{mathhub.sty}
% 
% \MakeShortVerb{\|}
%
% \def\omdoc{OMDoc}
% \def\latexml{{\LaTeX}ML}
% \title{MathHub Support for \sTeX\thanks{Version {\fileversion} (last revised
%        {\filedate})}}
%    \author{Michael Kohlhase\\
%            Jacobs University, Bremen\\
%            \url{http://kwarc.info/kohlhase}}
% \maketitle
%
% \begin{abstract}
%   The |sref| package is part of the {\sTeX} collection, a version of {\TeX/\LaTeX} that
%   allows to markup {\TeX/\LaTeX} documents semantically without leaving the document
%   format, essentially turning {\TeX/\LaTeX} into a document format for mathematical
%   knowledge management (MKM).
%
%   The |mathhub| packages extend \sTeX with support for the MathHub.info portal 
% \end{abstract}
%
%\tableofcontents\newpage
% 
% \section{Introduction}\label{sec:intro}
% 
% Much of the \sTeX content is hosted on \textsf{MathHub} (\url{http://MathHub.info}), a
% portal and archive for flexiformal mathematics. \textsf{MathHub} offers GIT repositories
% (public and private escrow) for mathematical documentation projects, online and offline
% authoring and document development infrastructure, and a rich, interactive reading
% interface. The |modules| package supports repository-sensitive operations on
% \textsf{MathHub}.
% 
% Note that \textsf{MathHub} has two-level repository names of the form
% \meta{group}|/|\meta{repo}, where \meta{group} is a \textsf{MathHub}-unique repository
% group and \meta{repo} a repository name that is \meta{group}-unique. The file and
% directory structure of a repository is arbitrary -- except that it starts with the
% directory |source| because they are Math Archives in the sense
% of~\cite{HorIacJuc:cscpnrr11}. But this structure can be hidden from the \sTeX author
% with \textsf{MathHub}-enabled versions of the \sTeX macros, which are defined in this
% package.
%
% \paragraph{Caveat} if you want to use the \textsf{MathHub} support macros (let's call
% them \textbf{mh-variants}), then every time a module is imported or a document fragment
% is included from another repos, the mh-variant |\importmhmodule| must be used, so that
% the ``current repository'' is set accordingly. To be exact, we only need to use
% mh-variants, if the imported module or included document fragment use mh-variants.
% 
% \section{The User Interface}\label{sec:user}
% 
% \subsection{Package Options}\label{sec:user:options}
% 
% none so far
%
% \subsection{\texttt{mh-modules}: MH Variants for Modules}
%
% The \DescribeMacro{\importmhmodule}|importmhmodule| macro is a variant of
% |\importmodule| with repository support. Instead of writing 
% \begin{verbatim}
% \defpath{MathHub}{/user/foo/lmh/MathHub}
% \importmodule[load=\MathHub{fooMH/bar/source/baz/foobar}]{foobar}
% \end{verbatim}
% we can simply write (assuming that |\MathHub| is defined as above)
% \begin{verbatim}
% \importmhmodule[repos=fooMH/bar,path=baz/foobar]{foobar}
% \end{verbatim}
% Note that the |\importmhmodule| form is more semantic, which allows more advanced
% document management features in \textsf{MathHub}. 
% 
% If |baz/foobar| is the ``current module'', i.e. if we are on the \textsf{MathHub} path
% \ldots|MathHub/fooMH/bar|\ldots, then stating the repository in the first optional
% argument is redundant, so we can just use
% \begin{verbatim}
% \importmhmodule[path=baz/foobar]{foobar}
% \end{verbatim}
% if no file needs to loaded, |\importmhmodule| is the same as |\importmodule|. 
%
% Of course, neither {\LaTeX} nor \latexml know about the repositories when they are
% called from a file system, so we can use the
% \DescribeMacro{\mhcurrentrepos}|\mhcurrentrepos| macro to tell them. But this is only
% needed to initialize the infrastructure in the driver file. In particular, we do not
% need to set it in in each module, since the |\importmhmodule| macro sets the current
% repository automatically.
% 
% The \DescribeMacro{\usemhmodule}|\usemhmodule| and
% \DescribeMacro{\adoptmhmodule}|\adoptmhmodule| macros are the analogs to |\usemodule|
% and |\adoptmodule|.
% 
% For this, the |modules| package supplies the mh-variants
% \DescribeMacro{\mhinputref}|\mhinputref| and \DescribeMacro{\mhinput}|\mhinput| of the
% |\inputref| macro introduced above and normal {\LaTeX} |\input| macro.
%
% \subsection{ \textsf{mh-mikoslides}: Support  for MiKo Slides}\label{sec:user:mikoslides}
% 
% The \DescribeMacro{\mhframeimage}|\mhframeimage| macro is a variant of |\frameimage|
% with repository support. Instead of writing
% \begin{verbatim}
% \defpath{MathHub}{/user/foo/lmh/MathHub}
% \frameimage{\MathHub{fooMH/bar/source/baz/foobar}}
% \end{verbatim}
% we can simply write (assuming that |\MathHub| is defined as above)
% \begin{verbatim}
% \mhframeimage[fooMH/bar]{baz/foobar}
% \end{verbatim}
% Note that the |\mhframeimage| form is more semantic, which allows more advanced document
% management features in \textsf{MathHub}.
% 
% If |baz/foobar| is the ``current module'', i.e. if we are on the \textsf{MathHub} path
% \ldots|MathHub/fooMH/bar|\ldots, then stating the repository in the first optional
% argument is redundant, so we can just use
% \begin{verbatim}
% \mhframeimage{baz/foobar}
% \end{verbatim}
%
% \section{Limitations}\label{sec:limitations}
% 
% In this section we document known limitations. If you want to help alleviate them,
% please feel free to contact the package author. Some of them are currently discussed in
% the \sTeX GitHub repository~\cite{sTeX:github:on}. 
% \begin{compactenum}
% \item none reported yet.
% \end{compactenum}
% 
% \StopEventually{\newpage\PrintIndex\newpage\PrintChanges\printbibliography}
% \newpage
%
% \section{Implementation}\label{sec:impl}
%
% The |sref| package generates two files: the {\LaTeX} package (all the code between
% {\textsf{$\langle$*package$\rangle$}} and {\textsf{$\langle$/package$\rangle$}}) and the
% {\latexml} bindings (between {\textsf{$\langle$*ltxml$\rangle$ and
%     $\langle$/ltxml$\rangle$}}). We keep the corresponding code fragments together,
% since the documentation applies to both of them and to prevent them from getting out of
% sync.
%
% We first set up header information for the {\latexml} binding file.
%    \begin{macrocode}
%<*ltxml>
package LaTeXML::Package::Pool;
use strict;
use LaTeXML::Package;
%</ltxml>
%    \end{macrocode}
% 
% \subsection{Package Options}\label{sec:impl:options}
%
% We declare some switches which will modify the behavior according to the package
% options. Generally, an option |xxx| will just set the appropriate switches to true
% (otherwise they stay false).\ednote{do we need this?}
%
%    \begin{macrocode}
%<*package>
\DeclareOption*{\PassOptionsToPackage{\CurrentOption}{metakeys}}
\ProcessOptions
%</package>
%<*ltxml>
DeclareOption(undef,sub {PassOptions('metakeys','sty',ToString(Digest(T_CS('\CurrentOption')))); });
%</ltxml>
%    \end{macrocode}
%
% Then we need to set up the packages by requiring the |metakeys| package~\ctancite{Kohlhase:metakeys}
% to be loaded (in the right version).
%
%    \begin{macrocode}
%<*package>
\RequirePackage{keyval}
%</package>
%<*ltxml>  
RequirePackage('keyval');
%</ltxml> 
%    \end{macrocode}
% 
% \subsection{General Infrastructure}
%
% \begin{macro}{\mhcurrentrepos}
%   |\mhcurrentrepos| is used to initialize the current repository. If the repos has
%   changed, it writes a call to the internal macro
%   \DescribeMacro{\@mhcurrentrepos}|\@mhcurrentrepos| for the aux file and calls it. So
%   that the |\importmodule| calls there work with the correct repos.
%    \begin{macrocode}
%<*package>
\newrobustcmd\mhcurrentrepos[1]{%
  \edef\@test{#1}% 
  \ifx\@test\mh@currentrepos% if new dir = old dir
    \relax% no need to change
  \else%
    \protected@write\@auxout{}{\string\@mhcurrentrepos{#1}}%
  \fi%
  \@mhcurrentrepos{#1}% define mh@currentrepos
}%
\newrobustcmd\@mhcurrentrepos[1]{\edef\mh@currentrepos{#1}}%
%</package>
%<*ltxml>
DefMacro('\mhcurrentrepos{}','\@mhcurrentrepos{#1}');
DefMacro('\@mhcurrentrepos{}','\def\mh@currentrepos{#1}\@@mhcurrentrepos{#1}');
DefConstructor('\@@mhcurrentrepos{}','',
  afterDigest => sub{ AssignValue('current_repos',ToString($_[1]->getArg(1)),'global'); } );
%</ltxml>#$
%    \end{macrocode}
% \end{macro}
% \begin{macro}{\libinput}
%   the |\libinput| macro inputs from the |lib| directory of the MathHub repository or the
%   |meta-inf/lib| repos of the group.
%    \begin{macrocode}
%<ltxml>RaxTeX('
%<*package|ltxml>
\def\modules@@first#1/#2;{#1}
\newcommand\libinput[1]{\def\@libfile{\MathHub{\mh@currentrepos/lib/#1}}%
\IfFileExists{\@libfile}{\input\@libfile}%
{\edef\@@group{\expandafter\modules@@first\mh@currentrepos;}
\edef\@inffile{\MathHub{\@@group/meta-inf/lib/#1}}
\IfFileExists{\@inffile}{\input{\@inffile}}%
{\PackageError{modules}
  {Library file missing, cannot input #1\MessageBreak%
    Both \@libfile.tex\MessageBreak and \@inffile.tex\MessageBreak do not exit}%
  {Check whether the file name is correct}}}}
%</package|ltxml>
%<ltxml>');
%    \end{macrocode}
% \end{macro}
%
% \subsection{MH Variants for Modules}
% 
% \begin{macro}{\importmhmodule}
%   The |\importmhmodule[|\meta{key=value list}|]{module}| saves the current value of |\mh@currentrepos| in a local macro
%   |\mh@@repos|, resets |\mh@currentrepos| to the new value if one is given in the
%   optional argument, and after importing resets |\mh@currentrepos| to the old value in
%   |\mh@@repos|. We do all the |\ifx| comparison with an |\expandafter|, since the values
%   may be passed on from other key bindings. Parameters will be passed to |\importmodule|.
%    \begin{macrocode}
%<*modules>
\srefaddidkey{importmhmodule}%
\addmetakey{importmhmodule}{repos}% saves the repo's path. E.g: smglom/numberfield
\addmetakey{importmhmodule}{path}% saves the module name. E.g: naturalnumbers
\addmetakey[sms]{importmhmodule}{ext}% saves the extension: E.g: tex
\addmetakey[false]{importmhmodule}{conservative}[true]%
\newrobustcmd\importmhmodule[2][]{%
  \metasetkeys{importmhmodule}{#1}%
  \ifx\importmhmodule@path\@empty% if module name is not set
    \importmodule[ext=\importmhmodule@ext,id=\importmhmodule@id]{#2}%
  \else%
    \edef\mh@@repos{\mh@currentrepos}% remember so that we can reset it. 
    \ifx\importmhmodule@repos\@empty% if in the same repos
      \relax% no need to change mh@currentrepos, i.e, current dirctory.
    \else%
      \mhcurrentrepos{\importmhmodule@repos}% change it. 
    \fi%
    \importmodule[load=\MathHub{\mh@currentrepos/source/\importmhmodule@path},%
    ext=\importmhmodule@ext,id=\importmhmodule@id]{#2}%
    \mhcurrentrepos{\mh@@repos}% after importing, reset to old value
  \fi%
  \ignorespaces%
}%
%    \end{macrocode}
% \end{macro}
% and now the analogs
% \begin{macro}{\usemhmodule}
%    \begin{macrocode}
\newrobustcmd\usemhmodule[2][]{%
  \metasetkeys{importmhmodule}{#1}%
  \ifx\importmhmodule@path\@empty%
    \usemodule[ext=\importmhmodule@ext,id=\importmhmodule@id]{#2}%
  \else%
    \edef\mh@@repos{\mh@currentrepos}%
    \ifx\importmhmodule@repos\@empty%
    \else%
      \mhcurrentrepos{\importmhmodule@repos}%
    \fi%
    \usemodule[load=\MathHub{\mh@currentrepos/source/\importmhmodule@path},ext=\importmhmodule@ext,id=\importmhmodule@id]{#2}%
    \mhcurrentrepos\mh@@repos%
  \fi%
  \ignorespaces%
}%
%    \end{macrocode}
% \end{macro}
%
% \begin{macro}{\adoptmhmodule}
%    \begin{macrocode}
\newrobustcmd\adoptmhmodule[2][]{%
  \metasetkeys{importmhmodule}{#1}%
  \ifx\importmhmodule@path\@empty
    \adoptmodule[ext=\importmhmodule@ext,id=\importmhmodule@id]{#2}%
  \else%
    \edef\mh@@repos{\mh@currentrepos}%
    \ifx\importmhmodule@repos\@empty%
    \else%
      \mhcurrentrepos{\importmhmodule@repos}%
    \fi%
    \adoptmodule[load=\MathHub{\mh@currentrepos/source/\importmhmodule@path},ext=\importmhmodule@ext,id=\importmhmodule@id]{#2}%
    \mhcurrentrepos\mh@@repos%
  \fi%
  \ignorespaces%
}%
%    \end{macrocode}
% \end{macro}
%
% \begin{macro}{\mhinputref}
%    \begin{macrocode}
\newrobustcmd\mhinputref[2][]{%
  \def\@repos{#1}%
  \edef\mh@@repos{\mh@currentrepos}%
  \ifx\@repos\@empty%
  \else%
    \mhcurrentrepos{#1}%
  \fi%
  \inputref{\MathHub{\mh@currentrepos/source/#2}}%
  \mhcurrentrepos\mh@@repos%
  \ignorespaces%
}%
%    \end{macrocode}
% \end{macro}
%
% \begin{macro}{\mhinput}
%    \begin{macrocode}
\let\mhinput\mhinputref%
%    \end{macrocode}
% \end{macro}
%
% \begin{environment}{importmhmodulevia}
%    \begin{macrocode}
\newenvironment{importmhmodulevia}[3][]{%
  \gdef\@@doit{\importmhmodule[#1]{#2}{#3}}%
  \ifmod@show\par\noindent importing module #2 via \@@doit\fi
}{%
  \aftergroup\@@doit\ifmod@show end import\fi%
}%
%    \end{macrocode}
% \end{environment}
%
%    \begin{macrocode}
\srefaddidkey{mhview}
\addmetakey{mhview}{display}
\addmetakey{mhview}{creators}
\addmetakey{mhview}{contributors}
\addmetakey{mhview}{srccite}
\addmetakey*{mhview}{title}
\addmetakey{mhview}{fromrepos}
\addmetakey{mhview}{torepos}
\addmetakey{mhview}{frompath}
\addmetakey{mhview}{topath}
\addmetakey[sms]{mhview}{ext}
%    \end{macrocode}
%
% \begin{environment}{mhview}
%   the MathHub version
%    \begin{macrocode}
\newenvironment{mhview}[3][]{% keys, from, to
  \metasetkeys{mhview}{#1}%
  \sref@target%
  \begin{@mhview}{#2}{#3}%
  \view@heading{#2}{#3}{\mhview@display}{\mhview@title}%
}{%
  \end{@mhview}%
  \ignorespaces%
}%
\ifmod@show\surroundwithmdframed{mhview}\fi
%    \end{macrocode}
% \end{environment}
%
% \begin{environment}{@mhview}
%   The |@mhview| does the actual bookkeeping at the module level.
%    \begin{macrocode}
\newenvironment{@mhview}[2]{%from, to
  \importmhmodule[repos=\mhview@fromrepos,path=\mhview@frompath,ext=\mhview@ext]{#1}%
  \importmhmodule[repos=\mhview@torepos,path=\mhview@topath,ext=\mhview@ext]{#2}%
}{}%
%    \end{macrocode}
% \end{environment}
% 
% \begin{environment}{mhviewsketch}
%   The |mhviewsketch| environment behaves like |mhview|, but only has text contents.
%    \begin{macrocode}
\newenvironment{mhviewsketch}[3][]{%
  \metasetkeys{mhview}{#1}%
  \sref@target%
  \begin{@mhview}{#2}{#3}%
  \view@heading{#2}{#3}{\mhview@display}{\mhview@title}%
}{%
  \end{@mhview}%
  \ignorespaces%
}%
\ifmod@show\surroundwithmdframed{mhviewsketch}\fi
%</modules>
%    \end{macrocode}
% \end{environment}
% \ednote{MK: sort these into the rest.}
%    \begin{macrocode}
%<*modules.ltxml>
DefKeyVal('mhview','id','Semiverbatim');
DefKeyVal('mhview','fromrepos','Semiverbatim');
DefKeyVal('mhview','torepos','Semiverbatim');
DefKeyVal('mhview','frompath','Semiverbatim');
DefKeyVal('mhview','topath','Semiverbatim');
DefKeyVal('mhview','title','Semiverbatim');
DefKeyVal('mhview','creators','Semiverbatim');
DefKeyVal('mhview','contributors','Semiverbatim');
DefKeyVal('mhview','display','Semiverbatim');
DefKeyVal('mhview','ext','Semiverbatim');
DefMacroI(T_CS('\begin{mhview}'),'OptionalKeyVals:mhview {}{}', sub {
  my ($gullet, $keyvals, $from_arg, $to_arg) = @_;
  my $from = ToString(Digest($from_arg));
  my $to = ToString(Digest($to_arg));
  AssignValue(from_module => $from);
  AssignValue(to_module => $to);
  my $from_repos = ToString(GetKeyVal($keyvals,'fromrepos'));
  my $to_repos = ToString(GetKeyVal($keyvals,'torepos'));
  my $repos = LookupValue('current_repos');
  my $from_path = ToString(GetKeyVal($keyvals,'frompath'));
  my $to_path = ToString(GetKeyVal($keyvals,'topath'));
  my $ext = ToString(GetKeyVal($keyvals,'ext')) if $keyvals;
  $ext = 'sms' unless $ext;
  my $current_repos = LookupValue('current_repos');
  if (!$from_repos) { $from_repos = $current_repos; }
  if (!$to_repos) { $to_repos = $current_repos; }
  return (
    Tokenize("\\importMHmoduleI[repos=$from_repos,path=$from_path,ext=$ext]{$from}")->unlist,
    Tokenize("\\importMHmoduleI[repos=$to_repos,path=$to_path,ext=$ext]{$to}")->unlist,
    Invocation(T_CS('\begin{viewenv}'),$keyvals,$from_arg,$to_arg)->unlist
  );
}); 
DefMacroI('\end{mhview}',undef,'\end{viewenv}');

DefMacroI(T_CS('\begin{mhviewsketch}'),'OptionalKeyVals:mhview {}{}', sub {
  my ($gullet, $keyvals, $from_arg, $to_arg) = @_;
  my $from = ToString(Digest($from_arg));
  my $to = ToString(Digest($to_arg));
  my $from_repos = ToString(GetKeyVal($keyvals,'fromrepos'));
  my $to_repos = ToString(GetKeyVal($keyvals,'torepos'));
  my $repos = LookupValue('current_repos');
  my $from_path = ToString(GetKeyVal($keyvals,'frompath'));
  my $to_path = ToString(GetKeyVal($keyvals,'topath'));
  my $ext = ToString(GetKeyVal($keyvals,'ext')) if $keyvals;
  $ext = 'sms' unless $ext;
  my $current_repos = LookupValue('current_repos');
  if (!$from_repos) { $from_repos = $current_repos; }
  if (!$to_repos) { $to_repos = $current_repos; }
  return (
    Tokenize("\\importMHmoduleI[repos=$from_repos,path=$from_path,ext=$ext]{$from}")->unlist,
    Tokenize("\\importMHmoduleI[repos=$to_repos,path=$to_path,ext=$ext]{$to}")->unlist,
    Invocation(T_CS('\begin{viewsketchenv}'),$keyvals,$from_arg,$to_arg)->unlist
  );
}); 
DefMacroI('\end{mhviewsketch}',undef,'\end{viewsketchenv}');

DefConstructor('\importmhmodule OptionalKeyVals:importmhmodule {}',
	       "<omdoc:imports "
	       . "from='?&GetKeyVal(#1,'load')(&canonical_omdoc_path(&GetKeyVal(#1,'load')))()###2'"
               . "?&defined(&GetKeyVal(#1,'conservative'))(load='&GetKeyVal(#1,'conservative')')()/>",
   afterDigest => \&importMHmoduleI);

DefConstructor('\usemhmodule OptionalKeyVals:importmhmodule {}',
   "<omdoc:uses from='?&GetKeyVal(#1,'load')(&canonical_omdoc_path(&GetKeyVal(#1,'load')))()###2'/>",
   afterDigest => \&importMHmoduleI);

DefConstructor('\adoptmhmodule OptionalKeyVals:importmhmodule {}',
   "<omdoc:adopts from='?&GetKeyVal(#1,'load')(&canonical_omdoc_path(&GetKeyVal(#1,'load')))()###2'/>",
   afterDigest => \&importMHmoduleI);

RawTeX('
\newcommand\mhinputref[2][]{\def\@repos{#1}%
\edef\mh@@repos{\mh@currentrepos}%
\ifx\@repos\@empty\else\mhcurrentrepos{#1}\fi%
\inputref{\MathHub{\mh@currentrepos/source/#2}}%
\mhcurrentrepos\mh@@repos}
\newcommand\mhinput[2][]{\def\@repos{#1}%
\edef\mh@@repos{\mh@currentrepos}%
\ifx\@repos\@empty\else\mhcurrentrepos{#1}\fi%
\input{\MathHub{\mh@currentrepos/source/#2}}%
\mhcurrentrepos\mh@@repos}
\newenvironment{importmhmodulevia}[3][]{\def\@repos{#1}%
\edef\mh@@repos{\mh@currentrepos}%
\ifx\@repos\@empty\else\mhcurrentrepos{#1}\fi%
\gdef\@@doit{\importmhmodule[#1]{#2}{#3}}
\begin{importmoduleenv}[load=\MathHub{\mh@currentrepos/source/#2}]{#3}}
{\end{importmoduleenv}\aftergroup\@@doit}
');
%</modules.ltxml>
%    \end{macrocode}
% 
% \subsection{ \textsf{mh-mikoslides}: Support  for MiKo Slides}\label{sec:impl:mikoslides}
% 
% \subsection{Support for \textsf{MathHub}}\label{sec:user:mathhub}
% 
% \begin{macro}{\mhframeimage}
%   Use the current value of |\mh@currentrepos| or the value of the |mhrepos| key if it is
%   given in |\frameimage|.
%    \begin{macrocode}
%<mikoslides>\addmetakey{Gin}{mhrepos}
%<mikoslides.ltxml>DefKeyVal('Gin','mhrepos','Semiverbatim');
%<mikoslides.ltxml>RawTeX('
%<*mikoslides.ltxml|mikoslides>
\newcommand\mhframeimage[2][]{%
  \metasetkeys{Gin}{#1}%
  \edef\mh@@repos{\mh@currentrepos}%
  \ifx\Gin@mhrepos\@empty%
    \frameimage[#1]{\MathHub{\mh@currentrepos/source/#2}}%
  \else%
    \frameimage[#1]{\MathHub{\Gin@mhrepos/source/#2}}%
  \fi%
}%
%</mikoslides.ltxml|mikoslides>
%<mikoslides.ltxml>');
%    \end{macrocode}
% \end{macro}
%
% \subsection{Finale}\label{sec:common}
%
% Finally, we need to terminate the file with a success mark for perl.
%    \begin{macrocode}
%<*ltxml>
1;
%</ltxml>
%    \end{macrocode}
% \Finale
\endinput
% \iffalse
%%% Local Variables: 
%%% mode: doctex
%%% TeX-master: t
%%% End: 
% \fi

