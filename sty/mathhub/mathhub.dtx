% \iffalse meta-comment
% MathHub support for sTeX
% Copyright (c) 2019 Michael Kohlhase, all rights reserved
%               this file is released under the
%               LaTeX Project Public License (LPPL)
%
% The original of this file is in the public repository at 
% http://github.com/sLaTeX/sTeX/
% \fi
%   
% \iffalse
%<package>\NeedsTeXFormat{LaTeX2e}[1999/12/01]
%<*driver>
\documentclass{ltxdoc}
\usepackage[utf8]{inputenc}
\usepackage{url,array,float,amssymb}
\usepackage[show]{ed}
\usepackage[hyperref=auto,style=alphabetic]{biblatex}
\addbibresource{kwarcpubs.bib}
\addbibresource{extpubs.bib}
\addbibresource{kwarccrossrefs.bib}
\addbibresource{extcrossrefs.bib}
\usepackage{stex-logo,mathhub}
\usepackage{ctangit}
\usepackage{hyperref}
\makeindex
\floatstyle{boxed}
\newfloat{exfig}{thp}{lop}
\floatname{exfig}{Example}
\newcommand\githubissue[1]{\cite{sTeX:github:on}, \hyperlink{https://github.com/sLaTeX/sTeX/issues/#1}{issue #1}}
\begin{document}
\RecordChanges
\DocInput{mathhub.dtx}
\end{document}
%</driver>
% \fi
% 
%\iffalse\CheckSum{438}\fi
% 
% \GetFileInfo{mathhub.sty}
% \changes{v1.0}{2015/11/09}{moved all MH functionality into one DTX file}
% \changes{v1.0}{2016/06/11}{Deprecated \texttt{mhviewsketch}}
% \changes{v1.1}{2017/11/09}{Changed the semantics of \texttt{libinput} to input first the
% \texttt{meta-inf}-level and then repos-level file; this allows more sharing and does not
% break MathHub content (only one of them currently exists)}
% \changes{v1.2}{2020/10/01}{got rid of the dependency on |localpaths.tex|}
% 
% \MakeShortVerb{\|}
%
% \def\omdoc{OMDoc}
% \def\latexml{{\LaTeX}ML}
% \title{MathHub Support for \sTeX\thanks{Version {\fileversion} (last revised
% {\filedate})}}
% \author{Michael Kohlhase\\
%   FAU Erlangen-N\"urnberg\\
%   \url{http://kwarc.info/kohlhase}}
% \maketitle
%
% \begin{abstract}
%   The |mathhub| package collection is part of the \sTeX collection, a version of
%   {\TeX/\LaTeX} that allows to markup {\TeX/\LaTeX} documents semantically without
%   leaving the document format, essentially turning {\TeX/\LaTeX} into a document format
%   for mathematical knowledge management (MKM).
%
%   The |mathhub| packages extend \sTeX with support for \textsf{MathHub} file system
%   layout, which has co-eveolved with the \textsf{MathHub.info} portal for active
%   documents, but is useful for organizing collections of \sTeX documents in its own
%   right.
% \end{abstract}
%
%\tableofcontents\newpage
% 
% \section{Introduction}\label{sec:intro}
% 
% As \sTeX files tend to be highly interlinked semantically one of the most important
% practical problems to solve for managing larger collections is the management of
% (relative or absolute) paths. The |mathhub| package provides an infrastructure for
% supporting a regular $\leadsto$ manageable file system layout schema that has
% co-eveolved with the \textsf{MathHub.info} portal for active documents, but is useful
% for organizing collections of \sTeX documents in its own right. In particular, since the
% layout scheme is supported by the |lmh| (\underline{l}ocal
% \underline{m}ath\underline{h}ub~\cite{mmt:lmh:on}), and |make| (a build
% system~\cite{mmt:buildsys:on} for MathHub archives) in MMT~\cite{uniformal:on} which
% automates many management tasks.  For instance, after installing the |mmt.jar|, the
% shell command |mmt lmh install| \meta{group}|/|\meta{arch} installs the installs the
% MathHub archive \meta{group}|/|\meta{arch} together with all of its dependencies and
% |mmt make pdflatex| \meta{file} generates PDF for the file \meta{file} (and
% generates/updates all auxiliary files necessary along the way). 
% 
% \textsf{MathHub} (\url{http://MathHub.info}), is a portal and archive for flexiformal
% mathematics. It hosts much of the \sTeX content \textsf{MathHub} on GIT repositories
% (public and private escrow) for mathematical documentation projects. \textsf{MathHub}
% supports online and offline (via |lmh|) authoring and document development
% infrastructure, and a rich, interactive reading interface.
% 
% The \textbf{\textsf{MathHub} file system layout} has a \textbf{\textsf{MathHub} root
% folder} (e.g. |~/localmh/MathHub|) which conibetains all \sTeX sources, which are in turn
% organized in \textbf{MathHub archives}~\cite{HorIacJuc:cscpnrr11}.  These are organized
% in a two-level folder system that is compatible by GIT repository managers like
% GitHub~\cite{GitHub:on} and GitLab~\cite{GitLab:on}. Even though it is not necessary for
% the |mathhub| package we will assume that these are GIT repositories, which have names
% of the form \meta{group}|/|\meta{arch}, where \meta{group} is a \textsf{MathHub}-unique
% repository group and \meta{arch} a MathHub archive name that is \meta{group}-unique.
%
% The MathHub archives have a prescribed structure; see~\cite{HorIacJuc:cscpnrr11} for
% details. For our purposes, we only need two aspects:
% \begin{itemize}
% \item the \sTeX sources are all in a top-level subdirectory |source|,
% \item there is a top-level sub-directory |META-INF| with a manifest file
% |MANIFEST.MF| which consists of lines of the form \meta{key}|: |\meta{values}.
% \end{itemize}
% For the purposes of the |mathhub| package we assume that the |MANIFEST.MF| file has at
% least the |id| key specified and the the value is exactly
% \meta{group}|/|\meta{arch}. Furthermore, we assume that the |MATHHUB| environment
% variable is set with the system path to the \textsf{MathHub} root folder.
%
% With this information the mechanics of the MathHub archive structure can be hidden from
% the \sTeX author with \textsf{MathHub}-enabled versions of the \sTeX macros (let's call
% them \textbf{mh-variants}), which are defined in the |mh|-packages of the |mathhub|
% bundle, which we document in this manual. The mechanics of the |mathhub| bundle is as
% follows: For most \sTeX package \meta{pack}|.sty| there is a |mh|-variant
% \meta{pack}|-mh.sty|, and \meta{pack}|.sty| takes the option |mh|. When that is given
% (by calling |\usepackage[mh]{|\meta{pack}|}|), then \meta{pack}|.sty| inputs
% \meta{pack}|-mh.sty| from the |mathhub| bundle, which augments the \meta{pack} package
% with MathHub functionality.
% 
% \section{The User Interface}\label{sec:user}
%
% We now document mh-variants of the \sTeX packages that have MathHub functionality.
% 
% \subsection{\texttt{mathhub.sty}: General Infrastructure}\label{sec:impl:infra}
%
% For the generation of absolute file paths, the |mathhub| package keeps track of the
% current archive. If this ever needs to be set manually, it can be declared by the
% \DescribeMacro{\mhcurrentrepos}|\mhcurrentrepos| macro relative to the \textsf{MathHub}
% root path. |\mhcurrentrepos{group/repos}| declare that it resides at the path
% \url{/user/foo/localmh/MathHub/group/repos} given that the \textsf{MathHub} root
% path is |/user/foo/| |localmh/mathhub|.
%
% Given a systematic grouping in the \textsf{MathHub} file layout scheme, \sTeX files in
% the same repository (and often even in the same group) share much of the preamble
% material. Thus it makes sense to centralize that in external (shared) files and situate
% it at the group and repository levels: at the group level. Fort the group level, the
% \textsf{MathHub} file system layout uses a specical repository
% \meta{group}|/meta-inf/lib| and at the repository level we use
% \meta{group}|/|\meta{repos}|/lib| for such files. The
% \DescribeMacro{\libinput}|\libinput| macro supports this practice:
% |\libinput{|\meta{filename}|}| macro inputs the files
% \meta{group}|/meta-inf/lib/|\meta{filename} and then
% \meta{group}|/|\meta{repos}|/lib/|\meta{filename} if they exist. Thus a typical
% top-level \sTeX file has the following lines in the preamble:
% \begin{verbatim}
% \libinput{preamble}
% \end{verbatim}
% The \DescribeMacro{\libusepackage}|\libusepackage| is analogous it allows to share
% {\LaTeX} package between MathHub archives. 
%
% \subsection{\texttt{omdoc-mh.sty}: MH Document Infrastructure}\label{sec:user:modules}
%
% The \DescribeMacro{\addmhbibresource}|\addmhbibresource| macro is a variant of
% |\addbibresource| from {bib\LaTeX} with repository support. Concretely,
% |\addmhbibresource[|\meta{repos}|]{|\meta{path}|}| expands to
% |\addbibresource{|\meta{MathHub}|/|\meta{reponame}|/|\meta{path}|}|, where
% \meta{reponame} is \meta{repo} if that is non-empty and the current repository
% else. Note that in contrast to the other MH variants, this does not add the |/source/|
% into the path, since {bib\TeX} files are often put into the |lib| directory parallel to
% |source|.
%
% 
% \subsection{\texttt{modules-mh.sty}: MH Variants for Modules}\label{sec:user:modules}
%
% The \DescribeMacro{\importmhmodule}|\importmhmodule| macro is a variant of
% |\importmodule| with repository support. Instead of writing 
% \begin{verbatim}
% \importmodule[load=\MathHub{fooMH/bar/source/baz/foobar}]{foobar}
% \end{verbatim}
% we can simply write (assuming that |\MathHub| is defined as above)
% \begin{verbatim}
% \importmhmodule[mhrepos=fooMH/bar,path=baz/foobar]{foobar}
% \end{verbatim}
% Note that the |\importmhmodule| form is more semantic, which allows more advanced
% document management features in \textsf{MathHub}. 
% 
% If |baz/foobar| is the ``current module'', i.e. if we are on the \textsf{MathHub} path
% \ldots|MathHub/fooMH/bar|\ldots, then stating the repository in the first optional
% argument is redundant, so we can just use one of the following forms
% \begin{verbatim}
% \importmhmodule[path=baz/foobar]{foobar}
% \importmhmodule[dir=baz]{foobar}
% \end{verbatim}
% if no file needs to loaded, |\importmhmodule| is the same as |\importmodule|. 
%
% Of course, {\LaTeX} does not know about the repositories when they are called from a
% file system, so we can use the \DescribeMacro{\mhcurrentrepos}|\mhcurrentrepos| macro to
% tell them. But this is usually only needed to initialize the infrastructure in the
% driver file. In particular, we do not need to set it in in each module, since the
% |\importmhmodule| macro sets the current repository automatically.
% 
% The \DescribeMacro{\usemhmodule}|\usemhmodule| is the analog to |\usemodule|.
% 
% For this, the |modules| package supplies the mh-variants
% \DescribeMacro{\mhinputref}|\mhinputref| and \DescribeMacro{\mhinput}|\mhinput| of the
% |\inputref| macro introduced above and normal {\LaTeX} |\input| macro. 
%
% \paragraph{Caveat} if you want to use the \textsf{MathHub} support macros, then every
% time a module is imported or a document fragment is included from another repository, the
% mh-variant |\importmhmodule| must be used, so that the ``current repository'' is set
% accordingly. To be exact, we only need to use mh-variants, if the imported module or
% included document fragment use mh-variants.
% 
% \subsection{\texttt{omtext-mh.sty}: MH Variants for OMText}\label{sec:user:modules}
%
% The \DescribeMacro{\cmhgraphics}|\cmhgraphics| macro is a variant of |\mycgraphics| with
% repository support. Instead of writing
% \begin{verbatim}
% \mycgraphics{\MathHub{fooMH/bar/source/baz/foobar}}
% \end{verbatim}
% we can simply write (assuming that |\MathHub| is defined as above)
% \begin{verbatim}
% \cmhgraphics[fooMH/bar]{baz/foobar}
% \end{verbatim}
% Note that the |\cmhgraphics| form is more semantic, which allows more advanced document
% management features in \textsf{MathHub}.
% 
% \subsection{\texttt{smultiling-mh.sty}: MH Variants for Multilinguality}\label{sec:user:smultiling}
%
% The \DescribeMacro{mhmodsig}|mhmodsig| and \DescribeMacro{mhmodnl}|mhmodnl| environments
% are the MH variants of the |modsig| and |modnl| environments from the |smultiing|
% package. Just as in the other MH packages, |mhmodnl| takes additional |mhrepos| and |path| keys
% and combine them to |load| key of |modnl|. Instead of writing 
% \begin{verbatim}
% \begin{modnl}[load=\MathHub{fooMH/bar/source/baz/foobar}]{foobar}{en}
% \end{verbatim}
% we can simply write (assuming that |\MathHub| is defined as above)
% \begin{verbatim}
% \begin{modnl}[mhrepos=fooMH/bar,path=baz/foobar]{foobar}{en}
% \end{verbatim}
%
% |mhmodsig| is just a notational variant of |modsig| that allows to keep the sources
% uniform.
%
% 
% \subsection{\texttt{structview-mh.sty}: MH Variants for Structures and Views}\label{sec:user:structview}
% \ednote{needs to be documented}
%
% \subsection{\textsf{mikoslides-mh.sty}: Support  for MiKo Slides}\label{sec:user:mikoslides}
% 
% The \DescribeMacro{\mhframeimage}|\mhframeimage| macro is a variant of |\frameimage|
% with repository support. Instead of writing
% \begin{verbatim}
% \frameimage{\MathHub{fooMH/bar/source/baz/foobar}}
% \end{verbatim}
% we can simply write (assuming that |\MathHub| is defined as above)
% \begin{verbatim}
% \mhframeimage[fooMH/bar]{baz/foobar}
% \end{verbatim}
% Note that the |\mhframeimage| form is more semantic, which allows more advanced document
% management features in \textsf{MathHub}.
% 
% If |baz/foobar| is the ``current module'', i.e. if we are on the \textsf{MathHub} path
% \ldots|MathHub/fooMH/bar|\ldots, then stating the repository in the first optional
% argument is redundant, so we can just use
% \begin{verbatim}
% \mhframeimage{baz/foobar}
% \end{verbatim}
%
% If we want to transclude a the contents of a file as a note, we can use the
% \DescribeMacro{\mhinputref*}|\mhinputref*| macro: |\mhinputref*[foo]{bar}| is equivalent
% to
%
%\begin{verbatim}
% \begin{note}
% \mhinputref[foo]{bar}
% \end{note}
% \end{verbatim}
%
% \subsection{\textsf{problem-mh.sty}: Support  for Problems}\label{sec:user:problem}
% 
% The \DescribeMacro{\includemhproblem}|\includemhproblem| macro is a variant of
% |\includeproblem| with repository support. Instead of writing
% \begin{verbatim}
% \includeproblem[pts=7]{\MathHub{fooMH/bar/source/baz/foobar}}
% \end{verbatim}
% we can simply write (assuming that |\MathHub| is defined as above)
% \begin{verbatim}
% \includemhproblem[mhrepos=fooMH/bar,pts=7]{baz/foobar}
% \end{verbatim}
% Note that the |\importmhproblem| form is more semantic, which allows more advanced
% document management features in \textsf{MathHub}. 
% 
% \subsection{\textsf{hwexam-mh.sty}: Support  for Assignments}\label{sec:user:hwexam}
% 
% The \DescribeMacro{\includemhassignment}|\includemhassignment| macro is a variant of
% |\includeassignment| with repository support. Instead of writing
% \begin{verbatim}
% \includeassignment[number=3]{\MathHub{fooMH/bar/source/baz/foobar}}
% \end{verbatim}
% we can simply write (assuming that |\MathHub| is defined as above)
% \begin{verbatim}
% \includemhassignment[mhrepos=fooMH/bar,number=3]{baz/foobar}
% \end{verbatim}
%
% \subsection{\textsf{lstmh.sty}: Support  for Listings}\label{sec:user:lst}
% 
% The \DescribeMacro{\lstinputmhlisting}|\lstinputmhlisting| macro is a variant of
% |\lstinputlisting| with repository support. Instead of writing
% \begin{verbatim}
% \lstinputlisting[language=XML]{\MathHub{fooMH/bar/source/baz/foobar.xml}}
% \end{verbatim}
% we can simply write (assuming that |\MathHub| is defined as above)
% \begin{verbatim}
% \lstinputmhlisting[mhrepos=fooMH/bar,language=XML]{baz/foobar.xml}
% \end{verbatim}
%
% \section{Limitations}\label{sec:limitations}
% 
% In this section we document known limitations. If you want to help alleviate them,
% please feel free to contact the package author. Some of them are currently discussed in
% the \sTeX GitHub repository~\cite{sTeX:github:on}. 
% \begin{enumerate}
% \item none reported yet.
% \end{enumerate}
% 
% \StopEventually{\newpage\PrintIndex\newpage\PrintChanges\printbibliography}
% \newpage
%
% \section{Implementation}\label{sec:impl}
%
% We need to set up the packages by requiring the |metakeys|
% package~\ctancite{Kohlhase:metakeys} to be loaded (in the right version).
%
%    \begin{macrocode}
%<*package>
\ProvidesPackage{mathhub}[2010/10/01 v1.2 Basic MathHub functionality]
\RequirePackage{keyval}
\RequirePackage{pathsuris}
%    \end{macrocode}
% 
% \subsection{\texttt{mathhub.sty}: General Infrastructure}\label{sec:impl:infra}
%
% \begin{macro}{\mhcurrentrepos}
%   |\mhcurrentrepos| is used to initialize the current repository.
%    \begin{macrocode}
\newcommand\mhcurrentrepos[1]{\edef\mh@currentrepos{#1}}
%    \end{macrocode}
% \end{macro}
%
% \begin{macro}{\libinput}
%   the |\libinput| macro inputs from the |lib| directory of the MathHub repository and
%   then the |meta-inf/lib| repository of the group, if they exist. Since in practice
%   nested libinputs may occur, we make sure that we stash the old values of |\mh@inffile|
%   and |\mh@libfile| and restore them at the end.
%    \begin{macrocode}
\def\modules@@first#1/#2;{#1}
\newcommand\libinput[1]{%
\ifcsvoid{mh@currentrepos}{%
  \PackageError{mathhub}{current MathHub repository not found}{}}%
  {}%
\edef\@mh@group{\expandafter\modules@@first\mh@currentrepos;}%
\let\orig@inffile\mh@inffile\let\orig@libfile\mh@libfile%
\def\mh@inffile{\MathHub{\@mh@group/meta-inf/lib/#1}}%
\def\mh@libfile{\MathHub{\mh@currentrepos/lib/#1}}%
\if@iswindows@%
  \path@to@windows\mh@inffile%
  \path@to@windows\mh@libfile%
\fi%
\IfFileExists\mh@inffile{\input\mh@inffile}{}%
\IfFileExists\mh@inffile{}{\IfFileExists\mh@libfile{}{%
  {\PackageError{mathhub}%
    {Library file missing; cannot input #1.tex\MessageBreak%
    Both \mh@libfile.tex\MessageBreak and \mh@inffile.tex\MessageBreak%
    do not exist}%
  {Check whether the file name is correct}}}}%
\IfFileExists\mh@libfile{\input\mh@libfile\relax}{}%
\let\mh@inffile\orig@inffile\let\mh@libfile\orig@libfile}
%    \end{macrocode}
% \end{macro}
%
% \begin{macro}{\libusepackage}
%   the |\libusepackage| is analogous to |\libinput|
%    \begin{macrocode}
\newcommand\libusepackage[2][]{%
\edef\@mh@group{\expandafter\modules@@first\mh@currentrepos;}%
\let\orig@inffile\mh@inffile\let\orig@libfile\mh@libfile%
\edef\mh@inffile{\MathHub{\@mh@group/meta-inf/lib/#2}}%
\edef\mh@libfile{\MathHub{\mh@currentrepos/lib/#2}}%
\if@iswindows@%
  \path@to@windows\mh@inffile%
  \path@to@windows\mh@libfile%
\fi%
\IfFileExists{\mh@inffile.sty}{\usepackage[#1]{\mh@inffile}}{}%
\IfFileExists {\mh@inffile.sty}{}{\IfFileExists{\mh@libfile.sty}{}{%
  {\PackageError{mathhub}%
    {Library file missing; cannot use package #2.sty\MessageBreak%
    Both \mh@libfile.sty\MessageBreak and \mh@inffile.sty\MessageBreak%
    do not exist}%
  {Check whether the file name is correct}}}}%
\IfFileExists\mh@libfile{\input\mh@libfile\relax}{}%
\let\mh@inffile\orig@inffile\let\mh@libfile\orig@libfile}
%    \end{macrocode}
% \end{macro}
%
% Generally, the {\TeX} formatter |pdflatex| needs to know the file system paths of the
% referenced \sTeX files: usually long relative paths. The |pathsuris|
% package~\ctancite{ZhaKoh:pus} from the \sTeX bundle makes this somewhat more palatable
% by supplying the |\defpath| macro, which we can use to set the \textsf{MathHub} root
% path, e.g. by |\defpath{MathHub}{/user/foo/localmh/MathHub}| (we will assume this
% setting for all examples below). Fortunately, we can compute this automatically.
% 
% We parse the |MATHHUB| environment variable via |kpsewhich| ({\LaTeX} can run
% this even in paranoid mode) and then set the |MathMub| path using |\defpath|. 
%
%    \begin{macrocode}

\kpsewhich\mathhub@path{--var-value MATHHUB}
\if@iswindows@\windows@to@path\mathhub@path\fi
\ifx\mathhub@path\@empty%
\PackageError{mathhub}
  {MATHHUB system variable not found or wrongly set}
  {use export MATHHUB="<path>", where <path> points your MathHub direcctory}
\else\defpath{MathHub}{\mathhub@path}\fi
\ifcsvoid{mathhub@path}{}{\message{^^JMATHHUB: \mathhub@path}}

%    \end{macrocode}
%\begin{macro}{\findmanifest}
% |\findmanifest{|\meta{path}|}| searches for a file |MANIFEST.MF| up and over \meta{path} in
% the file system tree.
%
%    \begin{macrocode}
\def\findmanifest#1{%
  \@cpath{#1}%
  \ifx\@CanPath\@Slash%
    \def\manifest@mf{}%
  \else\ifx\@CanPath\@empty%
      \def\manifest@mf{}%
  \else%
    \edef\@findmanifest@path{\@CanPath/MANIFEST.MF}%
    \if@iswindows@\path@to@windows\@findmanifest@path\fi%
    %\message{^^JHere: \@findmanifest@path^^J}
    \IfFileExists{\@findmanifest@path}{%
      %\message{MANIFEST.MF found at \@findmanifest@path}
      \edef\manifest@mf{\@findmanifest@path}%
      \xdef\temp@archive@dir{\expandafter\detokenize\expandafter{\@CanPath}}%
    }{%
    \edef\@findmanifest@path{\@CanPath/META-INF/MANIFEST.MF}%
    \if@iswindows@\path@to@windows\@findmanifest@path\fi%
    \IfFileExists{\@findmanifest@path}{%
      %\message{MANIFEST.MF found at \@findmanifest@path}
      \edef\manifest@mf{\@findmanifest@path}%
      \xdef\temp@archive@dir{\expandafter\detokenize\expandafter{\@CanPath}}%
    }{%
    \edef\@findmanifest@path{\@CanPath/meta-inf/MANIFEST.MF}%
    \if@iswindows@\path@to@windows\@findmanifest@path\fi%
    \IfFileExists{\@findmanifest@path}{%
      %\message{MANIFEST.MF found at \@findmanifest@path}
      \edef\manifest@mf{\@findmanifest@path}%
      \xdef\temp@archive@dir{\expandafter\detokenize\expandafter{\@CanPath}}%
    }{%
      \findmanifest{\@CanPath/..}%
    }}}%
  \fi\fi%
}
%    \end{macrocode}
%\end{macro}
% the next macro is a helper function for parsing |MANIFEST.MF|
%
%    \begin{macrocode}
\def\split@manifest@key{%
  \IfSubStr{\manifest@line}{\@Colon}{%
      \StrBefore{\manifest@line}{\@Colon}[\manifest@key]%
      \StrBehind{\manifest@line}{\@Colon}[\manifest@line]%
      \trimstring\manifest@line%
      \trimstring\manifest@key%
  }{%
      \def\manifest@key{}%
  }%
}
%    \end{macrocode}
%
% the next helper function iterates over lines in |MANIFEST.MF|
%
%    \begin{macrocode}
\def\parse@manifest@loop{%
  \ifeof\@manifest%
  \else%
    \read\@manifest to \manifest@line\relax%
    \edef\manifest@line{\expandafter\detokenize\expandafter{\manifest@line}}%
    \split@manifest@key%
    % id
    \IfStrEq\manifest@key{\detokenize{id}}{%
        %\message{archive id: \manifest@line}
        \xdef\manifest@mf@id{\manifest@line}%
    }{%
    % narration-base
    \IfStrEq\manifest@key{\detokenize{narration-base}}{%
        %\message{archive narration-base: \manifest@line}
        \xdef\manifest@mf@narr{\manifest@line}%
    }{%
    % namespace
    \IfStrEq\manifest@key{\detokenize{source-base}}{%
       % \message{archive source-base: \manifest@line}
        \xdef\manifest@mf@ns{\manifest@line}%
    }{%
    \IfStrEq\manifest@key{\detokenize{ns}}{%
        %\message{archive ns: \manifest@line}
        \xdef\manifest@mf@ns{\manifest@line}%
    }{%
    % dependencies
    \IfStrEq\manifest@key{\detokenize{dependencies}}{%
        %\message{archive dependencies: \manifest@line}
        \xdef\manifest@mf@deps{\manifest@line}%
    }{%
    }}}}}%
    \parse@manifest@loop%
  \fi%
}
%    \end{macrocode}
%\begin{macro}{\parsemanifest}
% |\parsemanifest{|\meta{macroname}|}{|\meta{path}|}| finds |MANIFEST.MF| via |\findmanifest{|\meta{path}|}|,
% and parses the file, storing the individual fields (|id|, |narr|, |ns| and |dependencies|) in
% \meta{macroname}|id|, \meta{macroname}|narr|, etc.
%    \begin{macrocode}
\newread\@manifest
\def\parsemanifest#1#2{%
  \gdef\temp@archive@dir{}%
  \findmanifest{#2}%
  \begingroup%
    \gdef\manifest@mf@id{}%
    \gdef\manifest@mf@narr{}%
    \gdef\manifest@mf@ns{}%
    \gdef\manifest@mf@deps{}%
    \openin\@manifest\manifest@mf\relax%
    \parse@manifest@loop%
    \closein\@manifest%
  \endgroup%
  \if@iswindows@\windows@to@path\manifest@mf\fi%
  \cslet{#1id}\manifest@mf@id%
  \cslet{#1narr}\manifest@mf@narr%
  \cslet{#1ns}\manifest@mf@ns%
  \cslet{#1deps}\manifest@mf@deps%
  \cslet{#1dir}\temp@archive@dir%
}
%    \end{macrocode}
%\end{macro}
%\begin{macro}{\setcurrentreposinfo}
% |\setcurrentreposinfo{|\meta{id}|}| sets the current repository to \meta{id}, checks if the
% |MANIFEST.MF| of this repository has already been read, and if not, find it, parses it and stores
% the values in |\currentrepos@|\meta{key}|@|\meta{id} for later retrieval.
%    \begin{macrocode}
\def\setcurrentreposinfo#1{%
  \ifcsdef{currentrepos@dir@#1}{%
    \mhcurrentrepos{#1}%
  }{%
    \parsemanifest{mathhub@archive@}{\MathHub{#1}}%
    \@setcurrentreposinfo%
  }%
  \ifcsvoid{mathhub@archive@dir}{\PackageError{mathhub}{No archive with %
    name #1 found!}{make sure that #1 is directly in your MATHHUB folder %
    and contains a MANIFEST.MF, either directly in #1 or in a meta-inf %
    subfolder.}}{}%
}

\def\@setcurrentreposinfo{%
  \mhcurrentrepos{\mathhub@archive@id}%
  \csxdef{currentrepos@dir@\mathhub@archive@id}{\mathhub@archive@dir}%
  \csxdef{currentrepos@narr@\mathhub@archive@id}{\mathhub@archive@narr}%
  \csxdef{currentrepos@ns@\mathhub@archive@id}{\mathhub@archive@ns}%
  \csxdef{currentrepos@deps@\mathhub@archive@id}{\mathhub@archive@deps}%
}
%    \end{macrocode}
%\end{macro}
% Finally -- and that is the ultimate goal of all of the above, we set the current repos.  
%    \begin{macrocode}
\parsemanifest{mathhub@archive@}\stex@maindir
\@setcurrentreposinfo
\message{^^J Current MathHub repository: \ifcsvoid{mh@currentrepos}{None}\mh@currentrepos}
%</package>
%    \end{macrocode}
%
% \subsection{\texttt{omdoc--mh.sty}: MH Document Infrastructure}\label{sec:impl:modules}
%
%    \begin{macrocode}
%<*omdoc>
\ProvidesPackage{omdoc-mh}[2019/03/20 v1.1 MathHub support for OMDoc Documents]
\RequirePackage{mathhub}
%    \end{macrocode}
%
% \begin{macro}{\addmhbibresource}
%    \begin{macrocode}
\newcommand\addmhbibresource[2][]{%
  \def\@repos{#1}%
  \ifx\@repos\@empty%
  \addbibresource{\MathHub{\mh@currentrepos/#2}}%
  \else
  \addbibresource{\MathHub{\@repos/#2}}%
  \fi%
  \ignorespacesandpars}%
%</omdoc>
%    \end{macrocode}
% \end{macro}
%
% \subsection{\texttt{modules-mh.sty}: MH Variants for Modules}\label{sec:impl:modules}
% 
%    \begin{macrocode}
%<*modules>
\ProvidesPackage{modules-mh}[2019/03/20 v1.1 MathHub support for the sTeX modules package]
\RequirePackage{mathhub}
%    \end{macrocode}
%
% \begin{macro}{\importmhmodule}
%   The |\importmhmodule[|\meta{key=value list}|]{module}| saves the current value of
%   |\mh@currentrepos| in a local macro |\mh@@repos|, resets |\mh@currentrepos| to the new
%   value if one is given in the optional argument, and after importing resets
%   |\mh@currentrepos| to the old value in |\mh@@repos|. We do all the |\ifx| comparison
%   with an |\expandafter|, since the values may be passed on from other key
%   bindings. Parameters will be passed to |\importmodule|.
%    \begin{macrocode}
\srefaddidkey{importmhmodule}%
\addmetakey{importmhmodule}{mhrepos}%
\addmetakey{importmhmodule}{path}%
\addmetakey{importmhmodule}{ext}% why does this exist?
\addmetakey{importmhmodule}{dir}%
\addmetakey[false]{importmhmodule}{conservative}[true]%
\newcommand\importmhmodule[2][]{%
  \usemodule@maybesetcodes
  \metasetkeys{importmhmodule}{#1}%
  \ifx\importmhmodule@dir\@empty%
    \edef\@path{\importmhmodule@path}%
  \else\edef\@path{\importmhmodule@dir/#2}\fi%
  \ifx\@path\@empty% if module name is not set
    \importmodule[id=\importmhmodule@id]{#2}%
  \else%
    \edef\mh@@repos{\mh@currentrepos}% remember so that we can reset it. 
    \ifx\importmhmodule@mhrepos\@empty% if in the same repos
      \relax% no need to change mh@currentrepos, i.e, current directory.
    \else%
      \setcurrentreposinfo{\importmhmodule@mhrepos}% change it. 
      \addto@thismodulex{\noexpand\setcurrentreposinfo{\importmhmodule@mhrepos}}%
    \fi%
    \importmodule[load=\MathHub{\mh@currentrepos/source/\@path},
                         id=\importmhmodule@id]{#2}%
    \setcurrentreposinfo{\mh@@repos}% after importing, reset to old value
    \addto@thismodulex{\noexpand\setcurrentreposinfo{\mh@@repos}}%
  \fi%
  \ignorespacesandpars%
}
%    \end{macrocode}
% \end{macro}
% and now the analogs
% \begin{macro}{\usemhmodule}
%    \begin{macrocode}
\newcommand\usemhmodule[2][]{%
\metasetkeys{importmhmodule}{#1}%
\ifx\importmhmodule@dir\@empty%
  \edef\@path{\importmhmodule@path}%
\else\edef\@path{\importmhmodule@dir/#2}\fi%
\ifx\@path\@empty%
  \usemodule[id=\importmhmodule@id]{#2}%
\else%
  \edef\mh@@repos{\mh@currentrepos}%
  \ifx\importmhmodule@mhrepos\@empty%
  \else\setcurrentreposinfo{\importmhmodule@mhrepos}\fi%
  \usemodule[load=\MathHub{\mh@currentrepos/source/\@path},%
                        id=\importmhmodule@id]{#2}%
  \setcurrentreposinfo\mh@@repos%
\fi%
\ignorespacesandpars}
%    \end{macrocode}
% \end{macro}
%
% \begin{macro}{\mhinputref}
%    \begin{macrocode}
\newcommand\mhinputref[2][]{%
\def\@repos{#1}%
\edef\mh@@repos{\mh@currentrepos}%
\ifx\@repos\@empty\else\setcurrentreposinfo{#1}\fi%
\inputref{\MathHub{\mh@currentrepos/source/#2}}%
\setcurrentreposinfo\mh@@repos%
\ignorespacesandpars}
%    \end{macrocode}
% \end{macro}
%
% \begin{macro}{\mhinput}
%    \begin{macrocode}
\let\mhinput\mhinputref%
%</modules>
%    \end{macrocode}
% \end{macro}
%
% \subsection{\texttt{omtext-mh.sty}: MH Variants for OMText}\label{sec:impl:mtext}
%
%    \begin{macrocode}
%<*omtext>
\ProvidesPackage{omtext-mh}[2019/03/20 v1.1 MathHub support for the sTeX omtext package]
\RequirePackage{mathhub}
%    \end{macrocode}
%
% \begin{macro}{\*mhgraphics}
%   Use the current value of |\mh@currentrepos| or the value of the |mhrepos| key if it is
%   given in |\my*graphics|.
%    \begin{macrocode}
\def\Gin@mhrepos{}
\define@key{Gin}{mhrepos}{\def\Gin@mhrepos{#1}}
\newcommand\mhgraphics[2][]{\setkeys{Gin}{#1}%
\edef\mh@@repos{\mh@currentrepos}%
\ifx\Gin@mhrepos\@empty\edef\temp@path{\MathHub{\mh@currentrepos/source/#2}}%\includegraphics[#1]{\MathHub{\mh@currentrepos/source/#2}}%
\else\edef\temp@path{\MathHub{\Gin@mhrepos/source/#2}}\fi%\includegraphics[#1]{\MathHub{\Gin@mhrepos/source/#2}}\fi
\if@iswindows@\path@to@windows\temp@path\fi
\includegraphics[#1]{\temp@path}
\def\Gin@mhrepos{}\setcurrentreposinfo\mh@@repos}
\newcommand\cmhgraphics[2][]{\begin{center}\mhgraphics[#1]{#2}\end{center}}
%    \end{macrocode}
% \end{macro}
%
% The following macros are deprecated. 
%
%    \begin{macrocode}
\newcommand\mhcgraphics[2][]{\begin{center}\mhgraphics[#1]{#2}\end{center}
  \PackageWarning{omtext-mh}{\protect\mhcgraphics\space is deprecated, use \protect\cmhgraphics\space instead}}
\newcommand\mhbgraphics[2][]{\fbox{\mhgraphics[#1]{#2}}
  \PackageWarning{omtext-mh}{\protect\mhbgraphics\space is deprecated, use
    \protect\mhgraphics\space and {center} instead}}
\newcommand\mhcbgraphics[2][]{\begin{center}\fbox{\mhgraphics[#1]{#2}}\end{center}
  \PackageWarning{omtext-mh}{\protect\mhcbgraphics\space is deprecated, use
    \protect\mhgraphics,\space {center}, and \protect\fbox\space instead}}
%</omtext>
%    \end{macrocode}
% 
% \subsection{\texttt{smultiling-mh.sty}: MH Variants for Multilinguality}\label{sec:impl:smultiling}
%
%    \begin{macrocode}
%<*smultiling>
\ProvidesPackage{smultiling-mh}[2019/03/20 v1.1 MathHub support for the sTeX smultiling package]
\RequirePackage{mathhub}
%    \end{macrocode}
%
% \begin{environment}{mhmodsig}
%    \begin{macrocode}
\newenvironment{mhmodsig}{\begin{modsig}}{\end{modsig}}
%    \end{macrocode}
% \end{environment}
%
% \begin{macro}{mhmodnl:*}
%    \begin{macrocode}
\addmetakey{mhmodnl}{mhrepos}
\addmetakey{mhmodnl}{path}
\addmetakey*{mhmodnl}{title}
\addmetakey*{mhmodnl}{creators}
\addmetakey*{mhmodnl}{contributors}
\addmetakey{mhmodnl}{srccite}
\addmetakey{primary}{mhmodnl}[yes]
%    \end{macrocode}
% \end{macro}
%
% \begin{environment}{mhmodnl}
%   The |mhmodnl| environment is just a layer over the |module| environment and the
%   |\importmhmodule| macro with the keys and language suitably adapted.
%    \begin{macrocode}
\newenvironment{mhmodnl}[3][]{\metasetkeys{mhmodnl}{#1}\def\@test{#1}%
\edef\@repos{\ifx\mhmodnl@mhrepos\@empty\mh@currentrepos\else\mhmodnl@mhrepos\fi}%
\edef\@load{\MathHub{\@repos/source/\ifx\mhmodnl@path\@empty #2\else\mhmodnl@path\fi}}%
\ifx\@test\@empty\begin{modnl}[load=\@load]{#2}{#3}\else\begin{modnl}[load=\@load,#1]{#2}{#3}\fi%
\ignorespacesandpars}
{\end{modnl}\ignorespacesandparsafterend}
%    \end{macrocode}
% \end{environment}
%
% \begin{environment}{mhviewsig}
%   The |mhviewsig| environment is just a layer over the |mhview| environment with the keys
%   suitably adapted.
%    \begin{macrocode}
\newenvironment{mhviewsig}[4][]{% keys, id, from, to
\def\@test{#1}\ifx\@test\@empty%
\begin{mhview}[id=#2]{#3}{#4}\else%
\begin{mhview}[id=#2,#1]{#3}{#4}\fi%
\ignorespacesandpars}
{\end{mhview}\ignorespacesandparsafterend}
%    \end{macrocode}
% \end{environment}
%
% \begin{environment}{mhviewnl}
%   The |mhviewnl| environment is just a layer over the |mhview| environment with the
%   keys and language suitably adapted.\ednote{MK: we have to do something about the
%   if@langfiles situation here. But this is non-trivial, since we do not know the current
%   path, to which we could append .\meta{lang}!}
%    \begin{macrocode}
\newenvironment{mhviewnl}[5][]{% keys, id, lang, from, to
\def\@test{#1}\ifx\@test\@empty%
\begin{mhview}[id=#2.#3]{#4}{#5}\else%
\begin{mhview}[id=#2.#3,#1]{#4}{#5}\fi%
\ignorespacesandpars}
{\end{mhview}\ignorespacesandparsafterend}
%</smultiling>
%    \end{macrocode}
% \end{environment}
%
% \subsection{\texttt{structview-mh.sty}: MH Variants for Structures and
% Views}\label{sec:impl:structview}
%
%    \begin{macrocode}
%<*structview>
\ProvidesPackage{structview-mh}[2019/03/20 v1.1 MathHub support for the sTeX structview package]
\RequirePackage{mathhub}
%    \end{macrocode}
%
% \begin{environment}{mhstructure}
%    \begin{macrocode}
\newenvironment{mhstructure}[3][]{%
  \gdef\@@doit{\importmhmodule[#1]{#3}}%
  \ifmod@show\par\noindent structure import "#2" from module #3 \@@doit\fi%
  \ignorespacesandpars}
{\aftergroup\@@doit\ifmod@show end import\fi%
  \ignorespacesandparsafterend}
%    \end{macrocode}
% \end{environment}
%
% \begin{environment}{importmhmodulevia}
%   this is now deprecated, we give an error
%    \begin{macrocode}
\newenvironment{importmhmodulevia}[2][]%
{\PackageError{structview-mh}%
  {The {importmhmodulevia} environment is deprecated}{use the {mhstructure} instead!}%
  \begin{mhstructure}[#1]{missing}{#2}}
{\end{mhstructure}}
%    \end{macrocode}
% \end{environment}
% 
%    \begin{macrocode}
\srefaddidkey{mhview}
\addmetakey{mhview}{display}
\addmetakey{mhview}{creators}
\addmetakey{mhview}{contributors}
\addmetakey{mhview}{srccite}
\addmetakey*{mhview}{title}
\addmetakey{mhview}{type}
\addmetakey{mhview}{fromrepos}
\addmetakey{mhview}{torepos}
\addmetakey{mhview}{frompath}
\addmetakey{mhview}{topath}
%    \end{macrocode}
%
% \begin{environment}{mhview}
%   the MathHub version
%    \begin{macrocode}
\newenvironment{mhview}[3][]% keys, from, to
{\metasetkeys{mhview}{#1}%
  \sref@target%
  \begin{@mhview}{#2}{#3}%
  \view@heading{#2}{#3}{\mhview@display}{\mhview@title}%
  \ignorespacesandpars}
{\end{@mhview}\ignorespacesandparsafterend}
\ifmod@show\surroundwithmdframed{mhview}\fi
%    \end{macrocode}
% \end{environment}
%
% \begin{environment}{@mhview}
%   The |@mhview| does the actual bookkeeping at the module level.
%    \begin{macrocode}
\newenvironment{@mhview}[2]{%from, to
  \usemhmodule[mhrepos=\mhview@fromrepos,path=\mhview@frompath]{#1}%
  \usemhmodule[mhrepos=\mhview@torepos,path=\mhview@topath]{#2}%
}{}%
%    \end{macrocode}
% \end{environment}
% 
% \begin{environment}{mhviewsketch}
%   The |mhviewsketch| environment is deprecated, we give an error
%    \begin{macrocode}
\newenvironment{mhviewsketch}[3][]%
{\PackageError{structview}%
  {The {mhviewsketch} environment is deprecated}{use the {mhview} instead!}%
  \begin{mhview}[#1]{#2}{#3}}
{\end{mhview}}
%    \end{macrocode}
% \end{environment}
% 
% \begin{environment}{mhinlineView}
%   Analogous modification to |inlineView|
%    \begin{macrocode}
\newenvironment{mhinlineView}[2][]% keys, source
{\metasetkeys{mhview}{#1}\sref@target%
  \importmhmodule[mhrepos=\mhview@fromrepos,path=\mhview@frompath]{#2}%
  \ignorespacesandpars}
{\ignorespacesandpars}
%    \end{macrocode}
% \end{environment}
%
% \begin{macro}{mhinlineview}
%    \begin{macrocode}
\newcommand\mhinlineview[3][]{\begin{mhinlineView}[#1]{#2}{\module@id}#3\end{mhinlineView}}
%</structview>
%    \end{macrocode}
% \end{macro}
%
% \subsection{ \textsf{mikoslides-mh.sty}: Support  for MiKo Slides}\label{sec:impl:mikoslides}
% 
%    \begin{macrocode}
%<*mikoslides>
\ProvidesPackage{mikoslides-mh}[2019/03/20 v1.1 MathHub support for the sTeX mikoslides package]
\RequirePackage{mathhub}
%    \end{macrocode}
%
% \begin{macro}{\mhframeimage}
%   Use the current value of |\mh@currentrepos| or the value of the |mhrepos| key if it is
%   given in |\frameimage|.
%    \begin{macrocode}
\def\Gin@mhrepos{}
\define@key{Gin}{mhrepos}{\def\Gin@mhrepos{#1}}
\newcommand\mhframeimage[2][]{%
  \setkeys{Gin}{#1}%
  \edef\mh@@repos{\mh@currentrepos}%
  \ifx\Gin@mhrepos\@empty%
    \edef\temp@path{\MathHub{\mh@currentrepos/source/#2}}%
  \else%
    \edef\temp@path{\MathHub{\Gin@mhrepos/source/#2}}%
  \fi%
  \if@iswindows@\path@to@windows\temp@path\fi%
  \frameimage[#1]{\temp@path}%
}%
%    \end{macrocode}
% \end{macro}
%
% \begin{macro}{\mhinputref*}
%    \begin{macrocode}
\let\orig@mhinputref\mhinputref
\def\mhinputref{\@ifstar\nmhinputref\orig@mhinputref}
\newcommand\nmhinputref[2][]{\ifnotes\orig@mhinputref[#1]{#2}\fi}
%    \end{macrocode}
% \end{macro}
%
% \begin{macro}{\mhexcursion}
%    \begin{macrocode}
\newcommand\activatemhexcursion[2][]{\ifstrempty{#1}%
  {\gappto\printexcursions{\mhinputref[\mh@currentrepos]{#2}}}%
  {\gappto\printexcursions{\mhinputref[#1]{#2}}}}
\newcommand\mhexcursion[4][]{% repos, label, path, text
  \activatemhexcursion[#1]{#3}\excursionref{#2}{#4}}%
%    \end{macrocode}
% \end{macro}
%
% \begin{macro}{\mhexcursiongroup}
\srefaddidkey{mhexcursiongroup}%
\addmetakey{mhexcursiongroup}{intro}%
\addmetakey{mhexcursiongroup}{mhrepos}%
\newcommand\mhexcursiongroup[1][]{%
  \metasetkeys{mhexcursiongroup}{#1}%
  \ifdefempty\printexcursions{}% only if there are excursions
  {\begin{omgroup}[#1]{Excursions}%
      \ifdefempty\mhexcursiongroup@intro{}%
      {\ifdefempty\mhexcursiongroup@mhrepos%
        {\mhinputref{\mhexcursiongroup@intro}}%
        {\mhinputref[\mhexcursiongroup@mhrepos]{\mhexcursiongroup@intro}}}%
      \printexcursions%
    \end{omgroup}}}
%</mikoslides>
%    \end{macrocode}
% \end{macro}
% \subsection{\textsf{problem-mh.sty}: Support  for Problems}\label{sec:impl:problem}
% 
%    \begin{macrocode}
%<*problem>
\ProvidesPackage{problem-mh}[2019/03/20 v1.1 MathHub support for the sTeX problem package]
\RequirePackage{mathhub}
%    \end{macrocode}
%
% \begin{macro}{\includemhproblem}
%   The |\includemhproblem| saves the current value of |\mh@currentrepos| in a local macro
%   |\mh@@repos|, resets |\mh@currentrepos| to the new value if one is given in the
%   optional argument, and after importing resets |\mh@currentrepos| to the old value in
%   |\mh@@repos|.
%    \begin{macrocode}
\addmetakey{inclprob}{mhrepos}
\newcommand\includemhproblem[2][]{\metasetkeys{inclprob}{#1}%
\edef\mh@@repos{\mh@currentrepos}%
\ifx\inclprob@mhrepos\@empty\else\setcurrentreposinfo\inclprob@mhrepos\fi%
\edef\temp@path{\MathHub{\mh@currentrepos/source/#2}}
\if@iswindows@\path@to@windows\temp@path\fi
\input{\temp@path}%
\setcurrentreposinfo\mh@@repos\clear@inclprob@keys}
%</problem>
%    \end{macrocode}
% \end{macro}
%
% \subsection{\textsf{hwexam-mh.sty}: Support  for Assignments}\label{sec:impl::hwexam}
% 
%    \begin{macrocode}
%<*hwexam>
\ProvidesPackage{hwexam-mh}[2019/03/20 v1.1 MathHub support for the sTeX hwexam package]
\RequirePackage{mathhub}
%    \end{macrocode}
%
% \begin{macro}{\inputmhassignment}
%   The |\inputmhassignment| saves the current value of |\mh@currentrepos| in a local macro
%   |\mh@@repos|, resets |\mh@currentrepos| to the new value if one is given in the
%   optional argument, and after importing resets |\mh@currentrepos| to the old value in
%   |\mh@@repos|.
%    \begin{macrocode}
\newcommand\inputmhassignment[2][]{\metasetkeys{inclassig}{#1}%
\edef\mh@@repos{\mh@currentrepos}%
\ifx\inclassig@mhrepos\@empty\else\setcurrentreposinfo\inclassig@mhrepos\fi%
\edef\temp@path{\MathHub{\mh@currentrepos/source/#2}}%
\if@iswindows@\path@to@windows\temp@path\fi%
\inputassignment[#1]{\temp@path}%
\setcurrentreposinfo\mh@@repos\clear@inclassig@keys}%
\newcommand\includemhassignment[2][]{\newpage\inputmhassignment[#1]{#2}}
%</hwexam>
%    \end{macrocode}
% \end{macro}
%
% \subsection{\textsf{tikzinput-mh.sty}: Support  for Assignments}\label{sec:impl:tikzinput}
% 
%    \begin{macrocode}
%<*tikzinput>
\ProvidesPackage{tikzinput-mh}[2019/03/20 v1.1 MathHub support for the sTeX tikzinput package]
\RequirePackage{mathhub}
\RequirePackage{pathsuris}
%    \end{macrocode}
%
%    \begin{macrocode}
\define@key{Gin}{mhrepos}{\def\Gin@mhrepos{#1}}
\newcommand\mhtikzinput[2][]{\def\Gin@mhrepos{}\setkeys{Gin}{#1}%
\edef\mh@@repos{\mh@currentrepos}%
\ifx\Gin@mhrepos\@empty\edef\temp@path{\MathHub{\mh@currentrepos/source/#2}}%
\else\setcurrentreposinfo\Gin@mhrepos\edef\temp@path{\MathHub{\Gin@mhrepos/source/#2}}\fi%
\if@iswindows@\path@to@windows\temp@path\fi%
\tikzinput[#1]{\temp@path}
\def\Gin@mhrepos{}\setcurrentreposinfo\mh@@repos}
\newcommand\cmhtikzinput[2][]{\begin{center}\mhtikzinput[#1]{#2}\end{center}}
%</tikzinput>
%    \end{macrocode}
%
% \subsection{\textsf{lstmh.sty}: Support  for Listings}\label{sec:impl:lst}
% 
%    \begin{macrocode}
%<*lst>
\ProvidesPackage{lstmh}[2019/03/20 v1.1 MathHub support for the listings package]
\RequirePackage{mathhub}
\RequirePackage{pathsuris}
\RequirePackage{listings}
%    \end{macrocode}
%
%    \begin{macrocode}
\define@key{lst}{mhrepos}{\def\lst@mhrepos{#1}}
\newcommand\lstinputmhlisting[2][]{\def\lst@mhrepos{}\setkeys{lst}{#1}%
\edef\mh@@repos{\mh@currentrepos}%
\ifx\lst@mhrepos\@empty\edef\temp@path{\MathHub{\mh@currentrepos/source/#2}}%
\else\edef\temp@path{\MathHub{\lst@mhrepos/source/#2}}\fi%
\if@iswindows@\path@to@windows\temp@path\fi%
\lstinputlisting[#1]{\temp@path}
\def\lst@mhrepos{}\setcurrentreposinfo\mh@@repos}
\newcommand\clstinputmhlisting[2][]{\begin{center}\lstinputmhlisting[#1]{#2}\end{center}}
%</lst>
%    \end{macrocode}
% \Finale
\endinput
% \iffalse
%%% Local Variables: 
%%% mode: doctex
%%% TeX-master: t
%%% End: 
% \fi


%  LocalWords:  iffalse NeedsTeXFormat mathhub.dtx mathhub.sty mhviewsketch omdoc latexml
%  LocalWords:  maketitle sref mathhub co-eveolved organizing tableofcontents newpage lmh
%  LocalWords:  leadsto HorIacJuc:cscpnrr11 textbf mh-variants mhcurrentrepos libinput
%  LocalWords:  mhcurrentrepos modules-mh.sty importmhmodule importmhmodule foobar ldots
%  LocalWords:  initialize usemhmodule usemhmodule usemodule mhinputref mhinputref keyval
%  LocalWords:  mhinput mhinput inputref mh-variant omtext-mh.sty  ZhaKoh:pus cmhgraphics
%  LocalWords:  mycgraphics smultiling-mh.sty mhmodsig structview-mh.sty mhframeimage ifx
%  LocalWords:  mikoslides-mh.sty mhframeimage frameimage problem-mh.sty includemhproblem
%  LocalWords:  includemhproblem includeproblem importmhproblem hwexam-mh.sty lstmh.sty
%  LocalWords:  includemhassignment includemhassignment includeassignment lstinputlisting
%  LocalWords:  lstinputmhlisting lstinputmhlisting printbibliography sec:impl metakeys
%  LocalWords:  ctancite newcommand mh@currentrepos expandafter modules-mh srefaddidkey
%  LocalWords:  addmetakey smglom numberfield metasetkeys importmhmodule@ext,id omtext
%  LocalWords:  ignorespaces mhrepos mhgraphics setkeys mygraphics mhbgraphics fbox 2,ext
%  LocalWords:  mhcbgraphics smultiling-mh mhmodnl srccite newenvironment tex,path mhview
%  LocalWords:  smultiling@language ignorespacesandpars ignorespacesandparsafterend 1,ext
%  LocalWords:  mhviewsig mhviewnl 3,ext structview-mh mhstructure gdef ifmod@show topath
%  LocalWords:  noindent aftergroup importmhmodulevia torepos surroundwithmdframed hwexam
%  LocalWords:  mhview@fromrepos,path mhview@frompath,ext mhview@torepos,path mikoslides
%  LocalWords:  mhview@topath,ext problem-mh inclprob inclassig inputmhassignment lst
%  LocalWords:  inputassignment tikzinput-mh.sty tikzinput pathsuris mhtikzinput lstmh
%  LocalWords:  cmhtikzinput doctex organized pdflatex defpath externalize texttt textsf
%  LocalWords:  flexiformal ednote libusepackage libusepackage mh.sty addmhbibresource
%  LocalWords:  addmhbibresource addbibresource Multilinguality modnl smultiing 2,ext
%  LocalWords:  transclude usepackage smultiling 1,ext 3,ext nmhinputref ifnotes 2,ext
%  LocalWords:  clstinputmhlisting cmhgraphics 1,ext 3,ext mhinlineView xparse libinputs
%  LocalWords:  mh@inffile mh@libfile orig@inffile orig@libfile kpsewhich MathMub ifeof
%  LocalWords:  ExplSyntaxOn ExplSyntaxOff trimstring xdef begingroup oldpercentcatcode
%  LocalWords:  catcode windowsstring detokenize endgroup csname endcsname newread 2,ext
%  LocalWords:  closein 1,ext 3,ext mmt.jar
