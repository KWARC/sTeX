% \iffalse meta-comment
% An Infrastructure for Mathematical Statements in sTeX
% Copyright (c) 2019 Michael Kohlhase, all rights reserved
%               this file is released under the
%               LaTeX Project Public License (LPPL)
%
% The original of this file is in the public repository at 
% http://github.com/sLaTeX/sTeX/
% \fi
%  
% \iffalse
%<package>\NeedsTeXFormat{LaTeX2e}[1999/12/01]
%<package>\ProvidesPackage{statements}[2020/04/27 v1.5 Semantic Markup for Statements]
%
%<*driver>
\documentclass{ltxdoc}
\usepackage[utf8]{inputenc}
\usepackage[T1]{fontenc}\usepackage{url,array,float,amsfonts}
\usepackage{statements,presentation}
\usepackage{paralist}
\usepackage[show]{ed}
\usepackage[hyperref=auto,style=alphabetic]{biblatex}
\addbibresource{kwarcpubs.bib}
\addbibresource{extpubs.bib}
\addbibresource{kwarccrossrefs.bib}
\addbibresource{extcrossrefs.bib}
\usepackage{stex-logo}
\usepackage{cmath}
\usepackage{ctangit}
\usepackage{hyperref}
\makeindex
\floatstyle{boxed}
\newfloat{exfig}{thp}{lop}
\floatname{exfig}{Example}
\def\githubissue#1{\cite{sTeX:github:on}, \hyperlink{https://github.com/sLaTeX/sTeX/issues/#1}{issue #1}}
\begin{document}
\RecordChanges
\DocInput{statements.dtx}
\end{document}
%</driver>
% \fi
% 
% \iffalse\CheckSum{809}\fi
%
% \changes{v0.9}{2005/06/14}{First Version with Documentation}
% \changes{v0.9a}{2005/07/01}{Completed Documentation}
% \changes{v0.9b}{2005/08/06}{Complete functionality and Updated Documentation}
% \changes{v0.9c}{2006/01/13}{more packaging}
% \changes{v0.9d}{2007/09/09}{moved omtext and friends to the omdoc package}
% \changes{v0.9d}{2007/09/09}{made dependence on the omdoc package explicit}
% \changes{v0.9d}{2007/09/09}{adding ids to many elements}
% \changes{v0.9e}{2008/05/27}{adding cross-references}
% \changes{v0.9e}{2008/09/29}{augmenting the index macros with optional values}
% \changes{v0.9f}{2008/12/04}{changed 'consymb' to 'symboldec' and documented it.}
% \changes{v0.9g}{2010/01/14}{the package is now based on {\texttt{ntheorem for presentation}}}
% \changes{v0.9g}{2010/01/19}{Added support for localization}
% \changes{v0.9g}{2010/02/23}{added {\texttt{\textbackslash symref}}}
% \changes{v1.0}{2010/06/18}{now based on  {\texttt{omtext}} package instead of {\texttt{omdoc}}}
% \changes{v1.0}{2010/07/13}{adding {\texttt{\textbackslash inlineex}}}
% \changes{v1.1}{2011/08/25}{renaming all convenience macros for {\texttt{\textbackslash
% definendum}} and {\texttt{\textbackslash termref}}}
% \changes{v1.1}{2012/07/06}{adding \texttt{\textbackslash usevocab} to example for importing}
% \changes{v1.1}{2013/05/17}{more support for types: \texttt{typedec} and
% \texttt{\textbackslash inlinetypedec}}
% \changes{v1.2}{2015/04/03}{adding optional last arg to \texttt{\textbackslash \*defi*}}
% \changes{v1.2}{2015/04/17}{adding \texttt{\textbackslash inlineass}}
% \changes{v1.2}{2015/09/05}{adding \texttt{\textbackslash defis} and friends}
% \changes{v1.3}{2017/07/31}{adding \texttt{\textbackslash Defi}, \texttt{\textbackslash Trefi} and friends}
% \changes{v1.4}{2017/10/15}{changing the optional argument of \texttt{\textbackslash
% defi} and friends to a keyval argument}
% \changes{v1.4}{2019/11/13}{adding \texttt{inlineAssertion}, \texttt{inlineDefinition},
% and \texttt{inlineExample} environment for block content}
% \changes{v1.5}{2020/04/27}{extending the first optional argument of
% \texttt{\textbackslash *trefi*} to work as in \texttt{\textbackslash mtref*} and
% deprecating the latter in favor of the former.}
% 
% \GetFileInfo{statements.sty}
% 
% \MakeShortVerb{\|}
% \def\scsys#1{{{\sc #1}}\index{#1@{\sc #1}}}
% \def\xml{\scsys{Xml}}
% \def\mathml{\scsys{MathML}}
% \def\omdoc{\scsys{OMDoc}}
% \def\openmath{\scsys{OpenMath}}
% \def\latexml{\scsys{LaTeXML}}
% \def\perl{\scsys{Perl}}
% \def\activemath{\scsys{ActiveMath}}
% \title{Semantic Markup for Mathematical Statements\thanks{Version {\fileversion} (last revised
%        {\filedate})}}
%    \author{Michael Kohlhase\\
%            FAU Erlangen-N\"urnberg\\
%            \url{http://kwarc.info/kohlhase}}
% \maketitle
%
% \begin{abstract}
%   The |statements| package is part of the {\stex} collection, a version of {\TeX/\LaTeX}
%   that allows to markup {\TeX/\LaTeX} documents semantically without leaving the
%   document format, essentially turning {\TeX/\LaTeX} into a document format for
%   mathematical knowledge management (MKM).
%
%   This package provides semantic markup facilities for mathematical statements like
%   Theorems, Lemmata, Axioms, Definitions, etc. in {\stex} files. This structure can be
%   used by MKM systems for added-value services, either directly from the {\sTeX}
%   sources, or after translation.
% \end{abstract}
% 
% \setcounter{tocdepth}{2}\tableofcontents\newpage
%
% \section{Introduction}\label{sec:intro}
%
% The motivation for the |statements| package is very similar to that for semantic macros
% in the |modules| package: We want to annotate the structural semantic properties of
% statements in the source, but present them as usual in the formatted documents. In
% contrast to the case for mathematical objects, the repertoire of mathematical statements
% and their structure is more or less fixed.
%
% This structure can be used by MKM systems for added-value services, either directly from
% the {\sTeX} sources, or after translation. Even though it is part of the {\stex}
% collection, it can be used independently, like it's sister package |sproofs|.
%
% {\stex}~\cite{Kohlhase:ulsmf08,sTeX:github:on} is a version of {\TeX/\LaTeX} that allows to
% markup {\TeX/\LaTeX} documents semantically without leaving the document format,
% essentially turning {\TeX/\LaTeX} into a document format for mathematical knowledge
% management (MKM). Currently the {\omdoc} format~\cite{Kohlhase:OMDoc1.2} is directly
% supported.
% 
% \section{The User Interface}\label{sec:user-interface}
% 
% The |statements| package supplies a semantically oriented infrastructure for marking up
% mathematical statements: fragments of natural language that state properties of
% mathematical objects, e.g. axioms, definitions, or theorems. The |statement| package
% provides an infrastructure for marking up the semantic relations between statements for
% the {\omdoc} transformation and uses the |ntheorem| package~\cite{MaySch:eltte09} for
% formatting (i.e. transformation to PDF).
%
% \subsection{Package Options}\label{sec:user:options}
% 
% The |statements| package provides the \DescribeMacro{defindex}|defindex| option to
% \sTeX. If this is set, then definienda are automatically passed into the index of the
% document. Furthermore, the |statements| package passes the
% \DescribeMacro{showmeta}|showmeta| to the |metakeys| package. If this is set, then the
% metadata keys are shown (see~\ctancite{Kohlhase:metakeys} for details and customization
% options). The |nontheorem| option tells statements not to load the |ntheorem| package --
% in case some other theorem package is already loaded; e.g. by the |beamer| package and
% we prefer that. Note that using the |nontheorem| option in a case where no theorem
% package is loaded will lead to errors.
%
% The \DescribeMacro{mh}|mh| option turns on MathHub support; see \ctancite{Kohlhase:mss}.
%
% \subsection{Statements}\label{sec:statements}
% 
% All the statements are marked up as environments, that take a |KeyVal| argument that
% allows to annotate semantic information. Generally, we distinguish two forms of
% statements:
% \begin{description}
% \item[{\twintoo{block}{statement}s}] have explicit discourse markers that delimit their
%   content in the surrounding text, e.g. the boldface word ``{\bf{Theorem}:}'' as a start
%   marker and a little line-end box as an end marker of a proof.
% \item[{\twintoo{flow}{statement}s}] do not have explicit markers, they are interspersed
%   with the surrounding text.
% \end{description}
% Since they have the same semantic status, they must both be marked up, but styled
% differently. We distinguish between these two presentational forms with the
% \DescribeMacro{display=}|display| key, which is allowed on all statement
% environments. If it has the value |block| (the default), then the statement will be
% presented in a paragraph of its own, have explicit discourse markers for its begin and
% end, possibly numbering, etc. If it has the value |flow|, then no extra presentation
% will be added the semantic information is invisible to the reader. Another key that is
% present on all statement environments in the \DescribeMacro{id=}|id| key it allows to
% identify the statement with a name and to reference it with the semantic referencing
% infrastructure provided by the |sref| package~\ctancite{Kohlhase:sref}.
%
% \subsubsection{Axioms and Assertions}\label{sec:user:axiomassertion}
% 
% The \DescribeEnv{assertion}|assertion| environment is used for marking up statements
% that can be justified from previously existing knowledge (usually marked with the
% monikers ``Theorem'', ``Lemma'', ``Proposition'', etc. in mathematical vernacular). The
% environment |assertion| is used for all of them, and the particular subtype of
% assertion is given in the \DescribeMacro{type=}|type| key. So instead of
% |\begin{Lemma}|\iffalse\end{Lemma}\fi we have to write
% |\begin{assertion}[type=lemma]|\iffalse\end{assertion}\fi (see
% Example~\ref{fig:assertion} for an example).
% \begin{exfig}
% \begin{verbatim}
% \begin{assertion}[id=sum-over-odds,type=lemma]
%   $\sum_{i=1}^n{2i-1}=n^2$
% \end{assertion}
% \end{verbatim}
% \vspace{-1em}will lead to the result\vspace{-2em}\par\noindent
% \begin{assertion}[id=sum-over-odds,type=lemma]
%   $\sum_{i=1}^n{2i-1}=n^2$
% \end{assertion}
% \caption{Semantic Markup for a Lemma in a {\texttt{module}} context}\label{fig:assertion}
% \end{exfig}
%
% Whether we will see the keyword ``Lemma'' will depend on the value of the optional
% |display| key.  In all of the |assertion| environments, the presentation expectation is
% that the text will be presented in italic font. The presentation (keywords, spacing, and
% numbering) of the |assertion| environment is delegated to a theorem styles from the
% |ntheorem| environment. For an assertion of type \meta{type} the |assertion| environment
% calls the |ST|\meta{type}|AssEnv| environment provided by the |statements| package; see
% Figure~\ref{fig:assertion-types} for a list of provided assertion types. Their
% formatting can be customized by redefining the |ST|\meta{type}|AssEnv| environment via
% the |\renewtheorem| command from the |ntheorem| package; see~\cite{MaySch:eltte09} for
% details.
%
% \begin{exfig}
%   \begin{tabular}{|l|l|}\hline
%     Value & Explanation \\\hline\hline
%     \textbf{theorem}, \textbf{proposition} 
%     & an important assertion with a proof\\\hline 
%     \multicolumn{2}{|p{12cm}|}{\footnotesize Note that the meaning of \textbf{theorem}
%       (in this case the existence of a proof) is not
%       enforced by {\omdoc} applications. It can be appropriate to give an assertion
%       the \textbf{theorem}, if the
%       author knows of a proof (e.g. in the literature), but has not formalized it in
%       {\omdoc} yet.}\\\hline\hline
%     \textbf{lemma} & a less important assertion with a proof\\\hline
%     \multicolumn{2}{|p{12cm}|}{\footnotesize The difference of importance specified
%     here is even softer than the other ones, since e.g. reusing
%     a mathematical paper as a chapter in a larger monograph, may make it necessary to
%     downgrade a theorem (e.g.  the main theorem of the paper) and give it the status of
%     a lemma in the overall work.}\\\hline\hline
%     \textbf{corollary} & a simple consequence\\\hline
%     \multicolumn{2}{|p{12cm}|}{\footnotesize An assertion is
%       sometimes marked as a corollary to some other statement, if the proof is
%       considered simple. This is often the case for important theorems that are simple
%       to get from technical lemmata.}\\\hline\hline
%     \textbf{postulate}, \textbf{conjecture}
%     & an assertion without proof or counter-exam\-ple\\\hline
%     \multicolumn{2}{|p{12cm}|}{\footnotesize Conjectures are assertions, whose
%       semantic value is not yet decided, but which the author considers likely to be
%       true. In particular, there is no proof or counter-example.}\\\hline\hline
%     \textbf{false-conjecture} 
%     & an assertion with a counter-example\\\hline
%     \multicolumn{2}{|p{12cm}|}{\footnotesize A conjecture that has proven to be false,
%       i.e. it has a counter-example. Such assertions are often kept for illustration and
%       historical purposes.}\\\hline\hline
%     \textbf{obligation}, \textbf{assumption} 
%     & an assertion on which a proof of another depends\\\hline
%     \multicolumn{2}{|p{12cm}|}{\footnotesize These kinds of assertions
%       are convenient during the exploration of a mathematical theory. They can be used
%       and proven later (or assumed as an axiom).}\\\hline\hline
%     \textbf{rule}
%     & a normative assertion\\\hline
%     \multicolumn{2}{|p{12cm}|}{\footnotesize These kinds of assertions can be interpreted
%     procedurally to trigger actions}\\\hline\hline
%     \textbf{observation}, \textbf{remark} & if everything else fails\\\hline
%     \multicolumn{2}{|p{12cm}|}{\footnotesize This type is the catch-all if none of the others
%       applies.}\\\hline 
%   \end{tabular}
% \caption{Types of Mathematical Assertions}\label{fig:assertion-types}
% \end{exfig}
%
% \DescribeEnv{axiom} The |axiom| environment is similar to |assertion|, but the content
% has a different ontological status: axioms are assumed without (formal) justification,
% whereas assertions are expected to be justified from other assertions, axioms or
% definitions. This environment relegates the formatting to the |STaxiomEnv| environment,
% which can be redefined for configuration.
% 
% \DescribeMacro{\inlineass} Sometimes we state mathematical properties in passing,
% e.g. in a phrase like ``\ldots $s(o)$ which is positive.''. For this we cannot use the
% |assertion| environment, which presupposes that its content gives all that is needed to
% understand the statement. In this situation, we just wrap the phrase in an |\inlineass|
% to mark it as an assertion. The |\inlinedef| macro accepts the same |id| and |for|
% keys in its optional argument, and additionally the |verbalizes| key which can be used
% to point to a full assertion of the concept somewhere else.
%
% The |\inlineass| macro has an environment
% version\DescribeMacro{inlineAssertion}|inlineAssertion|, which does the same, but can
% digest ``block content''. The most common case is when the inline definition contains a
% displayed equation (|\[...\]|). 
%
% \subsubsection{Symbols}\label{sec:user:symbol}
%
% \DescribeEnv{symboldec} The |symboldec| environment can be used for declaring concepts
% and symbols. Note the the |symdef| forms from the |modules| package will not do this
% automatically (but the |definition| environment and the |\inlinedef| macro will for all
% the definienda; see below). The |symboldec| environment takes an optional keywords
% argument with the keys |id|, |role|, |title| and |name|. The first is for general
% identification, the |role| specifies the {\openmath}/{\omdoc} role, which is one of
% |object|, |type|, |sort|, |binder|, |attribution|, |application|, |constant|,
% |semantic-attribution|, and |error| (see the {\omdoc} specification for details). The
% |name| key specifies the {\openmath} name of the symbol, it should coincide with the
% control sequence introduced by the corresponding |\symdef| (if one is present). The
% |title| key is for presenting the title of this symbol as in other statements. Usually,
% |axiom| and |symboldec| environments are used together as in Figure~\ref{fig:axioms}.
%
%\begin{exfig}
% \begin{verbatim}
% \symdef{zero}{0}
% \begin{symboldec}[name=zero,title=The number zero,type=constant]
%   The number zero, it is used as the base case of the inductive definition
%   of natural numbers via the Peano Axioms.
% \end{symboldec}
% 
% \symdef{succ}[1]{\prefix{s}{#1}}
% \begin{symboldec}[name=succ,title=The Successor Function,type=application]
%   The successor function, it is used for the step case of the inductive
%   definition of natural numbers via the Peano Axioms.
% \end{symboldec}
%
% \symdef{NaturalNumbers}{\mathbb{N}}
% \begin{symboldec}[name=succ,title=The Natural Numbers,type=constant]
%   The natural numbers inductively defined via the Peano Axioms.
% \end{symboldec}
% 
% \begin{axiom}[id=peano.P1,title=P1]
%   $\zero$ is a natural number.
% \end{axiom}
% ...
% \begin{axiom}[id=peano.P5,title=P5]
%   Any property $P$ such $P(\zero)$ and $P(\succ{k})$ whenever $P(k)$ 
%   holds for all $n$ in $\NaturalNumbers$
% \end{axiom}
% \end{verbatim}
% \vspace{-1em}will lead to the result\medskip\par\noindent
% \begin{module}[id=peano]
% \symdef{zero}{0}
% \begin{symboldec}[name=zero,title=The number zero,role=constant]
%   The number zero, it is used as the base case of the inductive definition
%   of natural numbers via the Peano Axioms.
% \end{symboldec}
% 
% \symdef{succ}[1]{\prefix{s}{#1}}
% \begin{symboldec}[name=succ,title=The Successor Function,role=application]
%   The successor function, it is used for the step case of the inductive
%   definition of natural numbers via the Peano Axioms.
% \end{symboldec}
%
% \symdef{NaturalNumbers}{\mathbb{N}}
% \begin{symboldec}[name=succ,title=The Natural Numbers,role=constant]
%   The natural numbers inductively defined via the Peano Axioms.
% \end{symboldec}
% 
% \begin{axiom}[id=peano.P1,title=P1]
%   $\zero$ is a natural number.
% \end{axiom}
% \ldots \stepcounter{STtheoremAssEnv}\stepcounter{STtheoremAssEnv}\stepcounter{STtheoremAssEnv}
% \begin{axiom}[id=peano.P5,title=P5]
%   Any property $P$ such $P(\zero)$ and $P(\succ{k})$ whenever $P(k)$ 
%   holds for all $n$ in $\NaturalNumbers$
% \end{axiom}
% \end{module}
% \caption{Semantic Markup for the Peano Axioms}\label{fig:axioms}
% \end{exfig}
%
% \subsubsection{Types}\label{sec:user:types}
% 
% In many cases, we can give additional information for symbols in the form of type
% assignments. \sTeX does not fix a type system, but allows types to be arbitrary
% mathematical objects that they can be defined in (imported) modules. The
% \DescribeMacro{\symtype}|\symtype| macro can be used to assign a type to a symbol: 
% \begin{quote}
%   |\symtype[|\meta{keys}|]{|\meta{sym}|}{|\meta{type}|}|
% \end{quote}
% assigns the type \meta{type} to a symbol with name \meta{sym}. For instance
%
% \begin{center}%
% |\symtype[id=plus-nat.type,system=sts]{plus}{\fntype{\Nat,\Nat}\Nat}|
% \end{center}
% 
% \noindent assigns the type $\mathbb{N}\times\mathbb{N}\rightarrow\mathbb{N}$ (in the
% |sts| type system) to the symbol |plus|. This states (type assignments are statements
% epistemologically) that addition is a binary function on natural numbers. The |\symtype|
% macro supports the keys |id| (for identifiers) and |system| for the type system.
% 
% Often, type assignments occur in informal context, where the type assignment is given by
% a natural language sentence or phrase. For this, the |statements| package supplies the
% \DescribeEnv{typedec}|typedec| environment and the
% \DescribeMacro{\inlinetypedec}|\inlinetypedec| macro. Both take an optional keyval
% argument followed by the type. The phrase/sentence is the body of the |typedec|
% environment and the last argument of the |\inlinetypedec| macro. The symbol name is
% given in via the |for| key. For convenience, the macro
% \DescribeMacro{\thedectype}|\thedectype| is bound to the type. So we can use
%\begin{verbatim}
% \begin{typedec}[for=plus,id=plus-nat.type]{\fntype{\Nat,\Nat}\Nat}
%   $+:\thedectype$ is a binary function on $\Nat$
% \end{typedec}
% \end{verbatim}
% instead of the |\symtype| above in an informal setting. 
% 
% \subsubsection{Definitions, and Definienda}\label{sec:definition}
% 
% \DescribeEnv{definition} The |definition| environment is used for marking up
% mathematical definitions. Its peculiarity is that it defines (i.e. gives a meaning to)
% new mathematical concepts or objects. These\DescribeMacro{\definiendum} are identified
% by the |\definiendum| macro, which is used as
% |\definiendum[|\meta{keys}|]{|\meta{text}|}|. Here, \meta{text} is the text that is to
% be emphasized in the presentation. |\definiendum| takes the key
% \DescribeMacro{lemma}|lemma| allows to specify the base form of the name of \meta{text}
% -- e.g. for referencing in a glossary or index.  The \DescribeMacro{name}|name| can be
% used for giving a system name of the symbol defined (for reference via |\termref|, see
% Section~\ref{sec:user:crossref}).  If the |name| key is not given, then the value of the
% |lemma| key is used as a system name instead if it is given, else \meta{text} is used
% as fallback. This is usually sufficient for most situations. More keys -- e.g. for
% specifying grammatical features of the term -- may come in the future.
%
%\begin{exfig}
%  \def\succ#1{s(#1)}
% \begin{verbatim}
% \symdef{one}{1}
% \begin{definition}[id=one.def,for=one]
%   $\notatiendum[one]{\one}$ is the successor of $\zero$
%   (formally: $\one :=\succ\zero$)
% \end{definition}
% \end{verbatim}
% \vspace{-1em}will lead to the result\medskip\par\noindent\vspace*{-1em}
% \begin{module}[id=onedef]
% \importmodule{peano}
% \symdef{one}{1}
% \begin{definition}[id=one.def,for=one]
%   $\notatiendum[one]{\one}$ is the successor of $\zero$
%   (formally: $\one :=\succ\zero$)
% \end{definition}
% \end{module}
% \caption{A Definition based on Figure {\ref{fig:axioms}}}\label{fig:definition}
% \end{exfig}
% The \DescribeMacro{\defi}|\defi{|\meta{text}|}| macro combines the functionality of the
% |\definiendum| macro with index markup from the |omdoc|
% package~\ctancite{Kohlhase:smomdl}: For definienda where the name and \meta{text} do not
% coincide, use
% \begin{center}
% |\defi[|name=\meta{name}|]{|\meta{text}|}[|\meta{indexkeys}|]|
% \end{center}
% to markup a definiendum \meta{text} with system name \meta{name} that appear in the
% index (where \meta{indexkeys} are passed to the |\omdoc@index*| macros from the |omtext|
% package) --- in other words in almost all definitions of single-word concepts. Again
% more keys for the first optional argument -- e.g. for specifying grammatical features of
% the term -- may come in the future.
%
% For definitions of functional objects, e.g. the image of a function, we have to extend
% the simple infrastructure by a facility of marking up argument: we have the situation in
% Figure~\ref{dig:pdefi}.
% 
% \begin{exfig}
% \begin{verbatim}
% \symde{imageof}[2]{#1(#2)}
% \symdef{image}[1]{\text{Im}(#1)}
% \begin{definintion}
%   Let $F:A\rightarrow B$ be a function and  $A'\sseteq A$, then  we call 
%   \begin{itemize}
%   \item $\imageof{f}A:=\{f(a)\in B\colon a\in A\}$ the
%     \defi[name=imageof]{image}$ \pdefi1[the set]{of}{$\primvar{A}$}
%     \pdefi2[the function]{under}{$f$}.
%   \item $\image{f}:=\imageof{f}A$ the \defi{image} $\image{f}$
%      \pdefi1[the function]{of}{$f$}
%   \end{itemize}
% \end{definintion}
% \end{verbatim}
% \begin{module}[id=foo]
% \begin{definition}[display=none]
% \newcommand\imageof[2]{#1(#2)}
% \newcommand\image[1]{\text{Im}(#1)}
% Let $F:A\rightarrow B$ be a function and  $A'\subseteq A$, then  we call 
% \begin{itemize}
% \item $\imageof{f}A:=\{f(a)\in B\colon a\in A\}$ the \defi[name=imageof]{image}
%    \pdefi1[the set]{of}{$\primvar{A}$} \pdefi2[the function]{under}{$f$}.
%  \item $\image{f}:=\imageof{f}A$ the \defi{image} $\image{f}$
%    \pdefi1[the function]{of}{$f$}
%  \end{itemize}
% \end{definition}
% \end{module}
% \caption{Definition Markup for Functional Objects}\label{fig:pdefi}
% \end{exfig}
%
% We also have the variants \DescribeMacro{\defii}|\defii|,
% \DescribeMacro{\defiii}|\defiii|, and \DescribeMacro{\defiv}|\defiv| for (adjectivized)
% multi-word compounds.  Note that if the definiendum contains semantic macros, then we
% need to specify the |loadmodules| key and also protect the semantic macro. For instance
% if |\eset| is the semantic macro for $\emptyset$, then we would use
% \begin{verbatim}
% \defii[name=eset-comp]{$\protect\eset$}{compatible}[loadmodules]
% \end{verbatim}
% for the definiendum markup. 
%
% \begin{exfig}
% \begin{verbatim}
% A \defi{graph} consists of \adefi{vertices}{vertex}  and \defis{edge}. 
% \end{verbatim}
% \caption{Definienda where Lemma and Text Form differ}\label{fig:lemma}
% \end{exfig}
%
% For the cases where the lemma and \meta{text} are different we can use the variants
% \DescribeMacro{\adefi}|\adefi|, \DescribeMacro{\adefii}|\adefii|,
% \DescribeMacro{\adefiii}|\adefiii|, and \DescribeMacro{\adefiv}|\adefiv| that have
% an additional first argument that allows to specify an alternative \meta{text}; see
% Figure~\ref{fig:defin}. The main use of these is to mark up inflected forms as in
% Figure~\ref{fig:lemma}.
%
% As the greatest number of these are plurals, which tends to be regular (e.g. adding a
% trailing ``s'' in English), we provide the variants \DescribeMacro{\defis}|\defis|,
% \DescribeMacro{\defiis}|\defiis|, \DescribeMacro{\defiiis}|\defiiis|, and
% \DescribeMacro{\defivs}|\defivs| for that case: |\defiis{simple}{group}| is
% equivalent to much longer |\adefii{simple groups}{simple}{group}| (but also see
% Figure~\ref{fig:lemma}).
% 
% \begin{exfig}
% \begin{tabular}{l|l|l}
%   \multicolumn{3}{l}{source}\\\hline
%   system name & result & index \\\hline\hline
%   \multicolumn{3}{l}{\texttt{\textbackslash defi\{concept\}}}\\\hline
%   |concept| & concept& concept\\\hline\hline
%   \multicolumn{3}{l}{\texttt{\textbackslash defi[name=csymbol]\{concept\}}}\\\hline
%   |csymbol| & concept & concept\\\hline\hline
%   \multicolumn{3}{l}{\texttt{\textbackslash adefi[name=csymbol]\{concepts\}\{concept\}}}\\\hline
%   |csymbol| & concepts & concept\\\hline\hline
%   \multicolumn{3}{l}{\texttt{\textbackslash defii\{concept\}\{group\}}}\\\hline
%   |concept-group| & concept group & concept group,  \\
%   && group - , concept\\\hline\hline
%   \multicolumn{3}{l}{\texttt{\textbackslash defiii\{small\}\{concept\}\{group\}}}\\\hline
%   |small-concept-group| & small concept group &  small concept group,  \\
%   && concept group - , small\\\hline
% \end{tabular}
% \caption{Some definienda with Index}\label{fig:defin}
% \end{exfig}
%
% For convenience, we also have capitalizing versions of all of the above:
% \DescribeMacro{\Defi*} |\Defi*| and \DescribeMacro{\Defi*}|\Defi*s|.
%
% Note that the |\definiendum|, |\defi*|, |\adefi*|, and |\defi*s|, macros can only be
% used inside the definitional situation, i.e. in a |definition| or |symboldec|
% environment or a |\inlinedef| macro. If you find yourself in a situation where you want
% to use it outside, you will most likely want to wrap the appropriate text fragment in a
% |\begin{definition}[display=flow]| ... and |\end{definition}|. For instance, we could
% continue the example in Figure~\ref{fig:axioms} with the |definition| environment in
% Figure~\ref{fig:definition}.
% 
% \DescribeMacro{\inlinedef} Sometimes we define mathematical concepts in passing, e.g. in
% a phrase like ``\ldots $s(o)$ which we call {\textbf{one}}.''. For this we cannot use
% the |definition| environment, which presupposes that its content gives all that is
% needed to understand the definition. But we do want to make use of the infrastructure
% introduced for the |definition| environment. In this situation, we just wrap the phrase
% in an |\inlinedef| macro that makes them available. The |\inlinedef| macro accepts the
% same |id| and |for| keys in its optional argument, and additionally the |verbalizes| key
% which can be used to point to a full definition of the concept somewhere else.
%
% The |\inlinedef| macro has an environment
% version\DescribeMacro{inlineDefinition}|inlineDefinition|, which does the same, but can
% digest ``block content''. The most common case is when the inline definition contains a
% displayed equation (|\[...\]|). 
%
% Note that definienda can only be referenced via a |\term| element, if they are only
% allowed inside a named module, i.e. a |module| environment with a name given by the
% |id=| key or the |theory=| key on is specified on the definitional environment.
%
% \subsubsection{Examples}\label{sec:user:example}
% 
% \DescribeEnv{example} The |example| environment is a generic statement environment,
% except that the |for| key should be given to specify the identifier what this is an
% example for. The |example| environment also expects a |type| key to be specified, so
% that we know whether this is an example or a counterexample.
% 
% \DescribeMacro{\inlineex} The |\inlineex| is analogous to |\inlinedef|, only that it is
% used for inline examples, e.g. ``\ldots mammals, e.g. goats''. Note that we have used an
% inline example for an inline example.  Like |\inlinedef|, the |\inlineex| macro has an
% environment version \DescribeMacro{inlineExample}|inlineExample| for ``block content''.
%
% \subsection{Cross-Referencing Symbols and Concepts}\label{sec:user:crossref}
%
% If we have defined a concept with the |\definiendum| macro, then we can mark up other
% occurrences of the term as referring to this concept. Note that this process cannot be
% fully automatized yet, since that would need advanced language technology to get around
% problems of disambiguation, inflection, and non-contiguous phrases\footnote{We do have a
%   program that helps annotate larger text collections spotting the easy cases; see
%   {\url{http://kwarc.info/projects/stex}} and look for the program
%   |termin|.}. Therefore, the \DescribeMacro{\termref}|\termref| can be used to make this
% information explicit. It takes the keys
% \begin{compactdesc}
% \item[\texttt{cdbase}] to specify a URI (a path actually, since {\LaTeX} cannot load
%   from URIs) where the module can be found.
% \item[\texttt{cd}] to specify the module in which the term is defined. If the |cd| key
%   is not given, then the current module is assumed. If no |cdbase| is specified (this is
%   the usual case), then the CD has to be imported via a |\importmodule| from the
%   |modules| package~\ctancite{KohAmb:smmssl}.
% \item[\texttt{name}] to specify the name of the definiendum (which is given in the body
%   of the |\definiendum| or the optional argument). If the |name| key is not specified,
%   then argument of the |\termref| macro is used.
% \item[\texttt{role}] is currently unused.
% \end{compactdesc}
% |\termref[cd=|\meta{cd}|,name=|\meta{name}|]{|\meta{text}|}| will just typeset the link
% text \meta{text} with (if the |hyperref| package is loaded) a hyperlink to the
% definition in module \meta{cd} that defines the concept \meta{name}, e.g. that contains
% |\defi[|name=\meta{name}|]{|\meta{text}|}|.
%
% Just as the |\definiendum| macro has the convenience variants |\defi| and |\?defi*|, the
% |\termref| has variants |\trefi|, |\trefii|, |\trefiii|, and |\trefiv| that take two and
% three arguments for the parts of the compositum. In the same module, concepts that are
% marked up by |\defi{|\meta{name}|}| in the definition can be referenced by
% \DescribeMacro{\trefi}|\trefi{|\meta{name}|}|. Here the link text is just
% \meta{name}. Concepts defined via |\defii{|\meta{first}|}{|\meta{second}|}| can be
% referenced by \DescribeMacro{\trefii}|\trefii{|\meta{first}|}{|\meta{second}|}| (with
% link text ``\meta{first} \meta{second}'') and analogously for
% |\defiii|/\DescribeMacro{\trefiii}|\trefiii| and % |\defiv|/\DescribeMacro{\trefiv}|\trefiv|.
% 
% For referencing terms outside the current module, the module name can be specified in
% the first optional argument of the |\*trefi*| macros. The first argument it syntactically
% optional to keep the parallelism to |\*defi*| and |\*trefi*|. To specify the |cdbase|, we
% have to resort to the |\termref| macro with the keyval arguments.
%
% For term references, where the symbol name and the verbalisation differ the situtation
% is more complex: The optional first argument can be used to specify the symbol via its
% name \meta{name} and module name \meta{mod} in a MMT URI \meta{mod}|?|\meta{name}. Note
% that MMT URIs can be relative:
% \begin{enumerate}
% \item |foo?bar| denotes the symbol |bar| from module |foo|
% \item |foo| the module |foo| (the symbol name is induced from the remaining
%   arguments of |\*trefi*|)
% \item |?bar| specifies symbol |bar| from the current module
% \end{enumerate}
% Note that the number suffix |i|/|ii|/|iii|/|iv| indicates the number of words in the
% actual language binding, not in the symbol name.
%
% Note that the |\termref| treatment above is natural for ``concepts'' declared by the
% |\termdef| macro from the |modules| package~\ctancite{KohAmb:smmssl}. Concepts are
% natural language names for mathematical objects. For ``symbols'', i.e. symbolic
% identifiers for mathematical objects used in mathematical formulae, we use the |\symdef|
% macro from the |modules| package. Sometimes, symbols also have an associated natural
% language concept, and we want to use the symbol name to reference it (instead of
% specifying |cd| and |name| which is more inconvenient). For this the |statements|
% package supplies the \DescribeMacro{\symref}|\symref| macro. Like |\termref|, and
% invocation of |\symref{|\meta{cseq}|}{|\meta{text}|}| will just typeset \meta{text} with
% a hyperlink to the relevant definition (i.e. the one that has the declaration
% |for=|\meta{cseq} in the metadata argument.)
% 
% The \DescribeMacro{\term}|\term| macro is a variant of the |\termref| macro that marks
% up a phrase as a (possible) term reference, which does not have a link \emph{yet}. This
% macro is a convenient placeholder for authoring, where a |\termref| annotation is
% (currently) too tedious or the link target has not been authored yet. It facilitates
% lazy flexiformalization workflows, where definitions for mathematical concepts are
% supplied or marked up by need (e.g. after a |grep| shows that the number of |\term|
% annotations of a concept is above a threshold). Editors or active documents can also
% support the |\term| macro like a wiki-like dangling link: a click on
% |\term{|\meta{phrase}|}| could generate a new editor buffer with a stub definition (an
% |definition| environment with |\definiendum| macro and appropriate metadata).\ednote{MK:
% we probably need multi-part variants for |?tref*|}
% 
% \subsection{Term Redefinition in Recaps}\label{sec:user:drefi}
%
% In many situations, the author ``recaps'' -- i.e. repeats, possibly in abbreviated or
% adapted form -- a definition from the literature.  Linguistically, these recaps often
% come in the form of definitions, but epistemically, the definienda are just term
% references to the definitions in the literature; we call them \textbf{definition
% recaps}, see \cite {IanKoh:fmkdam15} for a discussion.  To accomodate this, we supply
% the \DescribeMacro{recaps}|recaps| key for the |\definiendum| and thus |\defi*|
% macros. The full functionality:
% |\definiendum[name=foo,recaps=bar?foobar,lemma=FOO]{FOOta}|, which defines a symbol with
% name |foo| and lemma |FOO| in the current module, and which also recaps the symbol
% |foobar| from module |bar| and shows the definiendum |FOOta| -- ostensibly an inflected
% form of |FOO| -- is almost never needed.
%
% Therefore the |statements| package offers the \DescribeMacro{\drefi*}|\drefi*| macros
% and variants \DescribeMacro{\drefi*s}|\drefi*s| for plurals and
% \DescribeMacro{\Drefi*}|\Drefi*| for capitalization as above. These provide syntactic
% sugar for the most important forms: If \meta{uri} contains a |?|, then
% |\drefi[|\meta{uri}|]{|\meta{name}|}| abbreviates
% |\definiendum[recaps=|\meta{uri}|]{|\meta{name}|}| otherwise \meta{uri} represents a
% theory \meta{thy} and |\drefi[|\meta{thy}|]{|\meta{name}|}| abbreviates
% |\defiendum[recaps=|\meta{thy}|?|\meta{name}|]{|\meta{name}|}|.
%
% Figure~\ref{fig:drefi} shows a typical situation: We have an external theory |graphs| --
% here in the same file for example convenience, and a recap slide in a presentation,
% which imports or uses this theory. This has a |definition| which gives a telegraphic
% version of the graph definition from |graphs|. This assumes that theory |graphs| is
% known from above. An active document player could support this situation, by linking the
% |\drefi*|-encoded definienda and term references to them to their definitions from the
% |graphs| theory.
%
% \begin{exfig}
% \begin{verbatim}
% \begin{module}[id=graphs]
%   \begin{definition}
%     A \defi{graph} $G=\langle V,E\rangle$ consists of a set $V$ of
%     \adefi{vertices}{vertex} and a set $E\subseteq V\times V$ of
%     \defis{edge}.
%   \end{definition}
% \end{module}
% ...
% \begin{frame}
%   \frametitle{Preliminaries: The Relevant Notations}
%   \usemodule{graphs}
%   \begin{definition}[title=Recap: Graphs]
%     A \drefi[graphs]{graph} consists of \drefi[graphs?vertex]{vertices}
%     and \drefis[graphs]{edge}. 
%   \end{definition}
% \end{frame}
% \end{verbatim}
%   \caption{Redefinitions in Recaps}\label{fig:drefi}
% \end{exfig}
% 
% \section{Configuration of the Presentation}\label{sec:conf}
% 
% \DescribeMacro{\defemph} The |\defemph| macro is a configuration hook that allows to
% specify the style of presentation of the {\index*{definiendum}}. By default, it is set to
% |\bf| as a fallback, since we can be sure that this is always available. It can be
% customized by  redefinition: For instance |\renewcommand{\defemph}[1]{\emph{#1}}|,
% changes the default behavior to italics.
%
% \DescribeMacro{\termemph} The |\termenph| macro does the same for the style for
% |\termref|, it is empty by default. Note the term might carry an implicit hyper-reference
% to the defining occurrence and that the presentation engine might mark this up, changing
% this behavior.
%
% \DescribeMacro{\stDMemph} The |\stDMemph| macro does the same for the style for the
% markup of the discourse markers like ``Theorem''. If it is not defined, it is set to
% |\bf|; that allows to preset this in the class file. \ednote{function declarations}
% 
% Some authors like to lowercase the semantic references, i.e. use ``axiom 2.6'' instead
% of the default ``\sref{peano.P5}'' to refer to the last axiom in
% Figure~\ref{fig:axioms}. This can be achieved by redefining the
% \DescribeMacro{\STpresent}|\STpresent| macro, which is applied to the keyword of the
% |ST*Env| theorem environments.\ednote{this does not quite work as yet, since
%   \textbf{STpresent} is applied when the label is written. But we would really like to
%   have it applied when the reference is constructed. But for that we need to split the
%   label into keyword and number in package |sref|.}
% 
%   Finally, we provide configuration hooks in Figure~\ref{fig:hooks} for the statement
%   types provided by the |statement| package. These are mainly intended for package
%   authors building on |statements|, e.g. for multi-language support. The language
%   bindings are given in the |smultiling|~\ctancite{KohGin:smss} package not in
%   |statements| itself.
% 
%\begin{exfig}
% \begin{tabular}{lll}
% Environment & configuration macro & value\\\hline\hline
% \texttt{STtheoremAssEnv} & \texttt{\textbackslash st@theorem@kw} & \makeatletter\st@theorem@kw\\\hline
% \texttt{STlemmaAssEnv}  & \texttt{\textbackslash st@lemma@kw}  & \makeatletter\st@lemma@kw \\\hline
% \texttt{STpropositionAssEnv} & \texttt{\textbackslash st@proposition@kw} &  \makeatletter\st@proposition@kw \\\hline
% \texttt{STcorollaryAssEnv} & \texttt{\textbackslash st@corollary@kw} & \makeatletter\st@corollary@kw\\\hline
% \texttt{STconjectureAssEnv} & \texttt{\textbackslash st@conjecture@kw} & \makeatletter\st@conjecture@kw\\\hline
% \texttt{STfalseconjectureAssEnv} & \texttt{\textbackslash st@falseconjecture@kw} & \makeatletter\st@falseconjecture@kw\\\hline
% \texttt{STpostulateAssEnv} & \texttt{\textbackslash st@postulate@kw} & \makeatletter\st@postulate@kw\\\hline
% \texttt{STobligationAssEnv} & \texttt{\textbackslash st@obligation@kw} & \makeatletter\st@obligation@kw\\\hline
% \texttt{STassumptionAssEnv} & \texttt{\textbackslash st@assumption@kw} & \makeatletter\st@assumption@kw\\\hline
% \texttt{STobservationAssEnv} & \texttt{\textbackslash st@observation@kw} & \makeatletter\st@observation@kw\\\hline
% \texttt{STremarkAssEnv} & \texttt{\textbackslash st@remark@kw} & \makeatletter\st@remark@kw\\\hline
% \texttt{STruleAssEnv} & \texttt{\textbackslash st@rule@kw} & \makeatletter\st@rule@kw\\\hline
% \texttt{STexampleEnv} & \texttt{\textbackslash st@example@kw} & \makeatletter\st@example@kw\\\hline
% \texttt{STaxiomEnv} & \texttt{\textbackslash st@axiom@kw} & \makeatletter\st@axiom@kw\\\hline
% \texttt{STdefinitionEnv} & \texttt{\textbackslash st@definition@kw} & \makeatletter\st@definition@kw\\\hline
% \texttt{STnotationEnv} & \texttt{\textbackslash st@notation@kw} & \makeatletter\st@notation@kw
% \end{tabular}
% \caption{Configuration Hooks for statement types}\label{fig:hooks}
% \end{exfig}
%
% \section{Limitations}\label{sec:limitations}
% 
% In this section we document known limitations. If you want to help alleviate them,
% please feel free to contact the package author. Some of them are currently discussed in
% the \sTeX GitHub repository~\cite{sTeX:github:on}. 
% \begin{enumerate}
% \item none reported yet
% \end{enumerate}
% 
% \StopEventually{\newpage\PrintIndex\newpage\PrintChanges\printbibliography}
% 
% \section{The Implementation}\label{sec:impl} 
% 
% \subsection{Package Options}\label{sec:impl:options}
%
% We declare some switches which will modify the behavior according to the package
% options. Generally, an option |xxx| will just set the appropriate switches to true
% (otherwise they stay false). First we have the general options: |msection| specifies
% that theorems should be numbered in the |msection| counter provided by the |mikoslides|
% package/class.
%    \begin{macrocode}
%<*package>
\newif\if@modules@html@\@modules@html@true
\DeclareOption{omdocmode}{\@modules@html@false}
\newif\ifdef@index\def@indexfalse
\DeclareOption{defindex}{\def@indextrue}
\newif\if@nthm\@nthmtrue
\DeclareOption{nontheorem}{\@nthmfalse}
\newif\if@msection\@msectionfalse
\DeclareOption{msection}{\@msectiontrue}
\DeclareOption*{\PassOptionsToPackage{\CurrentOption}{omtext}}
\ProcessOptions
%    \end{macrocode}
% 
% The next measure is to ensure that some {\sTeX} packages are loaded: |omdoc| for the
% statement keys, |modules| since we need module identifiers for referencing. Furthermore,
% we need the |ntheorem| package for presenting statements.
%    \begin{macrocode}
\RequirePackage{omtext}
\RequirePackage[base]{babel}
\ifcsdef{proof}{\cslet{proof}{\relax}\cslet{endproof}{\relax}}{}% to redefine if necessary
\if@nthm
\RequirePackage[hyperref]{ntheorem}
\theoremstyle{plain}
\else
\RequirePackage{amsthm}
\fi
%    \end{macrocode}
% Now, we define an auxiliary function that lowercases strings
%    \begin{macrocode}
%    \end{macrocode}
% Sometimes it is necessary to fallback to symbol names in order to generate xml:id
% attributes. For this purpose, we define an auxiliary function which ensures the name
% receives a unique NCName equivalent.\ednote{Hard to be unique here, e.g. the names
% "foo\_bar" and "foo bar" would receive the same xml:id attributes... of course we can
% devise a more complex scheme for the symbol replacement.}
%    \begin{macrocode}
%    \end{macrocode}
%
% For the other languages, we set up triggers
%
%    \begin{macrocode}
\AfterBabelLanguage{ngerman}{\input{statements-ngerman.ldf}}
\AfterBabelLanguage{finnish}{\input{statements-finnish.ldf}}
\AfterBabelLanguage{french}{\input{statements-french.ldf}}
\AfterBabelLanguage{russian}{\input{statements-russian.ldf}}
%    \end{macrocode}
%
% \subsection{Statements}\label{sec:impl:statements}
%
% \begin{macro}{\STpresent}
%    \begin{macrocode}
\providecommand\STpresent[1]{#1}
%    \end{macrocode}
% \end{macro}
%
% \begin{macro}{\define@statement@env}
%    We define a meta-macro that allows us to define several variants of statements. Upon
%    beginning this environment, we first set the |KeyVal| attributes, then we decide
%    whether to print the discourse marker based on the value of the |display| key, then
%    (given the right Options were set), we show the semantic annotations, and finally
%    initialize the environment using the appropriate macro. Upon ending the environment,
%    we just run the respective termination macro.
%    \begin{macrocode}
\def\define@statement@env#1{%
\ifcsdef{#1}{\cslet{#1}{\relax}\cslet{end#1}{\relax}}{}% to redefine if necessary
\newenvironment{#1}[1][]{\metasetkeys{omtext}{##1}\sref@target\@in@omtexttrue%
\ifx\omtext@display\st@flow\def\@@env{omtext}\else\def\@@env{ST#1Env}%
\csname st@#1@initialize\endcsname\fi% display=flow
\ifx\omtext@title\@empty\begin{\@@env}\else\begin{\@@env}[\omtext@title]\fi%
\ifx\sref@id\@empty\sref@label@id{here}\else%
\sref@label@id{\STpresent{\csname st@#1@kw\endcsname}~\@currentlabel}\fi%
\strut\ignorespacesandpars}
{\csname st@#1@terminate\endcsname\end{\@@env}%
\omtext@post@skip\@in@omtextfalse}}
%    \end{macrocode}
% \end{macro}
%
% \begin{environment}{assertion}
%    \begin{macrocode}
\newenvironment{assertion}[1][]{\metasetkeys{omtext}{#1}\sref@target\@in@omtexttrue%
\ifx\omtext@display\st@flow\def\@@env{omtext}\else\def\@@env{ST\omtext@type AssEnv}\fi
\ifx\omtext@title\@empty\begin{\@@env}\else\begin{\@@env}[\omtext@title]\fi%
\ifx\sref@id\@empty\sref@label@id{here}\else%
\sref@label@id{\STpresent{\csname st@\omtext@type @kw\endcsname}~\@currentlabel}\fi}
{\end{\@@env}}
%    \end{macrocode}
% \end{environment}
%
% \begin{macro}{\st@*@kw}
%   We configure the default keywords for the various theorem environments. 
%    \begin{macrocode}
\def\st@theorem@kw{Theorem}
\def\st@lemma@kw{Lemma}
\def\st@proposition@kw{Proposition}
\def\st@corollary@kw{Corollary}
\def\st@conjecture@kw{Conjecture}
\def\st@falseconjecture@kw{Conjecture (false)}
\def\st@postulate@kw{Postulate}
\def\st@obligation@kw{Obligation}
\def\st@assumption@kw{Assumption}
\def\st@rule@kw{Rule}
\def\st@observation@kw{Observation}
\def\st@remark@kw{Remark}
%    \end{macrocode} 
%
% \end{macro}
% Then we configure the presentation of the theorem environments
%    \begin{macrocode}
\if@nthm
\theorembodyfont{\itshape}
\theoremheaderfont{\normalfont\bfseries}
\else
\theoremstyle{plain}
\fi
%    \end{macrocode}
%
% \begin{environment}{ST*AssEnv}
%   We define a number of internal assertion environments according to the values of its
%   |type| key. 
%    \begin{macrocode}
\if@msection
\newtheorem{STtheoremAssEnv}{\st@theorem@kw}[msection]
\else
\ifdef{\thesection}
{\newtheorem{STtheoremAssEnv}{\st@theorem@kw}[section]}
{\newtheorem{STtheoremAssEnv}{\st@theorem@kw}}
\fi
\newtheorem{STlemmaAssEnv}[STtheoremAssEnv]{\st@lemma@kw}
\newtheorem{STpropositionAssEnv}[STtheoremAssEnv]{\st@proposition@kw}
\newtheorem{STcorollaryAssEnv}[STtheoremAssEnv]{\st@corollary@kw}
\newtheorem{STconjectureAssEnv}[STtheoremAssEnv]{\st@conjecture@kw}
\newtheorem{STfalseconjectureAssEnv}[STtheoremAssEnv]{\st@falseconjecture@kw}
\newtheorem{STpostulateAssEnv}[STtheoremAssEnv]{\st@postulate@kw}
\newtheorem{STobligationAssEnv}[STtheoremAssEnv]{\st@obligation@kw}
\newtheorem{STassumptionAssEnv}[STtheoremAssEnv]{\st@assumption@kw}
\newtheorem{STobservationAssEnv}[STtheoremAssEnv]{\st@observation@kw}
\if@nthm\theorembodyfont{\rmfamily}\else\theoremstyle{definition}\fi
\newtheorem{STremarkAssEnv}[STtheoremAssEnv]{\st@remark@kw}
\newtheorem{STruleAssEnv}[STtheoremAssEnv]{\st@rule@kw}
%    \end{macrocode}
% \end{environment}
%
% \begin{environment}{example}
%    \begin{macrocode}
\def\st@example@initialize{}\def\st@example@terminate{}
\define@statement@env{example}
\def\st@example@kw{Example}
\newtheorem{STexampleEnv}[STtheoremAssEnv]{\st@example@kw}
%    \end{macrocode}
% \end{environment}
%
% \begin{environment}{axiom}
%    \begin{macrocode}
\def\st@axiom@initialize{}\def\st@axiom@terminate{}
\define@statement@env{axiom}
\def\st@axiom@kw{Axiom}
\newtheorem{STaxiomEnv}[STtheoremAssEnv]{\st@axiom@kw}
%    \end{macrocode}
% \end{environment}
%
% \begin{environment}{symboldec}
%   We use |\symdef@type| from the |modules| package as the visual cue. 
%    \begin{macrocode}
\srefaddidkey{symboldec}
\addmetakey{symboldec}{functions}
\addmetakey{symboldec}{role}
\addmetakey*{symboldec}{title}
\addmetakey*{symboldec}{name}
\addmetakey{symboldec}{subject}
\addmetakey*{symboldec}{display}
\newenvironment{symboldec}[1][]{\metasetkeys{symboldec}{#1}\sref@target\st@indeftrue%
\ifx\symboldec@display\st@flow\else{\noindent\stDMemph{\symdef@type} \symboldec@name:}\fi%
\ifx\symboldec@title\@empty~\else~(\stDMemph{\symboldec@title})\par\fi}{}
%    \end{macrocode}
% \end{environment}
% 
% \subsubsection{Types}\label{sec:impl:types}
% 
% \begin{macro}{\symtype}\ednote{MK@DG; the type element should percolate up.}
%    \begin{macrocode} 
\srefaddidkey{symtype}
\addmetakey*{symtype}{system}
\addmetakey*{symtype}{for}
\newcommand\type@type{Type}
\newcommand\symtype[3][]{\metasetkeys{symtype}{#1}\sref@target%
\noindent\type@type \ifx\symtype@\@empty\else (\symtype@system)\fi #2: $#3$}
%    \end{macrocode}
% \end{macro}
%
% \begin{macro}{\inlinetypedec}
%    \begin{macrocode} 
\newcommand\inlinetypedec[3][]{\metasetkeys{symtype}{#1}\sref@target{\def\thedectype{#2}#3}}
%    \end{macrocode}
% \end{macro}
%
% \begin{environment}{typedec}
%   We first define a theorem environment 
%    \begin{macrocode}
\def\st@typedec@kw{Type Declaration}
\newtheorem{STtypedecEnv}[STtheoremAssEnv]{\st@typedec@kw}
%    \end{macrocode}
% and then the environment itself.
%    \begin{macrocode}
\newenvironment{typedec}[2][]{\metasetkeys{omtext}{#1}\sref@target%
\def\thedectype{#2}%
\ifx\omtext@display\st@flow\def\@@env{omtext}\else\def\@@env{STtypedecEnv}\fi%
\ifx\omtext@title\@empty\begin{\@@env}\else\begin{\@@env}[\omtext@title]\fi%
\ifx\sref@id\@empty\else\label{typedec.\sref@id}\fi
\ifx\sref@id\@empty\sref@label@id{here}\else%
\sref@label@id{\STpresent{\csname st@typedec@kw\endcsname}~\@currentlabel}\fi%
\ignorespacesandpars}
{\end{\@@env}\omtext@post@skip}
%    \end{macrocode}
% \end{environment}
%
% \begin{environment}{definition}
%   The |definition| environment itself is quite similar to the other's but we need to set
%   the |\st@indef| switch to suppress warnings from |\st@def@target|.
%    \begin{macrocode}
\newif\ifst@indef\st@indeffalse
\ifcsdef{definition}{\cslet{definition}{\relax}\cslet{enddefinition}{\relax}}{}% to redefine if necessary
\newenvironment{definition}[1][]{\metasetkeys{omtext}{#1}\sref@target\st@indeftrue%
\ifx\omtext@display\st@flow\def\@@env{omtext}\else\def\@@env{STdefinitionEnv}\fi%
\ifx\omtext@title\@empty\begin{\@@env}\else\begin{\@@env}[\omtext@title]\fi%
\ifx\sref@id\@empty\sref@label@id{here}\else%
\sref@label@id{\STpresent{\csname st@definition@kw\endcsname}~\@currentlabel}\fi%
\ignorespacesandpars}
{\end{\@@env}}
\def\st@definition@kw{Definition}
\newtheorem{STdefinitionEnv}[STtheoremAssEnv]{\st@definition@kw}
%    \end{macrocode}
% \end{environment}
%
% \begin{environment}{notation}
%   We initialize the |\def\st@notation@initialize{}| here, and extend it with
%   functionality below. 
%    \begin{macrocode}
\def\notemph#1{#1}
\def\st@notation@terminate{}
\def\st@notation@initialize{}
\define@statement@env{notation}
\def\st@notation@kw{Notation}
\newtheorem{STnotationEnv}[STtheoremAssEnv]{\st@notation@kw}
%    \end{macrocode}
% \end{environment}
%
% \begin{macro}{\st@def@target}
%   the next macro is a variant of the |\sref@target| macro provided by the |sref| package
%   specialized for the use in the |\definiendum|, |\defi*|, |\Defi*|, |\defi*s|, and |\Defi*s|
%   macros. |\st@def@target{|\meta{opt}|}{|\meta{name}|}{|\meta{text}|}| makes a target
%   with label |sref@|\meta{opt}|@|\meta{modulename}|@target|, if \meta{opt} is non-empty,
%   else with the label |sref@|\meta{name}|@|\meta{modulename}|@target| (the first time it
%   encounters this symbol; i.e. if |\sref@|\meta{name}|@|\meta{modulename}|@defined| is
%   undefined). And it formats the |\defemph|-emphasized \meta{text}.  Also it generates
%   the necessary warnings for a definiendum-like macro.
%    \begin{macrocode}
\newcommand\st@def@target[3]{\edef\@symname{#1}\def\@verbname{#2}%
\ifst@indef% if we are in a definition or such
\@ifundefined{mod@id}% if we are not in a module
{\PackageWarning{statements}{definiendum in unidentified module\MessageBreak
\protect\definiendum, \protect\defi*,
\protect\Defi*, \protect\defi*s, \protect\Defi*s\MessageBreak
can only be referenced when called in a module with id key}}%
{% now we are in a module
\edef\@@cd{\ifx\omtext@theory\@empty\mod@id\else\omtext@theory\fi}%
\edef\@@name{\ifx\@symname\@empty\@verbname\else\@symname\fi}%
\defemph{\@ifundefined{sref@\@@name @\@@cd @defined}%
{\expandafter\sref@target@ifh{sref@\@@name @\@@cd @target}{#3}}%
{#3}}%
%\footnote{sTeX: target sref@\@@name @\@@cd @target}% for testing targets
\expandafter\gdef\csname sref@\@@name @\@@cd @defined\endcsname{yes}%
\ifmetakeys@showmeta\metakeys@show@keys{\@@cd}{name:\@@name}\fi}%
\else% st@indef: we are not in a definition or such
\PackageError{statements}%
{definiendum outside definition context\MessageBreak
\protect\definiendum, \protect\defi,
\protect\Defi, \protect\defi*s, \protect\Defi*s\MessageBreak
do not make sense semantically outside a definition.}
{Consider wrapping the defining phrase in a \protect\inlinedef}%
\fi}% st@indef
%    \end{macrocode}
% \end{macro}
%
%   The |\definiendum| and |\notatiendum| macros are very simple.
%
% \begin{macro}{\@termdef}
%   This macro is experimental, it is supposed to be invoked in |\definiendum| to define a
%   macro with the definiendum text, so that can be re-used later in term assignments (see
%   the |modules| package). But in the current context, where we rely on {\TeX} groupings
%   for visibility, this does not work, since the invocations of |\definiendum| are in
%   |definition| environments and thus one group level too low. Keeping this for future
%   reference.
%    \begin{macrocode}
\newcommand\@termdef[2][]{\def\@test{#1}%
\@ifundefined{mod@id}{}{\ifx\@test\@empty\def\@@name{#2}\else\def\@@name{#1}\fi%
\termdef{\mod@id @\@@name}{#2}}}
%    \end{macrocode}
% \end{macro}
%
% \begin{macro}{\definiendum}
%    We define the auxiliary |\@definiendum| macro which has two additionally arguments for
%    the default name and lemma, which can be overwritten by the keys in the first argument.
%    The actual |\definiendum| macro is just a special case without defaults. 
%    \begin{macrocode}
\addmetakey{definiendum}{name}
\addmetakey{definiendum}{lemma}
\addmetakey{definiendum}{recaps}
\newcommand\@definiendum[4][]{\metasetkeys{definiendum}{#1}% keys, text, default-name, default-lemma
\edef\@name{\ifx\definiendum@name\@empty #3\else\definiendum@name\fi}%
\edef\@lemma{\ifx\definiendum@lemma\@empty%
\ifx\definiendum@name\@empty #4\else\definiendum@name\fi%
\else\definiendum@lemma\fi}%
\st@def@target{\@name}{\@lemma}{#2}}
\newcommand\definiendum[2][]{\@definiendum[#1]{#2}{}{}}
%    \end{macrocode}
% \end{macro}
%
% \begin{macro}{\notatiendum}
%   the |notatiendum| macro also needs to be visible in the |notation| and |definition|
%   environments
%    \begin{macrocode}
\newcommand\notatiendum[2][]{\notemph{#2}}
%    \end{macrocode}
% \end{macro}
%
% We expand the {\latexml} bindings for |\defi|, |\defii|, |\defiii| and |\defiv| into two
% instances one will be used for the definition and the other for indexing.
%
% \begin{macro}{\adefi}
%    Again we split the |\adefi| macro into two parts: |\adef| does the definiendum bit and
%    |\@adefi| handles the last optional argument and does the indexing. We use the |\@verb|
%    macro to transport the verbalization. We also factor out the |\adefi@at| macro that sets
%    the |\@@at| register for the  index. This is re-used by |\adefi*| below. 
%    \begin{macrocode}
\newcommand\adefi[3][]{\def\@verb{#2}\@definiendum[#1]{#2}{#3}{#3}\@adefi}
\newcommand\adef@at{%
\edef\@@at{\ifx\definiendum@name\@empty%
\ifx\definienum@lemma\@empty\@verb\else\definiendum@lemma\fi%
\else\definiendum@name\fi}}
\newcommand\@adefi[1][]{\ifdef@index\adef@at\omdoc@indexi[at=\@@at,#1]{\@verb}\fi\xspace}
%    \end{macrocode}
% \end{macro}
%
% \begin{macro}{\defi}
%   We split the |\defi| macro in two: |\defi| does the definiendum bit and |\@defi|
%   handles the last optional argument and does the indexing. The information flow between
%   them goes via the local |\@phrase| macro.
%    \begin{macrocode}
\newcommand\defi[2][]{\adefi[#1]{#2}{#2}}
\newcommand\defis[2][]{\adefi[#1]{#2s}{#2}}
\newcommand\Defi[2][]{\adefi[#1]{\capitalize{#2}}{#2}}
\newcommand\Defis[2][]{\adefi[#1]{\capitalize{#2s}}{#2}}
%    \end{macrocode}
% \end{macro}
%
% \begin{macro}{\adefii}
%   analogous to |\adefi|
%    \begin{macrocode}
\newcommand\adefii[4][]{\def\@pone{#3}\def\@ptwo{#4}%
\def\@name{#3-#4}\def\@verb{#3 #4}%
\@definiendum[#1]{#2}{\@name}{\@verb}\@adefii}
\newcommand\@adefii[1][]{\ifdef@index\adef@at\omdoc@indexii[at=\@@at,#1]{\@pone}{\@ptwo}\fi\xspace}
%    \end{macrocode}
% \end{macro}
%
% \begin{macro}{\defii}
%    \begin{macrocode}
\newcommand\defii[3][]{\adefii[#1]{#2 #3}{#2}{#3}}
\newcommand\defiis[3][]{\adefii[#1]{#2 #3s}{#2}{#3}}
\newcommand\Defii[3][]{\adefii[#1]{\capitalize{#2 #3}}{#2}{#3}}
\newcommand\Defiis[3][]{\adefii[#1]{\capitalize{#2 #3s}}{#2}{#3}}
%    \end{macrocode}
% \end{macro}
%
% \begin{macro}{\adefiii}
%    \begin{macrocode}
\newcommand\adefiii[5][]{\def\@pone{#3}\def\@ptwo{#4}\def\@pthree{#5}%
\def\@verb{#3 #4 #5}\def\@name{#3-#4-#5}%
\@definiendum[#1]{#2}{\@name}{\@verb}\@adefiii}
\newcommand\@adefiii[1][]{\ifdef@index\adef@at\omdoc@indexiii[at=\@@at,#1]{\@pone}{\@ptwo}{\@pthree}\fi\xspace}
%    \end{macrocode}
% \end{macro}
%
% \begin{macro}{\defiii}
%   similar to |\defii|
%    \begin{macrocode}
\newcommand\defiii[4][]{\adefiii[#1]{#2 #3 #4}{#2}{#3}{#4}}
\newcommand\defiiis[4][]{\adefiii[#1]{#2 #3 #4s}{#2}{#3}{#4}}
\newcommand\Defiii[4][]{\adefiii[#1]{\capitalize{#2 #3 #4}}{#2}{#3}{#4}}
\newcommand\Defiiis[4][]{\adefiii[#1]{\capitalize{#2 #3 #4s}}{#2}{#3}{#4}}
%    \end{macrocode}
% \end{macro}
%
% \begin{macro}{\adefiv}
%    \begin{macrocode}
\newcommand\adefiv[6][]{\def\@pone{#3}\def\@ptwo{#4}\def\@pthree{#5}\def\@pfour{#6}%
\def\@name{#3-#4-#5-#6}\def\@verb{#3 #4 #5 #6}%
\@definiendum[#1]{#2}{\@name}{\@verb}\@adefiv}
\newcommand\@adefiv[1][]{\ifdef@index\adef@at\omdoc@indexiv[at=\@@at,#1]{\@pone}{\@ptwo}{\@pthree}{\@pfour}\fi\xspace}
%    \end{macrocode}
% \end{macro}
%
% \begin{macro}{\defiv}
%   similar to |\defiii|
%    \begin{macrocode}
\newcommand\defiv[5][]{\adefiv[#1]{#2 #3 #4 #5}{#2}{#3}{#4}{#5}}
\newcommand\defivs[5][]{\adefiv[#1]{#2 #3 #4 #5s}{#2}{#3}{#4}{#5}}
\newcommand\Defiv[5][]{\adefiv[#1]{\capitalize{#2 #3 #4 #5}}{#2}{#3}{#4}{#5}}
\newcommand\Defivs[5][]{\adefiv[#1]{\capitalize{#2 #3 #4 #5s}}{#2}{#3}{#4}{#5}}
%    \end{macrocode}
% \end{macro}
%
% \begin{macro}{\pdefi}
%    \begin{macrocode}
\newcommand\pdefi[1]{\@pdefi}
\newcommand\@pdefi[3][]{\defemph{#2 #3}}
%    \end{macrocode}
% \end{macro}
%
% \begin{macro}{\inlineex}
%    \begin{macrocode}
\newcommand\inlineex[2][]{\metasetkeys{omtext}{#1}%
\sref@target\sref@label@id{here}#2}
%    \end{macrocode}
% \end{macro}
%
% \begin{environment}{inlineExample}
%    \begin{macrocode}
\newenvironment{inlineExample}[1][]%
{\metasetkeys{omtext}{#1}\sref@target\sref@label@id{here}\ignorespacesandpars}%
{\ignorespacesandpars}%
%    \end{macrocode}
% \end{environment}
%
%
% \begin{macro}{\inlineass}
%    \begin{macrocode}
\newcommand\inlineass[2][]{\metasetkeys{omtext}{#1}%
\sref@target\sref@label@id{here}#2}
%    \end{macrocode}
% \end{macro}
%
% \begin{environment}{inlineAssertion}
%    \begin{macrocode}
\newenvironment{inlineAssertion}[1][]%
{\metasetkeys{omtext}{#1}\sref@target\sref@label@id{here}\ignorespacesandpars}%
{\ignorespacesandpars}%
%    \end{macrocode}
% \end{environment}
%
% \begin{macro}{\inlinedef}
%    \begin{macrocode}
\newcommand\inline@def@error{\if@in@omtext\else% we are not in an omtext or statement
\PackageError{modules}{\protect\inlinedef\space outside a statement!}%
{Try wrapping the paragraph in a\MessageBreak
\protect\begin{omtext}, \protect\begin{assertion}, \protect\begin{axiom}, ... \MessageBreak
whatever is suitable semantically}\fi}
\newcommand\inlinedef[2][]{\metasetkeys{omtext}{#1}%
\inline@def@error\sref@target\sref@label@id{here}\st@indeftrue #2}
%    \end{macrocode}
% \end{macro}
% 
% \begin{environment}{inlineDefinition}
%    \begin{macrocode}
\newenvironment{inlineDefinition}[1][]%
{\metasetkeys{omtext}{#1}\inline@def@error\sref@target\sref@label@id{here}\st@indeftrue\ignorespacesandpars}%
{\ignorespacesandpars}%
%    \end{macrocode}
% \end{environment}
%
% \subsection{Cross-Referencing Symbols and Concepts}\label{sec:impl:crossref}
%
% \begin{macro}{\termref}
%   |\termref[|\meta{keys}|]{|\meta{text}|}| makes a hyperlink with link text \meta{text}
%   to the definitional occurrence of the symbol specified by the |name|, |cd|, and
%   |cdbase| keys in \meta{kyes}. We first set sensible defaults if the keys are not
%   given. If the symbol is defined in the current document (i.e. if the macro
%   |\sref@|\meta{name}|@|\meta{cd}|@defined| is defined), then we make a local hyperref,
%   otherwise we punt to |\mod@termref|.
%    \begin{macrocode}
\addmetakey*{termref}{cd}
\addmetakey*{termref}{cdbase}
\addmetakey*{termref}{name}
\newcommand\termref[2][]{\metasetkeys{termref}{#1}%
\ifx\termref@cd\@empty\def\termref@cd{\mod@id}\fi%
\ifx\termref@name\@empty\def\termref@name{#2}\fi%
\@ifundefined{sref@\termref@name @\termref@cd @defined}%
{\ifx\termref@cdbase\@empty% external reference
\mod@termref\termref@cd\termref@name{#2}%
\else\sref@href@ifh\termref@cdbase{#2}%
\fi}%
{\def\@label{sref@\termref@name @\termref@cd @target}%
\sref@hlink@ifh\@label{#2}%\footnote{termref: internal reference to \@label}
}}
%    \end{macrocode}
% \end{macro}
%
% \begin{macro}{\@instring}
%   We first define an auxiliary conditional |\@instring| that checks of |?| is in
%   % the first argument.  |\@mtref| and |\@mdref| use it.
%    \begin{macrocode}
\def\@instring#1#2{TT\fi\begingroup\edef\x{\endgroup\noexpand\in@{#1}{#2}}\x\ifin@}
%    \end{macrocode}
% \end{macro}
%
% \begin{macro}{\@mtref}
%   |\@mtref{|\meta{symspec}|}{|\meta{text}|}{|\meta{name}|}| assembles a
%   |\termref[cd=|\meta{cd}|,name=|\meta{name'}|]| by splitting \meta{symspec} at the
%   |?| into \meta{cd} and \meta{name''}. If \meta{symspec} contains no |?|, it is
%   interpreted as a bare \meta{cd}  and \meta{name'} is \meta{name}. 
%    \begin{macrocode}
\newcommand\@@@mtref[3]{\def\@@cd{#1}\def\@@name{#2}%
\ifx\@@cd\@empty%
\ifx\@@name\@empty\termref[]{#3}\else\termref[name=\@@name]{#3}\fi%
\else%
\ifx\@@name\@empty\termref[cd=\@@cd]{#3}\else\termref[cd=\@@cd,name=\@@name]{#3}\fi%
\fi}
\def\@@mtref#1?#2\relax{\@@@mtref{#1}{#2}}
\newcommand\@mtref[3][]{\termemph{\if\@instring{?}{#1}\@@mtref #1\relax{#2}\else\termref[cd=#1,name=#3]{#2}\fi}}
%    \end{macrocode}
% \end{macro}
%
% \begin{macro}{\tref*}
%    \begin{macrocode}
\newcommand\trefi[2][]{\@mtref[#1]{#2}{#2}}
\newcommand\trefii[3][]{\@mtref[#1]{#2 #3}{#2-#3}}
\newcommand\trefiii[4][]{\@mtref[#1]{#2 #3 #4}{#2-#3-#4}}
\newcommand\trefiv[5][]{\@mtref[#1]{#2 #3 #4 #5}{#2-#3-#4-#5}}
\newcommand\trefis[2][]{\@mtref[#1]{#2s}{#2}}
\newcommand\trefiis[3][]{\@mtref[#1]{#2 #3s}{#2-#3}}
\newcommand\trefiiis[4][]{\@mtref[#1]{#2 #3 #4s}{#2-#3-#4}}
\newcommand\trefivs[5][]{\@mtref[#1]{#2 #3 #4 #5s}{#2-#3-#4-#5}}
%    \end{macrocode}
% \end{macro}
%
% \begin{macro}{\Tref*}
%    \begin{macrocode}
\newcommand\Trefi[2][]{\@mtref[#1]{\capitalize{#2}}{#2}}
\newcommand\Trefii[3][]{\@mtref[#1]{\capitalize{#2 #3}}{#2-#3}}
\newcommand\Trefiii[4][]{\@mtref[#1]{\capitalize{#2 #3 #4}}{#2-#3-#4}}
\newcommand\Trefiv[5][]{\@mtref[#1]{\capitalize{#2 #3 #4 #5}}{#2-#3-#4-#5}}
\newcommand\Trefis[2][]{\@mtref[#1]{\capitalize{#2s}}{#2}}
\newcommand\Trefiis[3][]{\@mtref[#1]{\capitalize{#2 #3s}}{#2-#3}}
\newcommand\Trefiiis[4][]{\@mtref[#1]{\capitalize{#2 #3 #4s}}{#2-#3-#4}}
\newcommand\Trefivs[5][]{\@mtref[#1]{\capitalize{#2 #3 #4 #5s}}{#2-#3-#4-#5}}
%    \end{macrocode}
% \end{macro}
%
% Now we care about the configuration switches, they are set to sensible values, if they
% are not defined already. These are just configuration parameters, which should not
% appear in documents, therefore we do not provide {\latexml} bindings for them.
% \begin{macro}{\*emph}
%    \begin{macrocode}
\providecommand{\termemph}[1]{#1}
\providecommand{\defemph}[1]{{\textbf{#1}}}
\providecommand{\stDMemph}[1]{{\textbf{#1}}}
%    \end{macrocode}
% \end{macro}
% 
% \begin{macro}{\term}
%   The |\term| macro is used for wiki-style dangling links with editor support.\ednote{MK: document above} 
%    \begin{macrocode}
\newcommand\term[2][]{\def\@test{#1}%
\ifx\@test\@empty\else
\@ifundefined{module@defs@#1}{\PackageWarning{statements}%
{{\protect\term} specifies module #1 which is not in
  scope\MessageBreak import it via e.g. via \protect\importmhmodule}}{}
\fi%
\PackageWarning{statements}%
{Dangling link (\protect\term) for "#2" still needs to be specified}%
\textcolor{blue}{\underline{#2}}}
%    \end{macrocode}
% \end{macro}
% \begin{macro}{\symref}
%   The |\symref| macros is quite simple, since we have done all the heavy lifting in the
%   |modules| package: we simply apply |\mod@symref@|\meta{arg1} to
%   \meta{arg2}.
%    \begin{macrocode}
\newcommand\symref[2]{\@nameuse{mod@symref@#1}{#2}}
%    \end{macrocode}
% \end{macro}
%X
% \subsection{Term Redefinition in Recaps}\label{sec:impl:drefi}
%
% \begin{macro}{\@mdref}
%   We first define an auxiliary macro |\@mdref|,  which checks for |?| in the first argument, if not it just calls
%   |\@definiendum|, otherwise it calls |\@mdref|, which assembles the |\@definiendum| after
%   splitting at the |?|.
%    \begin{macrocode}
\newcommand\@mdref[3][]{\def\@test{#1}%
\if\@instring{?}{#1}%
\ifx\@test\@empty\definiendum[name=#3]{#2}\else\definiendum[recaps=#1,name=#3]{#2}\fi%
\else%
\ifx\@test\@empty\definiendum[name=#3]{#2}\else\definiendum[recaps=#1?#3,name=#3]{#2}\fi%
\fi}
%    \end{macrocode}
%
% Actually, we do something else than specified above, but {\latexml} will do
% so.\ednote{or we will use that  } 
%
%    \begin{macrocode}
\newcommand\drefi[2][]{\defemph{\@mdref[#1]{#2}{#2}}}
\newcommand\drefii[3][]{\defemph{\@mdref[#1]{#2 #3}{#2-#3}}}
\newcommand\drefiii[4][]{\defemph{\@mdref[#1]{#2 #3 #4}{#2-#3-#4}}}
\newcommand\drefiv[5][]{\defemph{\@mdref[#1]{#2 #3 #4 #5}{#2-#3-#4-#5}}}
\newcommand\drefis[2][]{\defemph{\@mdref[#1]{#2s}{#2}}}
\newcommand\drefiis[3][]{\defemph{\@mdref[#1]{#2 #3s}{#2-#3}}}
\newcommand\drefiiis[4][]{\defemph{\@mdref[#1]{#2 #3 #4s}{#2-#3-#4}}}
\newcommand\drefivs[5][]{\defemph{\@mdref[#1]{#2 #3 #4 #5s}{#2-#3-#4-#5}}}
%    \end{macrocode}
% \end{macro}
%
% \begin{macro}{\Dref*}
%    \begin{macrocode}
\newcommand\Drefi[2][]{\defemph{\@mdref[#1]{\capitalize{#2}}{#2}}}
\newcommand\Drefii[3][]{\defemph{\@mdref[#1]{\capitalize{#2 #3}}{#2-#3}}}
\newcommand\Drefiii[4][]{\defemph{\@mdref[#1]{\capitalize{#2 #3 #4}}{#2-#3-#4}}}
\newcommand\Drefiv[5][]{\defemph{\@mdref[#1]{\capitalize{#2 #3 #4 #5}}{#2-#3-#4-#5}}}
\newcommand\Drefis[2][]{\defemph{\@mdref[#1]{\capitalize{#2s}}{#2}}}
\newcommand\Drefiis[3][]{\defemph{\@mdref[#1]{\capitalize{#2 #3s}}{#2-#3}}}
\newcommand\Drefiiis[4][]{\defemph{\@mdref[#1]{\capitalize{#2 #3 #4s}}{#2-#3-#4}}}
\newcommand\Drefivs[5][]{\defemph{\@mdref[#1]{\capitalize{#2 #3 #4 #5s}}{#2-#3-#4-#5}}}
%    \end{macrocode}
% \end{macro}
%
% \subsection{Deprecated Functionality}\label{sec:deprecated}
%
% In this section we centralize old interfaces that are only partially supported any
% more. 
% \begin{macro}{\*defi*}
%    \begin{macrocode}
\newcommand\defin[2][]{\defi[#1]{#2}%
\PackageWarning{statements}{\protect\defin\space is deprecated, use \protect\defi\space instead}}
\newcommand\twindef[3][]{\defii[#1]{#2}{#3}%
\PackageWarning{statements}{\protect\twindef\space is deprecated, use \protect\defii\space instead}}
\newcommand\atwindef[4][]{\defiii[#1]{#2}{#3}{#4}%
\PackageWarning{statements}{\protect\atwindef\space is deprecated, use \protect\defiii\space instead}}
\newcommand\definalt[3][]{\adefi[#1]{#2}{#3}%
\PackageWarning{statements}{\protect\definalt\space is deprecated, use \protect\adefi\space instead}}
\newcommand\twindefalt[4][]{\adefii[#1]{#2}{#3}{#4}%
\PackageWarning{statements}{\protect\twindefalt\space is deprecated, use \protect\adefii\space instead}}
\newcommand\atwindefalt[5][]{\adefiii[#1]{#2}{#3}{#4}{#5}%
\PackageWarning{statements}{\protect\atwindefalt\space is deprecated, use \protect\adefiii\space instead}}
\newcommand\@@atrefi[3][]{\def\@test{#1}%
  \ifx\@test\@empty\termref[name=#3]{#2}\else\termref[cd=#1,name=#3]{#2}\fi}
\newcommand\@atrefi[3][]{\termemph{\@@atrefi[#1]{#2}{#3}}}
\newcommand\twinref[3][]{\trefii[#1]{#2}{#3}%
\PackageWarning{statements}{\protect\twinref\space is deprecated, use \protect\trefii\space instead}}
\newcommand\atwinref[4][]{\@atrefiii[#1]{#2}{#3}{#4}%
\PackageWarning{statements}{\protect\atwindef\space is deprecated, use \protect\trefiii\space instead}}
%    \end{macrocode}
% \end{macro}
%
% \begin{macro}{\atref*}
%    \begin{macrocode}
\newcommand\atrefi[3][]{\@atrefi[#1]{#2}{#3}%
\PackageWarning{statements}{\protect\atrefi\space is deprecated, use \protect\trefi\space instead}}
\newcommand\atrefii[4][]{\@atrefi[#1]{#2}{#3-#4}%
\PackageWarning{statements}{\protect\atrefii\space is deprecated, use \protect\trefi\space instead}}
\newcommand\atrefiii[5][]{\@atrefi[#1]{#2}{#3-#4-#5}%
\PackageWarning{statements}{\protect\atrefiii\space is deprecated, use \protect\trefi\space instead}}
\newcommand\atrefiv[6][]{\@atrefi[#1]{#2}{#3-#4-#5-#6}%
\PackageWarning{statements}{\protect\atrefiv\space is deprecated, use \protect\trefi\space instead}}
%    \end{macrocode}
% \end{macro}
%
% \begin{macro}{\mtrefi*}
%    \begin{macrocode}
\newcommand\mtrefi[2][]{\trefi[#1]{#2}%
\PackageWarning{statements}{\protect\mtrefi\space is deprecated, use \protect\trefi\space instead}}
\newcommand\mtrefii[3][]{\trefii[#1]{#2}{#3}
\PackageWarning{statements}{\protect\mtrefii\space is deprecated, use \protect\trefii\space instead}}
\newcommand\mtrefiii[4][]{\trefiii[#1]{#2}{#3}{#4}
\PackageWarning{statements}{\protect\mtrefiii\space is deprecated, use \protect\trefiii\space instead}}
\newcommand\mtrefiv[5][]{\trefiv[#1]{#2}{#3}{#4}{#5}
\PackageWarning{statements}{\protect\mtrefiv\space is deprecated, use \protect\trefiv\space instead}}
%</package>
%    \end{macrocode}
% \end{macro}
% \Finale
%
\endinput
% \iffalse
%%% Local Variables: 
%%% mode: doctex
%%% TeX-master: t
%%% End: 
% \fi
% LocalWords:  GPL structuresharing STR dtx keyval env envfalse idfalse idtrue typedec st
% LocalWords:  displayfalse envtrue displaytrue forfalse typefalse titlefalse filedate eg
% LocalWords:   continuesfalse fortrue fromtrue typetrue titletrue CPERL omdoc thedectype
% LocalWords:   continuestrue symboldec omtext RequirePackage lowcase lc ToString termin
% LocalWords:  foreach hyperref href hlink DefStatement OptionalKeyVals ne NeedsTeXFormat
% LocalWords:  KeyVal xml CMP simpleDef PatternDef DefStatement PatternRule thedectype mh
% LocalWords:  requation PatternCMP RecDef DefConstructor keyvals defs psymbols tref eset
% LocalWords:  openElement symb closeElement ffor getValue attrs metadata undef rangle mh
% LocalWords:  afterDigestBegin setProperty AssignValue afterDigest definiendum rangle
% LocalWords:  cd addr LookupValue getArg toString idx idt definiendum ide idp st@flow
% LocalWords:  DefMacro args unlist inlinedef uri pdf afterOpen numberIt texttt XMath
% LocalWords:  iffalse consymb ntheorem textbackslash symref def scsys sc sc kw endinput
% LocalWords:  mathml openmath latexml activemath fileversion maketitle stex importmodule
% LocalWords:  setcounter tocdepth tableofcontents newpage sproofs ulsmf08 sref usevocab
% LocalWords:  MaySch eltte09 twintoo sref subsubsection exfig vspace vspace usemhvocab
% LocalWords:  noindent renewtheorem hline textbf textbf footnotesize ple peano inlineass
% LocalWords:  STaxiomEnv symdef medskip succ mathbb ldots stepcounter ednote usemhmodule
% LocalWords:  STtheoremAssEnv stepcounter STtheoremAssEnv stepcounter defin STtypedecEnv
% LocalWords:  STtheoremAssEnv notatiendum defin smomdl biblatex twindef cdbase defis
% LocalWords:  twindef atwindef atwindef adjectivized varaiants twindefalt cseq defindex
% LocalWords:  twindefalt atwindefalt atwindefalt csymbol definalt termref emph emptyset
% LocalWords:  termref compactdesc KohAmb smmssl twinref atwinref newpart impl customized
% LocalWords:  termdef defemph defemph renewcommand termemph termenph stDMemph formalized
% LocalWords:  stDMemph STpresent STpresent makeatletter STlemmaAssEnv textsf emphasized
% LocalWords:  STpropositionAssEnv STcorollaryAssEnv STconjectureAssEnv langle defiis
% LocalWords:  STfalseconjectureAssEnv STpostulateAssEnv STobligationAssEnv foo defiis
% LocalWords:  STassumptionAssEnv STobservationAssEnv STexampleEnv textsf ltxml defiiis
% LocalWords:  STdefinitionEnv STnotationEnv printbibliography langle ncname localization
% LocalWords:  theoremstyle sym newenvironment ifx csname endcsname inlineex loadmodules
% LocalWords:  currentlabel theorembodyfont itshape theoremheaderfont bfseries defiiis
% LocalWords:  normalfont newtheorem upshape srefaddidkey  definendum  customization gdef
% LocalWords:  newcommand indef newif ifst indeffalse indeftrue attr whatsit automatized
% LocalWords:  STdefinitionEnvKeyword notemph modulename ifundefined atwin defi behavior
% LocalWords:  expandafter providecommand nameuse doctex ctancite funval defii initialize
% LocalWords:  funsymbs findnodes symbolnode defsref showmeta showmeta sysname ngerman
% LocalWords:  defii defiii defiii adefi adefi adefii adefii adefiii adefiii STruleAssEnv
% LocalWords:  trefi trefii trefiii atref atrefi atrefii atrefiii conf metakeys setkeys
% LocalWords:  compactenum Deref metasetkeys addmetakey symtype ltx ltx sts specialized
% LocalWords:  ifmetakeys fntype rightarrow inlinetypedec inlinetypedec STremarkAssEnv
% LocalWords:  ignorespaces usemodule textcolor STtypedecEnvKeyword flexiformalization
%  LocalWords:  statements-ngerman.ldf omdoc@indexi xspace trefis trefiis trefiiis onedef
%  LocalWords:  importmhmodule nontheorem sum-over-odds,type zero,title zero,role defiv
%  LocalWords:  succ,title Function,role Numbers,role peano.P1,title peano.P5,title defiv
%  LocalWords:  plus-nat.type,system one.def,for adefiv adefiv defivs defivs capitalizing
%  LocalWords:  trefiv atrefiv IanKoh:fmkdam15 drefi drefi dref smultiling msection cslet
%  LocalWords:  mikoslides def@indexfalse def@indextrue if@nthm ifcsdef cslet endproof
%  LocalWords:  amsthm finnish statements-finnish.ldf statements-french.ldf thesection
%  LocalWords:  statements-russian.ldf rmfamily enddefinition omdoc@indexii omdoc@indexiv
%  LocalWords:  1,name trefivs mtref Mtref begingroup endgroup noexpand mdref drefii
%  LocalWords:  drefiii drefiv drefis drefiis drefiiis drefivs mtrefiv foo,recaps foobar
%  LocalWords:  verbalization capitalization KohGin:smss
