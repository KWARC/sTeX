% \iffalse meta-comment
% An Infrastructure for marking up Assignments
% $URL: https://svn.kwarc.info/repos/stex/trunk/sty/hwexam/hwexam.dtx $ 
% $Rev: 2352 $; last modified by $Author: kohlhase $ 
% $Date: 2013-09-15 10:23:18 +0200 (Sun, 15 Sep 2013) $
% Copyright (c) 2007 Michael Kohlhase, all rights reserved
%               this file is released under the
%               LaTeX Project Public License (LPPL)
% \fi
% 
% \iffalse
%<package|cls>\NeedsTeXFormat{LaTeX2e}[1999/12/01]
%<package>\ProvidesPackage{hwexam}[2013/12/12 v1.0 homework assignments and exams]
%<cls>\ProvidesClass{hwexam}[2013/12/12 v1.0 assignment and exam documents]
%
%<*driver>
\documentclass{ltxdoc}
\usepackage{url,float}
\usepackage{hwexam}
\usepackage[show]{ed}
\usepackage[hyperref=auto,style=alphabetic,backend=bibtex]{biblatex}
\bibliography{kwarc}
\usepackage{stex-logo}
\usepackage{../ctansvn}
\usepackage{hyperref}
\makeindex
\floatstyle{boxed}
\newfloat{exfig}{thp}{lop}
\floatname{exfig}{Example}
\def\tracissue#1{\cite{sTeX:online}, \hyperlink{http://trac.kwarc.info/sTeX/ticket/#1}{issue #1}}
\begin{document}\DocInput{hwexam.dtx}\end{document}
%</driver>
% \fi
%\iffalse\CheckSum{424}\fi
% 
% \changes{v0.9}{2006/09/18}{First Version with Documentation}
% \changes{v0.9a}{2010/06/25}{more semantic headers for exams}
% \changes{v0.9b}{2010/09/20}{adding \texttt{assignment.cls}}
% \changes{v0.9c}{2010/09/20}{renaming from \texttt{assignment} to \texttt{hwexam} to
% avoid name clashes with existing \texttt{assignment.cls} on CTAN.}
% \changes{v1.0}{2013/12/12}{adding MathHub support}
% 
% \GetFileInfo{hwexam.sty}
% 
% \MakeShortVerb{\|}
%\def\scsys#1{{{\sc #1}}\index{#1@{\sc #1}}}
% \def\latexml{\scsys{LaTeXML}}
%
% \title{\texttt{hwexam.sty/cls}: An Infrastructure for formatting Assignments 
%        and Exams\thanks{Version {\fileversion} (last revised {\filedate})}}
%    \author{Michael Kohlhase\\
%            Jacobs University, Bremen\\
%            \url{http://kwarc.info/kohlhase}}
% \maketitle
%
% \begin{abstract}
%   The |hwexam| package and class allows individual course assignment sheets and
%   compound assignment documents using problem files marked up with the |problem| package.
% \end{abstract}
% \setcounter{tocdepth}{2}\tableofcontents\newpage
%
%\section{Introduction}\label{sec:intro}
%
% The |hwexam| package and class supplies an infrastructure that allows to format
% nice-looking assignment sheets by simply including problems from problem files marked up
% with the |problem| package~\cite{Kohlhase:problem:ctan}.  It is designed to be
% compatible with |problems.sty|, and inherits some of the functionality.
% 
% \section{The User Interface}
% 
% \subsection{Package and Class Options}\label{sec:user:options}
% 
% The |hwexam| package and class take the options |solutions|, |notes|, |hints|, |pts|,
% |min|, and |boxed| that are just passed on to the |problems| package (cf. its
% documentation for a description of the intended behavior).
% 
% If the \DescribeMacro{showmeta}|showmeta| option is set, then the metadata keys are
% shown (see~\cite{Kohlhase:metakeys:ctan} for details and customization options).
% 
% The |hwexam| class additionally accepts the options |report|, |book|, |chapter|, |part|,
% and |showignores|, of the |omdoc| package~\cite{Kohlhase:smomdl:ctan} on which it is
% based and passes them on to that. For the |extrefs| option
% see~\cite{Kohlhase:sref:ctan}.
%
% \subsection{Assignments}
%
% This package supplies the \DescribeEnv{assignment}|assignment| environment that groups
% problems into assignment sheets. It takes an optional KeyVal argument with the keys
% \DescribeMacro{number}|number| (for the assignment number; if none is given, 1 is
% assumed as the default or --- in multi-assignment documents --- the ordinal of the
% |assignment| environment), \DescribeMacro{title}|title| (for the assignment title; this
% is referenced in the title of the assignment sheet), \DescribeMacro{type}|type| (for the
% assignment type; e.g. ``quiz'', or ``homework''), \DescribeMacro{given}|given| (for the
% date the assignment was given), and \DescribeMacro{due}|due| (for the date the
% assignment is due).
% 
% \subsection{Typesetting Exams}
%
% Furthermore, the |hwexam| package takes the option
% \DescribeMacro{multiple}|multiple| that allows to combine multiple assignment sheets into
% a compound document (the assignment sheets are treated as section, there is a table of
% contents, etc.). 
% 
% Finally, there is the option \DescribeMacro{test}|test| that modifies the behavior to
% facilitate formatting tests. Only in |test| mode, the macros |\testspace|,
% |\testnewpage|, and |\testemptypage| have an effect: they generate space for the
% students to solve the given problems. Thus they can be left in the {\LaTeX} source. 
%
% \DescribeMacro{\testspace}|\testspace| takes an argument that expands to a dimension,
% and leaves vertical space accordingly. \DescribeMacro{\testnewpage}|\testnewpage| makes
% a new page in |test| mode, and \DescribeMacro{\testemptypage}|\testemptypage| generates
% an empty page with the cautionary message that this page was intentionally left empty.
%
% Finally, the \DescribeEnv{testheading}|\testheading| takes an optional keyword argument
% where the keys \DescribeMacro{duration}|duration| specifies a string that specifies the
% duration of the test, \DescribeMacro{min}|min| specifies the equivalent in number of
% minutes, and \DescribeMacro{reqpts}|reqpts| the points that are required for a perfect
% grade.
% \begin{exfig}[ht]
% \makeatletter
% \@problem{1.1}{4}{10}
% \@problem{2.1}{4}{8}
% \@problem{2.2}{6}{10}
% \@problem{2.3}{6}{10}
% \@problem{3.1}{4}{8}
% \@problem{3.2}{4}{8}
% \@problem{3.3}{2}{4}
% \makeatother
% \begin{verbatim}
% \title{320101 General Computer Science (Fall 2010)}
% \begin{testheading}[duration=one hour,min=60,reqpts=27]
%   Good luck to all students!
% \end{testheading}
% \end{verbatim}
% \vspace*{-3ex}\hrule\vspace*{.5ex}  formats to\vspace*{1ex}
% \hrule\par\noindent\vspace*{2ex}
% \title{320101 General Computer Science (Fall 2010)}
% \begin{testheading}[duration=one hour,min=60,reqpts=27]
%   good luck
% \end{testheading}
% \caption{A generated test heading.}\label{fig:testheading}
% \end{exfig}
% 
% \subsection{Including Assignments}
%
% The \DescribeMacro{\includeassignment}|\includeassignment| macro can be used to include
% an assignment from another file. It takes an optional KeyVal argument and a second
% argument which is a path to the file containing the problem (the macro assumes that
% there is only one |assignment| environment in the included file).  The keys
% \DescribeMacro{number}|number|, \DescribeMacro{title}|title|,
% \DescribeMacro{type}|type|, \DescribeMacro{given}|given|, and \DescribeMacro{due}|due|
% are just as for the |assignment| environment and (if given) overwrite the ones specified
% in the |assignment| environment in the included file.
% 
% \subsection{Support for \textsf{MathHub}}\label{sec:user:mathhub}
% 
% Much of the \sTeX content is hosed on \textsf{MathHub} (\url{http://MathHub.info}), a
% portal and archive for flexiformal mathematics. \textsf{MathHub} offers GIT repositories
% (public and private escrow) for mathematical documentation projects, online and offline
% authoring and document development infrastructure, and a rich, interactive reading
% interface. The |modules| package supports repository-sensitive operations on
% \textsf{MathHub}. 
% 
% Note that \textsf{MathHub} has two-level repository names of the form
% \meta{group}|/|\meta{repo}, where \meta{group} is a \textsf{MathHub}-unique repository
% group and \meta{repo} a repository name that is \meta{group}-unique. The file and
% directory structure of a repository is arbitrary -- except that it starts with the
% directory |source| because they are Math Archives in the sense
% of~\cite{HorIacJuc:cscpnrr11}. But this structure can be hidden from the \sTeX author
% with \textsf{MathHub}-enabled versions of the |modules| macros.
% 
% The \DescribeMacro{\includemhassignment}|\includemhassignment| macro is a variant of
% |\importmodule| with repository support. Instead of writing 
% \begin{verbatim}
% \defpath{MathHub}{/user/foo/lmh/MathHub}
% \includeassignment[pts=7]{\MathHub{fooMH/bar/source/baz/foobar}}
% \end{verbatim}
% we can simply write (assuming that |\MathHub| is defined as above)
% \begin{verbatim}
% \includemhassignment[fooMH/bar]{baz/foobar}
% \end{verbatim}
% Note that the |\importmhmodule| form is more semantic, which allows more advanced
% document management features in \textsf{MathHub}. 
% 
% If |baz/foobar| is the ``current module'', i.e. if we are on the \textsf{MathHub} path
% \ldots|MathHub/fooMH/bar|\ldots, then stating the repository in the first optional
% argument is redundant, so we can just use
% \begin{verbatim}
% \includemhassignment{baz/foobar}
% \end{verbatim}
% Of course, neither {\LaTeX} nor \latexml know about the repositories when they are
% called from a file system, so we can use the |\mhcurrentrepos| macro from the |modules|
% package to tell them. But this is only needed to initialize the infrastructure in the
% driver file. In particular, we do not need to set it in in each module, since the
% |\importmhmodule| macro sets the current repository automatically.
%
% \paragraph{Caveat} if you want to use the \textsf{MathHub} support macros (let's call
% them mh-variants), then every time a module is imported or a document fragment is
% included from another repos, the mh-variant |\importmhmodule| must be used, so that the
% ``current repository'' is set accordingly. To be exact, we only need to use mh-variants,
% if the imported module or included document fragment use mh-variants.
%
% \section{Limitations}\label{sec:limitations}
% 
% In this section we document known limitations. If you want to help alleviate them,
% please feel free to contact the package author. Some of them are currently discussed in
% the \sTeX TRAC~\cite{sTeX:online}. 
% \begin{compactenum}
% \item none reported yet. 
% \end{compactenum}
% 
% \StopEventually{\newpage\PrintIndex\newpage\PrintChanges\printbibliography}\newpage
% \newpage
%
% \section{Implementation: The hwexam Class}\label{sec:impl:cls}
%
% The functionality is spread over the |hwexam| class and package. The class provides
% the |document| environment and pre-loads some convenience packages, whereas the package
% provides the concrete functionality.
% 
% |hwexam.dtx| generates four files: |hwexam.cls| (all the code between
% {\textsf{$\langle$*cls$\rangle$}} and {\textsf{$\langle$/cls$\rangle$}}), |hwexam.sty|
% (between {\textsf{$\langle$*package$\rangle$}} and
% {\textsf{$\langle$/package$\rangle$}}) and their {\latexml} bindings (between
% {\textsf{$\langle$*ltxml.cls$\rangle$}} and {\textsf{$\langle$/ltxml.cls$\rangle$}} and
% {\textsf{$\langle$*ltxml.sty$\rangle$}} and {\textsf{$\langle$/ltxml.sty$\rangle$
% respetively}}). We keep the corresponding code fragments together, since the
% documentation applies to both of them and to prevent them from getting out of sync.
%
% \subsection{Class Options}\label{sec:impl:cls:options}
%
% To initialize the |hwexam| class, we declare and process the necessary options by
% passing them to the respective packages and classes they come from.
% 
%    \begin{macrocode}
%<*cls>
\DeclareOption{test}{\PassOptionsToPackage{\CurrentOption}{hwexam}}
\DeclareOption{multiple}{\PassOptionsToPackage{\CurrentOption}{hwexam}}
\DeclareOption{showmeta}{\PassOptionsToPackage{\CurrentOption}{metakeys}}
\DeclareOption{extrefs}{\PassOptionsToPackage{\CurrentOption}{sref}}
\DeclareOption{notes}{\PassOptionsToPackage{\CurrentOption}{problem}}
\DeclareOption{hints}{\PassOptionsToPackage{\CurrentOption}{problem}}
\DeclareOption{solutions}{\PassOptionsToPackage{\CurrentOption}{problem}}
\DeclareOption{pts}{\PassOptionsToPackage{\CurrentOption}{problem}}
\DeclareOption{min}{\PassOptionsToPackage{\CurrentOption}{problem}}
\DeclareOption{boxed}{\PassOptionsToPackage{\CurrentOption}{problem}}
\DeclareOption{extract}{\PassOptionsToPackage{\CurrentOption}{problem}}
\DeclareOption*{\PassOptionsToClass{\CurrentOption}{omdoc}}
\ProcessOptions
%</cls>
%<*ltxml.cls>
# -*- CPERL -*-
package LaTeXML::Package::Pool;
use strict;
use LaTeXML::Package;
use LaTeXML::Util::Pathname;
use Cwd qw(cwd abs_path);
DeclareOption('test',,sub {PassOptions('hwexam','sty',ToString(Digest(T_CS('\CurrentOption')))); }); 
DeclareOption('multiple',sub {PassOptions('hwexam','sty',ToString(Digest(T_CS('\CurrentOption')))); }); 
DeclareOption('showmeta',sub {PassOptions('metakeys','sty',ToString(Digest(T_CS('\CurrentOption')))); }); 
DeclareOption('extrefs',sub {PassOptions('sref','sty',ToString(Digest(T_CS('\CurrentOption')))); }); 
DeclareOption('notes',sub {PassOptions('problem','sty',ToString(Digest(T_CS('\CurrentOption')))); }); 
DeclareOption('hints',sub {PassOptions('problem','sty',ToString(Digest(T_CS('\CurrentOption')))); }); 
DeclareOption('solutions',sub {PassOptions('problem','sty',ToString(Digest(T_CS('\CurrentOption')))); }); 
DeclareOption('pts',sub {PassOptions('problem','sty',ToString(Digest(T_CS('\CurrentOption')))); }); 
DeclareOption('min',sub {PassOptions('problem','sty',ToString(Digest(T_CS('\CurrentOption')))); }); 
DeclareOption('boxed',sub {PassOptions('problem','sty',ToString(Digest(T_CS('\CurrentOption')))); }); 
DeclareOption('extract',sub {PassOptions('problem','sty',ToString(Digest(T_CS('\CurrentOption')))); }); 
DeclareOption(undef,sub {PassOptions('omdoc','cls',ToString(Digest(T_CS('\CurrentOption')))); });
ProcessOptions();
%</ltxml.cls>
%    \end{macrocode}
%
% We load |article.cls|, and the desired packages. For the {\latexml} bindings, we make
% sure the right packages are loaded.
%
%    \begin{macrocode}
%<*cls>
\LoadClass{omdoc}
\RequirePackage{stex}
\RequirePackage{hwexam}
\RequirePackage{graphicx}
\RequirePackage{a4wide}
\RequirePackage{amssymb}
\RequirePackage{amstext}
\RequirePackage{amsmath}
%</cls>
%<*ltxml.cls>
LoadClass('omdoc');
RequirePackage('stex');
RequirePackage('hwexam');
RequirePackage('graphicx');
RequirePackage('amssymb');
RequirePackage('amstext');
RequirePackage('amsmath');
%</ltxml.cls>
%    \end{macrocode}
%
% \section{Implementation: The hwexam Package} 
%
% \subsection{Package Options}
%
% The first step is to declare (a few) package options that handle whether certain
% information is printed or not. Some come with their own conditionals that are set by the
% options, the rest is just passed on to the |problems| package.
%
%    \begin{macrocode}
%<*package>
\DeclareOption{showmeta}{\PassOptionsToPackage{\CurrentOption}{metakeys}}
\newif\iftest\testfalse
\newif\ifsolutions\solutionsfalse
\DeclareOption{test}{\testtrue\solutionsfalse}
\newif\ifmultiple\multiplefalse
\DeclareOption{multiple}{\multipletrue}
\DeclareOption*{\PassOptionsToPackage{\CurrentOption}{problem}}
\ProcessOptions
%</package>
%    \end{macrocode}
% Then we make sure that the necessary packages are loaded (in the right versions).
%    \begin{macrocode}
%<*package>
\RequirePackage{keyval}[1997/11/10]
\RequirePackage{problem}
%</package>
%    \end{macrocode}
%
% Here comes the equivalent header information for {\latexml}, we also initialize the
% package inclusions. Since {\latexml} does not handle options yet, we have nothing to
% do. 
%    \begin{macrocode}
%<*ltxml>
# -*- CPERL -*-
package LaTeXML::Package::Pool;
use strict;
use LaTeXML::Package;
RequirePackage('problem');
%</ltxml>
%    \end{macrocode}
%
% Then we register the namespace of the requirements ontology
%    \begin{macrocode}
%<*ltxml>
RegisterNamespace('assig'=>"http://omdoc.org/ontology/assignments#");
RegisterDocumentNamespace('assig'=>"http://omdoc.org/ontology/assignments#");
%</ltxml>
%    \end{macrocode}
%
% \subsection{Assignments}
%
% We will prepare the keyval support for the |assignment| environment.
%
%    \begin{macrocode}
%<*package>
\srefaddidkey{assig}
\addmetakey{assig}{number}
\addmetakey*{assig}{title}
\addmetakey{assig}{type}
\addmetakey{assig}{given}
\addmetakey{assig}{due}
%    \end{macrocode}
%
% The next three macros are intermediate functions that handle the case gracefully, where
% the respective token registers are undefined.
% 
% The |\given@due| macro prints information about the given and due status of the
% assignment. Its arguments specify the brackets. 
% 
%    \begin{macrocode}
\def\given@due#1#2{%
\ifx\assig@given\@empty
\ifx\assig@due\@empty\else#1 Due \assig@due #2\fi%
\else%assig@given non-empty
#1Given \assig@given%
\ifx\assig@due\@empty\else, Due \assig@due\fi #2\fi}
%    \end{macrocode}
% 
% With them, we can define the central |assignment| environment. This has two forms
% (separated by |\ifmultiple|) in one we make a title block for an assignment sheet, and
% in the other we make a section heading and add it to the table of
% contents.
% 
% \begin{macro}{assignment@titleblock}
%   This macro prints the title block of a section. If the |multiple| package option is
%   given we make a section heading out of this, and if not, a title block. Note that as
%   |problem|s are numbered by section, we also set the section counter in the latter
%   case.
%    \begin{macrocode}
\ifmultiple
\def\assignment@titleblock{%
\ifx\assig@number\@empty\stepcounter{section}\else\setcounter{section}{\assig@number}\fi%
\section*{\protect\document@hwexamtype~\arabic{section}:~\assig@title%
\ifx\assig@title\@empty\else\strut\\\fi\given@due()}%
\addcontentsline{toc}{section}%
{\document@hwexamtype~{\arabic{section}}:~%
\string\importmodules{\imported@modules}\assig@title}%
\setcounter{problem}{0}}
\else% multiple
\def\assignment@titleblock{%
\ifx\assig@number\@empty\stepcounter{section}\else\setcounter{section}{\assig@number}\fi%
\begin{center}\bf
\Large\@title\strut\\
\document@hwexamtype~\assig@number:~\assig@title\strut\\
\large{\given@due()}
\end{center}}
\fi %multiple
%    \end{macrocode}
% \end{macro}
% 
% \begin{macro}{assignment@process@keys}
%   this macro collects the keys from its argument and corrects them from the outside.
%    \begin{macrocode}
\def\assignment@process@keys#1{\metasetkeys{assig}{#1}
\ifx\inclassig@title\@empty\else\def\assig@title{\inclassig@title}\fi
\ifx\inclassig@type\@empty\else\def\assig@type{\inclassig@type}\fi
\ifx\inclassig@number\@empty\else\def\assig@number{\inclassig@number}\fi
\ifx\inclassig@due\@empty\else\def\assig@due{\inclassig@due}\fi
\ifx\inclassig@given\@empty\else\def\assig@given{\inclassig@given}\fi}
%    \end{macrocode}
% for this to work we need to define the |\inclassig| macros in case no
% |\includeassignment| is ever called. 
%    \begin{macrocode}
\def\inclassig@title{}
\def\inclassig@type{}
\def\inclassig@number{}
\def\inclassig@due{}
\def\inclassig@given{}
%    \end{macrocode}
% \end{macro}
%
% \begin{environment}{assignment}
%    \begin{macrocode}
\newenvironment{assignment}[1][]{\assignment@process@keys{#1}%
\assignment@titleblock%
\def\currentsectionlevel{assignment\xspace}%
\def\Currentsectionlevel{Assignment\xspace}%
\ignorespaces}{}
%</package>
%    \end{macrocode}
% 
%    \begin{macrocode}
%<*ltxml>
DefEnvironment('{assignment} OptionalKeyVals:assig',
  "<omdoc:omgroup ?&KeyVal(#1,'id')(xml:id='&KeyVal(#1,'id')')() "
  .  "assig:dummy='for the namespace'>"
 .  "<omdoc:metadata>"
  .    "<dc:title>"
  .       "Assignment ?&KeyVal(#1,'num')(&KeyVal(#1,'num').)()"
  .       "?&KeyVal(#1,'title')((&KeyVal(#1,'title')))"
  .    "</dc:title>"
  .    "?&KeyVal(#1,'given')(<omdoc:meta property='assig:given'>&KeyVal(#1,'given')</omdoc:meta>)()"
  .    "?&KeyVal(#1,'due')(<omdoc:meta property='assig:due'>&KeyVal(#1,'due')</omdoc:meta>)()"
  .    "?&KeyVal(#1,'pts')(<omdoc:meta property='assig:pts'>&KeyVal(#1,'pts')</omdoc:meta>)()"
  .  "</omdoc:metadata>"
  .  "#body"
  ."</omdoc:omgroup>\n",
  afterDigest=> sub {
    my ($stomach, $kv) = @_;
    my $kvi = LookupValue('inclassig');
    my @keys = qw(id num title pts given due);
    my @vals = $kvi && map($kvi->getValue($_), @keys);
    foreach my $i(0..$#vals) {
       $kv->setValue($keys[$i],$vals[$i]) if $vals[$i];
     }});#$
%</ltxml>
%    \end{macrocode}
% \end{environment}
% 
%    \begin{macrocode}
%<*package>
\def\assig@default@type{Assignment}
\addmetakey[\assig@default@type]{document}{hwexamtype}
%</package>
%    \end{macrocode}
% 
% \subsection{Including Assignments}
%
% The next command is essentially a glorified |\include| statement, it just sets some
% internal macros first that overwrite the local points, \ednote{these keys should be done
% with \texttt{\textbackslash addmetakey}}
% 
%    \begin{macrocode}
%<*package>
\addmetakey{inclassig}{number}
\addmetakey{inclassig}{title}
\addmetakey{inclassig}{type}
\addmetakey{inclassig}{given}
\addmetakey{inclassig}{due}
\newcommand\includeassignment[2][]{\metasetkeys{inclassig}{#1}\include{#2}}
\newcommand\inputassignment[2][]{\metasetkeys{inclassig}{#1}\input{#2}}
%</package>
%<*ltxml>
DefMacro('\includeassignment [] {}', sub {
  my ($stomach, $arg1, $arg2) = @_;
  AssignValue('inclassig',$arg1) if $arg1;
  (Invocation(T_CS('\input'),$arg2)->unlist);
});
DefMacro('\inputassignment [] {}','\input{#2}');
%</ltxml>
%    \end{macrocode}
%
% \subsection{Typesetting Exams}
%
%    \begin{macrocode}
%<*package>
\addmetakey{quizheading}{tas}
\newcommand\quizheading[1]{\def\@tas{#1}%
\large\noindent NAME: \hspace{8cm}  MAILBOX:\\[2ex]%
\ifx\@tas\@empty\else%
\noindent TA: \@for\@I:=\@tas\do{{\Large$\Box$}\@I\hspace*{1em}}\\[2ex]\fi}
%    \end{macrocode}
%    \begin{macrocode}
\addmetakey{testheading}{min}
\addmetakey{testheading}{duration}
\addmetakey{testheading}{reqpts}
\newenvironment{testheading}[1][]{\metasetkeys{testheading}{#1}
{\noindent\large{}Name: \hfill Matriculation Number:\hspace*{2cm}\strut\\[1ex]
\begin{center}\Large\textbf{\@title}\\[1ex]\large\@date\\[3ex]\end{center}  
{\textbf{You have 
\ifx\test@heading@duration\@empty\testheading@min minutes\else\testheading@duration\fi 
(sharp) for the test}};\\ Write the solutions to the sheet.}\par\noindent

\newcount\check@time\check@time=\testheading@min
\advance\check@time by -\theassignment@totalmin
The estimated time for solving this exam is {\theassignment@totalmin} minutes, 
leaving you {\the\check@time} minutes for revising your exam. 

\newcount\bonus@pts\bonus@pts=\theassignment@totalpts
\advance\bonus@pts by -\testheading@reqpts
You can reach {\theassignment@totalpts} points if you solve all problems. You will only need
{\testheading@reqpts} points for a perfect score, i.e.\ {\the\bonus@pts} points are
bonus points. \vfill
\begin{center}
  {\Large\em
%  You have ample time, so take it slow and avoid rushing to mistakes!\\[2ex]
  Different problems test different skills and knowledge, so do not get stuck on
  one problem.}\vfill\par\correction@table \\[3ex]
\end{center}}
{\newpage}
%</package>
%<*ltxml>
DefEnvironment('{testheading}OptionalKeyVals:omdoc','');
%</ltxml>
%    \end{macrocode}
%
%    \begin{macrocode}
%<*package>
\def\testspace#1{\iftest\vspace*{#1}\fi}
\def\testnewpage{\iftest\newpage\fi}
\def\testemptypage{\iftest\begin{center}This page was intentionally left
    blank for extra space\end{center}\vfill\eject\else\fi}
%</package>
%<*ltxml>
DefConstructor('\testspace{}','');
DefConstructor('\testnewpage','');
DefConstructor('\testemptypage','');
%</ltxml>
%    \end{macrocode}
%
% \begin{macro}{\@problem}
%   This macro acts on a problem's record in the |*.aux| file. Here we redefine it to
%   generate the correction table. 
%    \begin{macrocode}
%<*package>
\def\@problem#1#2#3{\stepcounter{assignment@probs}
\def\@test{#2}\ifx\@test\@empty\else\addtocounter{assignment@totalpts}{#2}\fi
\def\@test{#3}\ifx\@test\@empty\else\addtocounter{assignment@totalmin}{#3}\fi
\xdef\correction@probs{\correction@probs & #1}%
\xdef\correction@pts{\correction@pts & #2}
\xdef\correction@reached{\correction@reached &}}
%</package>
%    \end{macrocode}
% \end{macro}
% 
% \begin{macro}{\correction@table}
%   This macro generates the correction table
%    \begin{macrocode}
%<*package>
\newcounter{assignment@probs}
\newcounter{assignment@totalpts}
\newcounter{assignment@totalmin}
\def\correction@probs{prob.}%
\def\correction@pts{total}%
\def\correction@reached{reached}%
\stepcounter{assignment@probs}
\def\correction@table{\begin{tabular}{|l|*{\theassignment@probs}{c|}|p{3cm}|}\hline%
&\multicolumn{\theassignment@probs}{c||}%|
{\footnotesize To be used for grading, do not write here} &\\\hline
\correction@probs & Sum & grade\\\hline
\correction@pts &\theassignment@totalpts & \strut\hspace{3cm}\strut\\\hline
\correction@reached & & \\[.7cm]\hline
\end{tabular}}
%</package>
%    \end{macrocode}
% \end{macro}
% 
% \subsection{Support for \textsf{MathHub}}\label{sec:user:mathhub}
% 
% \begin{macro}{\includemhassignment}
%   The |\includemhassignment| saves the current value of |\mh@currentrepos| in a local macro
%   |\mh@@repos|, resets |\mh@currentrepos| to the new value if one is given in the
%   optional argument, and after importing resets |\mh@currentrepos| to the old value in
%   |\mh@@repos|.
%    \begin{macrocode}
%<ltxml>RawTeX('
%<*ltxml|package>
\newcommand\includemhassignment[2][]{\metasetkeys{inclassig}{#1}%
\edef\mh@@repos{\mh@currentrepos}%
\ifx\inclassig@mhrepos\@empty\else\mhcurrentrepos\inclassig@mhrepos\fi%
\input{\MathHub{\mh@currentrepos/source/#2}}%
\mhcurrentrepos\mh@@repos}
%</ltxml|package>
%<ltxml>');
%    \end{macrocode}
% \end{macro}
%
% \subsection{Leftovers}
%
% at some point, we may want to reactivate the logos font, then we use
% \begin{verbatim}
% here we define the logos that characterize the assignment
% \font\bierfont=../assignments/bierglas
% \font\denkerfont=../assignments/denker
% \font\uhrfont=../assignments/uhr
% \font\warnschildfont=../assignments/achtung
%
% \def\bierglas{{\bierfont\char65}}
% \def\denker{{\denkerfont\char65}}
% \def\uhr{{\uhrfont\char65}}
% \def\warnschild{{\warnschildfont\char 65}}
% \def\hardA{\warnschild}
% \def\longA{\uhr}
% \def\thinkA{\denker}
% \def\discussA{\bierglas}
% \end{verbatim}
%
% Finally, we need to terminate the file with a success mark for perl.
%    \begin{macrocode}
%<ltxml>1;
%    \end{macrocode}
% \Finale
\endinput
% \iffalse
% LocalWords:  GPL structuresharing STR iffalse cls NeedsTeXFormat hwexam hwexam.dtx sc
%%% Local Variables: 
%%% mode: doctex
%%% TeX-master: t
%%% End: 
% \fi
%  LocalWords:  texttt scsys sc latexml fileversion filedate maketitle setcounter newpage
%  LocalWords:  tocdepth tableofcontents pts showmeta showmeta showignores omdoc extrefs
%  LocalWords:  testspace testnewpage testemptypage testheading testheading reqpts reqpts
%  LocalWords:  exfig makeatletter makeatother vspace hrule vspace vspace noindent textsf
%  LocalWords:  includeassignment includeassignment HorIacJuc cscpnrr11 importmodule baz
%  LocalWords:  includemhassignment includemhassignment importmhmodule foobar ldots sref
%  LocalWords:  mhcurrentrepos mh-variants mh-variant compactenum printbibliography Cwd
%  LocalWords:  langle rangle langle rangle ltxml.cls ltxml.sty respetively metakeys qw
%  LocalWords:  cwd stex graphicx amssymb amstext amsmath newif iftest testfalse testtrue
%  LocalWords:  ifsolutions solutionsfalse ifmultiple multiplefalse multipletrue keyval
%  LocalWords:  ltxml assig srefaddidkey addmetakey ifx assignment@titleblock stepcounter
%  LocalWords:  document@hwexamtype importmodules metasetkeys inclassig@title inclassig
%  LocalWords:  inclassig@title inclassig@type inclassig@type inclassig@number xspace kv
%  LocalWords:  inclassig@number inclassig@due inclassig@due inclassig@given ignorespaces
%  LocalWords:  inclassig@given newenvironment currentsectionlevel OptionalKeyVals kvi
%  LocalWords:  omgroup vals hwexamtype ednote textbackslash newcommand inputassignment
%  LocalWords:  unlist quizheading tas hspace hfill textbf newcount vfill addtocounter
%  LocalWords:  theassignment@totalmin theassignment@totalpts assignment@probs xdef hline
%  LocalWords:  assignment@totalpts assignment@totalmin correction@probs correction@probs
%  LocalWords:  newcounter theassignment@probs footnotesize mh@currentrepos endinput
%  LocalWords:  inclassig@mhrepos inclassig@mhrepos doctex
