% \iffalse meta-comment
% A LaTeX Class and Package for the Smultiling Glossary
% Copyright (c) 2009 Michael Kohlhase, all rights reserved
%               this file is released under the
%               LaTeX Project Public License (LPPL)
%
% The original of this file is in the public repository at 
% http://github.com/KWARC/sTeX/
% \fi
%   
% \iffalse
%<package|cls>\NeedsTeXFormat{LaTeX2e}[1999/12/01]
%<cls>\ProvidesClass{smultiling}[2014/04/19 v0.1 Multilingual Support for sTeX]
%
%<*driver>
\documentclass{ltxdoc}
\usepackage{url,array,omdoc,omtext,float}
\usepackage[show]{ed}
\usepackage[hyperref=auto,style=alphabetic]{biblatex}
\addbibresource{kwarcpubs.bib}
\addbibresource{extpubs.bib}
\addbibresource{kwarccrossrefs.bib}
\addbibresource{extcrossrefs.bib}
\usepackage{stex-logo}
\usepackage{../ctangit}
\usepackage{hyperref}
\makeindex
\floatstyle{boxed}
\newfloat{exfig}{thp}{lop}
\floatname{exfig}{Example}
\def\tracissue#1{\cite{sTeX:online}, \hyperlink{hstp://trac.kwarc.info/sTeX/ticket/#1}{issue #1}}
\def\smglom{\textsf{SMGloM}\xspace}
\def\omdoc{\textsf{OMDoc}\xspace}
\def\latexml{{\LaTeX}ML\xspace}
\def\lmh{\textsf{lmh}\xspace}
\begin{document}\DocInput{smultiling.dtx}\end{document}
%</driver>
% \fi
% 
%\iffalse\CheckSum{382}\fi
% 
% \changes{v0.1}{2014/04/19}{First Version}
%
% 
% \MakeShortVerb{\|}
%
% \def\omdoc{OMDoc}
% \def\latexml{{\LaTeX}ML}
% \title{{\texttt{smultiling.sty}}: Multilinguality Support for \protect\sTeX}
%    \author{Michael Kohlhase, Deyan Ginev\\
%            Jacobs University, Bremen\\
%            \url{http://kwarc.info/kohlhase}}
% \maketitle
%
% \begin{abstract}
%   The |smultiling| package is part of the \sTeX collection, a version of {\TeX/\LaTeX}
%   that allows to markup {\TeX/\LaTeX} documents semantically without leaving the
%   document format, essentially turning {\TeX/\LaTeX} into a document format for
%   mathematical knowledge management (MKM).
%
%   The |smultiling| package adds multilinguality support for \sTeX, the idea is that
%   multilingual modules in \sTeX consist of a module signature together with multiple
%   language bindings that inherit symbols from it, which also account for cross-language
%   coordination.
% \end{abstract}
%
%\tableofcontents\newpage
% 
%\begin{omgroup}[id=sec:STR]{Introduction}
%  We have been using \sTeX as the encoding for the Semantic Multilingual Glossary of
%  Mathematics (\smglom; see~\cite{IanJucKoh:sps14}). The \smglom data model has been
%  taxing the representational capabilities of \sTeX with respect to multilingual support
%  and verbalization definitions; see~\cite{Kohlhase:dmesmgm14}, which we assume as
%  background reading for this note.
% 
% \begin{omgroup}{\protect\sTeX Module Signatures}
%   (monolingual) \sTeX had the intuition that the symbol definitions (|\symdef| and
%   |\symvariant|) are interspersed with the text and we generate \sTeX module signatures
%   (SMS |*.sms| files) from the \sTeX files. The SMS duplicate ``formal'' information
%   from the ``narrative'' \sTeX files. In the \smglom, we extend this idea by making the
%   the SMS primary objects that contain the language-independent part of the formal
%   structure conveyed by the \sTeX documents and there may be multiple narrative
%   ``language bindings'' that are translations of each other -- and as we do not want to
%   duplicate the formal parts, those are inherited from the SMS rather than written down
%   in the language binding itself. So instead of
% \begin{verbatim}
% \begin{module}[id=foo]
% \symdef{bar}{BAR}
% \begin{definition}[for=bar]
%   A \defiii{big}{array}{raster} ($\bar$) is a\ldots, it is much bigger 
%   than a \defiii[sar]{small}{array}{raster}.
% \end{definition}
% \end{module}
% \end{verbatim}
%
%   we now advocate the divided style in the listing below.
%
% \begin{verbatim}
% \usepackage[english,ngerman]{multiling}
% \begin{modsig}{foo}
% \symdef{bar}{BAR}
% \symbol{sar}
% \end{modsig}
%
% \begin{modnl}[creators=miko,primary]{foo}{en}
% \begin{definition}
%   A \defiii[bar]{big}{array}{raster} ($\bar$) is a\ldots, it is much bigger
%   than a \defiii[sar]{small}{array}{raster}. 
% \end{definition}
% \end{modnl}
%
% \begin{modnl}[creators=miko]{foo}{de}
% \begin{definition}
%   Ein \defiii[bar]{gro"ses}{Feld}{Raster} ($\bar$) ist ein\ldots, es
%   ist viel gr"o"ser als ein \defiii[sar]{kleines}{Feld}{Raster}. 
% \end{definition}
% \end{modnl}
% \end{verbatim}
%
%   There the |modsig| environment works exactly like the old |module| environment, only
%   that the |id| attribute has moved into the required argument -- anonymous module
%   signatures do not make sense. The |modnl| environment takes two arguments the first is
%   the name of the module signature it provides language bindings for and the second the
%   ISO 639 language specifier of the content language. We add the |primary| key |modnl|,
%   which can specify the primary language binding (the one the others translate from; and
%   which serves as the reference in case of translation conflicts).\ednote{@DG: This
%   needs to be implemented in LaTeXML}
% 
%   There is another difference in the multilingual encoding: All symbols are introduced
%   in the module signature, either by a |\symdef| or the new |\symbol| macro.
% 
%   We retain the old |module| environment as an intermediate stage. It is still useful
%   for monolingual texts. Note that for files with a module, we still have to extract
%   |*.sms| files. It is not completely clear yet, how to adapt the workflows. We clearly
%   need a \lmh or editor command that transfers an old-style module into a new-style
%   signature/binding combo to prepare it for multilingual treatment.
% \end{omgroup}
% \end{omgroup}
% 
% \begin{omgroup}[id=sec:user]{The User Interface}\
%
%   The |smultiling| package accepts all options of the |babel.sty| and just passes them
%   on to it. The options specify which languages can be used in the \sTeX language
%   bindings.
% 
% \end{omgroup}
% 
% \StopEventually{\newpage\PrintIndex\newpage\PrintChanges\printbibliography}\newpage
%
% \begin{omgroup}[id=sec:impl:cls]{Implementation}
%
%   Technically, the |smultiling| package is essentially a wrapper around the |babel|
%   package but allows specification of languages by their ISO 639 language codes.
% 
% \begin{omgroup}[id=sec:impl:cls:options]{Class Options}
%   To initialize the |smultiling| class, we pass on all options to |babel.cls| and record
%   which languages are loaded by defining |\smul@|\meta{language}|@loaded|
%   macros.\ednote{@DG: We also want to do that in \latexml}
% 
%   The \DescribeMacro{langfiles}|langfiles| option specifies that for a module
%   \meta{mod}, the module signature file has the name \meta{mod}|.tex| and the language
%   bindings of language with the ISO 639 language specifier \meta{lang} have the file
%   name \meta{mod}|.|\meta{lang}|.tex|.\ednote{implement other schemes, e.g. the onefile
%   scheme.}
% 
%    \begin{macrocode}
%<*sty>
\newif\if@langfiles\@langfilesfalse
\DeclareOption{langfiles}{\@langfilestrue}
\DeclareOption*{\PassOptionsToPackage{\CurrentOption}{babel}
\@namedef{smul@\CurrentOption @loaded}{yes}}
\ProcessOptions
%</sty>
%<*ltxml>
# -*- CPERL -*-
package LaTeXML::Package::Pool;
use strict;
use LaTeXML::Package;
DeclareOption('langfiles',sub {AssignValue('smultiling_langfiles',1,'global');});
DeclareOption(undef,sub {PassOptions('babel','sty',ToString(Digest(T_CS('\CurrentOption')))); });
ProcessOptions();
%</ltxml>
%    \end{macrocode}
%
% We load |babel.sty| 
%
%    \begin{macrocode}
%<*sty>
\RequirePackage{etoolbox}
\RequirePackage{babel}
%</sty>
%<*ltxml>
RequirePackage('babel');
%</ltxml>
%    \end{macrocode}
% \end{omgroup}
%
% \begin{omgroup}[id=sec:langbindings]{Handling Languages}
%
% \begin{macro}{\smg@select@language}
%   This macro selects one of the registered languages by its langauage code by setting
%   the internal |\smg@lang| macro to the argument and then runs the actual selection code
%   in |\smg@select@lang|. This internal code register is only initialized there, the code
%   is generated by the |\smg@register@language| macro below.
%    \begin{macrocode}
%<ltxml>RawTeX('
%<*sty|ltxml>
\newcommand\smg@select@lang{}
\newcommand\smg@select@language[1]{\def\smg@lang{#1}\smg@select@lang}
%    \end{macrocode}
% \end{macro}
%
% \begin{macro}{\smg@register@language}
%   |\smg@register@language{|\meta{lang}|}{|\meta{babel}|}| registers the |babel| language name
%   \meta{babel} with its ISO 639 languge code \meta{lang} by extending the
%   |\smg@select@language| macro. 
%    \begin{macrocode}
\newcommand\smg@register@language[2]%
{\@ifundefined{smul@#1@loaded}{}{\appto\smg@select@lang%
{\expandafter\ifstrequal\expandafter\smg@lang{#1}{\selectlanguage{#2}}{}}}}
%    \end{macrocode}
% \end{macro}
% Now we register a couple of languages for which we have |babel| support. Maybe we have
% to extend this list with others. But then we have to extend the mechanisms.
%    \begin{macrocode}
\smg@register@language{af}{afrikaans}
\smg@register@language{de}{ngerman}
\smg@register@language{fr}{french}%
\smg@register@language{he}{hebrew}
\smg@register@language{hu}{hungarian}
\smg@register@language{id}{indonesian}
\smg@register@language{ms}{malay}
\smg@register@language{nn}{nynorsk}
\smg@register@language{pt}{portuguese}
\smg@register@language{ru}{russian}
\smg@register@language{uk}{ukrainian}
\smg@register@language{en}{english}
\smg@register@language{es}{spanish}
\smg@register@language{sq}{albanian}
\smg@register@language{bg}{bulgarian}
\smg@register@language{ca}{catalan}
\smg@register@language{hr}{croatian}
\smg@register@language{cs}{czech}
\smg@register@language{da}{danish}
\smg@register@language{nl}{dutch}
\smg@register@language{eo}{esperanto}
\smg@register@language{et}{estonian}
\smg@register@language{fi}{finnish}
\smg@register@language{ka}{georgian}
\smg@register@language{el}{greek}
\smg@register@language{is}{icelandic}
\smg@register@language{it}{italian}
\smg@register@language{la}{latin}
\smg@register@language{no}{norsk}
\smg@register@language{pl}{polish}
\smg@register@language{sr}{serbian}
\smg@register@language{sk}{slovak}
\smg@register@language{sl}{slovenian}
\smg@register@language{sv}{swedish}
\smg@register@language{th}{thai}
\smg@register@language{tr}{turkish}
\smg@register@language{vi}{vietnamese}
\smg@register@language{cy}{welsh}
\smg@register@language{hi}{hindi}
%    \end{macrocode}
% \end{omgroup}
% 
% \begin{omgroup}{Signatures}
% 
% \begin{environment}{modsig}
%   The |modsig| environment is just a layer over the |module| environment. We also
%   redefine macros that may occur in module signatures so that they do not create markup.
%    \begin{macrocode}
\newenvironment{modsig}[2][]{%
\def\@test{#1}\ifx\@test\@empty\begin{module}[id=#2]\else\begin{module}[id=#2,#1]\fi}
{\end{module}}
%    \end{macrocode}
% \end{environment}
%
% \begin{environment}{viewsig}
%   The |viewsig| environment is just a layer over the |mhview| environment with the keys
%   suitably adapted.
%    \begin{macrocode}
\newenvironment{viewsig}[4][]{\def\@test{#1}\ifx\@test\@empty%
\begin{mhview}[id=#2,ext=tex]{#3}{#4}\else\begin{mhview}[id=#2,#1,ext=tex]{#3}{#4}\fi}
{\end{mhview}}
%<*sty|ltxml>
%<ltxml>');
%    \end{macrocode}
% \end{environment}
% \end{omgroup}
% 
% \begin{omgroup}[id=sec:langbindings]{Language Bindings}
% 
% \begin{macro}{modnl:*}
%    \begin{macrocode}
%<*sty>
\addmetakey{modnl}{load}
\addmetakey*{modnl}{title}
\addmetakey*{modnl}{creators}
\addmetakey*{modnl}{contributors}
\addmetakey{primary}{contributors}[yes]
%</sty>
%<*ltxml>
DefKeyVal('modnl','title','Semiverbatim');
DefKeyVal('modnl','load','Semiverbatim');
DefKeyVal('modnl','creators','Semiverbatim');
DefKeyVal('modnl','contributors','Semiverbatim');
DefKeyVal('modnl','primary','Semiverbatim');
%</ltxml>
%    \end{macrocode}
% \end{macro}
%
% \begin{environment}{modnl}
%   The |modnl| environment is just a layer over the |module| environment with the keys
%   and language suitably adapted.
%    \begin{macrocode}
%<*sty>
\newenvironment{modnl}[3][]{\metasetkeys{modnl}{#1}%
\smg@select@language{#3}%
\def\@test{#1}\ifx\@test\@empty\begin{module}[id=#2.#3]\else\begin{module}[id=#2.#3,#1]\fi%
\if@langfiles\importmodule[load=#2,ext=tex]{#2}\else
\ifx\modnl@load\@empty\importmodule{#2}\else\importmodule[ext=tex,load=\modnl@load]{#2}\fi%
\fi}
{\end{module}}
%</sty>
%<*ltxml>
DefEnvironment('{modnl} OptionalKeyVals:modnl {}{}',
         "<omdoc:theory "
         . 'xml:id="#2.#3">'
         .   "?&defined(&GetKeyVal(#1,'creators'))(<dc:creator>&GetKeyVal(#1,'creators')</dc:creator>)()"
         .   "?&defined(&GetKeyVal(#1,'title'))(<dc:title>&GetKeyVal(#1,'title')</dc:title>)()"
         .   "?&defined(&GetKeyVal(#1,'contributors'))(<dc:contributor>&GetKeyVal(#1,'contributors')</dc:contributor>)()"
         .   "#body"
         . "</omdoc:theory>",
  afterDigestBegin=>sub {
    my ($stomach, $whatsit) = @_;
    my $keyval = $whatsit->getArg(1);
    my $signature = ToString($whatsit->getArg(2));
    if ($keyval) {
      # If we're not given load, AND the langfiles option is in effect,
      # default to #2
      if ((! $keyval->getValue('load')) && (LookupValue('smultiling_langfiles'))) {
        $keyval->setValue('load',$signature); }
      # Always load a TeX file
      $keyval->setValue('ext','tex'); }
    importmoduleI($stomach,$whatsit)});
%</ltxml>%$
%    \end{macrocode}
% \end{environment}
%
% \end{omgroup}
% \end{omgroup} 
% \Finale
%<ltxml>1;
\endinput
% \iffalse
%%% Local Variables: 
%%% mode: doctex
%%% TeX-master: t
%%% End: 
% \fi

% LocalWords:  iffalse cls omdoc latexml texttt smlog.cls sref
% LocalWords:  maketitle newpage tableofcontents newpage omgroup ednote ltxml
% LocalWords:  printbibliography showmeta metakeys amstext ginput newcommand
% LocalWords:  module-defs gimport renewcommand langbindings gle newenvironment
% LocalWords:  doctex
