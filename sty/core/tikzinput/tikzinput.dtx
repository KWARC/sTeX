% \iffalse meta-comment
% A LaTeX Package for selective input of external TIKZ pictures
% Copyright (c) 2015 Michael Kohlhase, all rights reserved
%               this file is released under the
%               LaTeX Project Public License (LPPL)
%
% The original of this file is in the public repository at 
% http://github.com/KWARC/sTeX/
% \fi
%   
% \iffalse
%<package>\NeedsTeXFormat{LaTeX2e}[1999/12/01]
%<package>\ProvidesPackage{tikzinput}[2016/04/06 v1.0 Selective input of TIKZ pictures]
%
%<*driver>
\documentclass{ltxdoc}
\usepackage{url,array,float}
\usepackage{tikzinput,omdoc}
\usepackage[show]{ed}
\usepackage[hyperref=auto,style=alphabetic]{biblatex}
\addbibresource{kwarcpubs.bib}
\addbibresource{extpubs.bib}
\addbibresource{kwarccrossrefs.bib}
\addbibresource{extcrossrefs.bib}
\usepackage{stex-logo}
\usepackage{ctangit}
\usepackage{hyperref}
\makeindex
\floatstyle{boxed}
\newfloat{exfig}{thp}{lop}
\floatname{exfig}{Example}
\def\tracissue#1{\cite{sTeX:online}, \hyperlink{http://trac.kwarc.info/sTeX/ticket/#1}{issue #1}}
\begin{document}\DocInput{tikzinput.dtx}\end{document}
%</driver>
% \fi
% 
%\CheckSum{37}
% 
% \changes{v1.0}{2015/10/22}{self-documenting package}
% 
% \GetFileInfo{tikzinput.sty}
% 
% \MakeShortVerb{\|}
%
% \def\omdoc{OMDoc}
% \def\latexml{{\LaTeX}ML}
% \title{\texttt{tikzinput}: Selective Input of TIKZ Pictures\thanks{Version {\fileversion} (last revised
%        {\filedate})}}
%    \author{Michael Kohlhase\\
%            Jacobs University, Bremen\\
%            \url{http://kwarc.info/kohlhase}}
% \maketitle
%
% \begin{abstract}
%   Running |tikz| takes a lot of time in \latexml, therefore it is often more efficient
%   externalize the TIKZ pictures into separate (standalone) files, to let {\LaTeX} handle
%   the TIKZ pictures to generate an image, and just load it via the usual {\LaTeX}
%   graphics packages. The |tikzinput| package supports this workflow, and allows to
%   switch back to native TIKZ via a package option.
% \end{abstract}
%
%\tableofcontents\newpage
% 
%\begin{omgroup}[id=sec:STR]{Introduction}
%
%   Running |tikz| takes a lot of time in \latexml, therefore it is often more efficient
%   externalize the TIKZ pictures into separate (standalone) files, to let {\LaTeX} handle
%   the TIKZ pictures to generate an image, and just load it via the usual {\LaTeX}
%   graphics packages. The |tikzinput| package supports this workflow, and allows to
%   switch back to native TIKZ via a package option. 
% 
%   A side-effect of the workflow described above is that the TIKZ pictures can be
%   developed -- and formatted -- independently of the document they are intended
%   for. They can essentially be treated like an image file, which can be included into
%   multiple documents.
%
%\end{omgroup}
%
% \begin{omgroup}[id=sec:user]{The User Interface}
% 
% \begin{omgroup}[id=sec:user:options]{Package Options}
%
%   The behavior of the |tikzinput| package is determined by whether the
%   \DescribeMacro{image}|image| option is given. If it is not, then the |tikz| package is
%   loaded, all other options are passed on to it and |\tikzinput{|\meta{file}|}| inputs
%   the TIKZ file \meta{file}|.tex|; if not, only the |graphicx| package is loaded and
%   |\tikzinput{|\meta{file}|}| loads an image file \meta{file}|.|\meta{ext} generated
%   from \meta{file}|.tex|.
% 
% \end{omgroup}
% 
% \begin{omgroup}{Inputting Standalone TIKZ Pictures}
% 
%   The selective input functionality of the |tikzinput| package assumes that the TIKZ
%   pictures are externalized into a standalone picture file, such as the one
%   Example~\ref{fig:tikzpic}.
% 
% \begin{exfig}
% \begin{verbatim}
% \documentclass{standalone}
% \usepackage{tikz}
% \usetikzpackage{...}
% \begin{document}
%   \begin{tikzpicture}
%      ...
%   \end{tikzpicture}
% \end{document}
% \end{verbatim}
%  \caption{A Standalone TIKZ Picture File}\label{fig:tikzpic} 
% \end{exfig}
% \end{omgroup}
% \end{omgroup}
%
% The |standalone| class is a minimal {\LaTeX} class that when loaded in a document that
% uses the |standalone| package: the preamble and the |document| environment are
% disregarded during loading, so they do not pose any problems. In effect, an |\input| of
% the file in Figure~\ref{fig:tikzpic} only sees the |tikzpicture| environment, but the
% file itself is standalone in the sense that we can run {\LaTeX} over it separately,
% e.g. for generating an image file from it.
% 
% This is exactly where the |tikzinput| package comes in: it supplies the
% \DescribeMacro{\tikzinput}|\tikzinput| macro, which -- depending on the |image| option
% -- either directly inputs the TIKZ picture (source) or tries to load an image file
% generated from it.
% 
% Concretely, if the |image| option is not set for the |tikzinput| package, then
% |\tikzinput[|\meta{opt}|]{|\meta{file}|}| disregards the optional argument \meta{opt}
% and inputs \meta{file}|.tex| via |\input|. If it is,
% |\tikzinput[|\meta{opt}|]{|\meta{file}|}| expands to
% |\includegraphics[|\meta{opt}|]{|\meta{file}|}|.
% 
% \begin{omgroup}{Limitations}\label{sec:limitations}
% 
% In this section we document known limitations. If you want to help alleviate them,
% please feel free to contact the package author. Some of them are currently discussed in
% the \sTeX GitHub repository~\cite{sTeX:github:on}. 
% \begin{compactenum}
% \item none reported yet
% \end{compactenum}
% \end{omgroup}
% 
% \StopEventually{\newpage\PrintIndex\newpage\PrintChanges\printbibliography}\newpage
%
% \begin{omgroup}[id=sec:impl]{Implementation}
% 
% \begin{omgroup}{Package Options and Required Packages}
%
%   We define a new switch \DescribeMacro{\iftikzinput@image}|\iftikzinput@image| and the
%   |image| option. Apart from that we accept all options that might come our
%   way.\ednote{MK: Actually we would have liked to pass all options to TIKZ, but that
%   does not work, since that is specific about its options.}
%    \begin{macrocode}  
%<*package>
\newif\if@tikzinput@mh@\@tikzinput@mh@false
\DeclareOption{mh}{\@tikzinput@mh@true}
\newif\iftikzinput@image\tikzinput@imagefalse
\DeclareOption{image}{\tikzinput@imagetrue}
\DeclareOption*{}
\ProcessOptions
%    \end{macrocode}
% Next we require the packages we need, in the |image| case, we have to also provide
% ``empty'' versions of some TIKZ macros and environments that do not get defined as the
% |tikz| package is not loaded.
%    \begin{macrocode}  
\if@tikzinput@mh@\RequirePackage{tikzinput-mh}\fi
\iftikzinput@image
\RequirePackage{graphicx}
\providecommand\usetikzlibrary[1]{}
\else
\RequirePackage{tikz}
\RequirePackage{standalone}
\fi
%    \end{macrocode}
% \end{omgroup}
% 
% \begin{omgroup}{Inputting Standalone TIKZ Pictures}
%
% \begin{macro}{\tikzinput}
%   Depending on the |image| option, we do the necessary things.  
%    \begin{macrocode}  
\iftikzinput@image
\newcommand\tikzinput[2][]{\includegraphics[#1]{#2}}
\else
\newcommand\tikzinput[2][]{\input{#2}}
\fi
%    \end{macrocode}
% \end{macro}
% 
% \begin{macro}{\*tikzinput}
%   The variants we define in terms of |\tikzinput|.
%    \begin{macrocode}
\newcommand\ctikzinput[2][]{\begin{center}\tikzinput{#2}\end{center}}
%</package>
%    \end{macrocode}
% \end{macro}
% \end{omgroup}
% \end{omgroup}
% 
% \Finale
% \endinput
% Local Variables:
% mode: doctex
% TeX-master: t
% End:

%  LocalWords:  iffalse NeedsTeXFormat tikzinput tikzinput.dtx omdoc latexml texttt tikz
%  LocalWords:  maketitle externalize newpage tableofcontents newpage omgroup behavior
%  LocalWords:  graphicx externalized tikzpic exfig compactenum printbibliography impl
%  LocalWords:  ltxml iftikzinput@image iftikzinput@image ednote newif etoolbox doctex
%  LocalWords:  includegraphics providecommand usetikzlibrary ctikzinput
