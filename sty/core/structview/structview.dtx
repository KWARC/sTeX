% \iffalse meta-comment
% An Infrastructure for Structures and Views
% Copyright (C) 2004-2013 Michael Kohlhase, all rights reserved
%               this file is released under the
%               LaTeX Project Public License (LPPL)
%
% The original of this file is in the public repository at 
% http://github.com/KWARC/sTeX/
% \fi
%  
% \iffalse
%<package>\NeedsTeXFormat{LaTeX2e}[1999/12/01]
%<package>\ProvidesPackage{structview}[2016/04/05 v0.9 Structures and Views]
%
%<*driver>
\documentclass{ltxdoc}
\usepackage{stex-logo,structview}
\usepackage{url,array,float,textcomp}
\usepackage[show]{ed}
\usepackage[hyperref=auto,style=alphabetic]{biblatex}
\addbibresource{kwarcpubs.bib}
\addbibresource{extpubs.bib}
\addbibresource{kwarccrossrefs.bib}
\addbibresource{extcrossrefs.bib}
\usepackage{listings}
\usepackage{amsfonts}
\usepackage{ctangit}
\usepackage{xspace}
\usepackage{hyperref}
\makeindex
\floatstyle{boxed}
\newfloat{exfig}{thp}{lop}
\floatname{exfig}{Example}
\def\tracissue#1{\cite{sTeX:online}, \hyperlink{http://trac.kwarc.info/sTeX/ticket/#1}{issue #1}}
\begin{document}\DocInput{structview.dtx}\end{document} 
%</driver>
% \fi
% 
% \CheckSum{926}
%
% \changes{v0.9}{2015/11/05}{Moved out of the \texttt{modules} package}
% 
% \GetFileInfo{modules.sty}
% 
% \MakeShortVerb{\|}
%\def\scsys#1{{{\sc #1}}\index{#1@{\sc #1}}\xspace}
% \def\xml{\scsys{Xml}}
% \def\mathml{\scsys{MathML}}
% \def\omdoc{\scsys{OMDoc}}
% \def\openmath{\scsys{OpenMath}}
% \def\latexml{\scsys{LaTeXML}}
% \def\perl{\scsys{Perl}}
% \def\cmathml{Content-{\sc MathML}\index{Content {\sc MathML}}\index{MathML@{\sc MathML}!content}}
% \def\activemath{\scsys{ActiveMath}}
% \def\twin#1#2{\index{#1!#2}\index{#2!#1}}
% \def\twintoo#1#2{{#1 #2}\twin{#1}{#2}}
% \def\atwin#1#2#3{\index{#1!#2!#3}\index{#3!#2 (#1)}}
% \def\atwintoo#1#2#3{{#1 #2 #3}\atwin{#1}{#2}{#3}}
% \def\cT{\mathcal{T}}\def\cD{\mathcal{D}}
% \title{{\texttt{structview.sty}}: Structures and Views in {\stex}\thanks{Version {\fileversion} (last revised
%        {\filedate})}}
%    \author{Michael Kohlhase\\
%            Jacobs University, Bremen\\
%            \url{http://kwarc.info/kohlhase}}
% \maketitle
%
% \begin{abstract}
%   The |structview| package is part of the {\stex} collection, a version of {\TeX/\LaTeX}
%   that allows to markup {\TeX/\LaTeX} documents semantically without leaving the
%   document format, essentially turning {\TeX/\LaTeX} into a document format for
%   mathematical knowledge management (MKM).
%
%   This package supplies infrastructure for \omdoc structures and views: complex semantic
%   relations between modules/theories.
% \end{abstract}
%
% \newpage\setcounter{tocdepth}{2}\tableofcontents\newpage
%
% \section{Introduction}\label{sec:intro}
% \ednote{What are structures and views?}
% \section{The User Interface}
%
% The main contributions of the |modules| package are the |module| environment, which
% allows for lexical scoping of semantic macros with inheritance and the |\symdef| macro
% for declaration of semantic macros that underly the |module| scoping.
% 
% \subsection{Package Options}\label{sec:options}
%
% The |modules| package takes two options: If we set
% \DescribeMacro{showmods}|showmods|\ednote{This mechanism does not work yet, since we
% cannot disable it when importing modules and that leads to unwanted boxes. What we need
% to do instead is to tweak the sms utility to use an internal version that never shows
% anything during sms reading.}, then the views (see Section~\ref{sec:user:views}) are
% shown. If we set the \DescribeMacro{qualifiedimports}|qualifiedimports| option, then
% qualified imports are enabled. Qualified imports give more flexibility in module
% inheritance, but consume more internal memory. As qualified imports are not fully
% implemented at the moment, they are turned off by default see
% Limitation~\ref{sec:limitations:qualified-imports}. The option
% \DescribeMacro{noauxreq}|noauxreq| prohibits the registration of |\@requiremodules|
% commands in the |aux| file. They are necessary for preloading the module signatures so
% that entries in the table of contents can have semantic macros; but as they sometimes
% cause trouble the option allows to turn off preloading.
% 
% If the \DescribeMacro{showmeta}|showmeta| is set, then the metadata keys are shown
% (see~\cite{Kohlhase:metakeys:ctan} for details and customization options).
% 
% \subsection{Structures}\label{sec:user:structures}
% 
% The |\importmodule| macro has a variant
% \DescribeMacro{importmodulevia}|\importmodulevia| that allows the specification of a
% theory morphism to be applied.  |\importmodulevia{|\meta{thyid}|}{|\meta{assignments}|}|
% specifies the ``source theory'' via its identifier \meta{thyid} and the morphism by
% \meta{assignments}. There are four kinds:\ednote{MK: this needs to be consolidated and
% researched better.}
% \begin{compactdesc}
% \item[symbol assignments] via
%   \DescribeMacro{\vassign}|\vassign{|\meta{sym}|}{|\meta{exp}|}|, which defines the
%   symbol \meta{sym} introduced in the current theory by an expression \meta{exp} in the
%   source theory.
% \item[function assignments] via
%   \DescribeMacro{\fassign}|\fassign{|\meta{bvars}|}{|\meta{pat}|}{|\meta{exp}|}|, is a
%   variant which defines a function symbol \meta{sym} introduced in the current theory by
%   mapping a pattern expression \meta{pat} (\meta{sym} applied to \meta{bvars}) to an
%   expression \meta{exp} in the source theory on bound variables \meta{bvars}.
% \item[term assignments] via
%   \DescribeMacro{\tassign}|\tassign[|\meta{source-cd}|]{|\meta{tname}|}{|\meta{source-tname}|}|,
%   which assigns to the term with name \meta{tname} in the current theory a term with
%   name \meta{source-tname} in the theory \meta{source-cd} whose default value is the
%   source theory.
% \item[term text assignments] via
%   \DescribeMacro{\ttassign}|\tassign{|\meta{tname}|}{|\meta{text}|}|, which defines a
%   term with name \meta{tname} in the current theory via a definitional text.
% \end{compactdesc}
% 
%\begin{exfig}
% \begin{verbatim}
% \begin{module}[id=ring]
% \begin{importmodulevia}{monoid}
%   \vassign{rbase}\magbase
%   \fassign{a,b}{\rtimes{A}B}{\magmaop{a}b}
%   \vassign{rone}\monunit
% \end{importmodulevia}
% \symdef{rbase}{G}
% \symdef[name=rtimes]{rtimesOp}{\cdot}
% \symdef{rtimes}[2]{\infix\rtimesOp{#1}{#2}}
% \symdef{rone}{1}
% \begin{importmodulevia}{cgroup}
%   \vassign{rplus}\magmaop
%   \vassign{rzero}\monunit
%   \vassign{rinvOp}\cginvOp
% \end{importmodulevia}
% \symdef[name=rplus]{rplusOp}{+}
% \symdef{rplus}[2]{\infix\rplusOp{#1}{#2}}
% \symdef[name=rminus]{rminusOp}{-}
% \symdef{rminus}[1]{\infix\rminusOp{#1}{#2}}
% ...
% \end{module}
% \end{verbatim}
% \caption{A Module for Rings with inheritance from monoids and commutative groups}\label{fig:ring}
% \end{exfig}
%
% \subsection{Views}\label{sec:user:views}
% 
% A view is a mapping between modules, such that all model assumptions (axioms) of the
% source module are satisfied in the target module. \ednote{Document and make Examples}
%
% \section{Limitations \& Extensions}\label{sec:limitations}
% 
% In this section we will discuss limitations and possible extensions of the |modules|
% package. Any contributions and extension ideas are welcome; please discuss ideas,
% requests, fixes, etc on the {\sTeX} TRAC~\cite{sTeX:online}.
% 
% \section{The Implementation}\label{sec:impl} 
%
% \subsection{Package Options}\label{sec:impl:options}
% 
% We declare some switches which will modify the behavior according to the package
% options. Generally, an option |xxx| will just set the appropriate switches to true
% (otherwise they stay false). The options we are not using, we pass on to the |sref|
% package we require next.
%    \begin{macrocode}
%<*package>
\newif\if@structview@mh@\@structview@mh@false
\DeclareOption{mh}{\@structview@mh@true
\PassOptionsToPackage{\CurrentOption}{modules}}
\DeclareOption*{\PassOptionsToPackage{\CurrentOption}{modules}}
\ProcessOptions
%    \end{macrocode}
%
% The next measure is to ensure that the |sref| and |xcomment| packages are loaded (in the
% right version). For {\latexml}, we also initialize the package inclusions.
%    \begin{macrocode}
\if@structview@mh@\RequirePackage{structview-mh}\fi
\RequirePackage{modules}
%    \end{macrocode}
%
% \subsection{Structures}\label{sec:impl:structures}
% 
% \begin{macro}{\importmodulevia} 
%   The |importmodulevia| environment just calls |\importmodule|, but to get around the
%   group, we first define a local macro |\@@doit|, which does that and can be called with
%   an |\aftergroup| to escape the environment grouping introduced by |importmodulevia|.
%    \begin{macrocode}
\newenvironment{importmodulevia}[2][]{%
  \gdef\@@doit{\importmodule[#1]{#2}}%
  \ifmod@show\par\noindent importing module #2 via \@@doit\fi%
}{%
  \aftergroup\@@doit\ifmod@show end import\fi%
}%
%    \end{macrocode}
% \end{macro}
%
% \begin{environment}{\*assign}
%    \begin{macrocode}
\newrobustcmd\vassign[3][]{\ifmod@show\ensuremath{#2\mapsto #3}, \fi}%
\newrobustcmd\tassign[3][]{\ifmod@show #2\ensuremath{\mapsto} #3, \fi}%
\newrobustcmd\fassign[4][]{\ifmod@show \ensuremath{#3\mapsto #4}, \fi}%
\newrobustcmd\ttassign[3][]{\ifmod@show #2\ensuremath{\mapsto} ``#3'', \fi}%
%    \end{macrocode}
% \end{environment}
% 
% \subsection{Views}\label{sec:impl:views}
% 
% We first prepare the ground by defining the keys for the |view| environment.
%    \begin{macrocode}
\srefaddidkey{view}
\addmetakey*{view}{title}
\addmetakey{view}{display}
\addmetakey{view}{from}
\addmetakey{view}{to}
\addmetakey{view}{creators}
\addmetakey{view}{contributors}
\addmetakey{view}{srccite}
\addmetakey{view}{type}
\addmetakey[sms]{view}{ext}
%    \end{macrocode}
%
% \begin{macro}{\view@heading}
%   Then we make a convenience macro for the view heading. This can be customized. 
%    \begin{macrocode} 
\newcounter{view}[section]
\newrobustcmd\view@heading[4]{%
  \if@importing%
  \else%
    \stepcounter{view}%
    \edef\@display{#3}\edef\@title{#4}%
    \noindent%
      \ifx\@display\st@flow%
      \else%
        {\textbf{View} {\thesection.\theview} from \textsf{#1} to \textsf{#2}}%
        \sref@label@id{View \thesection.\theview}%
        \ifx\@title\@empty%
          \quad%
        \else%
          \quad(\@title)%
        \fi%
        \par\noindent%
      \fi%
      \ignorespaces%
  \fi%
}%ifmod@show
%    \end{macrocode}
% \end{macro}
%
% \begin{environment}{view}
%   The |view| environment relies on the |@view| environment (used also in the \sTeX
%   module signatures) for module bookkeeping and adds presentation (a heading and a box)
%   if the |showmods| option is set.
%    \begin{macrocode}
\newenvironment{view}[3][]{%
  \metasetkeys{view}{#1}%
  \sref@target%
  \begin{@view}{#2}{#3}%
  \view@heading{#2}{#3}{\view@display}{\view@title}%
}{%
  \end{@view}%
  \ignorespaces%
}%
\ifmod@show\surroundwithmdframed{view}\fi%
%    \end{macrocode}
% \end{environment}
%
% \begin{environment}{@view}
%   The |@view| does the actual bookkeeping at the module level.
%    \begin{macrocode}
\newenvironment{@view}[2]{%from, to
  \@importmodule[\view@from]{#1}{\view@ext}%
  \@importmodule[\view@to]{#2}{\view@ext}%
}{}%
%    \end{macrocode}
% \end{environment}
% 
% \begin{environment}{viewsketch}
%   The |viewsketch| environment behaves like |view|, but only has text contents.
%    \begin{macrocode}
\newenvironment{viewsketch}[3][]{%
  \metasetkeys{view}{#1}%
  \sref@target%
  \begin{@view}{#2}{#3}%
  \view@heading{#2}{#3}{\view@display}{\view@title}%
}{%
  \end{@view}%
}%
\ifmod@show\surroundwithmdframed{viewsketch}\fi%
%    \end{macrocode}
% \end{environment}
%
% \begin{macro}{\obligation}
%   The |\obligation| element does not do anything yet on the latexml
%   side.\ednote{document above}
%    \begin{macrocode}
\newrobustcmd\obligation[3][]{%
  \if@importing%
  \else Axiom #2 is proven by \sref{#3}%
  \fi%
}%
%</package>
%    \end{macrocode}
% \end{macro}
% \Finale
\endinput
%%% Local Variables: 
%%% mode: doctex
%%% TeX-master: t
%%% End: 
