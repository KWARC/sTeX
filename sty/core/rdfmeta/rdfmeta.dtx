% \iffalse meta-comment
% An Infrastructure for extending macros by general metadata keys
% Copyright (c) 2007 Michael Kohlhase, all rights reserved
%               this file is released under the
%               LaTeX Project Public License (LPPL)
% The original of this file is in the public repository at 
% http://github.com/KWARC/sTeX/
% \fi
% 
% \iffalse
%<*package>
\NeedsTeXFormat{LaTeX2e}[1999/12/01]
\ProvidesPackage{rdfmeta}[2016/04/07 v0.2 Metadata key upgrades]
%</package>
%<*driver>
\documentclass{ltxdoc}
\usepackage[utf8]{inputenc}
\usepackage[sectioning]{rdfmeta}
\usepackage{url,array,float,stex-logo}
\usepackage[show]{ed}
\usepackage{listings}
\lstset{basicstyle=\small\tt,columns=flexible}
\lstset{frame=none,numbers=none,lineskip=-.7ex,aboveskip=-.5ex,belowskip=-.5ex}
\usepackage[hyperref=auto,style=alphabetic]{biblatex}
\addbibresource{kwarcpubs.bib}
\addbibresource{extpubs.bib}
\addbibresource{kwarccrossrefs.bib}
\addbibresource{extcrossrefs.bib}
\usepackage{ctangit}
\usepackage{hyperref}
\makeindex
\floatstyle{boxed}
\newfloat{exfig}{thp}{lop}
\floatname{exfig}{Example}
\def\githubissue#1{\cite{sTeX:github:on}, \hyperlink{https://github.com/KWARC/sTeX/issues/#1}{issue #1}}
\def\omdoc{OMDoc\xspace}
\def\omdocv#1{OMDoc{#1}\xspace}
\begin{document}
\RecordChanges
\DocInput{rdfmeta.dtx}
\end{document}
%</driver>
% \fi
%
%\iffalse\CheckSum{175}\fi
%
% \changes{v0.1}{2010/03/05}{new}
% \changes{v0.2}{2010/09/01}{starred forms work now, more documentation}
% 
% \GetFileInfo{rdfmeta.sty}
% 
% \MakeShortVerb{\|}
% \def\latexml{{\LaTeX}ML}
%
% \title{RDFa Metadata in {\LaTeX}\thanks{Version {\fileversion} (last revised
%        {\filedate})}}
% \author{Michael Kohlhase \\
%         Jacobs University, Bremen\\
%         \url{http://kwarc.info/kohlhase}}
% \maketitle
% \begin{abstract}
%   The |rdfmeta| package allows mark up Ontology-based Metadata in {\LaTeX} documents
%   that can be harvested by automated tools or exported to PDF.
% \end{abstract}
%\tableofcontents
%\begin{center}\bf\Huge
%  Experimental!\\
%  do not use!
%\end{center}\newpage
%
% \section[id=intro]{Introduction}
%
% The |rdfmeta| package allows mark up extensible metadata in \stex documents, so that
% that it can be harvested by automated tools or exported to PDF. It is also intended to
% support the new metadata infrastructure for the {\omdoc} format~\cite{Kohlhase:OMDoc1.2}
% introduced in {\omdocv{1.3}}~\cite{Kohlhase:OMDoc1.3} (see~\cite{LK:MathOntoAuthDoc09}
% for the relevant ideas and and~\cite{KohKohLan:ssffld10:biblatex} for an application).
%
% Metadata are annotated as key value pairs in the semantic environments provided by
% \stex. In most markup formats, the metadata vocabularies are fixed by the language
% designer. In \stex, the |rdfmeta| package allows the user to extend the metadata
% vocabulary.
% 
% \begin{exfig}[ht]
% \begin{verbatim}
% \importmodule[../ontologies/cert]{certification}
% ... 
% \section[id=userreq,hasState=$\statedocrd{\tuev}$]{User Requirements}
% ...
% \end{verbatim}
% \vspace*{-1.5em}\hrule
% \lstinputlisting[language=XMl,morekeywords={imports,metadata,resource,link,omgroup},aboveskip=.5ex]{hasstate.omdoc}
% \caption{Metadata for Certification}\label{fig:hasState}
% \end{exfig}
% 
% Take, for instance, the case where we want to use metadata for the certification status
% of document fragments. In Figure~\ref{fig:hasState} we use the |hasState| key to say
% that a section has been approved by the T\"UV, a specific certification agency. There
% are two concerns here. First, the |hasState| key has to be introduced and given a
% meaning, and same for the (complex) value |\statedocrd{\tuev}|. This meaning is given in
% the |certifiation| ontology which we imported via the |\importmodule| command. The
% ontology can be marked up in \stex (see Figure~\ref{fig:certification}), with the
% exception that we use the |\keydef| macro for the definition of the |hasState| relation
% so that it also defines the key. For the details of this see the next section.
%
% \section[id=user]{User Interface}
%
% We now document the specifics of the environments and macros provided by the |rdfmeta|
% package from a user perspective. 
%
% \subsection[id=sec:user:options]{Package Options}
% 
% The |rdfmeta| package takes the option: \DescribeMacro{showmeta}|showmeta|. If this is
% set, then the metadata keys are shown (see~\cite{Kohlhase:metakeys:ctan} for details and
% customization options).
% 
% The remaining options can be used to specify metadata upgrades of standard keys. The
% \DescribeMacro{sectioning}|sectioning| option upgrades the |\part|, |\chapter|,
% |\section|, |\subsection|, |\subsubsection|, |\paragraph| macros (and of course their
% starred variants). 
%
% \subsection[id=sec:user:keydef]{Extending Macros and Environments by Metadata Keys}
% 
% The main user-visible feature of the |rdfmeta| package is the
% \DescribeMacro{\keydef}|keydef| macro. It takes two arguments, a ``key group
% identifier'' and a key name. In a nutshell, every \stex command that takes metadata keys
% comes with a ``key group identifier'' that identifies the set of admissible keys;
% see~\cite{Kohlhase:metakeys:ctan} for details on this concept. Figure~\ref{fig:keygroups}
% gives an overview over the key groups and their identifiers in \stex.
% \begin{figure}
% \begin{tabular}{|>{\tt}l|>{\tt}p{5.5cm}|>{\tt}p{2.4cm}|}\hline
%   \rmfamily\textbf{Key Group Identifier} & \rmfamily\textbf{Macros} & \rmfamily\textbf{Package/Class} \\\hline\hline 
%   dcm@person & DCMPerson & dcm.sty\\\hline
%   dcm@institution & DCMInstitution & dcm.sty\\\hline
%   dcm@sect & section & dcm.sty\\\hline
%   assig & assignment & hwexam.cls\\\hline
%   inclassig & includeassignment & hwexam.cls\\\hline
%   quizheading & quizheading & hwexam.cls\\\hline
%   testheading & quizheading & hwexam.cls\\\hline
%   module & module & modules.sty\\\hline
%   termdef & termdef & modules.sty \\\hline
%   view & view & modules.sty\\\hline
%   omgroup & omgroup & omdoc.sty\\\hline
%   ignore & ignore & omdoc.sty\\\hline
%   omtext & omtext, definition, axiom, assertion, example, inlinedef & omtext.sty, statements.sty\\\hline
%   phrase & phrase & omtext.sty\\\hline
%   problem & problem & problem.sty\\\hline
%   inclprob & includeproblem & problem.sty\\\hline
%   req & requirement & reqdoc.sty\\\hline
%   spf & sproof, spfcases, spfcase, spfstep, spfcomment & sproof.sty\\\hline
%   termref & termref & statements.sty \\\hline
%   symboldec & symboldec & statements.sty\\\hline
% \end{tabular}
% \caption{Key Group Identifiers in \protect\stex}\label{fig:keygroups}
% \end{figure}
%
% Semantically, |\keydef{|\meta{keygroup}|}{|\meta{key}|}| defines a symbol just like the
% |\symdef| macro from the |modules| package~\cite{KohAmb:smmssl:ctan}. But it also
% extends the syntax of \stex itself: it adds a key \meta{key} to \meta{keygroup}, which
% allows to state the corresponding metadata as a key/value pair in the \stex macro or
% environment. Following the ideas from~\cite{LK:MathOntoAuthDoc09}, the metadata is
% transformed to RDFa metadata~\cite{w3c:WD-rdfa-core-20100803} in {\omdoc}, where the
% identifiers of relations are exactly the symbols introduced by the corresponding
% |\keydef|.
% 
% \begin{exfig}[ht]
% \lstinputlisting[language=TeX,morekeywords={metalanguage}]{certification}
% \caption{A simple Ontology on Certification}\label{fig:certification}
% \end{exfig}
%
% In our example in Figure~\ref{fig:certification} we have defined a key |hasstate| in the
% |omtext| key group\footnote{For the \texttt{\textbackslash omtext} environment and key group
% see~\cite{Kohlhase:smmtf:ctan}} and a symbol |hasstate| via
% |\addkey{omtext}{hasstate}|. Furthermore, we have defined the meaning of the relation
% expressed by the |hasstate| symbol informally and specified some possible objects for
% the relation (that could of course have been done in other modules as well). We have
% made use of this metadata ontology and the new key |hasState| in the example in
% Figure~\ref{fig:hasState}.
%
% \subsection{Redefinitions of Common {\LaTeX} Macros and Environments} 
%
% The |rdfmeta| package redefines common {\LaTeX} commands (e.\,g.\ the sectioning macros)
% so that they include optional KeyVal arguments that can be extended by |\keydef|
% commands. With this extension, we can add RDFa metadata to any existing {\LaTeX}
% document and generate linked data (XHTML+RDFa documents) via the {\latexml}
% translator. 
% 
% \subsection[id=redefining]{Extending Packages with \texttt{rdfmeta}}
% 
% The |rdfameta| package also exposes its internal infrastructure for extending the
% redefinitions. Note that the upgrade macros can only be used in {\LaTeX} packages, as
% the macro names contain |@|. Consequently, this section is only addressed at package
% developers who want to extend existing (i.e. not written by them) packages with flexible
% metadata functionality.
% 
% \DescribeMacro{\rdfmeta@upgrade}|\rdfmeta@upgrade| is the basic upgrade macro. It takes
% an optional keyval argument an a command sequence \meta{cseq} as a proper argument and
% (if that is defined), redefines |\|\meta{cseq} to take a keyval argument.  There is a
% variant \DescribeMacro{\rdfmeta@upgrade*}|\rdfmeta@upgrade*| that has to be used to
% upgrade macros that have a starred form (e.g. |\section| and friends). Note that
% |\rdfmeta@upgrade*| upgrades both forms (e.g. |\section| and |\section*|).
%
% |\rdfmeta@upgrade| uses four keys to specify the behavior in the case the the macro to
% be upgraded already has an optional argument. For concreteness, we introduce them using
% the |\section| macro from standard {\LaTeX} as an example. |\section| has an optional
% argument for the ``short title'', which will appear in the table of contents. The
% \DescribeMacro{optarg}|optarg| key can be used to specify a key for the existing
% optional argument. Thus, after upgrading it via
% |\rdfmeta@upgrade*[optarg=short]{section}|, we can use the updated form
% |\section[short=|\meta{toctitle}|]{|\meta{title}|}| instead of the old
% |\section[|\meta{toctitle}|]{|\meta{title}|}|. Actually, this still has a problem: the
% |\section*| would also be given the |short| key and would be passed an optional argument
% (which it does not accept). To remedy this we can set the
% \DescribeMacro{optargstar}|optargstar| key to |no|. In summary, the correct upgrade
% command for |\section| and |\section*| would be
% \begin{verbatim}
% \rdfmeta@upgrade*[optarg=short,optargstar=no]{section}
% \end{verbatim}
% The |\rdfmeta@upgrade*| macro also initializes a metadata key-group (a named set of
% keys and their handlers; see~\cite{Kohlhase:metakeys:svn} for details) for the section
% macro with an |id| key for identification (see~\cite{Kohlhase:sref*} for
% details). Often, the name of the key-group is the same as the command sequence, so we
% take this as the default, if we want to specify a different metadata key-group name, we
% can do so with the \DescribeMacro{keygroup}|keygroup| key in |\rdfmeta@upgrade*|.
% 
% If \DescribeMacro{idlabel}|idlabel| is set to \meta{prefix}, then the {\LaTeX} label is
% set to the value \meta{prefix}|.|\meta{id}, where \meta{id} is the value given in the
% RDFa |id| key. This allows to use the normal {\LaTeX} referencing mechanism in addition
% to the semantic referencing mechanism provided by the |sref|
% package~\ctancite{Kohlhase:sref}.
% 
% \subsection[id=limitations]{Limitations}
% 
% In this section we document known limitations. If you want to help alleviate them,
% please feel free to contact the package author. Some of them are currently discussed in
% the \sTeX GitHub repository~\cite{sTeX:github:on}. 
%
% \begin{enumerate}
% \item Currently the coverage of the redefinitions of standard commands in the
%   \url{rdfmeta} package is minimal; we will extend this in the future.
% \item The |\rdfmeta@upgrade| macro only works with single arguments, this should be easy
%   to fix with |\case| for the argument string. 
% \item I am not sure |\rdfmeta@upgrade| works with environments.
% \item it would be convenient, if we had a macro |\keydefs|, which takes a list of
%   keygroups, so that we can define keys in multiple groups in one go,
%   e.g. |\keydefs{omtext,omgroup}{hasState}| in Figure~\ref{fig:certification}. But the
%   obvious ``solution''
% \begin{verbatim}
% \newcommand\keydefs[2]{\@for\@I:=#1\do{\keydef{#1}{#2}}}
% \end{verbatim}
% does not work for me. 
% \end{enumerate}
% 
% \StopEventually{\printbibliography}
% 
% \section[id=impl]{The Implementation}
% 
% \subsection{Package Options}\label{sec:impl:options}
%
% We declare some switches which will modify the behavior according to the package
% options. Generally, an option |xxx| will just set the appropriate switches to true
% (otherwise they stay false).\ednote{need an implementation for {\latexml}}
%
%    \begin{macrocode}
%<*package>
\newif\if@rdfmeta@sectioning\@rdfmeta@sectioningfalse
\DeclareOption{sectioning}{\@rdfmeta@sectioningtrue}
\DeclareOption*{\PassOptionsToPackage{\CurrentOption}{sref}
                           \PassOptionsToPackage{\CurrentOption}{modules}}
\ProcessOptions
%    \end{macrocode}
%
% The first measure is to ensure that the right packages are loaded. From the from {\stex}
% collection, we need the |sref| package (see~\ctancite{Kohlhase:sref}) for handling keys,
% the |modules| package for exporting the |\keydef| (see~\ctancite{KohAmb:smmssl}).
%
%    \begin{macrocode}
\RequirePackage{sref}
\RequirePackage{modules}
%    \end{macrocode}
% 
% and we define a macro \DescribeMacro{\rdfmeta@loaded}|rdfmeta@loaded| just for the
% purpose of determining whether the |rdfmeta| package is loaded. 
% 
% \begin{macro}{\rdfmeta@loaded}
%    \begin{macrocode}
\newcommand\rdfmeta@loaded{yes}
%    \end{macrocode}
% \end{macro}
%
% And another macro \DescribeMacro{\rdfmeta@sectioning}|rdfmeta@sectioning| to determine
% wether the sectioning macros have been redefined.
%
% \begin{macro}{\rdfmeta@loaded}
%    \begin{macrocode}
\if@rdfmeta@sectioning\newcommand\rdfmeta@sectioning{yes}\fi
%    \end{macrocode}
% \end{macro}
% 
% \subsection[id=impl:keydef]{Key Definitions}
%
% \begin{macro}{\keydef}
%   The |\keydef| macro is rather simple, we just add a key to the respective environment
%   and extend the export token register for the current module by an |\addmetakey|
%   instruction. 
%    \begin{macrocode}
\newcommand\keydef[2]{\addmetakey{#1}{#2}%
\expandafter\g@addto@macro\this@module{\addmetakey{#1}{#2}}}
%    \end{macrocode}
% \end{macro}
%
% \begin{macro}{\listkeydef}
%   The |\listkeydef| macro is analogous, but uses |\addmetalistkey| instead.
%   instruction. 
%    \begin{macrocode}
\newcommand\listkeydef[2]{\addmetalistkey{#1}{#2}%
\expandafter\g@addto@macro\this@module{\addmetalistkey{#1}{#2}}}
%    \end{macrocode}
% \end{macro}
%
% \subsection[id=impl:upgrade]{RDFa upgrade Facilities}
% 
% We first define the keys for the |\rdfmeta@upgrade| macro. 
%    \begin{macrocode}
\def\@yes@{yes}
\addmetakey*{upgrade}{idlabel}
\addmetakey*{upgrade}{optarg}
\addmetakey*[yes]{upgrade}{optargstar}
\addmetakey*{upgrade}{keygroup}
%    \end{macrocode}
%
% \begin{macro}{\rdfmeta@upgrade}
%   This upgrade macro gives extended functionality according to the optional keys. The
%   top-level invocation just differentiates on whether a star is following:
%    \begin{macrocode}
\def\rdfmeta@upgrade{\@ifstar\rdfmeta@upgrade@star\rdfmeta@upgrade@nostar}
%    \end{macrocode}
% Both cases are almost the same, they only differ in the third line where they call
% |\rdfmeta@upgrade@base| or |\rdfmeta@upgrade@base@star| defined above. In particular,
% both take the arguments originally intended for |\rdfmeta@upgrade|.
%    \begin{macrocode}
\newcommand\rdfmeta@upgrade@nostar[2][]{\metasetkeys{upgrade}{#1}%
\ifx\upgrade@keygroup\@empty\def\@@group{#2}\else\def\@@group{\upgrade@keygroup}\fi
\rdfmeta@upgrade@base{#2}{\@nameuse{\@@group @\upgrade@optarg}}}
%    \end{macrocode}
% They set the metakeys from the second argument, then set |\@@group| to be the intended
% group (if the |keygroup| key was specified, it takes precedence over the default
% |#2|).
%    \begin{macrocode}
\newcommand\rdfmeta@upgrade@star[2][]{\metasetkeys{upgrade}{#1}%
\ifx\upgrade@keygroup\@empty\def\@@group{#2}\else\def\@@group{\upgrade@keygroup}\fi
\rdfmeta@upgrade@base@star{#2}{\@nameuse{\@@group @\upgrade@optarg}}}
%    \end{macrocode}
% \end{macro}
%
% \begin{macro}{\rdfmeta@upgrade@base}
%   This auxiliary macro and is invoked as
%   |\rdfmeta@upgrade@base{|\meta{cseq}|}{|\meta{optarg}|}|, where \meta{cseq} is a
%   command sequence name. It checks if |\|\meta{cseq} is defined (if not it does
%   nothing), saves the old behavior of |\|\meta{cseq} as |\rdfmeta@|\meta{cseq}|@old|,
%   and then redefines |\|\meta{cseq} to take a keyval argument and passes \meta{optarg}
%   as the optional argument.
%    \begin{macrocode}
\newcommand\rdfmeta@upgrade@base[2]{\@ifundefined{#1}{}%
{\message{redefining macro #1,}
\ifx\upgrade@idlabel\@empty\srefaddidkey{#1}\else\srefaddidkey[prefix=\upgrade@idlabel]{#1}\fi%
\expandafter\let\csname rdfmeta@#1@old\expandafter\endcsname\csname #1\endcsname%
\expandafter\renewcommand\csname #1\endcsname[2][]%
{\metasetkeys{#1}{##1}\@nameuse{rdfmeta@#1@old}[#2]{##2}}
\addmetakey*\@@group{\upgrade@optarg}}}
%    \end{macrocode}
% \end{macro}
% 
% \begin{macro}{\rdfmeta@upgrade@base@star}
%   This is a variant of |\rdfmeta@upgrade@base|, which also takes care of the starred
%   variants of a macro.
%    \begin{macrocode}
\newcommand\rdfmeta@upgrade@base@star[2]{\@ifundefined{#1}{}%
{\message{redefining macros #1 and #1*,}
\ifx\upgrade@idlabel\@empty\srefaddidkey{#1}\else\srefaddidkey[prefix=\upgrade@idlabel]{#1}\fi%
\expandafter\let\csname rdfmeta@#1@old\expandafter\endcsname\csname #1\endcsname%
%    \end{macrocode}
% In this case, we cannot just use |\newcommand| for dealing with the optional argument
% because the star is between the command sequence and the arguments. So we make a case
% distinction on the presence of the star. 
% |\rdfmeta@|\meta{cseq}|@old|.
%    \begin{macrocode}
\expandafter\renewcommand\csname #1\endcsname%
{\@ifstar{\@nameuse{rdfmeta@#1@star}}{\@nameuse{rdfmeta@#1@nostar}}}%
%    \end{macrocode}
% the macros |\rdfmeta@|\meta{cseq}|@star| and |\rdfmeta@|\meta{cseq}|@nostar| that are
% defined in terms of |\rdfmeta@|\meta{cseq}|@old| handle the necessary cases. The second
% one is simple:
%    \begin{macrocode}
\expandafter\newcommand\csname rdfmeta@#1@nostar\endcsname[2][]%
{\metasetkeys{#1}{##1}\edef\@test{#2}%
\ifx\@test\@empty\@nameuse{rdfmeta@#1@old}{##2}%
\else\@nameuse{rdfmeta@#1@old}[#2]{##2}\fi}%
%    \end{macrocode}
% For |\rdfmeta@|\meta{cseq}|@star| we have to take care of the optional argument of the
% old macro: if the |optargstar| key was set, then we pass the second argument of
% |\rdfmeta@upgrade@base| as an optional argument to it as above. 
%    \begin{macrocode}
\ifx\upgrade@optargstar\@yes@%
\expandafter\newcommand\csname rdfmeta@#1@star\endcsname[2][]%
{\metasetkeys{#1}{##1}\@nameuse{rdfmeta@#1@old}*[#2]{##2}}%
\else%
\expandafter\newcommand\csname rdfmeta@#1@star\endcsname[2][]%
{\metasetkeys{#1}{##1}\@nameuse{rdfmeta@#1@old}*{##2}}%
\fi%
\addmetakey*\@@group{\upgrade@optarg}}}
%    \end{macrocode}
% \end{macro}
%
% \subsection[id=impl:redef]{Redefinitions}
%
% If the |sectioning| macro is set, we redefine the respective commands
%
%    \begin{macrocode}
\if@rdfmeta@sectioning
\message{redefining sectioning commands!}
\rdfmeta@upgrade*[optarg=short,optargstar=no]{part}
\rdfmeta@upgrade*[optarg=short,optargstar=no]{chapter}
\rdfmeta@upgrade*[optarg=short,optargstar=no]{section}
\rdfmeta@upgrade*[optarg=short,optargstar=no]{subsection}
\rdfmeta@upgrade*[optarg=short,optargstar=no]{subsubsection}
\rdfmeta@upgrade*[optarg=short,optargstar=no]{paragraph}
\fi
%</package>
%    \end{macrocode}
% \Finale
\endinput
% \iffalse
% LocalWords:  GPL structuresharing STR LaTeX dcm dtx keyval sref CPERL url qw
%%% Local Variables: 
%%% mode: doctex
%%% TeX-master: t
%%% End: 
% \fi
% LocalWords:  RequirePackage birthdate personaltitle academictitle workaddress
% LocalWords:  privaddress worktel privtel workfax privfax worktelfax noDelim
% LocalWords:  privtelfax getKeyValue valuelist ToString getValue affill STDERR
% LocalWords:  ExportMetadata AssignValue DCMperson DefConstructor afterDigest
% LocalWords:  getArg setValue FishOutMetadata keyvals foreach idlist tabline
% LocalWords:  LookupValue insertElement atabline bitabline shorttitle nc args
% LocalWords:  sharealike noderivativeworks DefMacro authorblock subsubsection
% LocalWords:  contribs OptionalKeyVals omgroup omdoc srcref xml RawTeX
% LocalWords:  openElement iffalse kohlhase Thu rdfmeta latexml fileversion omd
% LocalWords:  maketitle setcounter tocdepth tableofcontents newpage section
% LocalWords:  stex exfig vspace hrule lstinputlisting morekeywords aboveskip
% LocalWords:  hasstate.omdoc statedocrd tuev certifiation keydef ednote texttt
% LocalWords:  keysets rdfameta cseq idlabel ctancite ifundefined impl ltxml
% LocalWords:  printbibliography newcommand srefaddidkey expandafter csname ifx
% LocalWords:  expandafter endcsname csname endcsname renewcommand  showmeta
% LocalWords:  symdef redef doctex showmeta metakeys keyset textbackslash cert
% LocalWords:  MathOntoAuthDoc09 WD-rdfa-core-20100803 hasstate omtext omtext
% LocalWords:  addkey optarg optarg toctitle optargstar optargstar keygroup
% LocalWords:  oldpart textsf langle textsf langle newif sectioningfalse ifstar
% LocalWords:  sectioningtrue metasetkeys nameuse addmetakey nostar
