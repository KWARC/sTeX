% \iffalse meta-comment
% A LaTeX Class and Package for OMDoc Document Structures
% Copyright (c) 2016 Michael Kohlhase, all rights reserved
%               this file is released under the
%               LaTeX Project Public License (LPPL)
%
% The original of this file is in the public repository at 
% http://github.com/KWARC/sTeX/
% \fi
%   
% \iffalse
%<package|cls>\NeedsTeXFormat{LaTeX2e}[1999/12/01]
%<package>\ProvidesPackage{omdoc}[2018/12/03 v1.3 OMDoc document Structure]
%<cls>\ProvidesClass{omdoc}[2018/12/03 v1.3 OMDoc Documents]
%
%<*driver>
\documentclass{ltxdoc}
\usepackage[utf8]{inputenc}
\usepackage{url,array,float}
\usepackage[show]{ed}
\usepackage[hyperref=auto,style=alphabetic]{biblatex}
\addbibresource{kwarcpubs.bib}
\addbibresource{extpubs.bib}
\addbibresource{kwarccrossrefs.bib}
\addbibresource{extcrossrefs.bib}
\usepackage{stex-logo}
\usepackage{omdoc}
\usepackage{omtext}
\usepackage{ctangit}
\usepackage{hyperref}
\usepackage{paralist}
\makeindex
\floatstyle{boxed}
\newfloat{exfig}{thp}{lop}
\floatname{exfig}{Example}
\begin{document}
\RecordChanges
\DocInput{omdoc.dtx}
\end{document}
%</driver>
% \fi
% 
%\iffalse\CheckSum{385}\fi
%
% \changes{v0.1}{2006/1/17}{First Version}
% \changes{v0.2}{2006/7/11}{added OMDoc class}
% \changes{v0.3}{2007/09/09}{moved omtext and friends here from the statements package}
% \changes{v0.4}{2008/09/06}{added quotes}
% \changes{v0.5}{2008/03/28}{more package/class options}
% \changes{v0.7}{2009/11/24}{giving keyval arguments to the document environment}
% \changes{v1.0}{2010/05/25}{separated out \texttt{omtext.dtx}}
% \changes{v1.1}{2012/02/14}{integrated \texttt{etoolbox} package}
% \changes{v1.2}{2014/01/12}{front/backmatter}
% \changes{v1.3}{2015/11/18}{removing keyval arg from document in favor of
% \texttt{\textbackslash documentkeys} macro}
% \changes{v1.3}{2018/09/10}{global variables and switches}
% \changes{v1.3}{2018/12/03}{changed to keyval class/package options, allowed arbitrary classes}
%
% 
% \MakeShortVerb{\|}
%
% \def\omdoc{OMDoc}
% \def\latexml{{\LaTeX}ML}
% \title{{\texttt{omdoc.sty/cls}}: Semantic Markup for Open Mathematical Documents in {\LaTeX}}
%    \author{Michael Kohlhase\\
%            FAU Erlangen-N\"urnberg\\
%            \url{http://kwarc.info/kohlhase}}
% \maketitle
%
% \begin{abstract}
%   The |omdoc| package is part of the {\sTeX} collection, a version of {\TeX/\LaTeX} that
%   allows to markup {\TeX/\LaTeX} documents semantically without leaving the document
%   format, essentially turning {\TeX/\LaTeX} into a document format for mathematical
%   knowledge management (MKM).
%
%   This package supplies an infrastructure for writing {\omdoc} documents in {\LaTeX}.
%   This includes a simple structure sharing mechanism for {\sTeX} that allows to to move
%   from a copy-and-paste document development model to a copy-and-reference model, which
%   conserves space and simplifies document management. The augmented structure can be
%   used by MKM systems for added-value services, either directly from the {\sTeX}
%   sources, or after translation.
% \end{abstract}
%
%\newpage\tableofcontents\newpage
% 
%\begin{omgroup}[id=sec:STR]{Introduction}
% 
%  {\sTeX} is a version of {\TeX/\LaTeX} that allows to markup {\TeX/\LaTeX} documents
%  semantically without leaving the document format, essentially turning {\TeX/\LaTeX}
%  into a document format for mathematical knowledge management (MKM). The package
%  supports direct translation to the {\omdoc} format~\cite{Kohlhase:OMDoc1.2}
%
%  The |omdoc| package supplies macros and environment that allow to label document
%  fragments and to reference them later in the same document or in other documents. In
%  essence, this enhances the document-as-trees model to
%  documents-as-directed-acyclic-graphs (DAG) model. This structure can be used by MKM
%  systems for added-value services, either directly from the {\sTeX} sources, or after
%  translation. Currently, trans-document referencing provided by this package can only
%  be used in the {\sTeX} collection.
%
%  DAG models of documents allow to replace the ``Copy and Paste'' in the source document
%  with a label-and-reference model where document are shared in the document source and
%  the formatter does the copying during document
%  formatting/presentation.\ednote{integrate with latexml's XMRef in the Math mode.}
% \end{omgroup}
% 
% \begin{omgroup}[id=sec:user]{The User Interface}
% 
%   The |omdoc| package generates two files: |omdoc.cls|, and |omdoc.sty|. The {\omdoc}
%   class is a minimally changed variant of the standard |article| class that includes the
%   functionality provided by |omdoc.sty|. The rest of the documentation pertains to the
%   functionality introduced by |omdoc.sty|.
%
% \begin{omgroup}[id=sec:user:options]{Package and Class Options}
% 
%   The |omdoc| class accept the following options: 
%   \begin{center}
%     \begin{tabular}{|l|p{10cm}|}\hline
%         \texttt{class=\meta{name}} & load \meta{name}|.cls| instead of |article.cls|\\\hline 
%         \texttt{topsect=\meta{sect}} & The top-level sectioning level; the default for
%         \meta{sect} is \texttt{section}\\\hline 
%         \texttt{showignores} & show the the contents of the |ignore| environment after all \\\hline
%         \texttt{showmeta} & show the metadata; see |metakeys.sty|\\\hline
%         \texttt{showmods} & show modules; see |modules.sty|\\\hline
%         \texttt{extrefs} & allow external references; see |sref.sty|\\\hline
%         \texttt{defindex} & index definienda; see |statements.sty|\\\hline
%         \texttt{mh} & MathHub support; see~\cite{Kohlhase:mss:ctan}\\\hline
%     \end{tabular}
%   \end{center}
%   The |omdoc| package accepts the same except the first two.
% \end{omgroup}
% 
% \begin{omgroup}[id=sec:user:struct]{Document Structure}
% 
%   The top-level \DescribeEnv{document}|document| environment can be given key/value
%   information by the \DescribeMacro{\documentkeys}|\documentkeys| macro in the
%   preamble\footnote{We cannot patch the document environment to accept an optional
%   argument, since other packages we load already do; pity.}. This can be used to give
%   metadata about the document. For the moment only the \DescribeMacro{id}|id| key is
%   used to give an identifier to the \texttt{omdoc} element resulting from the {\latexml}
%   transformation.
% 
%   The structure of the document is given by the \DescribeEnv{omgroup}|omgroup|
%   environment just like in {\omdoc}. In the {\LaTeX} route, the |omgroup| environment is
%   flexibly mapped to sectioning commands, inducing the proper sectioning level from the
%   nesting of |omgroup| environments. Correspondingly, the |omgroup| environment takes an
%   optional key/value argument for metadata followed by a regular argument for the
%   (section) title of the omgroup. The optional metadata argument has the keys
%   \DescribeMacro{id}|id| for an identifier, \DescribeMacro{creators}|creators| and
%   \DescribeMacro{contributors}|contributors| for the Dublin Core
%   metadata~\cite{DCMI:dmt03}; see~\ctancite{Kohlhase:dcm} for details of the format. The
%   \DescribeMacro{short}|short| allows to give a short title for the generated
%   section. If the title contains semantic macros, they need to be protected by
%   |\protect|, and we need to give the \DescribeMacro{loadmodules}|loadmodules| key it
%   needs no value. For instance we would have
%   \begin{verbatim}
%   \begin{module}{foo}
%   \symdef{bar}{B^a_r}
%    ...
%   \begin{omgroup}[id=barderiv,loadmodules]
%     {Introducing $\protect\bar$ Derivations}
%   \end{verbatim}
% 
%   \sTeX automatically computes the sectioning level, from the nesting of |omgroup|
%   environments. But sometimes, we want to skip levels (e.g. to use a subsection* as an
%   introduction for a chapter). Therefore the |omdoc| package provides a variant
%   \DescribeEnv{blindomgroup}|blindomgroup| that does not produce markup, but increments
%   the sectioning level and logically groups document parts that belong together, but
%   where traditional document markup relies on convention rather than explicit
%   markup. The |blindomgroup| environment is useful e.g. for creating frontmatter at the
%   correct level. Example~\ref{fig:docstruct} shows a typical setup for the outer
%   document structure of a book with parts and chapters. We use two levels of
%   |blindomgroup|:
%   \begin{itemize}
%   \item The outer one groups the introductory parts of the book (which we assume to have
%     a sectioning hierarchy topping at the part level). This |blindomgroup| makes sure
%     that the introductory remarks become a ``chapter'' instead of a ``part''.
%   \item Th inner one groups the frontmatter\footnote{We shied away from redefining the
%     |frontmatter| to induce a blindomgroup, but this may be the ``right'' way to go in
%     the future.} and makes the preface of the book a section-level construct. Note that
%     here the |display=flow| on the |omgroup| environment prevents numbering as is
%     traditional for prefaces. 
%   \end{itemize}
%   \begin{exfig}
% \begin{verbatim}
% \begin{document}
% \begin{blindomgroup}
% \begin{blindomgroup}
% \begin{frontmatter}
% \maketitle\newpage
% \begin{omgroup}[display=flow]{Preface}
% ... <<preface>> ...
% \end{omgroup}
% \clearpage\setcounter{tocdepth}{4}\tableofcontents\clearpage
% \end{frontmatter}
% \end{blindomgroup}
% ... <<introductory remarks>> ...
% \end{blindomgroup}
% \begin{omgroup}{Introduction}
% ... <<intro>> ...
% \end{omgroup}
% ... <<more chapters>> ... 
% \bibliographystyle{alpha}\bibliography{kwarc}
% \end{document}
% \end{verbatim}\vspace*{-2em}
%   \caption{A typical Document Structure of a Book}\label{fig:docstruct}
% \end{exfig}
%
%
%   The \DescribeMacro{\currentsectionlevel}|\currentsectionlevel| macro supplies the name
%   of the current sectioning level, e.g. ``chapter'', or
%   ``subsection''. \DescribeMacro{\CurrentSectionLevel}|\CurrentSectionLevel| is the
%   capitalized variant. They are useful to write something like ``In this
%   |\currentsectionlevel|, we will\ldots'' in an |omgroup| environment, where we do not
%   know which sectioning level we will end up.
% \end{omgroup}
% 
% \begin{omgroup}[id=sec:user:ignore]{Ignoring Inputs}
% 
% The \DescribeEnv{ignore}|ignore| environment can be used for hiding text parts from the
% document structure. The body of the environment is not PDF or DVI output unless the
% \DescribeMacro{showignores}|showignores| option is given to the |omdoc| class or
% |package|. But in the generated {\omdoc} result, the body is marked up with a |ignore|
% element. This is useful in two situations. For
% \begin{description}
% \item[editing] One may want to hide unfinished or obsolete parts of a document
% \item[narrative/content markup] In {\stex} we mark up narrative-structured documents. In
%   the generated {\omdoc} documents we want to be able to cache content objects that are
%   not directly visible. For instance in the |statements|
%   package~\ctancite{Kohlhase:smms} we use the |\inlinedef| macro to mark up phrase-level
%   definitions, which verbalize more formal definitions. The latter can be hidden by an
%   ignore and referenced by the |verbalizes| key in |\inlinedef|.
% \end{description}
% 
% \end{omgroup}
%
% \begin{omgroup}[id=sec:user:sharing]{Structure Sharing}
%
%   The \DescribeMacro{\STRlabel}|\STRlabel| macro takes two arguments: a label and the
%   content and stores the the content for later use by
%   \DescribeMacro{\STRcopy}|\STRcopy[|\meta{URL}|]{|\meta{label}|}|, which expands to the
%   previously stored content. If the |\STRlabel| macro was in a different file, then we
%   can give a URL \meta{URL} that lets {\latexml} generate the correct reference.
%
%   \DescribeMacro{\STRsemantics} The |\STRlabel| macro has a variant |\STRsemantics|,
%   where the label argument is optional, and which takes a third argument, which is
%   ignored in {\LaTeX}. This allows to specify the meaning of the content (whatever that
%   may mean) in cases, where the source document is not formatted for presentation, but
%   is transformed into some content markup format.\ednote{document LMID und LMXREf here
%   if we decide to keep them.}
% \end{omgroup}
% 
% \begin{omgroup}[id=sec:user:gvars]{Global Variables}
% 
%   Text fragments and modules can be made more re-usable by the use of global
%   variables. For instance, the admin section of a course can be made course-independent
%   (and therefore re-usable) by using variables (actually token registers)
%   |courseAcronym| and |courseTitle| instead of the text itself. The variables can then
%   be set in the \sTeX preamble of the course notes file.
%   \DescribeMacro{\setSGvar}|\setSGvar{|\meta{vname}|}{|\meta{text}|}| to set the global
%   variable \meta{vname} to \meta{text} and
%   \DescribeMacro{\useSGvar}|\useSGvar{|\meta{vname}|}| to reference it.
%
%   With \DescribeMacro{\ifSGvar}|\ifSGvar| we can test for the contents of a global
%   variable: the macro call |\ifSGvar{|\meta{vname}|}{|\meta{val}|}{|\meta{ctext}|}|
%   tests the content of the global variable \meta{vname}, only if (after expansion) it is
%   equal to \meta{val}, the conditional text \meta{ctext} is formatted.
% \end{omgroup}
%
% \begin{omgroup}[id=sec:user:colors]{Colors}
% 
%   For convenience, the |omdoc| package defines a couple of color macros for the |color|
%   package: For instance \DescribeMacro{\blue}|\blue| abbreviates |\textcolor{blue}|, so
%   that |\blue{|\meta{something}|}| writes \meta{something} in blue. The macros
%   \DescribeMacro{\red}|\red| \DescribeMacro{...}|\green|, |\cyan|, |\magenta|, |\brown|,
%   |\yellow|, |\orange|, |\gray|, and finally \DescribeMacro{\black}|\black| are
%   analogous.
% \end{omgroup}
% \end{omgroup}
%
% \begin{omgroup}[id=sec:limitations]{Limitations}
% 
% In this section we document known limitations. If you want to help alleviate them,
% please feel free to contact the package author. Some of them are currently discussed in
% the \sTeX GitHub repository~\cite{sTeX:github:on}. 
% \begin{enumerate}
% \item when option |book| which uses |\pagestyle{headings}| is given and semantic macros
%   are given in the |omgroup| titles, then they sometimes are not defined by the time the
%   heading is formatted. Need to look into how the headings are made. 
% \end{enumerate}
% \end{omgroup}
% 
% \StopEventually{\newpage\PrintIndex\newpage\PrintChanges\newpage\printbibliography}\newpage
%
% \begin{omgroup}[id=sec:impl:cls]{Implementation: The OMDoc Class}
%
%   The functionality is spread over the |omdoc| class and package. The class provides the
%   |document| environment and the |omdoc| element corresponds to it, whereas the
%   package provides the concrete functionality.
% 
% \begin{omgroup}[id=sec:impl:cls:options]{Class Options}
%   To initialize the |omdoc| class, we declare and process the necessary options using
%   the |kvoptions| package for  key/value options handling. For
%   |omdoc.cls| this is quite simple. We have options |report| and |book|, which set the
%   \DescribeMacro{\omdoc@cls@class}|\omdoc@cls@class| macro and pass on the macro to |omdoc.sty|
%   for further processing. 
% 
%    \begin{macrocode}
%<*cls>
\RequirePackage{etoolbox}
\RequirePackage{kvoptions}
\SetupKeyvalOptions{family=omdoc@cls,prefix=omdoc@cls@}
\DeclareStringOption[article]{class}
\AddToKeyvalOption*{class}{\PassOptionsToPackage{class=\omdoc@cls@class}{omdoc}}
%    \end{macrocode}
%
% the following options are deprecated.
%
%    \begin{macrocode}
\DeclareVoidOption{report}{\def\omdoc@cls@class{report}%
\ClassWarning{omdoc}{the option 'report' is deprecated, use 'class=report', instead}}
\DeclareVoidOption{book}{\def\omdoc@cls@class{book}%
\ClassWarning{omdoc}{the option 'part' is deprecated, use 'class=book', instead}}
\DeclareVoidOption{bookpart}{\def\omdoc@cls@class{book}%
\PassOptionsToPackage{topsect=chapter}{omdoc}%
\ClassWarning{omdoc}{the option 'bookpart' is deprecated, use 'class=book,topsect=chapter', instead}}
%    \end{macrocode}
% the rest of the options are only passed on to |omdoc.sty| and the class selected by the
% first options.  We need to load the |etoolbox| package early for |\@xappto|. 
%    \begin{macrocode}
\def\@omdoc@cls@docopt{}
\DeclareDefaultOption{%
\ifx\@omdoc@cls@docopt\@empty%
\xdef\@omdoc@cls@docopt{\CurrentOption}%
\else\xappto\@omdoc@cls@docopt{,\CurrentOption}%
\fi}%
\PassOptionsToPackage{\CurrentOption}{omdoc}
\PassOptionsToPackage{\CurrentOption}{stex}
\ProcessKeyvalOptions{omdoc@cls}
%    \end{macrocode}
%
% We load |article.cls|, and the desired packages. For the {\latexml} bindings, we make
% sure the right packages are loaded.
%
%    \begin{macrocode}
\LoadClass[\@omdoc@cls@docopt]{\omdoc@cls@class}
\RequirePackage{omdoc}
\RequirePackage{stex}
%    \end{macrocode}
% \end{omgroup}
%
% \begin{omgroup}[id=sec:impl:cls:document]{Beefing up the \texttt{document} environment}
%
% Now, we will define the environments we need.  The top-level one is the |document|
% environment, which we redefined so that we can provide keyval arguments.
%
% \begin{environment}{document}
%   For the moment we do not use them on the {\LaTeX} level, but the document identifier
%   is picked up by {\latexml}.\ednote{faking documentkeys for now. @HANG, please implement}
%    \begin{macrocode}
\srefaddidkey{document}
\newcommand\documentkeys[1]{\metasetkeys{document}{#1}}
\let\orig@document=\document
\renewcommand{\document}[1][]{\metasetkeys{document}{#1}\orig@document}
%</cls>
%    \end{macrocode}
% \end{environment}
% \end{omgroup}
% \end{omgroup}
% 
% \begin{omgroup}[id=sec:impl:sty]{Implementation: OMDoc Package}
%
% \begin{omgroup}[id=sec:impl:options]{Package Options}
%
%   We declare some switches which will modify the behavior according to the package
%   options. Generally, an option |xxx| will just set the appropriate switches to true
%   (otherwise they stay false).
%
%    \begin{macrocode}
%<*package>
\RequirePackage{kvoptions}
\SetupKeyvalOptions{family=omdoc@sty,prefix=omdoc@sty@}
\DeclareBoolOption{mh}
\DeclareStringOption[article]{class}
\DeclareBoolOption{showignores}
\DeclareStringOption[section]{topsect}
\newcount\section@level 
\DeclareDefaultOption{\PassOptionsToPackage{\CurrentOption}{sref}}
\ProcessKeyvalOptions{omdoc@sty}
%    \end{macrocode}
%
% Then we need to set up the packages by requiring the |sref| package to be loaded.
%
%    \begin{macrocode}
\ifomdoc@sty@mh\RequirePackage{omdoc-mh}\fi
\RequirePackage{sref}
\RequirePackage{xspace}
\RequirePackage{comment}
\RequirePackage{pathsuris}
%    \end{macrocode}
%
% Finally, we set the \DescribeMacro{\section@level}|\section@level| macro that governs
% sectioning. The default is two (corresponding to the |article| class), then we set the
% defaults for the standard classes |book| and |report| and then we take care of the
% levels passed in via the |topsect| option.
%
%    \begin{macrocode}
\section@level=2
\ifdefstring{\omdoc@sty@class}{book}{\section@level=0}{}
\ifdefstring{\omdoc@sty@class}{report}{\section@level=0}{}
\ifdefstring{\omdoc@sty@topsect}{part}{\section@level=0}{}
\ifdefstring{\omdoc@sty@topsect}{chapter}{\section@level=1}{}
%    \end{macrocode}
%
% \end{omgroup}
% 
% \begin{omgroup}[id=sec:impl:struct]{Document Structure}
% 
%   The structure of the document is given by the |omgroup| environment just like in
%   OMDoc. The hierarchy is adjusted automatically according to the {\LaTeX} class in
%   effect. 
% \begin{macro}{\currentsectionlevel}
%   For the |\currentsectionlevel| and |\Currentsectionlevel| macros we use an internal
%   macro |\current@section@level| that only contains the keyword (no markup). We
%   initialize it with ``document'' as a default. In the generated OMDoc, we only generate
%   a text element of class |omdoc_currentsectionlevel|, wich will be instantiated by CSS
%   later.\ednote{MK: we may have to experiment with the more powerful uppercasing macro
%   from \texttt{mfirstuc.sty} once we internationalize.}
%    \begin{macrocode}
\def\current@section@level{document}%
\newcommand\currentsectionlevel{\lowercase\expandafter{\current@section@level}\xspace}%
\newcommand\Currentsectionlevel{\expandafter\MakeUppercase\current@section@level\xspace}%
%    \end{macrocode}
% \end{macro}
% 
% \begin{environment}{blindomgroup}
% \begin{macrocode}
\newcommand\at@begin@blindomgroup[1]{}
\newenvironment{blindomgroup}
{\advance\section@level by 1\at@begin@blindomgroup\setion@level}
{\advance\section@level by -1}
%    \end{macrocode}
% \end{environment}
%
% \begin{macro}{\omgroup@nonum}
%   convenience macro: |\omgroup@nonum{|\meta{level}|}{|\meta{title}|}| makes an unnumbered
%   sectioning with title \meta{title} at level \meta{level}.
%    \begin{macrocode}
\newcommand\omgroup@nonum[2]{%
\ifx\hyper@anchor\@undefined\else\phantomsection\fi%
\addcontentsline{toc}{#1}{#2}\@nameuse{#1}*{#2}}
%    \end{macrocode}
% \end{macro}
%
% \begin{macro}{\omgroup@num}
%   convenience macro: |\omgroup@nonum{|\meta{level}|}{|\meta{title}|}| makes numbered
%   sectioning with title \meta{title} at level \meta{level}. We have to check the |short|
%   key was given in the |omgroup| environment and -- if it is use it. But how to do that
%   depends on whether the |rdfmeta| package has been loaded. In the end we call
%   |\sref@label@id| to enable crossreferencing.
%    \begin{macrocode}
\newcommand\omgroup@num[2]{%
\edef\@@ID{\sref@id}
\ifx\omgroup@short\@empty% no short title
\@nameuse{#1}{#2}%
\else% we have a short title
\@ifundefined{rdfmeta@sectioning}%
  {\@nameuse{#1}[\omgroup@short]{#2}}%
  {\@nameuse{rdfmeta@#1@old}[\omgroup@short]{#2}}%
\fi%
\sref@label@id@arg{\omdoc@sect@name~\@nameuse{the#1}}\@@ID}
%    \end{macrocode}
% \end{macro}
%
% \begin{environment}{omgroup}
%    \begin{macrocode}
\def\@true{true}
\def\@false{false}
\srefaddidkey{omgroup}
\addmetakey{omgroup}{date}
\addmetakey{omgroup}{creators}
\addmetakey{omgroup}{contributors}
\addmetakey{omgroup}{srccite}
\addmetakey{omgroup}{type}
\addmetakey*{omgroup}{short}
\addmetakey*{omgroup}{display}
\addmetakey[false]{omgroup}{loadmodules}[true]
%    \end{macrocode}
% we define a switch for numbering lines and a hook for the beginning of groups: The
% \DescribeMacro{\at@begin@omgroup}|\at@begin@omgroup| macro allows customization. It is
% run at the beginning of the |omgroup|, i.e. after the section heading.
%    \begin{macrocode}
\newif\if@@num\@@numtrue
\newif\if@mainmatter\@mainmattertrue
\newcommand\at@begin@omgroup[3][]{}
%    \end{macrocode}
%
% Then we define a helper macro that takes care of the sectioning magic. It comes with its
% own key/value interface for customization.
%
%    \begin{macrocode}
\addmetakey{omdoc@sect}{name}
\addmetakey[false]{omdoc@sect}{clear}[true]
\addmetakey{omdoc@sect}{ref}
\addmetakey[false]{omdoc@sect}{num}[true]
\newcommand\omdoc@sectioning[3][]{\metasetkeys{omdoc@sect}{#1}%
\ifx\omdoc@sect@clear\@true\cleardoublepage\fi%
\if@@num% numbering not overridden by frontmatter, etc.
\ifx\omdoc@sect@num\@true\omgroup@num{#2}{#3}\else\omgroup@nonum{#2}{#3}\fi%
\def\current@section@level{\omdoc@sect@name}%
\else\omgroup@nonum{#2}{#3}%
\fi}% if@@num
%    \end{macrocode}
% and another one, if redefines the |\addtocontentsline| macro of {\LaTeX} to import the
% respective macros. It takes as an argument a list of module names.\ednote{MK: the
% extension sms is hard-coded here, but should not be. This will not work in multilingual
% settings.}
%    \begin{macrocode}
\newcommand\omgroup@redefine@addtocontents[1]{%
\edef\@@import{#1}%
\@for\@I:=\@@import\do{%
\edef\@path{\csname module@\@I  @path\endcsname}%
\@ifundefined{tf@toc}\relax%
     {\protected@write\tf@toc{}{\string\@requiremodules{\@path}{sms}}}}
\ifx\hyper@anchor\@undefined% hyperref.sty loaded?
\def\addcontentsline##1##2##3{%
\addtocontents{##1}{\protect\contentsline{##2}{\string\withusedmodules{#1}{##3}}{\thepage}}}
\else% hyperref.sty not loaded
\def\addcontentsline##1##2##3{%
\addtocontents{##1}{\protect\contentsline{##2}{\string\withusedmodules{#1}{##3}}{\thepage}{\@currentHref}}}%
\fi}% hypreref.sty loaded?
%    \end{macrocode}
% now the |omgroup| environment itself. This takes care of the table of contents via the
% helper macro above and then selects the appropriate sectioning command from
% |article.cls|.
%    \begin{macrocode}
\newenvironment{omgroup}[2][]% keys, title
{\metasetkeys{omgroup}{#1}\sref@target%
\ifx\omgroup@display\st@flow\@@numfalse\fi
\if@mainmatter\else\@@numfalse\fi
%    \end{macrocode}
% If the |loadmodules| key is set on |\begin{omgroup}|, we redefine the |\addcontetsline|
%   macro that determines how the sectioning commands below construct the entries for the
%   table of contents.
%    \begin{macrocode}
\ifx\omgroup@loadmodules\@true%
\omgroup@redefine@addtocontents{\@ifundefined{mod@id}\used@modules%
{\@ifundefined{module@\mod@id @path}{\used@modules}\mod@id}}\fi%
%    \end{macrocode}
% now we only need to construct the right sectioning depending on the value of
% |\section@level|.
%    \begin{macrocode}
\advance\section@level by 1\relax%
\ifcase\section@level%
\or\omdoc@sectioning[name=Part,clear,num]{part}{#2}%
\or\omdoc@sectioning[name=Chapter,clear,num]{chapter}{#2}%
\or\omdoc@sectioning[name=Section,num]{section}{#2}%
\or\omdoc@sectioning[name=Subsection,num]{subsection}{#2}%
\or\omdoc@sectioning[name=Subsubsection,num]{subsubsection}{#2}%
\or\omdoc@sectioning[name=Paragraph,ref=this paragraph]{paragraph}{#2}%
\or\omdoc@sectioning[name=Subparagraph,ref=this subparagraph]{paragraph}{#2}%
\fi% \ifcase
\at@begin@omgroup[#1]\section@level{#2}}% for customization
{\advance\section@level by -1}
%    \end{macrocode}
% \end{environment}
% 
% \end{omgroup}
%
% \begin{omgroup}[id=sec:user:docmatter]{Front and Backmatter}
% 
%   Index markup is provided by the |omtext| package~\cite{Kohlhase:smmtf:ctan}, so in the
%   |omdoc| package we only need to supply the corresponding |\printindex| command, if it
%   is not already defined
% \begin{macro}{\printindex}
%    \begin{macrocode}
\providecommand\printindex{\IfFileExists{\jobname.ind}{\input{\jobname.ind}}{}}
%</package>
%    \end{macrocode}
% \end{macro}
% 
% \begin{environment}{frontmatter}
%   some classes (e.g. |book.cls|) already have |\frontmatter|, |\mainmatter|, and
%   |\backmatter| macros. As we want to define |frontmatter| and |backmatter|
%   environments, we save their behavior (possibly defining it) in |orig@*matter| macros
%   and make them undefined (so that we can define the environments).
%    \begin{macrocode}
%<*cls>
\ifcsdef{frontmatter}% to redefine if necessary
  {\cslet{orig@frontmatter}{\frontmatter}\cslet{frontmatter}{\relax}}
  {\cslet{orig@frontmatter}{\clearpage\@mainmatterfalse\pagenumbering{roman}}}
\ifcsdef{backmatter}% to redefine if necessary
  {\cslet{orig@backmatter}{\backmatter}\cslet{backmatter}{\relax}}
  {\cslet{orig@backmatter}{\clearpage\@mainmatterfalse\pagenumbering{roman}}}
%    \end{macrocode}
% \end{environment}
% 
% \begin{environment}{frontmatter}
%   now we can define the environments
%    \begin{macrocode}
\newenvironment{frontmatter}
{\orig@frontmatter}
{\ifcsdef{mainmatter}{}{\clearpage\@mainmattertrue\pagenumbering{arabic}}}
%    \end{macrocode}
% \end{environment}
%
% \begin{environment}{backmatter}
%   some classes (e.g. |book.cls|) already have a |\backmatter| macro. We use that
%   (possibly defining it). 
%    \begin{macrocode}
\newenvironment{backmatter}
{\orig@backmatter}
{\ifcsdef{mainmatter}{}{\clearpage\@mainmattertrue\pagenumbering{arabic}}}
%</cls> 
%    \end{macrocode}
% \end{environment}
% \end{omgroup}
%
% \begin{omgroup}[id=sec:impl:ignore]{Ignoring Inputs}
% \begin{environment}{ignore}
%    \begin{macrocode}
%<*package>
\ifomdoc@sty@showignores
\addmetakey{ignore}{type}
\addmetakey{ignore}{comment}
\newenvironment{ignore}[1][]
{\metasetkeys{ignore}{#1}\textless\ignore@type\textgreater\bgroup\itshape}
{\egroup\textless/\ignore@type\textgreater}
\renewenvironment{ignore}{}{}\else\excludecomment{ignore}\fi
%    \end{macrocode}
% \end{environment}
% \end{omgroup}
% 
% \begin{omgroup}[id=sec:impl:share]{Structure Sharing}
%   \ednote{The following is simply copied over from the |latexml| package, which we
%   eliminated, we should integrate better.}
%    \begin{macrocode}
\providecommand{\lxDocumentID}[1]{}%
\def\LXMID#1#2{\expandafter\gdef\csname xmarg#1\endcsname{#2}\csname xmarg#1\endcsname}
\def\LXMRef#1{\csname xmarg#1\endcsname}
%    \end{macrocode}
%
% \begin{macro}{\STRlabel}
%    The main macro, it it used to attach a label to some text expansion. Later on, using the
%    |\STRcopy| macro, the author can use this label to get the expansion originally assigned.
%    \begin{macrocode}
\long\def\STRlabel#1#2{\STRlabeldef{#1}{#2}{#2}}
%    \end{macrocode}
% \end{macro}
% 
% \begin{macro}{\STRcopy}
%   The |\STRcopy| macro is used to call the expansion of a given label. In case the label
%   is not defined it will issue a warning.\ednote{MK: we need to do something about the
%   ref!}
%    \begin{macrocode}
\newcommand\STRcopy[2][]{\expandafter\ifx\csname STR@#2\endcsname\relax
\message{STR warning: reference #2 undefined!}
\else\csname STR@#2\endcsname\fi}
%    \end{macrocode}
% \end{macro}
%
% \begin{macro}{\STRsemantics}
%    if we have a presentation form and a semantic form, then we can use
%    \begin{macrocode}
\newcommand\STRsemantics[3][]{#2\def\@test{#1}\ifx\@test\@empty\STRlabeldef{#1}{#2}\fi}
%    \end{macrocode}
% \end{macro}
%
% \begin{macro}{\STRlabeldef}
%    This is the macro that does the actual labeling. Is it called inside |\STRlabel|
%    \begin{macrocode}
\def\STRlabeldef#1{\expandafter\gdef\csname STR@#1\endcsname}
%    \end{macrocode}
% \end{macro}
% \end{omgroup}
% 
% \begin{omgroup}[id=sec:impl:gvars]{Global Variables}
%
% \begin{macro}{\setSGvar}
%   set a global variable
%    \begin{macrocode}
\newcommand\setSGvar[1]{\@namedef{sTeX@Gvar@#1}}
%    \end{macrocode}
% \end{macro}
%
% \begin{macro}{\useSGvar}
%   use a global variable
%    \begin{macrocode}
\newcommand\useSGvar[1]{\@nameuse{sTeX@Gvar@#1}}
%    \end{macrocode}
% \end{macro}
%
% \begin{macro}{\ifSGvar}
%   set a global variable
%    \begin{macrocode}
\newcommand\ifSGvar[3]{\def\@test{#2}\expandafter\ifx\csname sTeX@Gvar@#1\endcsname\@test #3\fi}
%    \end{macrocode}
% \end{macro}
%
% \end{omgroup}
%
% \begin{omgroup}[id=sec:impl:colors]{Colors}
% 
% \begin{environment}{blue, red, green, magenta}
%    We will use the following abbreviations for colors from |color.sty|
%    \begin{macrocode}
\def\black#1{\textcolor{black}{#1}}
\def\gray#1{\textcolor{gray}{#1}}
\def\blue#1{\textcolor{blue}{#1}}
\def\red#1{\textcolor{red}{#1}}
\def\green#1{\textcolor{green}{#1}}
\def\cyan#1{\textcolor{cyan}{#1}}
\def\magenta#1{\textcolor{magenta}{#1}}
\def\brown#1{\textcolor{brown}{#1}}
\def\yellow#1{\textcolor{yellow}{#1}}
\def\orange#1{\textcolor{orange}{#1}}
%</package>
%    \end{macrocode}
% \end{environment}
% \end{omgroup}
% \end{omgroup}
% \Finale
\endinput
% \iffalse
%%% Local Variables: 
%%% mode: doctex
%%% TeX-master: t
%%% End: 
% \fi
% LocalWords:  GPL structuresharing STR omdoc dtx stex CPERL LoadClass url dc filedate om 
% LocalWords:  RequirePackage RegisterNamespace namespace xsl DocType ltxml dtd DAG hline
% LocalWords:  ltx DefEnvironment beforeDigest AssignValue inPreamble getGullet rangle
% LocalWords:  afterDigest keyval omgroup DefKeyVal Semiverbatim KeyVal srcf frontmatter
% LocalWords:  OptionalKeyVals DefParameterType IfBeginFollows skipSpaces CMP rangle xdef
% LocalWords:  ifNext DefMacro needwrapper unlist DefConstructor omtext bgroup showmods
% LocalWords:  useCMPItemizations RefStepItemCounter egroup beginItemize li di pathsuris
% LocalWords:  beforeDigestEnd dt autoclose ul ol dl env showignores srcref Cwd rdfmeta
% LocalWords:  afterOpen LastSeenCMP autoClose DefCMPEnvironment proto ToString st@flow
% LocalWords:  addAttribute nlex nlcex omdocColorMacro args tok MergeFont qw setion@level
% LocalWords:  TokenizeInternal toString isMath foreach maybeCloseElement id'd Backmatter
% LocalWords:  autoOpen minipage footnotesize scriptsize numberIt whatsit href endinput
% LocalWords:  getAttribute setAttribute OMDoc RelaxNGSchema noindex xml lec KeyVals
% LocalWords:  Subsubsection useDefaultItemizations refundefinedtrue sblockquote defindex
% LocalWords:  DefCMPConstructor sinlinequote idx idt ide idp emph  extrefs sref Tokenize
% LocalWords:  flushleft flushright DeclareOption PassOptions undef cls iffalse noauxreq
% LocalWords:  ProcessOptions subparagraph ignoresfalse ignorestrue texttt ttin behavior
% LocalWords:  texttt latexml fileversion maketitle newpage tableofcontents cwd srccite
% LocalWords:  newpage ednote ctancite dmt03 smms inlinedef STRlabel STRcopy loadmodules
% LocalWords:  STRlabel STRsemantics STRsemantics textcolor printbibliography loadmodules
% LocalWords:  textsf langle textsf langle respetively orig renewcommand cdir capitalized
% LocalWords:  baseuri baseuri baselocal baselocal SOURCEFILE cooluri newif ifx tf@toc
% LocalWords:  SOURCEBASE chapterfalse partfalse newcount ifshow chaptertrue initialize
% LocalWords:  parttrue srefaddidkey newenvironment textbf compactenum showmeta tf@toc
% LocalWords:  noindent noindent ignorespaces ifnum thepart thechapter regexp color.sty
% LocalWords:  thesection thesubsection thesubsubsection needswrapper itshape xmarg xmarg
% LocalWords:  textgreater renewenvironment excludecomment STRlabeldef csname Section,num
% LocalWords:  expandafter endcsname xref newcommand gdef doctex metakeys Hacky arabic
% LocalWords:  metasetkeys addmetakey printindex providecommand jobname.ind importmodules
% LocalWords:  jobname.ind tocdepth hateq ensuremath xspace hatequiv equiv NeedsTeXFormat
% LocalWords:  textleadsto leadsto etoolbox blindomgroup blindomgroup docstruct setSGvar
% LocalWords:  compactitem exfig vspace currentsectionlevel currentsectionlevel setSGvar
% LocalWords:  ldots URLBASE ifclass bookfalse booktrue currentsetionlevel thedocument@ID
% LocalWords:  nonum phantomsection nameuse numtrue numfalse contentsline unnum vname
% LocalWords:  thepage hypreref.sty ifcase cleardoublepage frontmatterfalse customization
% LocalWords:  frontmattertrue pagenumbering setcounter hyperref.sty addcontetsline ctext
%  LocalWords:  mfirstuc.sty internationalize documentkeys withusedmodules Part,clear,num

% \endinput
% Local Variables:
% mode: doctex
% TeX-master: t
% End:
%  LocalWords:  crossreferencing Chapter,clear,num Subsection,num Subsubsection,num cslet
%  LocalWords:  Paragraph,ref Subparagraph,ref useSGvar useSGvar ifSGvar ifSGvar topsect
%  LocalWords:  sTeX@Gvar kvoptions omdoc@cls,prefix book,topsect xappto omdoc@sty,prefix
%  LocalWords:  ifdefstring ifcsdef cslet
