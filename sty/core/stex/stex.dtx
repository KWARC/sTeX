% \iffalse meta-comment
% The sTeX packages all at once
% Copyright (c) 2006-2008 Michael Kohlhase, all rights reserved
%               this file is released under the
%               LaTeX Project Public License (LPPL)
% The original of this file is in the public repository at 
% http://github.com/KWARC/sTeX/
% \fi
% 
% \iffalse
%<package|logo>\NeedsTeXFormat{LaTeX2e}[1999/12/01]
%<package>\ProvidesPackage{stex}[2016/04/04 v1.0 Semantic Markup]
%<logo>\ProvidesPackage{stex-logo}[2016/04/04 v1.0 sTeX Logo]
%
%<*driver>
\documentclass{ltxdoc}
\usepackage{url,float,xspace,tikz,listings}
\usepackage[solutions,hints,notes]{problem}
\usepackage[show]{ed}
\usepackage[hyperref=auto,style=alphabetic]{biblatex}
\addbibresource{kwarcpubs.bib}
\addbibresource{extpubs.bib}
\addbibresource{kwarccrossrefs.bib}
\addbibresource{extcrossrefs.bib}
\usepackage{stex-logo,lstomdoc}
\usepackage{ctangit}
\usepackage{hyperref}
\makeindex
\def\latexml{\hbox{{\LaTeX}ML}\xspace}
\floatstyle{boxed}
\newfloat{exfig}{thp}{lop}
\floatname{exfig}{Example}
\def\tracissue#1{\cite{sTeX:online}, \hyperlink{http://trac.kwarc.info/sTeX/ticket/#1}{issue #1}}
\def\xml{XML\xspace}
\def\xslt{XSLT\xspace}
\def\mathml{MathML\xspace}
\def\omdoc{OMDoc\xspace}
\def\smglom{SMGloM\xspace}
\begin{document}\DocInput{stex.dtx}\end{document}
%</driver>
% \fi
% 
% \iffalse\CheckSum{36}\fi
%
% \changes{v1.0}{2015/11/19}{self-documenting package}
% 
% \GetFileInfo{stex.sty}
% 
% \MakeShortVerb{\|}
% \title{Semantic Markup  in {\TeX/\LaTeX}\thanks{Version {\fileversion} (last revised
% \filedate)}}
% \author{Michael Kohlhase\\
% Jacobs University, Bremen\\
% \url{http://kwarc.info/kohlhase}}
% \maketitle
%
% \begin{abstract}
%   We present a collection of {\TeX} macro packages that allow to markup {\TeX/\LaTeX}
%   documents semantically without leaving the document format, essentially turning
%   {\TeX/\LaTeX} into a document format for mathematical knowledge management (MKM).
% \end{abstract}
% \setcounter{tocdepth}{2}\tableofcontents\newpage
%
% \section{Introduction}
%
% The last few years have seen the emergence of various content-oriented {\xml}-based,
% content-oriented markup languages for mathematics on the web, e.g.
% OpenMath~\cite{BusCapCar:2oms04}, content MathML~\cite{CarIon:MathML03}, or our own
% {\omdoc}~\cite{Kohlhase:OMDoc1.2}. These representation languages for mathematics, that
% make the structure of the mathematical knowledge in a document explicit enough that
% machines can operate on it. Other examples of content-oriented formats for mathematics
% include the various logic-based languages found in automated reasoning tools
% (see~\cite{RobVor:hoar01} for an overview), program specification languages (see
% e.g.~\cite{Bergstra:as89}).
%
% The promise if these content-oriented approaches is that various tasks involved in ``doing
% mathematics'' (e.g. search, navigation, cross-referencing, quality control, user-adaptive
% presentation, proving, simulation) can be machine-supported, and thus the working
% mathematician is relieved to do what humans can still do infinitely better than machines:
% The creative part of mathematics --- inventing interesting mathematical objects,
% conjecturing about their properties and coming up with creative ideas for proving these
% conjectures. However, before these promises can be delivered upon (there is even a
% conference series~\cite{MKM-IG-Meetings:online} studying ``Mathematical Knowledge
% Management (MKM)''), large bodies of mathematical knowledge have to be converted into
% content form.
%
% Even though {\mathml} is viewed by most as the coming standard for representing
% mathematics on the web and in scientific publications, it has not not fully taken off in
% practice. One of the reasons for that may be that the technical communities that need
% high-quality methods for publishing mathematics already have an established method which
% yields excellent results: the {\TeX/\LaTeX} system: and a large part of mathematical
% knowledge is prepared in the form of {\TeX}/{\LaTeX} documents.
%
% {\TeX}~\cite{Knuth:ttb84} is a document presentation format that combines complex
% page-description primitives with a powerful macro-expansion facility, which is utilized in
% {\LaTeX} (essentially a set of {\TeX} macro packages, see~\cite{Lamport:ladps94}) to
% achieve more content-oriented markup that can be adapted to particular tastes via
% specialized document styles. It is safe to say that {\LaTeX} largely restricts content
% markup to the document structure\footnote{supplying macros e.g. for sections, paragraphs,
%   theorems, definitions, etc.}, and graphics, leaving the user with the presentational
% {\TeX} primitives for mathematical formulae. Therefore, even though {\LaTeX} goes a great
% step into the direction of an MKM format, it is not, as it lacks infrastructure for
% marking up the functional structure of formulae and mathematical statements, and their
% dependence on and contribution to the mathematical context.
%
% \subsection{The {\xml} vs. {\TeX/\LaTeX} Formats and Workflows}
%
% {\mathml} is an {\xml}-based markup format for mathematical formulae, it is standardized
% by the World Wide Web Consortium in {\cite{CarIon:MathML03}}, and is supported by the
% major browsers. The {\mathml} format comes in two integrated components: presentation
% {\mathml} presentation MathML and content {\mathml} content MathML. The former provides
% a comprehensive set of layout primitives for presenting the visual appearance of
% mathematical formulae, and the second one the functional/logical structure of the
% conveyed mathematical objects. For all practical concerns, presentation {\mathml} is
% equivalent to the math mode of {\TeX}. The text mode facilitates of {\TeX} (and the
% multitude of {\LaTeX} classes) are relegated to other {\xml} formats, which embed
% {\mathml}.
% 
% The programming language constructs of {\TeX} (i.e. the macro definition
% facilities\footnote{We count the parser manipulation facilities of {\TeX}, e.g. category
%   code changes into the programming facilities as well, these are of course impossible for
%   {\mathml}, since it is bound to {\xml} syntax.}) are relegated to the {\xml}
% programming languages that can be used to develop language extensions. 
% transformation language {\xslt}~\cite{Deach:exls99,Kay:xpr00} or proper {\xml}-enabled
% The {\xml}-based syntax and the separation of the presentational-, functional- and
% programming/extensibility concerns in {\mathml} has some distinct advantages over the
% integrated approach in {\TeX/\LaTeX} on the services side: {\mathml} gives us better
% \begin{itemize}
% \item integration with web-based publishing,
% \item accessibility to disabled persons, e.g. (well-written) {\mathml} contains enough
%   structural information to supports screen readers.
% \item reusability, searchabiliby and integration with mathematical software systems
%   (e.g. copy-and-paste to computer algebra systems), and
% \item validation and plausibility checking.
% \end{itemize}
% 
% On the other hand, {\TeX/\LaTeX}/s adaptable syntax and tightly integrated programming
% features within has distinct advantages on the authoring side:
%  
% \begin{itemize}
% \item The {\TeX/\LaTeX} syntax is much more compact than {\mathml} (see the difference in
%   Figure~\ref{fig:mathml-sum} and Equation ~\ref{eq:cmathml-sum}), and if needed, the
%   community develops {\LaTeX} packages that supply new functionality in with a succinct
%   and intuitive syntax.
% \item The user can define ad-hoc abbreviations and bind them to new control sequences to
%   structure the source code.
% \item The {\TeX/\LaTeX} community has a vast collection of language extensions and best
%   practice examples for every conceivable publication purpose and an established and very
%   active developer community that supports these.
% \item There is a host of software systems centered around the {\TeX/\LaTeX} language that
%   make authoring content easier: many editors have special modes for {\LaTeX}, there are
%   spelling/style/grammar checkers, transformers to other markup formats, etc.
% \end{itemize}
%
% In other words, the technical community is is heavily invested in the whole
% {\index*{workflow}}, and technical know-how about the format permeates the
% community. Since all of this would need to be re-established for a {\mathml}-based
% workflow, the technical community is slow to take up {\mathml} over {\TeX/\LaTeX}, even in
% light of the advantages detailed above.
% 
% \subsection{A {\LaTeX}-based Workflow for {\xml}-based Mathematical Documents}
% 
% An elegant way of sidestepping most of the problems inherent in transitioning from a
% {\LaTeX}-based to an {\xml}-based workflow is to combine both and take advantage of the
% respective advantages.
% 
% The key ingredient in this approach is a system that can transform {\TeX\LaTeX} documents
% to their corresponding {\xml}-based counterparts. That way, {\xml}-documents can be
% authored and prototyped in the {\LaTeX} workflow, and transformed to {\xml} for
% publication and added-value services, combining the two workflows. 
% 
% There are various attempts to solve the {\TeX/\LaTeX} to {\xml} transformation problem
% (see ~\cite{StaGinDav:maacl09} for an overview); the most mature is probably Bruce
% Miller's {\latexml} system~\cite{Miller:latexml:online}. It consists of two parts: a
% re-implementation of the {\TeX} {\index*{analyzer}} with all of it's intricacies, and a
% extensible {\xml} emitter (the component that assembles the output of the parser). Since
% the {\LaTeX} style files are (ultimately) programmed in {\TeX}, the {\TeX} analyzer can
% handle all {\TeX} extensions, including all of {\LaTeX}. Thus the {\latexml} parser can
% handle all of {\TeX/\LaTeX}, if the emitter is extensible, which is guaranteed by the
% {\latexml} binding language: To transform a {\TeX/\LaTeX} document to a given {\xml}
% format, all {\TeX} extensions\footnote{i.e. all macros, environments, and syntax
%   extensions used int the source document} must have ``{\latexml}
% bindings''\index{LaTeXML}{binding}, i.e. a directive to the {\latexml} emitter that
% specifies the target representation in {\xml}.
%
% \section{The Packages of the \protect\stex Collection}\label{sec:packages}
%
% In the following, we will shortly preview the packages and classes in the {\stex}
% collection. They all provide part of the solution of representing semantic structure in
% the {\TeX/\LaTeX} workflow. We will group them by the conceptual level they
% address\ednote{come up with a nice overview figure here!}
%
% \subsection{Content Markup of Mathematical Formulae in {\TeX/\LaTeX}}
%
% The first two packages are concerned basically with the math mode in {\TeX},
% i.e. mathematical formulae. The underlying problem is that run-of-the-mill {\TeX/\LaTeX}
% only specifies the presentation (i.e. what formulae look like) and not their content
% (their functional structure). Unfortunately, there are no good methods (yet) to infer the
% latter from the former, but there are ways to get presentation from content.
% 
% Consider for instance the following ``standard notations''\footnote{The first one is
%   standard e.g. in Germany and the US, and the last one in France} for binomial
% coefficients: $\left(n\atop k\right)$, $_nC^k$, $\mathcal{C}^n_k$ all mean the same thing:
% $n!\over k!(n-k)!$. This shows that we cannot hope to reliably recover the functional
% structure (in our case the fact that the expression is constructed by applying the
% binomial function to the arguments $n$ and $k$) from the presentation alone.
% 
% The solution to this problem is to dump the extra work on the author (after all she
% knows what she is talking about) and give them the chance to specify the intended
% structure. The markup infrastructure supplied by the {\stex} collection lets the author
% do this without changing\footnote{However, semantic annotation will make the author more
% aware of the functional structure of the document and thus may in fact entice the author
% to use presentation in a more consistent way than she would usually have.} the visual
% appearance, so that the {\LaTeX} workflow is not disrupted. . We speak of semantic
% preloading for this process and call our collection of macro packages {\stex} (Semantic
% {\TeX}). For instance, we can now write
% \begin{equation}\label{eq:cmathml-sum}
%   \verb|\CSumLimits{k}1\infty{\Cexp{x}k}| \quad\hbox{instead of the usual}\quad
%   \verb|\sum_{k=1}^\infty x^k|
% \end{equation}
%
% In the first form, we specify that you are applying a function (|CSumLimits| $\hat=$ Sum
% with Limits) to 4 arguments:
% \begin{inparaenum}[\em i)]
% \item the bound variable $k$ (that runs from)
% \item the number 1 (to)
% \item $\infty$ (to infinity summing the terms)
%   \item \verb|\Cexp{x}k| (i.e. x to the
%   power k).
% \end{inparaenum}
% In the second form, we merely specify hat {\LaTeX} should draw a capital Sigma character
% ($\Sigma$) with a lower index which is an equation $k=1$ and an upper index $\infty$. Then
% it should place next to it an $x$ with an upper index $k$.
%
% Of course human readers (that understand the math) can infer the content structure from
% this presentation, but the {\latexml} converter (who does not understand the math) cannot,
% but we want to have the content {\mathml} expression in Figure~\ref{fig:mathml-sum}.
% \begin{exfig}
% \begin{lstlisting}[language=MathML,belowskip=-1ex,aboveskip=-1ex]
%  <math xmlns="http://www.w3.org/1998/Math/MathML">
%    <bind> 
%      <apply><sumlimits/><cn>1</cn><infinity/></apply>
%      <bvar><ci>k</ci></bvar>
%      <apply><exp/><ci>x</ci><ci>k</ci></apply>
%     </bind>
%  </math>
% \end{lstlisting}
%   \caption{Content {\mathml} Form of $\sum_{k=1}^\infty x^k$}\label{fig:mathml-sum}
% \end{exfig}
% 
% Obviously, a converter can infer this from the first {\LaTeX} structure with the help of
% the curly braces that indicate the argument structure, but not from the second (because it
% does not understand the math).
%
% The nice thing about the \verb|cmathml| infrastructure is that you can still run {\LaTeX}
% over the first form and get the same formula in the DVI file that you would have gotten
% from running it over the second form. That means, if the author is prepared to write the
% mathematical formulae a little differently in her {\LaTeX} sources, then she can use them
% in {\xml} and {\LaTeX} at the same time.
%
% \subsubsection{{\texttt{cmathml}}: Encoding Content {\mathml} in {\TeX/\LaTeX}}
%
% The {\texttt{cmathml}} package (see~\ctancite{Kohlhase:tbscml}) provides a set of macros that
% correspond to the K-14 fragment of mathematics (Kindergarten to undergraduate college
% level ($\hat=14^{th}$ grade)). We have already seen an example above in equation
% (\ref{eq:cmathml-sum}), where the content markup in {\TeX} corresponds to a content
% {\mathml}-expression (and can actually be translated to this by the {\latexml} system.)
% However, the content {\mathml} vocabulary is fixed in the {\mathml} specification and
% limited to the K-14 fragment; the notation of mathematics of course is much larger and
% extensible on the fly.
%
% \subsubsection{{\texttt{presentation}}: Flexible Presentation for Semantic Macros}
%
% The {\texttt{presentation}} package (see~\ctancite{Kohlhase:ipsmsl}) supplies an
% infrastructure that allows to specify the presentation of semantic macros, including
% preference-based bracket elision. This allows to markup the functional structure of
% mathematical formulae without having to lose high-quality human-oriented presentation in
% {\LaTeX}. Moreover, the notation definitions can be used by MKM systems for added-value
% services, either directly from the {\sTeX} sources, or after translation.
% \begin{figure}[ht]\centering
% \begin{tikzpicture}[xscale=1.1]\tt
%   \node (metakeys) at (0,0) {metakeys};
%   \node (cpath) at (-2,0) {cpath};
%   \node (presentation) at (2.5,0) {presentation};
%
%   \node (sref) at (0,1) {sref};
%   \node (cmath) at (2.5,1) {cmath};
%
%   \node (rdfmeta) at (-2,2) {rdfmeta};
%   \node (modules) at (0,2) {modules};
%   \node (omdoc) at (1.5,2)  {omdoc};
%   \node (sproof) at (3,2)  {sproof};
%
%   \node (wa) at (-2,3) {workaddress};
%   \node (omtext) at (0,3) {omtext};
%   \node (structview) at (3,3)  {structview};
%
%   \node (dcm) at (-2,4) {dcm};
%   \node (statements) at (0,4) {statements};
%   \node (stex-logo) at (3.5,4)  {stex-logo};
%
%   \node (problem) at (4.5,5) {problem};
%   \node (tikzinput) at (2.5,5) {tikzinput};
%   \node (stex) at (0,5) {stex};
%   \node (smultiling) at (-2,5) {smultiling};
%
%   \node (smglomsty) at (-2,6) {smglom.sty};
%   \node (mikoslidessty) at (.5,6) {mikoslides.sty};
%   \node (hwexamsty) at (4.5,6) {hwexam.sty};
%
%   \node (smglomcls) at (-2,7) {smglom.cls};
%   \node (mikoslidescls) at (.5,7) {mikoslides.cls};
%   \node (hwexamcls) at (4.5,7) {hwexam.cls};
%   \node (omdoccls) at (2.5,6.5) {omdoc.cls};
%
%   \draw[->] (sref) -- (metakeys);
%   \draw[->] (cmath) -- (presentation);
%   \draw[->] (rdfmeta) -- (sref);
%   \draw[->] (wa) -- (modules);
%   \draw[->] (modules) -- (sref);
%   \draw[->] (modules) -- (cpath);
%   \draw[->] (omdoc) -- (sref);
%   \draw[->] (sproof) -- (sref);
%   \draw[->] (dcm) to[bend right=70] (rdfmeta);
%   \draw[->] (dcm) -- (wa);
%   \draw[->] (omtext) -- (modules);
%   \draw[->] (statements) -- (omtext);
%   \draw[->] (stex) -- (statements);
%   \draw[->] (stex) -- (dcm);
%   \draw[->] (stex) to[bend left=5] (sproof);
%   \draw[->] (stex) to[bend left=5] (structview);
%   \draw[->] (structview) -- (modules);
%   \draw[->] (stex) to[bend left=15] (cmath);
%   \draw[->] (stex) to[bend left=20] (omdoc);
%   \draw[->] (stex) --  (stex-logo);
%   \draw[->] (problem) -- (omtext);
%   \draw[->] (smglomsty) -- (smultiling);
%   \draw[->] (smglomsty) -- (statements);
%   \draw[->] (smglomcls) -- (smglomsty);
%   \draw[->] (smglomcls) -- (stex);
%   \draw[->] (mikoslidescls) -- (mikoslidessty);
%   \draw[->] (mikoslidescls) -- (smglomsty);
%   \draw[->] (mikoslidessty) -- (tikzinput);
%   \draw[->] (mikoslidessty) -- (stex);
%   \draw[->] (mikoslidessty) -- (smglomsty);
%   \draw[->] (hwexamcls) -- (hwexamsty);
%   \draw[->] (hwexamsty) -- (problem);
%   \draw[->] (omdoccls) to[bend right=20] (omdoc);
%
%   \draw[->] (hwexamcls) -- (tikzinput);
%   \draw[->] (hwexamcls) to[bend right=10] (omdoccls);
%   \draw[->] (mikoslidescls) to[bend left=10] (omdoccls);
%   \draw[->] (smglomcls) to[bend right=10] (omdoccls);
% \end{tikzpicture}
% \caption{The \protect\sTeX packages and their dependencies.}
% \end{figure}
% \subsection{Mathematical Statements}
%
% \subsubsection{{\texttt{statements}}: Extending Content Macros for Mathematical Notation}
% 
% The \texttt{statements} package (see\ctancite{Kohlhase:smms}) provides semantic markup
% facilities for mathematical statements like Theorems, Lemmata, Axioms, Definitions,
% etc. in {\stex} files. This structure can be used by MKM systems for added-value services,
% either directly from the {\sTeX} sources, or after translation.
%
% \subsubsection{{\texttt{sproof}}: Extending Content Macros for Mathematical Notation}
%
% The \texttt{sproof} package (see~\ctancite{Kohlhase:smp})supplies macros and environment
% that allow to annotate the structure of mathematical proofs in {\stex} files. This
% structure can be used by MKM systems for added-value services, either directly from the
% {\sTeX} sources, or after translation.
%
% \subsection{Context Markup for Mathematics}
%
% \subsubsection{{\texttt{modules}}: Extending Content Macros for Mathematical Notation}
% 
% The \texttt{modules} package (see~\ctancite{KohAmb:smmssl}) supplies a definition
% mechanism for semantic macros and a non-standard scoping construct for them, which is
% oriented at the semantic dependency relation rather than the document structure. This
% structure can be used by MKM systems for added-value services, either directly from the
% {\sTeX} sources, or after translation.
%
% \subsection{Mathematical Document Classes}
%
% \subsubsection{OMDoc Documents}
%
% The \texttt{omdoc} package provides an infrastructure that allows to markup {\omdoc}
% documents in {\LaTeX}. It provides \texttt{omdoc.cls}, a class with the and
% {\texttt{omdocdoc.sty}}
%
% \subsection{Conclusion}\label{sec:concl}
%
% The {\stex} collection provides a set of semantic macros that extends the familiar and
% time-tried {\LaTeX} workflow in academics until the last step of Internet publication of
% the material. For instance, an {\smglom} module can be authored and maintained in
% {\LaTeX} using a simple text editor, a process most academics in technical subjects are
% well familiar with. Only in a last publishing step (which is fully automatic) does it get
% transformed into the {\xml} world, which is unfamiliar to most academics. 
%
% Thus, {\stex} can serve as a conceptual interface between the document author and MKM
% systems: Technically, the semantically preloaded {\LaTeX} documents are transformed into
% the (usually {\xml}-based) MKM representation formats, but conceptually, the ability to
% semantically annotate the source document is sufficient.
% 
% The {\stex} macro packages have been validated together with a case
% study~\cite{Kohlhase04:stex}, where we semantically preload the course materials for a
% two-semester course in Computer Science at Jacobs University Bremen and transform them to
% the {\omdoc} MKM format.
%
% \subsection{Licensing, Download and Setup}\label{sec:setup}
% 
% The {\stex} packages are licensed under the {\LaTeX} Project Public License~\cite{LPPL},
% which basically means that they can be downloaded, used, copied, and even modified by
% anyone under a set of simple conditions (e.g. if you modify you have to distribute under a
% different name). 
%
% The {\stex} packages and classes are available from the Comprehensive {\TeX} Archive
% Network (CTAN~\cite{CTAN:on}) and are part of the primary {\TeX/\LaTeX} distributions
% (e.g. TeXlive~\cite{TeXLive:on} and MikTeX~\cite{MiKTeX:on}). The development version is
% on GitHub~\cite{sTeX:github:on}, it can cloned or forked from the repository URL
% \begin{center}
%   \url{https://github.com/KWARC/sTeX.git}
% \end{center}
% It is usually a good idea to enlarge the internal memory allocation of the \TeX/\LaTeX executables. This can be done by
% adding the following configurations in \texttt{texmf.cnf} (or changing them, if they
% alreday exist). Note that you will probably need \texttt{sudo} to do this. 
% \begin{lstlisting}
% max_in_open = 50        % simultaneous input files and error insertions, 
% param_size = 20000      % simultaneous macro parameters, also applies to MP
% nest_size = 1000        % simultaneous semantic levels (e.g., groups)
% stack_size = 10000      % simultaneous input sources
% main_memory = 12000000
% \end{lstlisting}
% After that, you have to run the 
% \begin{lstlisting}
% sudo fmtutil-sys --all
% \end{lstlisting}
%
% \section{Utilities}\label{sec:utilities}
%
% To simplify dealing with {\stex} documents, we are providing a small collection of command
% line utilities, which we will describe here. For details and downloads go to
% {\url{http://kwarc.info/projects/stex}}.
%
% \begin{description}
% \item[{\texttt{msplit}}] splits an {\stex} file into smaller ones (one module per file)
% \item[{\texttt{rf}}] computes the ``reuse factor'', i.e. how often {\stex} modules are reused
%   over a collection of documents 
% \item[{\texttt{sgraph}}] visualizes the module graph
% \item[{\texttt{sms}}] computes the {\stex} module signatures for a give {\stex} file
% \item[{\texttt{bms}}] proposes a sensible module structure for an un-annotated {\stex} file
% \end{description}
%
% \StopEventually{\newpage\PrintChanges}
% \newpage
%
% \section{The Implementation}\label{sec:implementation}
%
% \subsection{Package Options}\label{sec:impl:options}
% 
% The first step is to declare (a few) package options that handle whether certain
% information is printed or not. They all come with their own conditionals that are set by
% the options.
%
%    \begin{macrocode}
%<*package>
\DeclareOption*{\PassOptionsToPackage{\CurrentOption}{statements}
                           \PassOptionsToPackage{\CurrentOption}{structview}
                           \PassOptionsToPackage{\CurrentOption}{sproofs}
                           \PassOptionsToPackage{\CurrentOption}{omdoc}
                           \PassOptionsToPackage{\CurrentOption}{cmath}
                           \PassOptionsToPackage{\CurrentOption}{dcm}}
\ProcessOptions
%    \end{macrocode}
%
% Then we make sure that the necessary packages are loaded (in the right versions).
%    \begin{macrocode}
\RequirePackage{stex-logo}
\RequirePackage{statements}
\RequirePackage{structview}
\RequirePackage{sproof}
\RequirePackage{omdoc}
\RequirePackage{cmath}
\RequirePackage{dcm}
%</package>
%    \end{macrocode}
%
% \subsection{The \protect\sTeX Logo}\label{sec:impl:ids}
%
% To provide default identifiers, we tag all elements that allow |xml:id| attributes by
% executing the |numberIt| procedure from |omdoc.sty.ltxml|.
% 
%    \begin{macrocode}
%<*logo>
\RequirePackage{xspace}
\def\stex{\@ifundefined{texorpdfstring}{\let\texorpdfstring\@firstoftwo}{}\texorpdfstring{\raisebox{-.5ex}S\kern-.5ex\TeX}{sTeX}\xspace}
\def\sTeX{\stex}
%</logo>
%    \end{macrocode}
% \Finale
\endinput
% \iffalse
% LocalWords:  GPL structuresharing STR dtx pts keyval xcomment CPERL DefKeyVal iffalse
%%% Local Variables: 
%%% mode: doctex
%%% TeX-master: t
%%% End: 
% \fi
% LocalWords:  RequirePackage Semiverbatim DefEnvironment OptionalKeyVals soln texttt baz
% LocalWords:  exnote DefConstructor inclprob NeedsTeXFormat omd.sty textbackslash exfig
%  LocalWords:  stopsolution fileversion filedate maketitle setcounter tocdepth newpage
%  LocalWords:  tableofcontents showmeta showmeta solutionstrue usepackage minipage hrule
%  LocalWords:  linewidth elefants.prob Elefants smallskip noindent textbf startsolutions
%  LocalWords:  startsolutions stopsolutions stopsolutions includeproblem includeproblem
%  LocalWords:  textsf HorIacJuc cscpnrr11 includemhproblem includemhproblem importmodule
%  LocalWords:  importmhmodule foobar ldots latexml mhcurrentrepos mh-variants mh-variant
%  LocalWords:  compactenum langle rangle langle rangle ltxml metakeys newif ifexnotes rm
%  LocalWords:  exnotesfalse exnotestrue ifhints hintsfalse hintstrue ifsolutions ifpts
%  LocalWords:  solutionsfalse ptsfalse ptstrue ifmin minfalse mintrue ifboxed boxedfalse
%  LocalWords:  boxedtrue sref mdframed marginpar prob srefaddidkey addmetakey refnum kv
%  LocalWords:  newcounter ifx thesection theproblem hfill newenvironment metasetkeys ltx
%  LocalWords:  stepcounter currentsectionlevel xspace ignorespaces surroundwithmdframed
%  LocalWords:  omdoc autoopen autoclose solvedinminutes kvi qw vals newcommand exhint
%  LocalWords:  specialcomment excludecomment mhrepos xref marginpar addtocounter doctex
%  LocalWords:  mh@currentrepos endinput
%
