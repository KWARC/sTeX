% \iffalse meta-comment
% A LaTeX Class and Package for the SMGloM Glossary
% Copyright (c) 2009 Michael Kohlhase, all rights reserved
%               this file is released under the
%               LaTeX Project Public License (LPPL)
%
% The original of this file is in the public repository at 
% http://github.com/KWARC/sTeX/
% \fi
%   
% \iffalse
%<package|cls>\NeedsTeXFormat{LaTeX2e}[1999/12/01]
%<cls>\ProvidesClass{smglom}[2013/05/27 v0.1 Semantic Multilingual Glossary for Math]
%<sty>\ProvidesPackage{smglom}[2013/05/27 v0.1 Semantic Multilingual Glossary for Math]
%
%<*driver>
\documentclass{ltxdoc}
\usepackage{url,array,omdoc,omtext,float}
\usepackage[show]{ed}
\usepackage[hyperref=auto,style=alphabetic]{biblatex}
\bibliography{kwarc}
\usepackage{stex-logo}
\usepackage{../ctangit}
\usepackage{hyperref}
\makeindex
\floatstyle{boxed}
\newfloat{exfig}{thp}{lop}
\floatname{exfig}{Example}
\def\tracissue#1{\cite{sTeX:online}, \hyperlink{hstp://trac.kwarc.info/sTeX/ticket/#1}{issue #1}}
\begin{document}\DocInput{smglom.dtx}\end{document}
%</driver>
% \fi
% 
%\iffalse\CheckSum{382}\fi
% 
% \changes{v0.1}{2013/12/17}{First Version}
% \changes{v0.1}{2014/2/17}{Renamed ad split into \texttt{smglom.sty} and \texttt{smglom.cls}}
%
% 
% \MakeShortVerb{\|}
%
% \def\omdoc{OMDoc}
% \def\latexml{{\LaTeX}ML}
% \title{{\texttt{smglom.cls/sty}}: Semantic Multilingual Glossary for Math}
%    \author{Michael Kohlhase\\
%            Jacobs University, Bremen\\
%            \url{http://kwarc.info/kohlhase}}
% \maketitle
%
% \begin{abstract}
%   The |smglom| package is part of the {\sTeX} collection, a version of {\TeX/\LaTeX}
%   that allows to markup {\TeX/\LaTeX} documents semantically without leaving the
%   document format, essentially turning {\TeX/\LaTeX} into a document format for
%   mathematical knowledge management (MKM).
%
%   This package supplies an infrastructure for writing {\omdoc} glossary entries. 
% \end{abstract}
%
%\tableofcontents\newpage
% 
%\begin{omgroup}[id=sec:STR]{Introduction}
%
% \end{omgroup}
% 
% \begin{omgroup}[id=sec:user]{The User Interface}\
% 
%
% \begin{omgroup}[id=sec:user:options]{Package and Class Options}
%   |smglom.cls| accepts all options of the |omdoc.cls| and |article.cls| and just passes
%   them on to these. 
% \end{omgroup}
% 
% \end{omgroup}
% 
% \StopEventually{\newpage\PrintIndex\newpage\PrintChanges\printbibliography}\newpage
%
% \begin{omgroup}[id=sec:impl:cls]{Implementation: The SMGloM Class}
%
% \begin{omgroup}[id=sec:impl:cls:options]{Class Options}
% To initialize the |smglom| class, we pass on all options to |omdoc.cls|
% 
%    \begin{macrocode}
%<*cls>
\DeclareOption*{\PassOptionsToClass{\CurrentOption}{omdoc}}
\ProcessOptions
%</cls>
%<*ltxml.cls|ltxml.sty>
# -*- CPERL -*-
package LaTeXML::Package::Pool;
use strict;
use LaTeXML::Package;
DeclareOption(undef,sub {PassOptions('article','cls',ToString(Digest(T_CS('\CurrentOption')))); });
ProcessOptions();
%</ltxml.cls|ltxml.sty>
%    \end{macrocode}
%
% We load |omdoc.cls|, and the desired packages. For the {\latexml} bindings, we make
% sure the right packages are loaded.
%
%    \begin{macrocode}
%<*cls>
\LoadClass{omdoc}
\RequirePackage{smglom}
%</cls>
%<*sty>
\RequirePackage{amstext}
\RequirePackage{modules}
\RequirePackage{dcm}
\RequirePackage{statements}
\RequirePackage{sproof}
\RequirePackage{cmath}
\RequirePackage{presentation}
\RequirePackage{amsfonts}
%</sty>
%<*ltxml.cls>
LoadClass('omdoc');
RequirePackage('smglom');
%</ltxml.cls>
%<*ltxml.sty>
RequirePackage('amstext');
RequirePackage('modules'); 
RequirePackage('dcm'); 
RequirePackage('statements');
RequirePackage('cmath');
RequirePackage('presentation');
RequirePackage('amsfonts');
%</ltxml.sty>
%    \end{macrocode}
% \end{omgroup}
%
% \begin{omgroup}[id=sec:module-defs]{For Module Definitions}
%
% \begin{macro}{gimport}
%   just a shortcut
%    \begin{macrocode}
%<ltxml.sty>RawTeX('
%<*sty|ltxml.sty>
\newcommand\gimport[2][]{\def\@test{#1}%
\edef\mh@@repos{\mh@currentrepos}%
\ifx\@test\@empty\importmhmodule[repos=\mh@@repos,ext=tex,path=#2]{#2}%
\else\importmhmodule[repos=#1,ext=tex,path=#2]{#2}\fi
\mhcurrentrepos\mh@@repos\ignorespaces}
%    \end{macrocode}
% \end{macro}
%
% \begin{macro}{guse}
%   just a shortcut
%    \begin{macrocode}
\newcommand\guse[2][]{\def\@test{#1}%
\edef\mh@@repos{\mh@currentrepos}%
\ifx\@test\@empty\usemhmodule[repos=\mh@@repos,ext=tex,path=#2]{#2}%
\else\usemhmodule[repos=#1,ext=tex,path=#2]{#2}\fi
\mhcurrentrepos\mh@@repos\ignorespaces}
%    \end{macrocode}
% \end{macro}
%
% \begin{macro}{gadopt}
%   just a shortcut
%    \begin{macrocode}
\newcommand\gadopt[2][]{\def\@test{#1}%
\edef\mh@@repos{\mh@currentrepos}%
\ifx\@test\@empty\adoptmhmodule[repos=\mh@@repos,ext=tex,path=#2]{#2}%
\else\adoptmhmodule[repos=#1,ext=tex,path=#2]{#2}\fi
\mhcurrentrepos\mh@@repos\ignorespaces}
%    \end{macrocode}
% \end{macro}
%
% \begin{environment}{gview}
%   The |gview| environment is just a layer over the |view| environment with the keys
%   suitably adapted.
%    \begin{macrocode}
\newenvironment{gview}[3][]{\metasetkeys{mhview}{#1}\def\@test{#1}%
\edef\from@repos{\ifx\mhview@fromrepos\@empty\mh@currentrepos\else\mhview@fromrepos\fi}%
\edef\to@repos{\ifx\mhview@torepos\@empty\mh@currentrepos\else\mhview@torepos\fi}%
\ifx\@test\@empty%
\begin{mhview}[fromrepos=\from@repos,frompath=#2,torepos=\to@repos,topath=#3,ext=tex]{#2}{#3}%
\else%
\begin{mhview}[fromrepos=\from@repos,frompath=#2,torepos=\to@repos,topath=#3,ext=tex]{#2}{#3}%
\fi}
{\end{mhview}}
%</sty|ltxml.sty>
%<ltxml.sty>');
%    \end{macrocode}
% \end{environment}
%
% \begin{macro}{symbol}
%   has a starred form for primary symbols.
%    \begin{macrocode}
%<*sty>
\def\symbol{\@ifstar{\@symbol}{\@symbol@star}}
\def\@symbol#1{\if@importing\else Symbol: \textsf{#1}\fi}
\def\@symbol@star#1{\if@importing\else Primary Symbol: \textsf{#1}\fi}
%</sty>
%<*ltxml.sty>
DefConstructor('\symbol OptionalMatch:* {}',
     "<omdoc:symbol ?#1(role='primary')(role='secondary') name='#2'/>");
%</ltxml.sty>
%    \end{macrocode}
% \end{macro}
%
% \begin{macro}{*nym}
%    \begin{macrocode}
%<*sty>
\newcommand\hypernym[3][]{\if@importing\else\par\noindent #2 is a hypernym of #3\fi}
\newcommand\hyponym[3][]{\if@importing\else\par\noindent #2 is a hyponym of  #3\fi}
\newcommand\meronym[3][]{\if@importing\else\par\noindent #2 is a meronym of #3\fi}
%</sty>
%<*ltxml.sty>
DefConstructor('\hypernym [] {}{}',"");
DefConstructor('\hyponym [] {}{}',"");
DefConstructor('\meronym [] {}{}',"");
%</ltxml.sty>
%    \end{macrocode}
% \end{macro}
%
% \begin{macro}{\MSC}
%   to define the Math Subject Classification, \ednote{MK: what to do for the LaTeXML side?}
%    \begin{macrocode}
%<*sty>
\newcommand\MSC[1]{\if@importing\else MSC: #1\fi}
%</sty>
%<*ltxml.sty>
DefConstructor('\MSC{}',"");
%</ltxml.sty>
%    \end{macrocode}
% \end{macro}
% \end{omgroup}
% 
% \begin{omgroup}[id=sec:langbindings]{For Language Bindings}
%
% This functionality must be moved to the |smultiling| package. 
% 
% \begin{environment}{gve}
%   The |gve| environment is just a layer over the |mhviewsketch| environment with the keys
%   suitably adapted.
%    \begin{macrocode}
%<ltxml.sty>RawTeX('
%<*sty|ltxml.sty>
\newenvironment{gve}[5][]{\metasetkeys{mhview}{#1}\def\@test{#1}%
\edef\from@repos{\ifx\mhview@fromrepos\@empty\mh@currentrepos\else\mhview@fromrepos\fi}%
\edef\to@repos{\ifx\mhview@torepos\@empty\mh@currentrepos\else\mhview@torepos\fi}%
\ifx\@test\@empty%
\begin{mhviewsketch}[id=#2.#3,fromrepos=\from@repos,frompath=#2,torepos=\to@repos,topath=#3,ext=tex]{#4}{#5}%
\else%
\begin{mhviewsketch}[id=#2.#3,fromrepos=\from@repos,frompath=#2,torepos=\to@repos,topath=#3,ext=tex]{#4}{#5}%
\fi
\smg@select@language{#3}}
{\end{mhviewsketch}}
%</sty|ltxml.sty>
%<ltxml.sty>');
%    \end{macrocode}
% \end{environment}
% \end{omgroup}
% \end{omgroup} 
% \Finale
\endinput
% \iffalse
%%% Local Variables: 
%%% mode: doctex
%%% TeX-master: t
%%% End: 
% \fi

% LocalWords:  iffalse cls smglo smglo.dtx omdoc latexml texttt smlog.cls sref SMGloM
% LocalWords:  maketitle newpage tableofcontents newpage omgroup ednote ltxml smglom.dtx
% LocalWords:  printbibliography showmeta metakeys amstext ginput newcommand sproof cmath
% LocalWords:  module-defs gimport renewcommand langbindings gle newenvironment amsfonts
% LocalWords:  doctex NeedsTeXFormat langfiles ngerman smultiling
